P1 und P2 sind bereits mit dem Server verbunden. P1 kann entweder mit einem anderen Mensch-Spieler spielen, oder einen KI-Spieler zu seiner Lobby hinzufügen. Sind beide Spieler in der Lobby und signalisieren dem Server, dass sie bereit sind, so kann entweder P1 oder P2 die Spielpartie starten.\\
Zum Erstellen einer neuen Lobby gehört auch die Konfiguration des Editors mit Szenario, Partie-Konfiguration und Charakter-Beschreibung. Diese wurde hier nicht explizit aufgeführt. \\
\begin{tikzpicture}
\begin{umlseqdiag} 
	\umlactor[class=Mensch-Spieler]{P1}
	\umlobject[no ddots]{Server}
	\umlactor[class=Spieler]{P2}

	
	\begin{umlcall}[op=newLobby(String), dt=5]{P1}{Server}
		\begin{umlcall}[type=return, op=lobbyCreated]{Server}{P1}
		\end{umlcall}
		
		\begin{umlcall}[op=joinLobby(String)]{P1}{Server}
			\begin{umlcall}[type=return, op=lobbyInfo]{Server}{P1}
			\end{umlcall}
		\end{umlcall}
	
	
		\begin{umlfragment}[type=alt, label=P2 is KI, inner xsep=10]
			\begin{umlcall}[op=addKiToLobby, dt = 10]{P1}{Server}
			\end{umlcall}
			\begin{umlcall}[op=lobbyInfo]{Server}{P2}
			\end{umlcall}
			
			\umlfpart[P2 is Mensch]
		
			\begin{umlcall}[op=joinLobby(String)]{P2}{Server}
				\begin{umlcall}[type=return, op=lobbyInfo]{Server}{P2}
				\end{umlcall}
			\end{umlcall}
		\end{umlfragment}
		
	\end{umlcall}
	
	
	
	\begin{umlcall}[op=ready]{P1}{Server}
	\begin{umlcall}[op=ready, dt=10]{P2}{Server}
	\end{umlcall}
	\end{umlcall}
	
	\begin{umlfragment}[type=alt, label=P1, inner xsep=7] 
		\begin{umlcall}[op=start, dt=10]{P1}{Server} 
		\end{umlcall}
	
		\umlfpart[P2]
		
		\begin{umlcall}[op=start, dt=20]{P2}{Server}
		\end{umlcall}
	\end{umlfragment}
	
	
    
\end{umlseqdiag}
\end{tikzpicture}