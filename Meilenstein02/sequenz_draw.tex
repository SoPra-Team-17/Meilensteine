In diesem Diagramm wird die Wahl- und Ausrüstungsphase durch ein Sequenzdiagramm beschrieben.\\

\begin{tikzpicture}
	\begin{umlseqdiag}
		\umlactor[class=Spieler]{P1}
		\umlactor[no ddots]{Server}
		\umlactor[class=Spieler]{P2}

		\begin{umlfragment}[type=Wahlphase,inner xsep=7]
			\begin{umlcall}[op={3 Char./ 3 Gadgets},dt=10,return={Gewähltes Gadget/Char.},padding=5]{Server}{P1}
			\end{umlcall}
			\begin{umlcall}[op={3 Char./ 3 Gadgets},dt=-6,return={Gewähltes Gadget/Char.},padding=5]{Server}{P2}
			\end{umlcall}
		\end{umlfragment}

		

		\begin{umlfragment}[type=Ausrüstungsphase,inner xsep=7]
			\begin{umlcall}[op={Gadget auf Char.},dt=10,return={Bestätigung},padding=5]{P1}{Server}
			\end{umlcall}
			\begin{umlcall}[op={Gadget auf Char.},dt=10,return={Bestätigung},padding=5]{P2}{Server}
			\end{umlcall}
		\end{umlfragment}
	\end{umlseqdiag}
\end{tikzpicture}

In der Wahlphase bietet der Server jedem Spieler drei Charaktere und drei Gadgets zur Auswahl an. 
Es wird dabei durch den Server sichergestellt, dass die Spieler nicht das gleiche Gadget oder den gleichen Charakter wählen können.\\
Eine vom Spieler zusammengestellte Fraktion besteht aus zwei bis vier Charakteren.
Insgesamt wird die im Diagramm beschriebene Wahlphase acht mal wiederholt.\\

In der Ausrüstungsphase muss jeder Spieler seine gewählten Gadgets einem seiner Agenten zuordnen. Die im Diagramm beschriebene Ausrüstungsphase wird solange wiederholt, bis alle Gadgets einem Charakter zugeordnet sind.
