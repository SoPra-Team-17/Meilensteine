Das untenstehende Zustandsdiagramm gibt einen kleinen Überblick über die grobe Funktionsweise des KI-Clients. 

\begin{tikzpicture} 
    \umlstateinitial[y=4, name=initial]
    \begin{umlstate}[name=running]{running}
        \begin{umlstate}[name=waiting, y=0]{Warte}
        \end{umlstate}

        \begin{umlstate}[name=analyzing, y=-4]{Analysiere Spielzustand}
        \end{umlstate}

        \begin{umlstate}[name=planning, y=-8]{Plane nächste Aktion}
        \end{umlstate}

        \umlHVHtrans[arm1=5, arg={Aktion Senden}, pos=0.5]{planning}{waiting}
        \umltrans[arg={Aktion Empfangen}, pos=0.5]{waiting}{analyzing}
        \umltrans{analyzing}{planning}

    \end{umlstate}

    \umltrans[arg={verbinde/}, pos=0.3]{initial}{waiting}
    \umlstatefinal[name=final, x=-6, y=-1]
    \umlHVtrans[arg={Spielende erreicht/}, pos=0.6]{waiting}{final}
\end{tikzpicture}
