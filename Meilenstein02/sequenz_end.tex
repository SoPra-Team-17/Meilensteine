Sobald ein Sieger feststeht wird diese Information je nach Art des Teilnehmers verarbeitet. Anschließend verlassen die Teilnehmer das Spiel und landen erneut bei \glqq\ref{Sequ_Lobby} \nameref{Sequ_Lobby}\grqq, wenn sie nicht nicht die Verbindung zum Server beenden.

\begin{tikzpicture} 
    \begin{umlseqdiag} 
        \umlactor[class=Zuschauer]{W}
        \umlobject[no ddots]{Server}
        \umlactor[class=Spieler]{P}
        
        \begin{umlcall}[op={action(result)}]{Server}{P}
        \begin{umlcall}[type=return, op={action(result)}]{Server}{W}
            
            \begin{umlfragment}[type=alt, label=Winner, inner xsep=8]
             
             	\begin{umlcallself}[op=displayWinner]{W}
             	\end{umlcallself}
	             	
             	
             	\begin{umlfragment}[type=alt, label=Mensch, inner xsep=7]
	             	\begin{umlcallself}[op=displayWinner, dt=5]{P}
	             	\end{umlcallself}
	             	
	             	\umlfpart[KI]
	             	
	             	\begin{umlcallself}[op=infoWinner]{P}
	             	\end{umlcallself}
             	\end{umlfragment}
             	
             	\begin{umlcall}[op={leaveGame}, dt=7]{W}{Server}
             	\begin{umlcall}[op={leaveGame}, dt=10]{P}{Server}
             	
				\end{umlcall}
				\end{umlcall}

            \end{umlfragment}
            
            
            \begin{umlfragment}[type=opt, inner xsep=7]
	            \begin{umlcall}[op={deregister}, dt=7]{W}{Server}
		            \begin{umlcall}[op={deregister}, dt=10]{P}{Server}
		            \end{umlcall}
	            \end{umlcall}
            \end{umlfragment}
        
            
		\end{umlcall}
        \end{umlcall}

    \end{umlseqdiag} 
\end{tikzpicture}
