In diesem Zustandsdiagramm werden abstrakt die Zustände beschrieben, welche während eines Spiels auftreten können.
\tikzset{singlestate/.style={draw,fill=yellow!20, rounded corners, align=center}}

\begin{tikzpicture}
	\umlstateinitial[name=initial, x=0, y=0]
	\umlstatefinal[name=final,x=8,y=-14]
	\begin{umlstate}[x=4, y=-10, name=turnp1]{Spieler 1 ist am Zug}
	\end{umlstate}

	\begin{umlstate}[x=12,y=-10,name=turnp2]{Spieler 2 ist am Zug}
	\end{umlstate}

	\begin{umlstate}[x=8,y=-6,name=apbp]{Bestimmen der AP/BP}
		\umlstatetext{Do/Bestimmen der AP und BP \\Abhängig von den \\gewählten Charaktere}
	\end{umlstate}

	\node[singlestate] at (3.5,-16)(charAction){Charakter gewählt};
	\node[singlestate] at (12.5,-16)(charAction2){Charakter gewählt};

	\umlHVtrans[anchor2=90]{initial}{apbp}

	\umlHVtrans[arg=Zug beendet,align=right,pos=1.5,anchor1=0,anchor2=260]{turnp1}{apbp}
	\umlHVtrans[arg=Zug beendet,align=left,pos=1.5,anchor1=180,anchor2=280]{turnp2}{apbp}

	\umlHVtrans[arg={Nächster Spieler},anchor1=175,anchor2=130,pos=0.05,align=right]{apbp}{turnp1}
	\umlHVtrans[arg={Nächster Spieler},anchor2=65,pos=0.8]{apbp}{turnp2}

	\umlHVtrans[arg={Siegbedingung},pos=1.5,align=right,anchor1=-20,anchor2=135]{turnp1}{final}
	\umlHVtrans[arg={Siegbedingung},pos=1.5,anchor1=200,anchor2=45,align=left]{turnp2}{final}
	
	\umltrans[arg={Überlanges Spiel  }, recursive=170|190|1.5cm, pos=2.1,align=right, recursive direction=left to left]{turnp1}{turnp1}
	\umltrans[arg=Überlanges Spiel, recursive=10|-10|1.5cm, pos=1.5, recursive direction=left to left]{turnp2}{turnp2}


	\umlHVtrans[arg=Charakter ausgewählt,anchor1=248,anchor2=100,align=right,pos=1.5]{turnp1}{charAction}
	\umlHVtrans[arg=Aktion/Bewegung,pos=1.5,align=left]{charAction}{turnp1}

	\umlHVtrans[arg=Charakter ausgewählt,anchor1=298,anchor2=80 ,align=left,pos=1.5]{turnp2}{charAction2}
	\umlHVtrans[arg=Aktion/Bewegung,align=right,pos=1.5]{charAction2}{turnp2}

\end{tikzpicture}
