In diesem Zustandsdiagramm werden abstrakt die Zustände beschrieben, welche während eines Spiels auftreten können.

\begin{tikzpicture}
	\umlstateinitial[name=initial, x=0, y=0]
	\umlstatefinal[name=final,x=8,y=-11]
	\begin{umlstate}[x=4, y=-7, name=turnp1]{Spieler 1 ist am Zug}
	\end{umlstate}

	\begin{umlstate}[x=12,y=-7,name=turnp2]{Spieler 2 ist am Zug}
	\end{umlstate}

	\begin{umlstate}[x=8,y=-3,name=apbp]{Bestimmen der AP/BP}
		\umlstatetext{Do/Bestimmen der AP und BP \\Abhängig von den \\gewählten Charaktere}
	\end{umlstate}
	
	\umlHVtrans[anchor2=145]{initial}{turnp1}
	\umltrans[arg={Zug beenden},pos=0.5,anchor1=10,anchor2=170]{turnp1}{turnp2}
	\umltrans[arg={Zug beenden},pos=0.5,anchor1=-170,anchor2=-10]{turnp2}{turnp1}

	\umltrans[arg=Zug beendet]{turnp1}{apbp}
	\umltrans[arg=Zug beendet]{turnp2}{apbp}
	\umlHVtrans[arg={Nächster Spieler},anchor1=175,anchor2=110,pos=0.4]{apbp}{turnp1}
	\umlHVtrans[arg={Nächster Spieler},anchor2=65,pos=0.8]{apbp}{turnp2}
	\umltrans[arg={Siegbedingung eingetreten},pos=0.5]{turnp1}{final}
	\umltrans[arg={Siegbedingung eingetreten},pos=0.5]{turnp2}{final}
	
	\umltrans[arg=Bewegen, recursive=155|165|3.5cm, pos=1.5, recursive direction=left to left]{turnp1}{turnp1}
	\umltrans[arg=Aktion, recursive=175|185|3.5cm, pos=1.5, recursive direction=left to left]{turnp1}{turnp1}
	\umltrans[arg=Überlanges Spiel, recursive=195|205|3.5cm, pos=1.5, recursive direction=left to left]{turnp1}{turnp1}

	\umltrans[arg=Bewegen, recursive=25|15|3.5cm, pos=1.5, recursive direction=left to left]{turnp2}{turnp2}
	\umltrans[arg=Aktion, recursive=5|-5|3.5cm, pos=1.5, recursive direction=left to left]{turnp2}{turnp2}
	\umltrans[arg=Überlanges Spiel, recursive=-15|-25|3.5cm, pos=1.5, recursive direction=left to left]{turnp2}{turnp2}
\end{tikzpicture}
