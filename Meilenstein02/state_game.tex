In diesem Zustandsdiagramm werden abstrakt die Zustände beschrieben, welche während eines Spiels auftreten können.

\begin{tikzpicture}
	\umlstateinitial[name=initial, x=0, y=0]
	\umlstatefinal[name=final,x=8,y=-11]
	\begin{umlstate}[x=4, y=-7, name=turnp1]{Spieler 1 ist am Zug}
	\end{umlstate}

	\begin{umlstate}[x=12,y=-7,name=turnp2]{Spieler 2 ist am Zug}
	\end{umlstate}

	\begin{umlstate}[x=8,y=-3,name=apbp]{Bestimmen der AP/BP}
		\umlstatetext{Do/Bestimmen der AP und BP \\Abhängig von den \\gewählten Charaktere}
	\end{umlstate}
	
	\umlHVtrans[anchor2=145]{initial}{turnp1}
	%\umltrans[arg={Zug beenden},pos=0.5,anchor1=10,anchor2=170]{turnp1}{turnp2}
	%\umltrans[arg={Zug beenden},pos=0.5,anchor1=-170,anchor2=-10]{turnp2}{turnp1}

	\umltrans[arg=Zug beendet,align=right,pos=0.6]{turnp1}{apbp}
	\umltrans[arg=Zug beendet,align=left,pos=0.6]{turnp2}{apbp}
	\umlHVtrans[arg={Nächster Spieler},anchor1=175,anchor2=130,pos=0.05,align=right]{apbp}{turnp1}
	\umlHVtrans[arg={Nächster Spieler},anchor2=65,pos=0.8]{apbp}{turnp2}
	\umltrans[arg={Siegbedingung},pos=0.5,align=left,anchor1=270,anchor2=180]{turnp1}{final}
	\umltrans[arg={Siegbedingung},pos=0.5,anchor1=270,anchor2=0,align=right]{turnp2}{final}
	
	\umltrans[arg={Bewegen  }, recursive=155|165|1.5cm, pos=2.1,align=right, recursive direction=left to left]{turnp1}{turnp1}
	\umltrans[arg={Aktion  }, recursive=175|185|1.5cm, pos=2.1,align=right, recursive direction=left to left]{turnp1}{turnp1}
	\umltrans[arg={Überlanges Spiel  }, recursive=195|205|1.5cm, pos=2.1,align=right, recursive direction=left to left]{turnp1}{turnp1}

	\umltrans[arg=Bewegen, recursive=25|15|1.5cm, pos=1.5,recursive direction=left to left]{turnp2}{turnp2}
	\umltrans[arg=Aktion, recursive=5|-5|1.5cm, pos=1.5, recursive direction=left to left]{turnp2}{turnp2}
	\umltrans[arg=Überlanges Spiel, recursive=-15|-25|1.5cm, pos=1.5, recursive direction=left to left]{turnp2}{turnp2}
\end{tikzpicture}
