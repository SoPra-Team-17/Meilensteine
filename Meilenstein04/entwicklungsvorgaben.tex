Der Server und der KI-Client werden in C++ implementiert. Der Client wird in Python implementiert.\\
Die Dokumentation des Systems umfasst ausführliche, englische Dokumentation des Quellcodes, inklusive Doxygen, ein Entwicklerhandbuch, welches die Entscheidungen der Archiktektur beinhaltet und ein Benutzerhandbuch, welches die Verwendung der Software darstellt.\\
\\
Folgende Software wird bei der Entwicklung verwendet:
\begin{itemize}
\item Git als Versionierungssoftware
	
\item SonarQube bzw. Pylint als Qualitätssicherungssoftware
	
\item Docker als Containervirtualisierungssoftware 
\end{itemize}
Die Software wird von den sechs beteiligten Entwicklern mit Scrum als Umsetzung von agilen Methoden entwickelt. Die Tutorien erfüllen die Rolle von Sprint Meetings, der Tutor nimmt hierbei die Rolle des Product-Owners ein. Die Issue-Funktion von Git stellt das Scrum-Board dar.\\
SonarQube bzw. Pylint wird unterstützend verwendet um die Qualität vom Source-Code bei den Begutachtungen in den Sprint-Reviews zu angemessen zu analysieren.
Außerdem müssen die Kriterien der Codeanalyse-Tools erfüllt werden, bevor ein Pull-Request gemerged werden kann.\\
Die Komponenten Benutzer-Client, KI-Client und Server werden im Team implementiert, die Komponente Editor wird auf der Messe erworben.\\
Die Entwicklung der Netzwerkkommunkation wird auf dem Netzwerkprotokoll basieren, das vom Standardisierungskomitee erstellt wird.\\
Um die Übertragbarkeit des Systems im Bezug auf Hardware zu gewährleisten, wird das fertige Produkt in einem Docker-Container übergeben.\\

