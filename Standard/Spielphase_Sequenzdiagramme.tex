\documentclass[12pt,a4paper]{scrbook}

\usepackage{tikz-uml}
\usepackage[ngerman]{babel}

\begin{document}


\begin{tikzpicture}
	\begin{umlseqdiag}
	\umlactor[class=Client]{Spieler}
	\umlobject[no ddots]{Server}
	\umlactor[class=Client]{Zuschauer}
	
	
	\begin{umlfragment}[type=loop, label=nächster Charakter]
		
		\begin{umlcallself}[op=Vorbereitung, dt=7.5]{Server}
		\end{umlcallself}
		\begin{umlcall}[op=Spielstatusnachricht]{Server}{Spieler}
		\end{umlcall}
		\begin{umlcall}[op=Spielstatusnachricht]{Server}{Zuschauer}
		\end{umlcall}
		
		\begin{umlfragment}[type=alt, label=Spieler ist am Zug, inner ysep=7.5]
		
			\begin{umlfragment}[type=loop, label=nächste Aktion, inner xsep = 7.5]
				\begin{umlcall}[op=Aktionsnachricht, dt=22.5]{Spieler}{Server}
				\end{umlcall}
				\begin{umlcallself}[op=Validierung]{Server}
				\end{umlcallself}
						
				\begin{umlfragment}[type=opt, label=nicht valide, inner xsep=10]
					\begin{umlcall}[op=Strike, dt=7.5]{Server}{Spieler}
					\end{umlcall}
				\end{umlfragment}
				
			\end{umlfragment}
			
						
			\umlfpart[Spieler ist nicht am Zug]
		
			\begin{umlcallself}[op=wait, dt = 7.5]{Spieler}
			\end{umlcallself}
		\end{umlfragment}
		
		\begin{umlcall}[op=Spielstatusnachricht, dt=30]{Server}{Spieler}
		\end{umlcall}
		\begin{umlcall}[op=Spielstatusnachricht]{Server}{Zuschauer}
		\end{umlcall}
		
	\end{umlfragment}
	
	
	\end{umlseqdiag}
\end{tikzpicture}


\end{document}
