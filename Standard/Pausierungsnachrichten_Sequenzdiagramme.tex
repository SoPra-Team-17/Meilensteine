\documentclass[12pt,a4paper]{scrbook}


\usepackage{tikz}
\usepackage{tikz-uml}
\usepackage[ngerman]{babel}

\begin{document}

\begin{tikzpicture}

\begin{umlseqdiag}

	\umlactor[x = 0, class=Client]{Spieler 1}
	\umlactor[x = 3.5, class=Client]{Spieler 2}
	\umlobject[x = 8.5, no ddots]{Server}
	\umlactor[x = 13.5, class=Client]{Zuschauer/KI}
	
	
	\begin{umlfragment}[type=alt, label=Spieler 1 m\"ochte pausieren, inner ysep=5]	
		
		\begin{umlcall}[op=Pausierungsnachricht, dt = 13.0]{Spieler 1}{Server}
		\end{umlcall}
		
		\umlfpart[Spieler 2 m\"ochte pausieren]
		
		
		\begin{umlcall}[op=Pausierungsnachricht, dt=20]{Spieler 2}{Server}
		\end{umlcall}
		
	\end{umlfragment}
	
	\begin{umlcall}[op=Pausierungsbest\"atigung, dt = -5]{Server}{Spieler 1}
	\end{umlcall}
	\begin{umlcall}[op=Pausierungsbest\"atigung, dt = -12]{Server}{Spieler 2}
	\end{umlcall}
	\begin{umlcall}[op=Pausierungsbest\"atigung, dt = -12]{Server}{Zuschauer/KI}
	\end{umlcall}


	\begin{umlfragment}[type=alt, label=Spieler 1 m\"ochte das Spiel fortsetzen, inner ysep=5]	
		
		\begin{umlcall}[op=Fortsetzungsnachricht, dt = 23]{Spieler 1}{Server}
		\end{umlcall}
		
		\umlfpart[Spieler 2 m\"ochte das Spiel fortsetzen]
		
		
		\begin{umlcall}[op=Fortsetzungsnachricht, dt = 26]{Spieler 2}{Server}
		\end{umlcall}
		
	\end{umlfragment}

	\begin{umlcall}[op=Fortsetzungsbest\"atigung, dt=-5]{Server}{Spieler 1}
	\end{umlcall}
	\begin{umlcall}[op=Fortsetzungsbest\"atigung, dt=-12]{Server}{Spieler 2}
	\end{umlcall}
	\begin{umlcall}[op=Fortsetzungsbest\"atigung, dt=-12]{Server}{Zuschauer/KI}
	\end{umlcall}

\end{umlseqdiag}

\end{tikzpicture}

\end{document}
