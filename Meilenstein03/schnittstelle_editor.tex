Erstellt ein Mensch-Client über die Lobbyübersicht eine neue Lobby, so muss er die Konfiguration des Editors über eine graphische Oberfläche durchführen (FA-E \ref{e-gui}, FA-E \ref{e-szenarioedit}, FA-E \ref{e-partieedit}, FA-E \ref{e-charedit}).

\begin{figure}[h]
	\centering
	\tikzstyle{block} = [draw, rectangle, 
	minimum height=3em, minimum width=6em, very thick, align = center]
	\begin{tikzpicture}[auto, node distance=2cm,>=latex']
	
	\node[block] (Editor) at (0, 7) {<<Dialog>>\\ \textbf{Editor}};
	
	\node[block, above left = 1.0cm and 1.0cm of Editor] (Help) {<<Popup>>\\ \textbf{Help}\\ (QA10)};
	
	\node[block, below left = 3.0cm and 3.0cm of Editor] (Szenario) {<<Dialog>>\\ \textbf{Szenario-Editor}};
	
	\node[block, below = 3.0cm of Editor] (Partie) {<<Dialog>>\\ \textbf{Partie-Editor}};
	
	\node[block, below right = 3.0cm and 3.0cm of Editor] (Charakter) {<<Dialog>>\\ \textbf{Charakter-Editor}};
	
	\node[block, above right = 1.0cm and 1.5cm of Editor] (Lobby) {<<Dialog>>\\ \textbf{create Lobby}};
	
	
	\draw[thick, ->, shorten >=1mm,shorten <=1mm] (Editor.west) -| (Help.south) node [near start, xshift=3mm] {\small Help};
	\draw[thick, ->, shorten >=1mm,shorten <=1mm] (Help.east) -| ([xshift= -0.5cm]Editor.north) node [near start, xshift=3mm] {\small Close};
	
	\draw[thick, ->, shorten >=1mm,shorten <=1mm] (Editor.east) -| (Lobby.south) node [near start, xshift=1.5mm] {\small safe $\vert\vert$ Back};
	\draw[thick, ->, shorten >=1mm,shorten <=1mm] (Lobby.west) -| ([xshift= 0.5cm]Editor.north) node [near start, xshift=3mm] {\small configure};
	
	\draw[thick, ->, shorten >=1mm,shorten <=1mm] ([xshift = -0.5cm]Editor.south) -- (Szenario.north) node [near start, xshift=-30mm, yshift = -5mm] {\small Szenario};
	\draw[thick, ->, shorten >=1mm,shorten <=1mm] (Szenario.east) -- (Partie.west) node [near start, xshift=3mm] {\small Partie};
	\draw[thick, ->, shorten >=1mm,shorten <=1mm] (Partie.east) -- (Charakter.west) node [near start, xshift=3mm] {\small Charakter};
	\draw[thick, ->, shorten >=1mm,shorten <=1mm] (Charakter.north) -- ([xshift = 0.5cm]Editor.south) node [near start, xshift=4mm, yshift = 10mm] {\small Editor};
	
	
	
	\end{tikzpicture}
	\caption{graphische Schnittstelle des Editors}
\end{figure}

Um die Konfiguration, die im Editor vorgenommen wurde, über die Websocket-Schnittstelle des Mensch-Clients mit dem Server zu kommunizieren, verfügt der Editor über eine JSON-Schnittstelle (FA-E \ref{e-json-encoding}, FA-E \ref{e-json-decoding}).