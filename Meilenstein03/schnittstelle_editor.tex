Erstellt ein Mensch-Client über die Lobbyübersicht eine neue Lobby, so muss er die Konfiguration des Editors über eine graphische Oberfläche durchführen (FA-E \ref{e-gui}, FA-E \ref{e-szenarioedit}, FA-E \ref{e-partieedit}, FA-E \ref{e-charedit}).

TODO: Dialogstrukturdiagramm (new Lobby -> Dialog Editor -> Szenario -> Partiekonfiguration(Zeit für Phasen, Wahrscheinlichkeiten) -> CharakterBeschreibung -> back / new Lobby) einfache Benutzung QA10

\mbox{}\\
Um die Konfiguration, die im Editor vorgenommen wurde, über die Websocket-Schnittstelle des Mensch-Clients mit dem Server zu kommunizieren, verfügt der Editor über eine JSON-Schnittstelle (FA-E \ref{e-json-encoding}, FA-E \ref{e-json-decoding}).