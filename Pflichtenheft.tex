\documentclass{uulm-assignment}

\usepackage{import}
\usepackage{tabularx}
\usepackage{listings}
\usepackage{todonotes}
\usepackage{graphicx}
\usepackage{siunitx}
\usepackage{placeins}
\usepackage{tikz}
\usepackage{tikz-uml}

\setboolean{showsolutions}{false}

\ifthenelse{\boolean{showsolutions}}{
\newcommand\mitloesung{1}%
}{
\newcommand\mitloesung{0}%
}


% Für Korrektur-Kommentare in roten Boxen:
\newcommand{\flo}[1]{
    \fcolorbox{purple}{pink}{\sffamily\scriptsize\bfseries\textcolor{black}{Flo:}} {\sffamily\bfseries\textcolor{purple}{#1}}
}

% Entnommen und angepasst von https://github.com/SoPra-Team-10
\newcommand{\Fachwissen}[7] {
		\begin{tabular}{|p{2,25cm}|p{12cm}|}
			\hline
			\textbf{Begriff} & \textbf{#1} \\
			\hline
			\textbf{Beschreibung} & #2 \\ 
			\hline
			\textbf{Ist-ein} & #3 \\
			\hline
			\textbf{Kann-sein} & #4 \\ 
			\hline
			\textbf{Aspekt} & #5 \\ 
			\hline
			\textbf{Bemerkung} & #6 \\ 
			\hline
			\textbf{Beispiel} & #7 \\
			\hline
		\end{tabular}
}

\newcommand{\FachwissenSyn}[8] {
		\begin{tabular}{|p{2,25cm}|p{12cm}|}
			\hline
			\textbf{Begriff} & \textbf{#1} \\
			\hline
			\textbf{Beschreibung} & #2 \\ 
			\hline
			\textbf{Ist-ein} & #3 \\
			\hline
			\textbf{Kann-sein} & #4 \\ 
			\hline
			\textbf{Aspekt} & #5 \\ 
			\hline
			\textbf{Bemerkung} & #6 \\ 
			\hline
			\textbf{Beispiel} & #7 \\
			\hline
			\textbf{Synonym} & #8 \\
			\hline
		\end{tabular}
}



\hypersetup{colorlinks=false,urlcolor=uulm-in}

\faculty{Institut für Softwaretechnik und Programmiersprachen\hspace{0.05cm}}
\course{Softwaregrundprojekt}
\semester{\hspace{0.05cm}WiSe 2019/20}
\supervisor{\textbf{} \hspace{7.9cm} Prof. Dr. Matthias Tichy, Florian Ege, Dennis Jehle}
 
%\assignmentdeadline{}         % Abgabedatum: XYZ
%\assignmentduration{15 Minuten} % Bearbeitungsdauer: XYZ
% \studentdata                      % Name & Matrikelnummer Feld

\assignmenttype{}
\assignmentno{}
\title{Pflichtenheft Team 17}

\begin{document}

\maketitle

\tableofcontents

\clearpage
\section{Kontextanalyse}
Der Zweck dieses Dokuments ist es, eine detaillierte Beschreibung der Anforderungen, sowie der Benutzerschnittstelle
für die Anwendung No time to spy bereitzustellen. Es wird abgegrenzt, welche Anforderungen erfüllt werden müssen,
damit die entwickelte Anwendung vom Kunden akzeptiert wird.


\subsection{Anwendungsbereiche}
Das Spiel No time to spy ist eine C++-Anwendung, die es dem Benutzer ermöglicht, ein 2D-Spiel zu spielen.


\clearpage
\section{Fachwissen}
Diese Tabelle enthält Abkürzungen und domänenspezifische Begriffe, die im Dokument verwendet werden


\clearpage
\section{Domänenmodell}
\FloatBarrier
\begin{figure}[h]
    \centering
    \includegraphics[width=\textwidth]{domainmodel.pdf}
    \caption{Domänenmodell}
\end{figure}

\FloatBarrier

\begin{figure}[h]
    \centering
    \includegraphics[width=\textwidth]{fields.pdf}
    \caption{Domänenmodell: Feld}
\end{figure}
\FloatBarrier


\begin{figure}[h]
    \centering
    \includegraphics[width=\textwidth]{gadgets.pdf}
    \caption{Domänenmodell: Gadget}
\end{figure}
\FloatBarrier



\clearpage
\section{Spezifische Anforderungen}
Dieser Abschnitt enthält alle spezifischen Anforderungen des Systems. Er bietet eine detaillierte
Beschreibung des Systems und seiner Funktionen.


\subsection{Akteure}
%Hier kommen die Akteure rein

\begin{tabularx}{16cm}{l|X}
	\textbf{Akteur} & \textbf{Server} \\
	\hline
	Rolle & Zentrale Software-Einheit zur Kommunikation mit den Clients.\\ 
	\hline
	Aufgabe & Der Server ist Vermittler in der Kommunikation zwischen den Clients. Er geht auf Anfrage Verbindungen mit Clients ein und kann diese auch wieder beenden. Während ein Client mit dem Server verbunden ist, bearbeitet und validiert der Server alle Anfragen des Clients und sorgt dafür, dass der Client alle notwendigen Informationen zum aktuellen Stand erhält.\\ 
\end{tabularx}

\begin{tabularx}{16cm}{l|X}
	\textbf{Akteur} & \textbf{Client} \\
	\hline
	Rolle & Lokale Software-Einheit zur Kommunikation mit dem Server \\ 
	\hline
	Aufgabe & Da Clients nicht direkt miteinander kommunizieren können, kommuniziert der Client mit einem Server. Der Client kann eine Verbindung zum Server auf- und auch wieder abbauen. Ein verbundener Client kann Informationen weitergeben, anfragen und erhalten.\\ 
\end{tabularx}

\begin{tabularx}{16cm}{l|X}
	\textbf{Akteur} & \textbf{Benutzer-Client} \\
	\hline
	Rolle & Client, der von einem Menschen bedient wird.\\ 
	\hline
	Aufgabe & Der Benutzer-Client stellt eine graphische Oberfläche zur Bedienung zur Verfügung. Er ermöglicht die Bedienung des Editors und die Teilnahme an einer Spielpartie.\\ 
\end{tabularx}

\begin{tabularx}{16cm}{l|X}
	\textbf{Akteur} & \textbf{KI-Client} \\
	\hline
	Rolle & Client, der von einem Computer gesteuert wird.\\ 
	\hline
	Aufgabe & Der KI-Client kann von einem Benutzer-Client als Spieler zur Spielpartie hinzugefügt werden und benutzt keine graphische Oberfläche.\\ 
\end{tabularx}

\begin{tabularx}{16cm}{l|X}
	\textbf{Akteur} & \textbf{Teilnehmer} \\
	\hline
	Rolle & Alle Akteure, die an einer Spielpartie teilnehmen.\\ 
	\hline
	Aufgabe & Teilnehmer erhalten Informationen zur Spielpartie vom Server und können die Spielpartie jederzeit wieder verlassen.\\ 
\end{tabularx}

\begin{tabularx}{16cm}{l|X}
	\textbf{Akteur} & \textbf{Zuschauer} \\
	\hline
	Rolle & Benutzer-Client, der passiv an einer Spielpartie teilnimmt.\\ 
	\hline
	Aufgabe & Über die graphische Oberfläche des Benutzer-Clients, kann der Benutzer das laufende Spiel beobachten, aber nicht eingreifen. \\ 
\end{tabularx}

\begin{tabularx}{16cm}{l|X}
	\textbf{Akteur} & \textbf{Spieler} \\
	\hline
	Rolle & Client, der aktiv an einer Spielpartei teilnimmt.\\ 
	\hline
	Aufgabe & Wenn ein Spieler an der Reihe ist, führt er durch Steuerung seiner Charaktere valide Spielzüge aus und informiert den Server über seine Aktionen. Ziel des Spielers ist es, die Spielpartie zu gewinnen.\\ 
\end{tabularx}

\begin{tabularx}{16cm}{l|X}
	\textbf{Akteur} & \textbf{Benutzer-Spieler} \\
	\hline
	Rolle & Spieler, der ein Benutzer-Client ist.\\ 
	\hline
	Aufgabe & Ein Benutzer-Spieler spielt die Spielpartie über die graphische Oberfläche.\\ 
\end{tabularx}

\begin{tabularx}{16cm}{l|X}
	\textbf{Akteur} & \textbf{KI-Spieler} \\
	\hline
	Rolle & Spieler, der ein KI-Client ist.\\ 
	\hline
	Aufgabe & Ein KI-Spieler spielt die Spielpartie unter Verwendung von künstlicher Intelligenz.\\ 
\end{tabularx}

\begin{tabularx}{16cm}{l|X}
	\textbf{Akteur} & \textbf{Agent} \\
	\hline
	Rolle & Charakter in einer Spielpartie, der von einem Spieler gesteuert wird.\\ 
	\hline
	Aufgabe & Agenten sind Charaktere, die während der Spielpartie von Spielern gesteuert werden.\\ 
\end{tabularx}

\begin{tabularx}{16cm}{l|X}
	\textbf{Akteur} & \textbf{NPC} \\
	\hline
	Rolle & Charakter in einer Spielpartie, der von keinem Spieler gesteuert werden kann.\\ 
	\hline
	Aufgabe & NPCs sind Charaktere, die vom Server gesteuert werden. Der Server informiert die Teilnehmer über die Spielzüge der NPCs.\\ 
\end{tabularx}

\begin{tabularx}{16cm}{l|X}
	\textbf{Akteur} & \textbf{Auftraggeber} \\
	\hline
	Rolle & Florian Ege, stellvertretend für das Institut für Softwaretechnik und Programmiersprachen\\ 
	\hline
	Aufgabe & Gibt die Anforderungen für \glqq No Time To Spy\grqq vor und nimmt das fertige Produkt ab.\\ 
\end{tabularx}

\begin{tabularx}{16cm}{l|X}
	\textbf{Akteur} & \textbf{Vertreter des Auftraggebers} \\
	\hline
	Rolle & Tutor Ismail Temel\\ 
	\hline
	Aufgabe & Dient als Vermittler zwischen Entwicklern und Auftraggeber.\\ 
\end{tabularx}

\begin{tabularx}{16cm}{l|X}
\textbf{Akteur} & \textbf{Entwickler} \\
\hline
Rolle & SoPra-Team-17: Dominik Authaler, Lukas Bleile, Marco Deuscher, Jonas Otto, Carolin Schindler, Dominik Tabib Khoie\\ 
\hline
Aufgabe & Agile Entwicklung von \glqq No Time To Spy\grqq nach den Anforderungen des Auftraggebers.\\ 
\end{tabularx}


\clearpage
\subsection{Funktionale Anforderungen}

Dieser Abschnitt enthält alle Anforderungen, die die grundlegenden Aktionen des Softwaresystems
spezifizieren.

\subsubsection{Server}

\begin{tabularx}{16cm}{l|X}
\textbf{ID} & \textbf{FA-S} \\
\hline
TITEL: & Partie Konfiguration \\
\hline
BESCHREIBUNG: & Der Server muss beim Start die Partie Konfiguration aus einer Datei lesen. \\
\hline
BEGRÜNDUNG: & Die Konfiguration soll für das Balancing des Spiels einfach angepasst werden können. \\
\hline
ABHÄNGIGKEITEN: & \\
\end{tabularx}

\begin{tabularx}{16cm}{l|X}
\textbf{ID} & \textbf{FA-S} \\
\hline
TITEL: & Szenario Konfiguration \\
\hline
BESCHREIBUNG: & Der Server muss beim Start die Szenario Konfiguration (Spielfeld) aus einer Datei lesen. \\
\hline
BEGRÜNDUNG: & Das Spielfeld soll veräderbar sein. \\
\hline
ABHÄNGIGKEITEN: & \\
\end{tabularx}

\begin{tabularx}{16cm}{l|X}
\textbf{ID} & \textbf{FA-S} \\
\hline
TITEL: & Charakter Konfiguration \\
\hline
BESCHREIBUNG: & Der Server muss beim Start die Konfiguration der Charaktere aus einer Datei lesen. \\
\hline
BEGRÜNDUNG: & \\
\hline
ABHÄNGIGKEITEN: & \\
\end{tabularx}

\begin{tabularx}{16cm}{l|X}
\textbf{ID} & \textbf{FA-S} \\
\hline
TITEL: & Client Verbindung \\
\hline
BESCHREIBUNG: & Der Server muss mittels Websockets über TCP eine Verbindung mit dem Client halten.
Die Verbindung wird vom Client initiiert. Bei Verbindungsabbruch muss die Verbindung wiederhergestellt werden. \\
\hline
BEGRÜNDUNG: & Integrale Funktionalität des Servers \\
\hline
ABHÄNGIGKEITEN: & \\
\end{tabularx}

\begin{tabularx}{16cm}{l|X}
\textbf{ID} & \textbf{FA-S} \\
\hline
TITEL: & Zuschauer \\
\hline
BESCHREIBUNG: & Ein Client kann sich als Zuschauer verbinden, bekommt also Updates über den Spielzustand, kann aber nicht aktiv am Spiel teilnehmen. \\
\hline
BEGRÜNDUNG: & \\
\hline
ABHÄNGIGKEITEN: & \\
\end{tabularx}

\begin{tabularx}{16cm}{l|X}
\textbf{ID} & \textbf{FA-S} \\
\hline
TITEL: & Client Timeout \\
\hline
BESCHREIBUNG: & \\
\hline
BEGRÜNDUNG: & \\
\hline
ABHÄNGIGKEITEN: & \\
\end{tabularx}

\begin{tabularx}{16cm}{l|X}
\textbf{ID} & \textbf{FA-S} \\
\hline
TITEL: & Verspätete Nachrichten \\
\hline
BESCHREIBUNG: & \\
\hline
BEGRÜNDUNG: & \\
\hline
ABHÄNGIGKEITEN: & \\
\end{tabularx}

\begin{tabularx}{16cm}{l|X}
\textbf{ID} & \textbf{FA-S} \\
\hline
TITEL: & Spiel Durchführung \\
\hline
BESCHREIBUNG: & \\
\hline
BEGRÜNDUNG: & \\
\hline
ABHÄNGIGKEITEN: & \\
\end{tabularx}

\begin{tabularx}{16cm}{l|X}
\textbf{ID} & \textbf{FA-S} \\
\hline
TITEL: & Spielpause \\
\hline
BESCHREIBUNG: & \\
\hline
BEGRÜNDUNG: & \\
\hline
ABHÄNGIGKEITEN: & \\
\end{tabularx}

\begin{tabularx}{16cm}{l|X}
\textbf{ID} & \textbf{FA-S} \\
\hline
TITEL: & Erkennung von Regelverstößen \\
\hline
BESCHREIBUNG: & \\
\hline
BEGRÜNDUNG: & \\
\hline
ABHÄNGIGKEITEN: & \\
\end{tabularx}

\begin{tabularx}{16cm}{l|X}
\textbf{ID} & \textbf{FA-S} \\
\hline
TITEL: & Senden des Spielzustands \\
\hline
BESCHREIBUNG: & \\
\hline
BEGRÜNDUNG: & \\
\hline
ABHÄNGIGKEITEN: & \\
\end{tabularx}

\begin{tabularx}{16cm}{l|X}
\textbf{ID} & \textbf{FA-S} \\
\hline
TITEL: & Gewinner Erkennung \\
\hline
BESCHREIBUNG: & \\
\hline
BEGRÜNDUNG: & \\
\hline
ABHÄNGIGKEITEN: & \\
\end{tabularx}

\begin{tabularx}{16cm}{l|X}
\textbf{ID} & \textbf{FA-S} \\
\hline
TITEL: & Erstellen von Replay Files \\
\hline
BESCHREIBUNG: & \\
\hline
BEGRÜNDUNG: & \\
\hline
ABHÄNGIGKEITEN: & \\
\end{tabularx}

\begin{tabularx}{16cm}{l|X}
\textbf{ID} & \textbf{FA-S} \\
\hline
TITEL: & JSON Encoding \\
\hline
BESCHREIBUNG: & \\
\hline
BEGRÜNDUNG: & \\
\hline
ABHÄNGIGKEITEN: & \\
\end{tabularx}

\begin{tabularx}{16cm}{l|X}
\textbf{ID} & \textbf{FA-S} \\
\hline
TITEL: & JSON Decoding \\
\hline
BESCHREIBUNG: & \\
\hline
BEGRÜNDUNG: & \\
\hline
ABHÄNGIGKEITEN: & \\
\end{tabularx}

\begin{tabularx}{16cm}{l|X}
\textbf{ID} & \textbf{FA-S} \\
\hline
TITEL: & Websockets \\
\hline
BESCHREIBUNG: & \\
\hline
BEGRÜNDUNG: & \\
\hline
ABHÄNGIGKEITEN: & \\
\end{tabularx}



\clearpage
\subsubsection{Client}
\begin{tabularx}{16cm}{l|X}
\textbf{ID} & \textbf{FA-C} \\
\hline
TITEL: & Optional: Intro \\
\hline 
BESCHREIBUNG: & Beim Starten des Programms soll ein Film ablaufen, der den Titel des Spiels präsentiert und die Teammitglieder nennt. Dieser kann jederzeit mittels Tastendruck abgebrochen werden. Nach Ende des Films oder Abbruch gelangt der Nutzer ins Hauptmenü. \\
\hline
BEGRÜNDUNG: & Der Film versteckt etwaige Ladezeiten beim Starten und informiert den Spieler über das Spiel und die Autoren.\\
\hline
ABHÄNGIGKEITEN: & \\
\end{tabularx}

\begin{tabularx}{16cm}{l|X}
\textbf{ID} & \textbf{FA-C} \\
\hline
TITEL: & Hauptmenü \\
\hline 
BESCHREIBUNG: & Im Hauptmenü hat der Nutzer die Möglichkeit, seinen Namen festzulegen, einem Server durch Eingabe der Adresse beizutreten, oder das Spiel zu beenden.\\ 
\hline
BEGRÜNDUNG: & Vor dem Spielen muss der Spieler eine Verbindung zum Server aufbauen. \\
\hline
ABHÄNGIGKEITEN: & \\
\end{tabularx}

\begin{tabularx}{16cm}{l|X}
\textbf{ID} & \textbf{FA-C} \\
\hline
TITEL: & Beitreten als Mitspieler \\
\hline 
BESCHREIBUNG: & Nach Verbinden mit dem Server kann der Spieler die Rolle als aktiver Spieler wählen. Dies wird dem Server mitgeteilt. \\
\hline
BEGRÜNDUNG: & Es muss zwischen aktiven Spielern und passiven Zuschauern unterschieden werden. \\
\hline
ABHÄNGIGKEITEN: & \\
\end{tabularx}

\begin{tabularx}{16cm}{l|X}
\textbf{ID} & \textbf{FA-C} \\
\hline
TITEL: & Beitreten als Zuschauer \\
\hline 
BESCHREIBUNG: & Nach Verbinden mit dem Server kann der Spieler die Rolle als passiver Zuschauer wählen. Dies wird dem Server mitgeteilt. \\
\hline
BEGRÜNDUNG: & Es muss zwischen aktiven Spielern und passiven Zuschauern unterschieden werden. \\
\hline
ABHÄNGIGKEITEN: & \\
\end{tabularx}

\begin{tabularx}{16cm}{l|X}
\textbf{ID} & \textbf{FA-C} \\
\hline
TITEL: & Registrieren als menschlicher Spieler \\
\hline 
BESCHREIBUNG: & Beim Verbinden mit dem Server muss diesem mitgeteilt werden, dass es sich um einen menschlichen Spieler handelt. \\
\hline
BEGRÜNDUNG: & Es muss zwischen menschlichen Spielern und KI Spielern unterschieden werden. \\
\hline
ABHÄNGIGKEITEN: & \\
\end{tabularx}

\begin{tabularx}{16cm}{l|X}
\textbf{ID} & \textbf{FA-C} \\
\hline
TITEL: & Nutzername \\
\hline 
BESCHREIBUNG: & Der Client muss es dem Nutzer ermöglichen einen Nutzernamen zu wählen. Dieser soll vom Server validiert werden und wird eventuell nicht akzeptiert. \\
\hline
BEGRÜNDUNG: & Beide Mitspieler werden durch Nutzernamen identifiziert. \\
\hline
ABHÄNGIGKEITEN: & \\
\end{tabularx}

\begin{tabularx}{16cm}{l|X}
\textbf{ID} & \textbf{FA-C} \\
\hline
TITEL: & Spiel Anzeige \\
\hline 
BESCHREIBUNG: & Der Client muss dem Benutzer den Aktuellen Zustand des Spiels jederzeit darstellen. Dazu gehört:
\begin{itemize}
    \item Charaktere der eigenen Fraktion
    \item Eigenschaften der Charaktere
    \item Gadgets der Charaktere
    \item Spielfeld mit Spielern und Gadgets
\end{itemize} \\
\hline
BEGRÜNDUNG: & Integraler Teil des Spiels \\
\hline
ABHÄNGIGKEITEN: & \\
\end{tabularx}

\begin{tabularx}{16cm}{l|X}
\textbf{ID} & \textbf{FA-C} \\
\hline
TITEL: & Interaktion mit dem Spiel \\
\hline 
BESCHREIBUNG: & Der Client muss dem Spieler ermöglichen, Aktionen auf dem Spielfeld durchzuführen, wenn der Spieler am Zug ist. \\
\hline
BEGRÜNDUNG: & Integraler Bestandteil des Spiels \\
\hline
ABHÄNGIGKEITEN: & \\
\end{tabularx}

\begin{tabularx}{16cm}{l|X}
\textbf{ID} & \textbf{FA-C} \\
\hline
TITEL: & Optional: Hilfefunktion \\
\hline 
BESCHREIBUNG: & Der Client soll den Spieler durch Vorschläge für Aktionen unterstützen. \\
\hline
BEGRÜNDUNG: & Dies vereinfacht das Erlernen des Spiels und das Kennenlernen der verschiedenen Aktionen. \\
\hline
ABHÄNGIGKEITEN: & \\
\end{tabularx}

\begin{tabularx}{16cm}{l|X}
\textbf{ID} & \textbf{FA-C} \\
\hline
TITEL: & Optional: Hotkeys \\
\hline 
BESCHREIBUNG: & Die Aktionen während dem Spiel sollen durch Hotkeys ausführbar sein. \\
\hline
BEGRÜNDUNG: & Dies dient einer komfortableren Bedienung des Spiels. \\
\hline
ABHÄNGIGKEITEN: & \\
\end{tabularx}

\begin{tabularx}{16cm}{l|X}
\textbf{ID} & \textbf{FA-C} \\
\hline
TITEL: & Animation der Aktionen \\
\hline 
BESCHREIBUNG: & Die Aktionen einer Runde sollen animiert dargestellt werden. \\
\hline
BEGRÜNDUNG: & Einfacheres Nachvollziehen der eigenen und gegnerischen Aktionen \\
\hline
ABHÄNGIGKEITEN: & \\
\end{tabularx}

\begin{tabularx}{16cm}{l|X}
\textbf{ID} & \textbf{FA-C} \\
\hline
TITEL: & Wunsch auf Pausieren \\
\hline 
BESCHREIBUNG: & Der Client ermöglicht es dem Spieler, ein Pausieren der Runde zu beantragen. \\
\hline
BEGRÜNDUNG: & Für bessere Immersion kann der Benutzer während dem Spielen selbst Cocktails trinken, was eventuelle Pausen fürs Nachfüllen bedingt. \\
\hline
ABHÄNGIGKEITEN: & \\
\end{tabularx}

\begin{tabularx}{16cm}{l|X}
\textbf{ID} & \textbf{FA-C} \\
\hline
TITEL: & Wunsch auf Wiederaufnahme des Spiels \\
\hline 
BESCHREIBUNG: & Der Client ermöglicht es dem Spieler, die Wiederaufnahme des Spiels (Beenden der Pause) zu beantragen. \\
\hline
BEGRÜNDUNG: & Ermöglicht Beenden der Pause \\
\hline
ABHÄNGIGKEITEN: & \\
\end{tabularx}

\begin{tabularx}{16cm}{l|X}
\textbf{ID} & \textbf{FA-C} \\
\hline
TITEL: & Persistente Session \\
\hline 
BESCHREIBUNG: & Ein Abbruch der Verbindung zum Server darf nicht zum Beenden des Spiels führen. Der Client muss versuchen, die Verbindung wiederherzustellen und im Erfolgsfall das Spiel fortsetzen. \\
\hline
BEGRÜNDUNG: & Mit Verbindungsabbrüchen ist z.B bei der Verwendung von WiFi zu rechnen, es soll trotzdem möglich sein zu spielen. \\
\hline
ABHÄNGIGKEITEN: & \\
\end{tabularx}

\begin{tabularx}{16cm}{l|X}
\textbf{ID} & \textbf{FA-C} \\
\hline
TITEL: & Gewinneranzeige \\
\hline 
BESCHREIBUNG: & Wenn durch den Server ein Gewinner festgestellt wird, muss dies im Client dem Spieler präsentiert werden. \\
\hline
BEGRÜNDUNG: & Integraler Bestandteil des Spiels \\
\hline
ABHÄNGIGKEITEN: & \\
\end{tabularx}

\begin{tabularx}{16cm}{l|X}
\textbf{ID} & \textbf{FA-C} \\
\hline
TITEL: & Optional: Statistik \\
\hline 
BESCHREIBUNG: & Neben der Anzeige des Gewinners sollen zusätzlich Statistiken zum Spielverlauf angezeigt werden. \\
\hline
BEGRÜNDUNG: & Information des Spielers \\
\hline
ABHÄNGIGKEITEN: & \\
\end{tabularx}

\begin{tabularx}{16cm}{l|X}
\textbf{ID} & \textbf{FA-C} \\
\hline
TITEL: & Optional: Replay \\
\hline 
BESCHREIBUNG: & Der Client kann statt eines aktiven Spiels auch eine vom Server erstellte Aufzeichnung eines vergangenen Spiels anzeigen. Hierbei ist keinerlei Interaktion möglich. \\
\hline
BEGRÜNDUNG: & Erneutes Anschauen besonders interessanter Partien \\
\hline
ABHÄNGIGKEITEN: & \\
\end{tabularx}


\clearpage
\subsubsection{KI-Client}
\begin{tabularx}{16cm}{l|X}
\refstepcounter{table}\label{ki-session}
\textbf{ID} & \textbf{FA-KI \arabic{table}} \\
\hline
TITEL: & Websocket Verbindung \\
\hline 
BESCHREIBUNG: & Der KI-Client muss eine Websocket Verbindung zum Server aufbauen können. Darüber können JSON encodierte Nachrichten geschickt werden. \\
\hline
BEGRÜNDUNG: & Kommunikation mit dem Server \\
\hline
ABHÄNGIGKEITEN: & \\
\end{tabularx}

\begin{tabularx}{16cm}{l|X}
\refstepcounter{table}\label{ki-register}
\textbf{ID} & \textbf{FA-KI \arabic{table}} \\
\hline
TITEL: & Registrieren als KI \\
\hline 
BESCHREIBUNG: & Der KI-Client muss sich nach der Verbindung mit dem Server als KI identifizieren. \\
\hline
BEGRÜNDUNG: & Anzeige für den Spielers, ob er gegen eine KI oder einen menschlichen Gegner spielt. \\
\hline
ABHÄNGIGKEITEN: & FA-KI \ref{ki-session} \\
\end{tabularx}

\begin{tabularx}{16cm}{l|X}
\refstepcounter{table}\label{ki-cli}
\textbf{ID} & \textbf{FA-KI \arabic{table}} \\
\hline
TITEL: & Commandline Argumente \\
\hline 
BESCHREIBUNG: & Der KI-Client muss Kommandozeilen Argumente unterstützen um Parameter zu setzen. Die Argumente sind vom Standardisierungskomitee definiert. \\
\hline
BEGRÜNDUNG: & Nicht-interaktive Benutzung, Konfiguration des Clients. \\
\hline
ABHÄNGIGKEITEN: & \\
\end{tabularx}

\begin{tabularx}{16cm}{l|X}
\refstepcounter{table}\label{ki-config}
\textbf{ID} & \textbf{FA-KI \arabic{table}} \\
\hline
TITEL: & Konfigurationsdatei \\
\hline 
BESCHREIBUNG: & Die Parameter des Commandline Interface müssen auch über eine Konfigurationsdatei konfigurierbar sein. \\
\hline
BEGRÜNDUNG: & Persistentes Speichern der Parameter \\
\hline
ABHÄNGIGKEITEN: & FA-KI \ref{ki-cli}\\
\end{tabularx}

\begin{tabularx}{16cm}{l|X}
\refstepcounter{table}\label{ki-intelligenz}
\textbf{ID} & \textbf{FA-KI \arabic{table}} \\
\hline
TITEL: & Intelligenzstufen \\
\hline 
BESCHREIBUNG: & Der KI-Client muss die Möglichkeit besitzen, zwischen mehreren Strategien zu wechseln abhängig von der eingestellten Intelligenzstufe. Die Intelligenzstufe kann durch ein Commandline Argument oder in der Konfigurationsdatei festgelegt werden. \\
\hline
BEGRÜNDUNG: & Unterschiedlicher Schwierigkeitsgrad für Anfänger vs. fortgeschrittene Spieler \\
\hline
ABHÄNGIGKEITEN: & FA-KI \ref{ki-cli}, \ref{ki-config}\\
\end{tabularx}

\begin{tabularx}{16cm}{l|X}
\refstepcounter{table}\label{ki-actions}
\textbf{ID} & \textbf{FA-KI \arabic{table}} \\
\hline
TITEL: & Aktionen \\
\hline 
BESCHREIBUNG: & Die KI muss in jeder Runde regelkonforme und sinnvolle Aktionen bestimmen, und diese dem Server senden. \\
\hline
BEGRÜNDUNG: & Grundlegende Funktionalität der KI \\
\hline
ABHÄNGIGKEITEN: & FA-KI \ref{ki-session}\\
\end{tabularx}

\begin{tabularx}{16cm}{l|X}
\refstepcounter{table}\label{ki-api}
\textbf{ID} & \textbf{FA-KI \arabic{table}} \\
\hline
TITEL: & Optional: API \\
\hline 
BESCHREIBUNG: & Die KI soll eine Programmierschnittstelle zur Anbindung an den Benutzer Client besitzen. \\
\hline
BEGRÜNDUNG: & Verwenden der KI zum Vorschlagen von Aktionen für den Spieler \\
\hline
ABHÄNGIGKEITEN: & FA-KI \ref{ki-actions}\\
\end{tabularx}

\subsubsection{Editor}
\begin{tabularx}{16cm}{l|X}
\refstepcounter{table}\label{e-json-encoding}
\textbf{ID} & \textbf{FA-E \arabic{table}} \\
\hline
TITEL: & JSON Encoding \\
\hline
BESCHREIBUNG: & Der Editor muss die Funktionalität besitzen, Konfigurationsdateien für Szenarios, Partien und Charaktere im JSON Format zu encodieren. \\
\hline
BEGRÜNDUNG: & Notwendig zum Speichern der Konfiguration. \\
\hline
PRIORITÄT: & ++\\ 
\hline
ABHÄNGIGKEITEN: & \\
\end{tabularx}

\begin{tabularx}{16cm}{l|X}
\refstepcounter{table}\label{e-json-decoding}
\textbf{ID} & \textbf{FA-E \arabic{table}} \\
\hline
TITEL: & JSON Decoding \\
\hline
BESCHREIBUNG: & Der Editor muss die Funktionalität besitzen, Konfigurationsdateien für Szenarios, Partien und Charaktere aus dem JSON Format zu decodieren. \\
\hline
BEGRÜNDUNG: & Notwendig zum laden der Konfiguration. \\
\hline
PRIORITÄT: & ++\\
\hline
ABHÄNGIGKEITEN: & \\
\end{tabularx}

\begin{tabularx}{16cm}{l|X}
\refstepcounter{table}\label{e-gui}
\textbf{ID} & \textbf{FA-E \arabic{table}} \\
\hline
TITEL: & GUI \\
\hline
BESCHREIBUNG: & Der Editor muss über eine grafische Benutzeroberfläche zum Editieren der Konfiguration besitzen. \\
\hline
BEGRÜNDUNG: & Grundlegende Funktionalität \\
\hline
PRIORITÄT: & +\\
\hline
ABHÄNGIGKEITEN: & \\
\end{tabularx}

\begin{tabularx}{16cm}{l|X}
\refstepcounter{table}\label{e-szenarioedit}
\textbf{ID} & \textbf{FA-E \arabic{table}} \\
\hline
TITEL: & Szenario Editor \\
\hline
BESCHREIBUNG: & Benutzeroberfläche zum Editieren des Szenarios mit Darstellung äquivalent zum Benutzer Client des Spiels \\
\hline
BEGRÜNDUNG: & Grundlegende Funktionalität \\
\hline
PRIORITÄT: & 0\\
\hline
ABHÄNGIGKEITEN: & FA-E \ref{e-gui}\\
\end{tabularx}

\begin{tabularx}{16cm}{l|X}
\refstepcounter{table}\label{e-partieedit}
\textbf{ID} & \textbf{FA-E \arabic{table}} \\
\hline
TITEL: & Partie Editor \\
\hline
BESCHREIBUNG: & Im Partie Editor werden die Partie Konfigurationen (Zeit für Phasen, Wahrscheinlichkeiten) in tabellarischer Form editiert. \\
\hline
BEGRÜNDUNG: & Grundlegende Funktionalität \\
\hline
PRIORITÄT: & +\\
\hline
ABHÄNGIGKEITEN: & FA-E \ref{e-gui}\\
\end{tabularx}

\begin{tabularx}{16cm}{l|X}
\refstepcounter{table}\label{e-charedit}
\textbf{ID} & \textbf{FA-E \arabic{table}} \\
\hline
TITEL: & Charakter Editor \\
\hline
BESCHREIBUNG: & Der Charakter Editor zeigt die existierenden Charaktere in der Konfiguration an und erlaubt es, bestehende und neue Charaktere zu editieren. \\
\hline
BEGRÜNDUNG: & Grundlegende Funktionalität \\
\hline
PRIORITÄT: & -\\
\hline
ABHÄNGIGKEITEN: & FA-E \ref{e-gui}\\
\end{tabularx}


\clearpage
\subsubsection{Allgemeine funktionale Anforderungen}
%Hier kommen die generellen FA rein

\begin{tabularx}{16cm}{l|X}
	\refstepcounter{table}\label{Felder}
	\textbf{ID} & \textbf{FA-G \arabic{table}} \\
	\hline
	TITEL: & Felder \\
	\hline
	BESCHREIBUNG: & Die schachbrettartigen Felder, aus denen das Spielbrett aufgebaut ist, können freier Raum sein, oder mit Hindernissen oder Objekten besetzt sein. \\
	\hline
	BEGRÜNDUNG: & Unterschiedliche Arten von Feldern erlauben den Charakteren das Fortbewegen und interagieren auf dem Spielbrett. \\
	\hline
	ABHÄNGIGKEITEN: & FA-G \ref{Spielbrett} \\
\end{tabularx}

\begin{tabularx}{16cm}{l|X}
	\refstepcounter{table}\label{Spielbrett}
	\textbf{ID} & \textbf{FA-G \arabic{table}} \\
	\hline
	TITEL: & Spielbrett \\
	\hline
	BESCHREIBUNG: & Das Spiel findet auf einem Spielbrett statt, welches aus den Feldern eines kartesischen Gitters aufgebaut ist. \\
	\hline
	BEGRÜNDUNG: & Der Aufbau aus Feldern in einem kartesischen Gitter ermöglicht die Fortbewegung der Charaktere durch Bewegungspunkte. \\
	\hline
	ABHÄNGIGKEITEN: & FA-G \ref{Felder} \\
\end{tabularx}

\begin{tabularx}{16cm}{l|X}
	\refstepcounter{table}\label{Entfernung}
	\textbf{ID} & \textbf{FA-G \arabic{table}} \\
	\hline
	TITEL: & Entfernung \\
	\hline
	BESCHREIBUNG: & Die Entfernung zwischen zwei Feldern A und B ist die minimale Anzahl an Schritten auf Nachbarfelder (in alle acht Richtungen), die benötigt wird um von A nach B zu gelangen. \\
	\hline
	BEGRÜNDUNG: & Mit der Entfernung zwischen zwei Feldern A und B wird berechnet, wieviele Bewegungspunkte ein Charakter benötigt, um von Feld A zu Feld B zu gelangen. \\
	\hline
	ABHÄNGIGKEITEN: & FA-G \ref{Felder}, FA-G \ref{Spielbrett} \\
\end{tabularx}

\begin{tabularx}{16cm}{l|X}
	\refstepcounter{table}\label{Sichtlinie}
	\textbf{ID} & \textbf{FA-G \arabic{table}} \\
	\hline
	TITEL: & Sichtlinie \\
	\hline
	BESCHREIBUNG: & Es existiert eine Sichtlinie zwischen zwei Feldern A und B, wenn gilt: Betrachtet man die Verbindungslinie vom Mittelpunkt von A zum Mittelpunkt von B, so müssen alle Felder, die von dieser Verbindunglinie geschnitten werden Felder sein, die die Sichtlinie nicht blockieren. Dann ist B von A aus sichtbar. Felder, die von dieser Verbindungslinie nur tangiert werden blockieren die Sicht nicht. \\
	\hline
	BEGRÜNDUNG: & Dadurch wird festgestellt, ob ein Feld B von einem Feld A aus sichtbar ist, also ob ein Charakter auf Feld A sehen kann, was sich auf Feld B befindet. \\
	\hline
	ABHÄNGIGKEITEN: & FA-G \ref{Felder}, FA-G \ref{Spielbrett} \\
\end{tabularx}

\begin{tabularx}{16cm}{l|X}
	\refstepcounter{table}\label{Freies Feld}
	\textbf{ID} & \textbf{FA-G \arabic{table}} \\
	\hline
	TITEL: & Freies Feld \\
	\hline
	BESCHREIBUNG: & Charaktere können auf einem Freien Feld stehen oder darüber hinweglaufen. Es blockiert die Sichtlinie nicht. \\
	\hline
	BEGRÜNDUNG: & Freie Felder dienen der Fortbewegung der Charaktere. \\
	\hline
	ABHÄNGIGKEITEN: & FA-G \ref{Felder}, FA-G \ref{Sichtlinie} \\
\end{tabularx}

\begin{tabularx}{16cm}{l|X}
	\refstepcounter{table}\label{Wand}
	\textbf{ID} & \textbf{FA-G \arabic{table}} \\
	\hline
	TITEL: & Wand \\
	\hline
	BESCHREIBUNG: & Ein Feld, auf welchem sich eine Wand befindet ist nicht betretbar und es blockiert die Sichtlinie. \\
	\hline
	BEGRÜNDUNG: & Durch Wände können große Räume getrennt werden oder neue Räume entstehen. \\
	\hline
	ABHÄNGIGKEITEN: & FA-G \ref{Felder}, FA-G \ref{Sichtlinie} \\
\end{tabularx}

\begin{tabularx}{16cm}{l|X}
	\refstepcounter{table}\label{Kamin-Feld}
	\textbf{ID} & \textbf{FA-G \arabic{table}} \\
	\hline
	TITEL: & Kamin-Feld \\
	\hline
	BESCHREIBUNG: & Kamin-Felder sind nicht betretbar und sie blockieren die Sichtlinie. Befindet sich ein Charakter mit klammen Klamotten am Anfang einer Runde auf einem Nachbarfeld eines Kamin-Feldes, so wird die Klamme Klamotten-Eigenschaft am Ende der Runde entfernt. \\
	\hline
	BEGRÜNDUNG: & Der Kamin trocknet die klammen Klamotten, dadurch kann ein Charakter diese Eigenschaft los werden. \\
	\hline
	ABHÄNGIGKEITEN: & FA-G \ref{Felder}, FA-G \ref{Sichtlinie} \\
\end{tabularx}

\begin{tabularx}{16cm}{l|X}
	\refstepcounter{table}\label{Sitzplatz}
	\textbf{ID} & \textbf{FA-G \arabic{table}} \\
	\hline
	TITEL: & Sitzplatz \\
	\hline
	BESCHREIBUNG: & Auf einem solchen Feld befindet sich eine Sitzgelegenheit, bspw. ein Sessel, Barhocker, etc. Das Feld ist betretbar und blockiert die Sichtlinie nicht. Befindet sich ein Charakter am Anfang einer Runde auf einem Sitzplatz-Feld, so werden seine Health Points am Ende der Runde auf den Maximalwert aufgefüllt. \\
	\hline
	BEGRÜNDUNG: & Dadurch können Charaktere mit wenigen Health Points diese wieder füllen. \\
	\hline
	ABHÄNGIGKEITEN: & FA-G \ref{Felder}, FA-G \ref{Sichtlinie} \\
\end{tabularx}

\begin{tabularx}{16cm}{l|X}
	\refstepcounter{table}\label{Spielchips}
	\textbf{ID} & \textbf{FA-G \arabic{table}} \\
	\hline
	TITEL: & Spielchips \\
	\hline
	BESCHREIBUNG: & Charaktere haben die Möglichkeit an Roulette-Tischen Spielchips zu gewinnen. Alle gesammelten Spielchips einer Fraktion werden am Ende einer Partie in Intelligence Points umgerechnet. \\
	\hline
	BEGRÜNDUNG: & Dadurch wird das Spiel spannender, da die Spielchips den Spielausgang beeinflussen. \\
	\hline
	ABHÄNGIGKEITEN: & FA-G \ref{Roulette-Tisch} \\
\end{tabularx}

\begin{tabularx}{16cm}{l|X}
	\refstepcounter{table}\label{Roulette-Tisch}
	\textbf{ID} & \textbf{FA-G \arabic{table}} \\
	\hline
	TITEL: & Roulette-Tisch \\
	\hline
	BESCHREIBUNG: & Ein Feld, auf welchem ein Roulette-Tisch steht ist nicht betretbar und es blockiert die Sichtlinie nicht. Wenn ein Charakter auf einem Nachbarfeld eines Roulette-Tisches steht, kann er als Aktion einmal Roulette spielen. Jeder Roulette-Tisch verfügt zu Beginn einer Partie über eine im Szenario festgelegte individuelle Anzahl an Chips. \\
	\hline
	BEGRÜNDUNG: & An Roulette-Tischen haben Charaktere die Chance Chips zu gewinnen bzw. zu verlieren. \\
	\hline
	ABHÄNGIGKEITEN: & FA-G \ref{Felder}, FA-G \ref{Sichtlinie} \\
\end{tabularx}

\begin{tabularx}{16cm}{l|X}
	\refstepcounter{table}\label{Bar-Tisch}
	\textbf{ID} & \textbf{FA-G \arabic{table}} \\
	\hline
	TITEL: & Bar-Tisch \\
	\hline
	BESCHREIBUNG: & Felder, auf denen sich ein Bar-Tisch befindet sind nicht betretbar und sie blockieren die Sichtlinie nicht. Zu Beginn jeder Runde erscheint auf jedem leeren Bar-Tisch ein Cocktail. \\
	\hline
	BEGRÜNDUNG: & An Bar-Tischen können die Charaktere Cocktails aufnehmen. \\
	\hline
	ABHÄNGIGKEITEN: & FA-G \ref{Felder}, FA-G \ref{Cocktail aufnehmen}, FA-G \ref{Sichtlinie} \\
\end{tabularx}

\begin{tabularx}{16cm}{l|X}
	\refstepcounter{table}\label{Tresor}
	\textbf{ID} & \textbf{FA-G \arabic{table}} \\
	\hline
	TITEL: & Tresor \\
	\hline
	BESCHREIBUNG: & Felder, auf denen sich ein Tresor befindet, sind nicht betretbar und sie blockieren die Sichtlinie nicht. Tresore können Geheiminformationen oder Gadgets enthalten. Die Tresore sind eindeutig und sichtbar durchnummeriert (1, 2, ...). \\
	\hline
	BEGRÜNDUNG: & An Tresoren können Charaktere Geheimnisse oder Gadgets erhalten. \\
	\hline
	ABHÄNGIGKEITEN: & FA-G \ref{Felder}, FA-G \ref{Sichtlinie},  \todo[inline]{2.4 Gadgets, 2.7 Geheimnisse} \\
\end{tabularx}

\begin{tabularx}{16cm}{l|X}
	\refstepcounter{table}\label{Charakter}
	\textbf{ID} & \textbf{FA-G \arabic{table}} \\
	\hline
	TITEL: & Charakter \\
	\hline
	BESCHREIBUNG: & Ein Charakter besitzt folgende Werte, die normalerweise für die gegnerische Fraktion nicht sichtbar sind: 
	\begin{itemize}
		\item Name
		\item Beschreibung
		\item Position
		\item Fraktion
		\item Bewegungspunkte (BP) und Aktionspunkte (AP)
		\item Health Points (HP)
		\item Intelligence Points (IP)
		\item Eigenschaften
		\item Inventar
		\item hält Cocktail
		\item Spielchips
	\end{itemize}
	\\
	\hline
	BEGRÜNDUNG: & Jeder Charakter soll individuell sein und sich über das Spiel hinweg verändern können.\\
	\hline
	ABHÄNGIGKEITEN: & FA-G \ref{Charakter-NameBeschreibung}, FA-G \ref{Charakter-Position}, FA-G \ref{Charakter-Fraktion},  FA-G \ref{BP und AP}, FA-G \ref{HP}, FA-G \ref{IP}, FA-G \ref{Eigenschaften}, FA-G \ref{Inventar}, FA-G \ref{haelt Cocktail}, FA-G \ref{Spielchips} \\
\end{tabularx}

\begin{tabularx}{16cm}{l|X}
	\refstepcounter{table}\label{Charakter-NameBeschreibung}
	\textbf{ID} & \textbf{FA-G \arabic{table}} \\
	\hline
	TITEL: & Charakter-Name und -Beschreibung \\
	\hline
	BESCHREIBUNG: & Jeder Charakter hat einen eindeutigen Namen und die Charakter-Beschreibung beschreibt den Charakter aus \glqq James Bond\grqq in wenigen Sätzen.\\
	\hline
	BEGRÜNDUNG: & Zusätzliche Informationen für den Spieler. \\
	\hline
	ABHÄNGIGKEITEN: & \\
\end{tabularx}

\begin{tabularx}{16cm}{l|X}
	\refstepcounter{table}\label{Charakter-Position}
	\textbf{ID} & \textbf{FA-G \arabic{table}} \\
	\hline
	TITEL: & Charakter-Position \\
	\hline
	BESCHREIBUNG: & Gibt die Position des Charakters auf dem Spielfeld in x- und y-Koordinaten an.\\
	\hline
	BEGRÜNDUNG: & Hält fest, wo sich der Charakter auf dem Spielfeld befindet.\\
	\hline
	ABHÄNGIGKEITEN: & FA-G \ref{Exfiltratation},  \todo[inline]{Spielfeld, 2.9. Beginn der Partie, 2.5 Bewegung, }\\
\end{tabularx}

\begin{tabularx}{16cm}{l|X}
	\refstepcounter{table}\label{Charakter-Fraktion}
	\textbf{ID} & \textbf{FA-G \arabic{table}} \\
	\hline
	TITEL: & Charakter-Fraktion \\
	\hline
	BESCHREIBUNG: & Gibt an, zu welcher Fraktion der Charakter gehört. Mögliche Fraktionen sind Spieler1, Spieler2 oder NPC.\\
	\hline
	BEGRÜNDUNG: & Hält fest, von wem der Charakter zu steuern ist.\\
	\hline
	ABHÄNGIGKEITEN: & \todo[inline]{2.8.1 Wahlphase}\\
\end{tabularx}

\begin{tabularx}{16cm}{l|X}
	\refstepcounter{table}\label{BP und AP}
	\textbf{ID} & \textbf{FA-G \arabic{table}} \\
	\hline
	TITEL: & Bewegungspunkte (BP) und Aktionspunkte (AP) \\
	\hline
	BESCHREIBUNG: & Während eine Charakter am Zug ist, kann er BP für Bewegungen auf dem Spielfeld und AP für Aktionen einsetzen.
	Zu Beginn eines Zuges erhält er jeweils Punkte. Nach Beenden eines Zuges verfallen übrig gebliebene Punkte. BP und AP können nicht negativ sein.\\
	\hline
	BEGRÜNDUNG: & Hält fest, wie viele Bewegungen und Aktionen der Charakter in diesem Spielzug noch ausführen kann. \\
	\hline
	ABHÄNGIGKEITEN: & FA-G \ref{Flinkheit}, FA-G \ref{Schwerfaelligkeit}, FA-G \ref{Behaendigkeit}, FA-G \ref{Behaebigkeit}, FA-G \ref{Agilitaet} \todo[inline]{2.10.1 Züge, 2.5 Bewegung, 2.6 Aktionen}\\
\end{tabularx}

\begin{tabularx}{16cm}{l|X}
	\refstepcounter{table}\label{HP}
	\textbf{ID} & \textbf{FA-G \arabic{table}} \\
	\hline
	TITEL: & Health Points (HP) \\
	\hline
	BESCHREIBUNG: & HP geben Auskunft über den Gesundheitszustand des Charakters. Zu Spielbeginn werden die HP auf 100 gesetzt. Während des Spiels sorgen verschiedene Aktionen dafür, dass HP hinzugefügt oder abgezogen werden. Werden HP abgezogen, spricht man von Schaden. Die Punkte nehmen Werte zwischen 0 und 100 an.\\
	\hline
	BEGRÜNDUNG: & Zeigt Auswirkung verschiedener Aktionen auf den Charakter. \\
	\hline
	ABHÄNGIGKEITEN: & FA-G \ref{Robuster Magen}, FA-G \ref{Zaehigkeit} \todo[inline]{2.6 Aktionen, 2.4 Gadgets}\\
\end{tabularx}

\begin{tabularx}{16cm}{l|X}
	\refstepcounter{table}\label{Exfiltratation}
	\textbf{ID} & \textbf{FA-G \arabic{table}} \\
	\hline
	TITEL: & Exfiltration \\
	\hline
	BESCHREIBUNG: & Sinken die HP eines Charakter auf 0, so wird eine Exfiltration durchgeführt. Dabei wird der Charakter auf ein zufällig gewähltes freies Sitzplatz-Feld versetzt und seine HP auf 1 gesetzt. Ist kein freier Sitzplatz vorhanden, so wird ein Sitzplatz zufällig ausgewählt und die Person, die dort saß, wird auf ein zufälliges freies Nachbarfeld des Sitzplatzes platziert.\\
	\hline
	BEGRÜNDUNG: & HP sollen nicht 0 sein.\\
	\hline
	ABHÄNGIGKEITEN: & FA-G \ref{HP} \todo[inline]{2.12 Zufall und Alternativen}\\
\end{tabularx}

\begin{tabularx}{16cm}{l|X}
	\refstepcounter{table}\label{IP}
	\textbf{ID} & \textbf{FA-G \arabic{table}} \\
	\hline
	TITEL: & Intelligence Points (IP) \\
	\hline
	BESCHREIBUNG: & IP geben Auskunft, über die Spionagefähigkeiten des Charakter. Zu Beginn besitzen die IP den Wert 0.\\
	\hline
	BEGRÜNDUNG: & Punkte, die später für Sieg relevant sind.\\
	\hline
	ABHÄNGIGKEITEN: & \todo[inline]{2.4. Gaspatronen-Lippenstift, 2.4 Wanze und Ohrstöpsel, 2.4. Chicken Feed, 2.6 Roulette?, 2.7 Geheimnisse, Abhängigkeit von 2.11}\\
\end{tabularx}


\begin{tabularx}{16cm}{l|X}
	\refstepcounter{table}\label{Eigenschaften}
	\textbf{ID} & \textbf{FA-G \arabic{table}} \\
	\hline
	TITEL: & Eigenschaften \\
	\hline
	BESCHREIBUNG: & Sind entweder dauerhafte Fähigkeiten eines Charakters oder vorübergehende Zustände. Fähigkeiten kommen passiv zum Tragen oder ermöglichen dem Charakter bestimmte Aktionen. Zustände werden durch Aktionen erworben oder verloren.\\
	\hline
	BEGRÜNDUNG: & Fähigkeiten sorgen für individuelle Charaktere und Eigenschaften halten den aktuellen Zustand des Charakters fest. \\
	\hline
	ABHÄNGIGKEITEN: & FA-G \ref{Flinkheit}, FA-G \ref{Schwerfaelligkeit}, FA-G \ref{Behaendigkeit}, FA-G \ref{Behaebigkeit}, FA-G \ref{Agilitaet}, FA-G \ref{Glueckspilz}, FA-G \ref{Pechvogel}, FA-G \ref{Klamme Klamotten}, FA-G \ref{Konstant Klamme Klamotten}, FA-G \ref{Robuster Magen}, FA-G \ref{Zaehigkeit}, FA-G \ref{Babysitter}, FA-G \ref{Honey Trap}, FA-G \ref{Bang and Burn}, FA-G \ref{Flaps and Seals}, FA-G \ref{Tradecraft}, FA-G \ref{Observation} \\
\end{tabularx}

\begin{tabularx}{16cm}{l|X}
	\refstepcounter{table}\label{Inventar}
	\textbf{ID} & \textbf{FA-G \arabic{table}} \\
	\hline
	TITEL: & Inventar \\
	\hline
	BESCHREIBUNG: & Im Inventar sind alle Gadgets aufgelistet, die der Charakter aktuell bei sich trägt.\\
	\hline
	BEGRÜNDUNG: & Hält fest, welche Gadgets der Charakter nutzen kann.\\
	\hline
	ABHÄNGIGKEITEN: & \todo[inline]{2.4 Gadgets}\\
\end{tabularx}

\begin{tabularx}{16cm}{l|X}
	\refstepcounter{table}\label{haelt Cocktail}
	\textbf{ID} & \textbf{FA-G \arabic{table}} \\
	\hline
	TITEL: & hält Cocktail \\
	\hline
	BESCHREIBUNG: & Gibt an, ob der Charakter einen Cocktail in seiner Hand hält oder nicht. Ein Charakter kann maximal einen Cocktail in der Hand halten. Wird ein Cocktail in der Hand gehalten, so sind Aktionen mit diesem möglich. \\
	\hline
	BEGRÜNDUNG: & Hält fest, ob Interaktion mit einem Cocktail möglich ist oder nicht.\\
	\hline
	ABHÄNGIGKEITEN: & \todo[inline]{2.6 Cocktail}\\
\end{tabularx}

\begin{tabularx}{16cm}{l|X}
	\refstepcounter{table}\label{Spielchips}
	\textbf{ID} & \textbf{FA-G \arabic{table}} \\
	\hline
	TITEL: & Spielchips \\
	\hline
	BESCHREIBUNG: & Zu Beginn des Spiels besitzt jeder Charakter 10 Spielchips. Mit diesen kann er an Roulette-Tischen spielen und dadurch seine Anzahl an Spielchips erhöhen oder verringern.\\
	\hline
	BEGRÜNDUNG: & Werden benötigt, um Roulette zu spielen.\\
	\hline
	ABHÄNGIGKEITEN: & \todo[inline]{2.6 Roulette}\\
\end{tabularx}

\begin{tabularx}{16cm}{l|X}
	\refstepcounter{table}\label{Flinkheit}
	\textbf{ID} & \textbf{FA-G \arabic{table}} \\
	\hline
	TITEL: & Flinkheit \\
	\hline
	BESCHREIBUNG: & Besitzt ein Charakter die Fähigkeit Flinkheit, so erhält er in jeder Runde drei Bewegungspunkte.\\
	\hline
	BEGRÜNDUNG: & Unterschiedliche Anzahl an BP und AP je nach Charakter.\\
	\hline
	ABHÄNGIGKEITEN: & \\
\end{tabularx}

\begin{tabularx}{16cm}{l|X}
	\refstepcounter{table}\label{Schwerfaelligkeit}
	\textbf{ID} & \textbf{FA-G \arabic{table}} \\
	\hline
	TITEL: & Schwerfälligkeit \\
	\hline
	BESCHREIBUNG: & Besitzt ein Charakter die Fähigkeit Schwerfälligkeit, so erhält er in jeder Runde einen Bewegungspunkt.\\
	\hline
	BEGRÜNDUNG: & Unterschiedliche Anzahl an BP und AP je nach Charakter.\\
	\hline
	ABHÄNGIGKEITEN: & \\
\end{tabularx}

\begin{tabularx}{16cm}{l|X}
	\refstepcounter{table}\label{Behaendigkeit}
	\textbf{ID} & \textbf{FA-G \arabic{table}} \\
	\hline
	TITEL: & Behändigkeit \\
	\hline
	BESCHREIBUNG: & Besitzt ein Charakter die Fähigkeit Behändigkeit, so erhält er in jeder Runde zwei Aktionspunkte.\\
	\hline
	BEGRÜNDUNG: & Unterschiedliche Anzahl an BP und AP je nach Charakter.\\
	\hline
	ABHÄNGIGKEITEN: & \\
\end{tabularx}

\begin{tabularx}{16cm}{l|X}
	\refstepcounter{table}\label{Behaebigkeit}
	\textbf{ID} & \textbf{FA-G \arabic{table}} \\
	\hline
	TITEL: & Behäbigkeit \\
	\hline
	BESCHREIBUNG: & Besitzt ein Charakter die Fähigkeit Behäbigkeit, so wird ihm zu Beginn jeder Runde zufällig entweder ein Bewegungspunkt oder einer Aktionspunkt abgezogen.\\
	\hline
	BEGRÜNDUNG: & Unterschiedliche Anzahl an BP und AP je nach Charakter.\\
	\hline
	ABHÄNGIGKEITEN: & \todo[inline]{2.12 Zufall}\\
\end{tabularx}

\begin{tabularx}{16cm}{l|X}
	\refstepcounter{table}\label{Agilitaet}
	\textbf{ID} & \textbf{FA-G \arabic{table}} \\
	\hline
	TITEL: & Agilität \\
	\hline
	BESCHREIBUNG: & Besitzt ein Charakter die Fähigkeit Agilität, so wird ihm zu Beginn jeder Runde zufällig entweder ein Bewegungspunkt oder einer Aktionspunkt hinzugefügt.\\
	\hline
	BEGRÜNDUNG: & Unterschiedliche Anzahl an BP und AP je nach Charakter.\\
	\hline
	ABHÄNGIGKEITEN: & \todo[inline]{2.12 Zufall}\\
\end{tabularx}

\begin{tabularx}{16cm}{l|X}
	\refstepcounter{table}\label{Glueckspilz}
	\textbf{ID} & \textbf{FA-G \arabic{table}} \\
	\hline
	TITEL: & Glückspilz \\
	\hline
	BESCHREIBUNG: & Besitzt ein Charakter die Fähigkeit Glückspilz, so beträgt seine Gewinnchance beim Roulette $\frac{23}{37}$.\\
	\hline
	BEGRÜNDUNG: & Unterschiedliche Gewinnchancen beim Roulette je nach Charakter.\\
	\hline
	ABHÄNGIGKEITEN: & \todo[inline]{2.12 Zufall, Abhängigkeit von 2.6 Roulette}\\
\end{tabularx}

\begin{tabularx}{16cm}{l|X}
	\refstepcounter{table}\label{Pechvogel}
	\textbf{ID} & \textbf{FA-G \arabic{table}} \\
	\hline
	TITEL: & Pechvogel \\
	\hline
	BESCHREIBUNG: & Besitzt ein Charakter die Fähigkeit Pechvogel, so beträgt seine Gewinnchance beim Roulette $\frac{13}{37}$.\\
	\hline
	BEGRÜNDUNG: & Unterschiedliche Gewinnchancen beim Roulette je nach Charakter. \\
	\hline
	ABHÄNGIGKEITEN: & \todo[inline]{2.12 Zufall, Abhängigkeit von 2.6 Roulette}\\
\end{tabularx}

\begin{tabularx}{16cm}{l|X}
	\refstepcounter{table}\label{Klamme Klamotten}
	\textbf{ID} & \textbf{FA-G \arabic{table}} \\
	\hline
	TITEL: & Klamme Klamotten \\
	\hline
	BESCHREIBUNG: & Besitzt ein Charakter den Zustand Klamme Klamotten, so halbiert sich seine Erfolgswahrscheinlichkeit bei einer Wahrscheinlichkeitsprobe. \\
	\hline
	BEGRÜNDUNG: & Unterschiedliche Erfolgswahrscheinlichkeiten je nach Zustand des Charakters.\\
	\hline
	ABHÄNGIGKEITEN: & FA-G \ref{Konstant Klamme Klamotten} \todo[inline]{2.1 Kamin-Feld, 2.4 Föhn, 2.6 Cocktail, 2.12 Zufall}\\
\end{tabularx}

\begin{tabularx}{16cm}{l|X}
	\refstepcounter{table}\label{Konstant Klamme Klamotten}
	\textbf{ID} & \textbf{FA-G \arabic{table}} \\
	\hline
	TITEL: & Konstant Klamme Klamotten \\
	\hline
	BESCHREIBUNG: & Besitzt ein Charakter die Fähigkeit Konstant Klamme Klamotten, so hat er dauerhaft den Zustand Klamme Klamotten.\\
	\hline
	BEGRÜNDUNG: & Unterschiedliche Erfolgswahrscheinlichkeiten je nach Charakter.\\
	\hline
	ABHÄNGIGKEITEN: & \\
\end{tabularx}

\begin{tabularx}{16cm}{l|X}
	\refstepcounter{table}\label{Robuster Magen}
	\textbf{ID} & \textbf{FA-G \arabic{table}} \\
	\hline
	TITEL: & Robuster Magen \\
	\hline
	BESCHREIBUNG: & Besitzt ein Charakter die Fähigkeit Robuster Magen, so erhält er die doppelte Anzahl Cocktail-HP und nur den halben Gift-Cocktail-Schaden durch vergiftete Cocktails.\\
	\hline
	BEGRÜNDUNG: & Unterschiedliche Health Points je nach Charakter.\\
	\hline
	ABHÄNGIGKEITEN: & \todo[inline]{Cocktail-HP, Gift-Cocktail-Schaden}\\
\end{tabularx}

\begin{tabularx}{16cm}{l|X}
	\refstepcounter{table}\label{Zaehigkeit}
	\textbf{ID} & \textbf{FA-G \arabic{table}} \\
	\hline
	TITEL: & Zähigkeit \\
	\hline
	BESCHREIBUNG: & Besitzt ein Charakter die Fähigkeit Zähigkeit, so wird jeder Schaden, der nicht durch vergiftete Cocktails entsteht, um die Hälfte reduziert.\\
	\hline
	BEGRÜNDUNG: & Unterschiedliche Health Points je nach Charakter.\\
	\hline
	ABHÄNGIGKEITEN: & \\
\end{tabularx}

\begin{tabularx}{16cm}{l|X}
	\refstepcounter{table}\label{Babysitter}
	\textbf{ID} & \textbf{FA-G \arabic{table}} \\
	\hline
	TITEL: & Babysitter \\
	\hline
	BESCHREIBUNG: & Besitzt ein Charakter die Fähigkeit Babysitter, so wehrt er Angriffe auf benachbarte Charaktere der eigenen Fraktion mit der vorgegebenen Babysitter-Wahrscheinlichkeit ab. Bei einem Angriff wird zuerst die Wahrscheinlichkeitsprobe durch den Angreifer gemacht und wenn diese erfolgreich ist wird die Babysitter Fähigkeit eingesetzt. Ist der Charakter mit der Fähigkeit Babysitter erfolgreich, so misslingt der Angriff, ohne dass die gegnerische Fraktion etwas von der Fähigkeit erfährt, ansonsten wird der Angriff durchgeführt.\\
	\hline
	BEGRÜNDUNG: & Ermöglicht es Angriffen zu entgehen.\\
	\hline
	ABHÄNGIGKEITEN: & \todo[inline]{Babysitter-Wahrscheinlichkeit, Wahrscheinlichkeitsprobe, Abhängigkeit von 2.6 Aktion, 2.4 Gadgets}\\
\end{tabularx}

\begin{tabularx}{16cm}{l|X}
	\refstepcounter{table}\label{Honey Trap}
	\textbf{ID} & \textbf{FA-G \arabic{table}} \\
	\hline
	TITEL: & Honey Trap \\
	\hline
	BESCHREIBUNG: & Besitzt ein Charakter die Fähigkeit Honey Trap, so werden mit der vorgegebenen Honey-Trap-Wahrscheinlichkeit Angriffe anstatt auf diesen Charakter auf einen zufälligen anderen Charakter ausgeübt, insofern dieser andere Charakter ebenfalls Ziel der Aktion hätte sein können.\\
	\hline
	BEGRÜNDUNG: & Ermöglicht es Angriffen zu entgehen.\\
	\hline
	ABHÄNGIGKEITEN: & \todo[inline]{Honey-Trap-Wahrscheinlichkeit}\\
\end{tabularx}

\begin{tabularx}{16cm}{l|X}
	\refstepcounter{table}\label{Bang and Burn}
	\textbf{ID} & \textbf{FA-G \arabic{table}} \\
	\hline
	TITEL: & Bang and Burn \\
	\hline
	BESCHREIBUNG: & Besitzt ein Charakter die Fähigkeit Bang and Burn, so kann er einen benachbarten Roulette-Tisch unbrauchbar machen. Auf diesem kann dann nicht mehr gespielt werden.\\
	\hline
	BEGRÜNDUNG: & Verhindert das Sammeln von Spielchips, die später über den Sieg mitentscheiden.\\
	\hline
	ABHÄNGIGKEITEN: & \todo[inline]{Abhängigkeit von 2.1 Roulette-Tisch}\\
\end{tabularx}

\begin{tabularx}{16cm}{l|X}
\refstepcounter{table}\label{Flaps and Seals}
\textbf{ID} & \textbf{FA-G \arabic{table}} \\
\hline
TITEL: & Flaps and Seals \\
\hline
BESCHREIBUNG: & Besitzt ein Charakter die Fähigkeit Flaps and Seals, so kann er in einen Tresor spicken, der zwei Felder von ihm entfernt und somit nicht auf einem Nachbarfeld steht.\\
\hline
BEGRÜNDUNG: & Weniger Bewegungspunkte notwendig, um zu Tresor zu gelangen.\\
\hline
ABHÄNGIGKEITEN: & \todo[inline]{2.4 Maulwürfel, Abhängigkeit von 2.6 Tresor-Spicken}\\
\end{tabularx}


\begin{tabularx}{16cm}{l|X}
\refstepcounter{table}\label{Tradecraft}
\textbf{ID} & \textbf{FA-G \arabic{table}} \\
\hline
TITEL: & Tradecraft \\
\hline
BESCHREIBUNG: & Besitzt ein Charakter die Fähigkeit Tradecraft, so wiederholt er eine fehlgeschlagene Wahrscheinlichkeitsprobe einer Aktion einmal.\\
\hline
BEGRÜNDUNG: & Höhere Chance für eine erfolgreiche Aktion.\\
\hline
ABHÄNGIGKEITEN: & \todo[inline]{Wahrscheinlichkeitsprobe, 2.4 Maulwuerfel}\\
\end{tabularx}

\begin{tabularx}{16cm}{l|X}
\refstepcounter{table}\label{Observation}
\textbf{ID} & \textbf{FA-G \arabic{table}} \\
\hline
TITEL: & Observation \\
\hline
BESCHREIBUNG: & Besitzt ein Charakter die Fähigkeit Observation, so kann er diese als Aktion gegen einen anderen Charakter in Sichtlinie ausführen. Dabei wird mit der vorgegebenen Observation-Erfolg-Wahrscheinlichkeit aufgedeckt, ob der Charakter zur gegnerischen Fraktion gehört oder nicht. Die Aktion bleibt vom observierten Charakter unbemerkt. \\
\hline
BEGRÜNDUNG: & Möglichkeit um herauszufinden, welche Charaktere zur gegnerischen Fraktion gehören und welche NPCs sind.\\
\hline
ABHÄNGIGKEITEN: & \todo[inline]{Observation-Erfolg-Wahrscheinlichkeit, 2.4 Maulwuerfel}\\
\end{tabularx}

\begin{tabularx}{16cm}{l|X}
	\refstepcounter{table}\label{Gadgets}
	\textbf{ID} & \textbf{FA-G \arabic{table}} \\
	\hline
	TITEL: & Gadgets \\
	\hline
	BESCHREIBUNG: & Charaktere können Gadgets in ihrem Inventar bei sich tragen. Sie verschaffen ihrem Besitzer bestimmte Eigenschaften oder ermöglichen bestimmte Aktionen. Jedes Gadget kommt höchstens einmal im Spiel vor und kann, falls nicht anders festgelegt, mehrmals verwendet werden. \\
	\hline
	BEGRÜNDUNG: & Dadurch wird das Spiel spannender, denn die Gadgets beeinflussen den Spielverlauf. \\
	\hline
	ABHÄNGIGKEITEN: & FA-G \ref{Charakter}, FA-G \ref{Inventar}, FA-G \ref{Eigenschaften}, FA-G \ref{Aktion durchführen} \\
\end{tabularx}

\begin{tabularx}{16cm}{l|X}
	\refstepcounter{table}\label{Akku-Foehn}
	\textbf{ID} & \textbf{FA-G \arabic{table}} \\
	\hline
	TITEL: & Akku-Föhn \\
	\hline
	BESCHREIBUNG: & Wird der Akku-Föhn von einem Charakter als Aktion an sich selbst oder einem benachbarten Charakter angewendet, so verliert dieser Ziel-Charakter die Klamme Klamotten-Eigenschaft. Der Akku-Föhn kann beliebig oft eingesetzt werden. \\
	\hline
	BEGRÜNDUNG: & Das ermöglicht es den Charakteren die Klamme Klamotten-Eigenschaft loszuwerden.  \\
	\hline
	ABHÄNGIGKEITEN: & FA-G \ref{Gadgets}, FA-G \ref{Charakter}, FA-G \ref{Aktion durchführen} \\
\end{tabularx}

\begin{tabularx}{16cm}{l|X}
	\refstepcounter{table}\label{Maulwuerfel}
	\textbf{ID} & \textbf{FA-G \arabic{table}} \\
	\hline
	TITEL: & Maulwürfel \\
	\hline
	BESCHREIBUNG: & Hat ein Charakter den Maulwürfel im Inventar, so werden die Eigenschaften Tradecraft, Flaps and Seals und Observation deaktiviert, falls er sie hat. Er verliert diese Eigenschaften vorübergehend und bekommt sie wieder, wenn er den Maulwürfel losgeworden ist. Mit einer Aktion kann ein Charakter den Maulwürfel auf ein beliebiges Nicht-Wand-Feld in Sichtweite und Maulwürfel-Wurfweite werfen. Befindet sich auf diesem Zielfeld ein Charakter, nimmt dieser den Maulwürfel in sein Inventar auf. Ansonsten prallt der Maulwürfel von dem Zielfeld ab und landet im Inventar des sich am nächsten befindenden Charakters. Falls der Maulwürfel im Inventar eines NPC ist, macht dieser bei seinem nächsten Zug einen Maulwürfelwurf auf ein zufälliges Zielfeld. \\
	\hline
	BEGRÜNDUNG: &  Der Maulwürfel deaktiviert bei dem Charakter, der ihn im Inventar hat positive Eigenschaften und bringt dadurch Spannung in das Spiel. \\
	\hline
	ABHÄNGIGKEITEN: & FA-G \ref{Gadgets}, FA-G \ref{Charakter}, FA-G \ref{Inventar},FA-G \ref{Eigenschaften}, \todo[inline]{Maulwürfel-Wurfweite} \\
\end{tabularx}

\begin{tabularx}{16cm}{l|X}
	\refstepcounter{table}\label{Technicolor-Prisma}
	\textbf{ID} & \textbf{FA-G \arabic{table}} \\
	\hline
	TITEL: & Technicolor-Prisma \\
	\hline
	BESCHREIBUNG: & Durch eine Aktion kann das Technicolor-Prisma an einem Roulette-Tisch installiert werden und vertauscht dann die Farben rot und schwarz. Dadurch wird das Resultat beim Roulette-Spielen negiert, d.h. ein Charakter verliert, wenn er normalerweise gewonnen hätte und umgekehrt. Damit sind die Erfolgswahrscheinlichkeiten für Glückspilze und Pechvögel beim Roulette-Spielen vertauscht. Ein Technicolor-Prisma kann nur einmal verwendet werden. \\
	\hline
	BEGRÜNDUNG: &  Durch das Technicolor-Prisma wird das Roulette-Spielen beeinflusst. \\
	\hline
	ABHÄNGIGKEITEN: & FA-G \ref{Gadgets}, FA-G \ref{Roulette spielen}, \\
\end{tabularx}

\begin{tabularx}{16cm}{l|X}
	\refstepcounter{table}\label{Charakter Liste}
	\textbf{ID} & \textbf{FA-G \arabic{table}} \\
	\hline
	TITEL: & Charakter Liste \\
	\hline
	BESCHREIBUNG: & Eine Liste, die alle spielbaren Charaktere mit ihren Fähigkeiten beinhaltet. Die Charaktere, die im Lastenheft unter \glqq A Einige Vorschläge für Charaktere\grqq aufgeführt sind, sollen mindestens in der Liste enthalten sein.\\
	\hline
	BEGRÜNDUNG: & Aus dieser Liste können Charaktere zum Spielen gewählt werden.\\
	\hline
	ABHÄNGIGKEITEN: & FA-G \ref{Charakter} \todo[inline]{2.8.1 Wahlphase}\\
\end{tabularx}

\begin{tabularx}{16cm}{l|X}
	\refstepcounter{table}\label{Bewegung durchfuehren}
	\textbf{ID} & \textbf{FA-G\arabic{table}} \\
	\hline
	TITEL: & Bewegung durchführen \\
	\hline
	BESCHREIBUNG: & Innerhalb eines Zuges kann ein Charakter, der noch mehr als einen Bewegungspunkt besitzt, eine Bewegung durchführen. Das bedeutet, dass der Charakter sich von dem Spielfeld seiner aktuellen Position auf ein angrenzendes betretbares Spielfeld bewegen kann.
	Jedes Spielfeld, welches mit einer Seite oder einer Ecke das Spielfeld der aktuellen Position berührt, ist ein angrenzendes Spielfeld. Dementsprechend sind Bewegungen in horizontaler, vertikaler und diagonaler Linie möglich.
	Wenn der Client die Anweisung gibt, den Charakter auf ein angrenzendes Feld zu bewegen, welches nicht betretbar ist, so darf der Charakter sich danach nicht auf diesem Feld befinden, sondern muss auf dem Spielfeld der aktuellen Position bleiben. In diesem Fall wurde die  Bewegung nicht erfolgreich durchgeführt. In der graphischen Darstellung kann dem Benutzer mit einer Animation angezeigt werden, dass die Bewegung nicht erfolgreich war bzw. dass es nicht möglich ist, das gewählte Spielfeld zu betreten.
 Bei erfolgreicher Durchführung einer Bewegung muss dem Charakter am Zug die Anzahl der Bewegungspunkte um 1 reduziert werden. Bei nicht erfolgreicher Durchführung darf sich die Anzahl der Bewegungspunkte  nicht verändern.\\
	\hline
	BEGRÜNDUNG: & Die Bedingungen für das Bewegen eines Charakters müssen eindeutig definiert sein.\\
	\hline
	ABHÄNGIGKEITEN: & FA-G\ref{Charakter} \todo[inline]{2.8.1 Wahlphase}\\
\end{tabularx}

\begin{tabularx}{16cm}{l|X}
	\refstepcounter{table}\label{Draengeln}
	\textbf{ID} & \textbf{FA-G\arabic{table}} \\
	\hline
	TITEL: & Drängeln \\
	\hline
	BESCHREIBUNG: & Wenn ein Charakter sich auf ein Feld bewegt, auf dem bereits ein anderer Charakter 		befindet, so tauschen die beiden Charaktere Plätze. Das heißt, der Charakter der die Bewegung durchgeführt hat steht auf dem Feld, auf das er sich bewegen wollte. Dies ist das Feld, auf dem der andere Charakter vor der Bewegung stand. Der andere Charakter befindet sich, nach dem Durchführen der Bewegung, auf dem Feld, auf dem der aktive Charakter ursprünglich stand. \\
	\hline
	BEGRÜNDUNG: & Um das Verhalten im Fall der Bewegung auf ein Spielfeld, welches mit einem anderen 			Charakter besetzt ist, eindeutig zu definieren.\\
	\hline
	ABHÄNGIGKEITEN: & FA-G\ref{Bewegung durchfuehren} \todo[inline]{2.8.1 Wahlphase}\\
\end{tabularx}

\begin{tabularx}{16cm}{l|X}
	\refstepcounter{table}\label{Aktion durchfuehren}
	\textbf{ID} & \textbf{FA-G\arabic{table}} \\
	\hline
	TITEL: & Aktion durchführen \\
	\hline
	BESCHREIBUNG: & Innerhalb eines Zuges kann ein Charakter, der noch mehr als einen Aktionspunkt hat und die Ausführungsbedingungen einer spezifischen Aktion erfüllt, diese spezifische Aktion ausführen.
	 Wenn der Client eine gültigen Aktionsbefehl eingibt, so gilt die Ausführung der Aktion als erfolgreich. Bei der Eingabe eines ungültigen Aktionsbefehls gilt die Ausführung der Aktions als nicht erfolgreich.
	 Bei erfolgreicher Ausführung der Aktion muss dem Charakter am Zug die Anzahl der Aktionspunkte um 1 reduziert werden und die Konsequenzen der spezifischen Aktion müssen den Zustand des Spiels entsprechend ändern. Bei nicht erfolgreicher Ausführung darf sich die Anzahl der Aktionspunkte und der Zustand des Spiels nicht ändern.  \\
	\hline
	BEGRÜNDUNG: & Es muss genau definiert sein, unter welchen Bedingungen das Ausführen einer Aktion möglich ist und welche Konsequenzen die Aktion hat.\\
	\hline
	ABHÄNGIGKEITEN: & FA-G\ref{Bewegung durchfuehren} \todo[inline]{2.8.1 Wahlphase}\\
\end{tabularx}


\begin{tabularx}{16cm}{l|X}
	\refstepcounter{table}\label{Gadget verwenden}
	\textbf{ID} & \textbf{FA-G\arabic{table}} \\
	\hline
	TITEL: & Gadget verwenden \\
	\hline
	BESCHREIBUNG: & Wenn sich ein verwendbares Gadget im Inventar eines Charakters befindet, dieser Charakter am Zug ist und die Ausführungsbedingungen des Gadgets für diesen Charakter erfüllt sind, so kann der Charakter als Aktion das Gadget verwenden. \\
	 
	\hline
	BEGRÜNDUNG: & Das Verwenden von Gadgets stellt eine Aktion dar.\\
	\hline
	ABHÄNGIGKEITEN: & FA-G\ref{Aktion durchfuehren}  \todo[inline]{2.8.1 Wahlphase}\\
\end{tabularx}

\begin{tabularx}{16cm}{l|X}
	\refstepcounter{table}\label{Roulette spielen}
	\textbf{ID} & \textbf{FA-G\arabic{table}} \\
	\hline
	TITEL: & Roulette spielen \\
	\hline
	BESCHREIBUNG: & Ein Charakter, der sich innerhalb eines Zuges auf einem Feld befindet, welches an ein Feld mit Roulette-Tisch angrenzt, der kann als Aktion einmal Roulette spielen. Der Client muss dazu einen Betrag an Spielchips eingeben. Dieser Betrag hat den Mindestwert 1 und einen Höchstwert. Der Höchstwert ist der kleinere der beiden folgenden Werte: die Anzahl der Spielchips, welche der Charakter im Moment besitzt; die Anzahl der Spielchips, die am Tisch verfügbar sind. Die eingegebene Betrag stellt den Einsatz des Charakters dar. Wenn der Charakter gewinnt, dann wird der Spielchips-Wert des Charakters um den Einsatz-Wert erhöht und der Spielchips-Wert des Roulette-Tisches um den Einsatz-Wert verringert. Wenn der Charakter verliert, dann wird der Spielchips-Wert des Charakters um den Einsatz-Wert reduziert und der Spielchipswert des Roulette-Tisches um den Einsatz-Wert erhöht. Ob ein Charakter gewinnt wird durch die Roulette-Gewinn-Wahrscheinlichkeit bestimmt.\\
	 
	\hline
	BEGRÜNDUNG: & Aufgrund der Spielmechanik ist Roulette spielen eine Aktion. Das Verhalten beim Ausführen dieser Aktion muss eindeutig definiert sein.\\
	\hline
	ABHÄNGIGKEITEN: & FA-G\ref{Aktion durchfuehren} \todo[inline]{2.8.1 Wahlphase}\\
\end{tabularx}

\begin{tabularx}{16cm}{l|X}
	\refstepcounter{table}\label{Cocktail aufnehmen}
	\textbf{ID} & \textbf{FA-G\arabic{table}} \\
	\hline
	TITEL: & Cocktail aufnehmen \\
	\hline
	BESCHREIBUNG: & Wenn ein Charakter am Zug ist, noch mehr als 0 Aktionpunkte hat, sich auf einem Feld befindet, welches an einen Bar-Tisch angrenzt, sich auf diesem Bartisch zu diesem Zeitpunkt ein Cocktail befindet und er selbst im Moment keinen Cocktail in der Hand hält, so kann der Charakter als Aktion diesen Cocktail aufnehmen. Danach befindet sich der Cocktail in der Hand des Charakters. Diese Information muss für alle Clients sichtbar sein.\\
	 
	\hline
	BEGRÜNDUNG: & Aufgrund der Spielmechanik ist das Aufnehmen eines Cocktails eine Aktion. Das Verhalten beim Ausführen dieser Aktion muss eindeutig definiert sein.\\
	\hline
	ABHÄNGIGKEITEN: & FA-G\ref{Aktion durchfuehren}  \todo[inline]{2.8.1 Wahlphase}\\
\end{tabularx}


\clearpage
\subsection{Nichtfunktionale Anforderungen}
Dieser Abschnitt spezifiziert die Qualitätsanforderungen (QA) an das Softwaresystem.


\clearpage
\todo{Vielleicht die Diagramme thematisch und nicht nach Typ ordnen}
\section{Anwendungsfälle des Systems}
\subsection{Verbindungsverwaltung zwischen Client und Server}
Der Client ist entweder ein KI-Client, der von Software gesteuert ist oder ein Benutzer-Client, der von einem Menschen gesteuert wird. Deshalb gibt es eine extend-Relation von den beiden Akteuren 'KI-Client' und 'Benutzer-Client' zum abstrakten Client. \\
Nachdem der Client die Verbindung aufgebaut hat, muss die Verbindung von Client und Server gehalten werden, deshalb ist der Anwendungsfall 'Verbindung halten' mit einer include-Relation an den Anwendungsfall 'Verbindung initiieren' gebunden. \\
Die Dateien des Spiels müssen im JSON-Format über die Verbindung übertragen werden, deshalb die include-Relation vom Anwendungsfall 'Verbindung halten' zum Anwendungsfall 'JSON-Codierung'. \\
Falls die Verbindung abbricht, muss die Verbindung wieder aufgebaut werden, damit die Verbindung gehalten ist. Deshalb gibt es eine include-Relation vom Anwendungsfall 'Verbindung halten' zum Anwendungsfall 'Verbindung wieder aufbauen'. \\
Wenn der Client eine Verbindung zum Server initiiert, muss der Client dem Server übermitteln, ob es sich um einen KI-Client oder um einen von Menschen gesteuerten Client handelt. Deshalb gibt es eine include-Relation vom Anwendungsfall 'Verbindung initiieren' zum Anwendungsfall 'Registrieren des Client-Typs'. \\
Falls der Client während dem Halten der Verbindung die Anwortfrist nicht einhält, so soll der Server die Verbindung zum Client trennen. Deshalb gibt es eine bedingte extend-Relation vom Anwendungsfall 'Verbindung trennen' zum Anwendungsfall 'Verbindung halten'. \\


Anforderungsabdeckungen:\\

Verbindung initiieren

FA-C \ref{c-session} %FA-C 33 Websocket Verbindung
FA-KI \ref{ki-session} %FA-KI 39 Websocket Verbindung

Verbindung halten

FA-S \ref{s-clientconnection} %FA-S 4 Client Verbindung
FA-S \ref{s-websockets} %FA-S 19 Websockets
FA-C \ref{c-session} %FA-C 33 Websocket Verbindung
FA-KI \ref{ki-session} %FA-KI 39 Websocket Verbindung

Verbindung wieder aufbauen

FA-S \ref{s-clientconnection} %FA-S 4 Client Verbindung
FA-C \ref{c-persistentsession} %FA-C 34 Persistente Session

Verbindung trennen

FA-S \ref{s-timeout} %FA-S 6 Client Timeout

JSON-Codierung

FA-S \ref{s-json-encoding} %FA-S 17 JSON Encoding
FA-S \ref{s-json-decoding} %FA-S 18 JSON Decoding
FA-C \ref{c-session} %FA-C 33 Websocket Verbindung
FA-KI \ref{ki-session} %FA-KI 39 Websocket Verbindung

Registrieren des Client-Typs

FA-C \ref{c-join-human} %FA-C 24 Registrieren als menschlicher Spieler
FA-KI \ref{ki-register} %FA-KI 40 Registrieren als KI

\begin{figure}
  \centering
  \includegraphics[width=\textwidth]{Meilenstein02/use_case_network.pdf}
  \caption{Anwendungsfälle Netzwerkverbindungsmanagement}
\end{figure}


\section{Abläufe im System}
\subsection{Verbindungsaufbau zum Server}
Beim Aufbau einer Verbindung mit dem Server, muss der Spieler seinen Anzeige-Namen angeben und ob er ein Mensch- oder ein KI-Spieler ist. \\
\begin{tikzpicture} 
    \begin{umlseqdiag} 
        \umlactor[class=Spieler]{P}
        \umlobject[no ddots]{Server}
        
        \begin{umlcall}[op={register}, return={Success}]{P}{Server}
            \begin{umlcall}[op={name(String)}]{P}{Server} 
            \end{umlcall}
            
            \begin{umlfragment}[type=alt, label=Mensch, inner xsep=7] 
                \begin{umlcall}[op={type("Mensch")}]{P}{Server} 
                \end{umlcall}

                \umlfpart[KI]

                \begin{umlcall}[op={type("KI")}]{P}{Server}
                \end{umlcall}
            \end{umlfragment}

        \end{umlcall}

    \end{umlseqdiag} 
\end{tikzpicture}


\clearpage
\subsection{Durchführen einer Aktion}
In jeder Runde legt der Server eine Zugreihenfolge für die Charaktere fest und berechnet deren Bewegungs- und Aktionspunkte. Hier ist gerade ein Charakter aus der Fraktion von P1 am Zug. Ein Zug kann entweder eine Bewegung oder eine Aktion sein. Die dargestellte Sequenz wiederholt sich solange bis der Charakter von P1 keine BP und AP mehr hat oder P1 den Zug beendet. Anschließend ist der nächste Charakter an der Reihe.\\

\begin{tikzpicture}
    \begin{umlseqdiag}
        \umlactor[class=Client]{P1}
        \umlobject[no ddots]{Server}
        \umlactor[class=Client]{P2}

        \begin{umlcall}[op=zug]{P1}{Server}

            \begin{umlcallself}[op=validate]{Server}
            \end{umlcallself}

            \begin{umlfragment}[type=alt, label=fail, inner xsep=8]

                \begin{umlcall}[type=return, op=fail]{Server}{P1}
                \end{umlcall}

                \umlfpart[success]

                \begin{umlcall}[op=zug(result)]{Server}{P2}
                \end{umlcall}

                \begin{umlcall}[type=return, op=zug(result)]{Server}{P1}
                \end{umlcall}

            \end{umlfragment}

        \end{umlcall}

    \end{umlseqdiag}
\end{tikzpicture}


\subsection{Zustände im Spiel}
In diesem Zustandsdiagramm werden abstrakt die Zustände beschrieben, welche während eines Spiels auftreten können.
\tikzset{singlestate/.style={draw,fill=yellow!20, rounded corners, align=center}}

\begin{tikzpicture}
	\umlstateinitial[name=initial, x=0, y=0]
	\umlstatefinal[name=final,x=8,y=-14]
	\begin{umlstate}[x=4, y=-10, name=turnp1]{Spieler 1 ist am Zug}
	\end{umlstate}

	\begin{umlstate}[x=12,y=-10,name=turnp2]{Spieler 2 ist am Zug}
	\end{umlstate}

	\begin{umlstate}[x=8,y=-6,name=apbp]{Bestimmen der AP/BP}
		\umlstatetext{Do/Bestimmen der AP und BP \\Abhängig von den \\gewählten Charaktere}
	\end{umlstate}

	\node[singlestate] at (3.5,-16)(charAction){Charakter gewählt};
	\node[singlestate] at (12.5,-16)(charAction2){Charakter gewählt};

	\umlHVtrans[anchor2=90]{initial}{apbp}

	\umlHVtrans[arg=Zug beendet,align=right,pos=1.5,anchor1=0,anchor2=260]{turnp1}{apbp}
	\umlHVtrans[arg=Zug beendet,align=left,pos=1.5,anchor1=180,anchor2=280]{turnp2}{apbp}

	\umlHVtrans[arg={Nächster Spieler},anchor1=175,anchor2=130,pos=0.05,align=right]{apbp}{turnp1}
	\umlHVtrans[arg={Nächster Spieler},anchor2=65,pos=0.8]{apbp}{turnp2}

	\umlHVtrans[arg={Siegbedingung},pos=1.5,align=right,anchor1=-20,anchor2=135]{turnp1}{final}
	\umlHVtrans[arg={Siegbedingung},pos=1.5,anchor1=200,anchor2=45,align=left]{turnp2}{final}
	
	\umltrans[arg={Überlanges Spiel  }, recursive=170|190|1.5cm, pos=2.1,align=right, recursive direction=left to left]{turnp1}{turnp1}
	\umltrans[arg=Überlanges Spiel, recursive=10|-10|1.5cm, pos=1.5, recursive direction=left to left]{turnp2}{turnp2}


	\umlHVtrans[arg=Charakter ausgewählt,anchor1=248,anchor2=100,align=right,pos=1.5]{turnp1}{charAction}
	\umlHVtrans[arg=Aktion/Bewegung,pos=1.5,align=left]{charAction}{turnp1}

	\umlHVtrans[arg=Charakter ausgewählt,anchor1=298,anchor2=80 ,align=left,pos=1.5]{turnp2}{charAction2}
	\umlHVtrans[arg=Aktion/Bewegung,align=right,pos=1.5]{charAction2}{turnp2}

\end{tikzpicture}

\clearpage

\clearpage
\subsection{Zustände des Clients}
%Found at: %https://tex.stackexchange.com/questions/420217/change-style-of-basic-states-in-tikz-uml-package
\tikzset{singlestate/.style={draw,fill=yellow!20, rounded corners, align=center}}

\begin{tikzpicture}
	\umlstateinitial[name=initial, x=0, y=0]
	\begin{umlstate}[x=7, y=-2, name=menu]{Hauptmenü}
		\umlstatetext{}
		\umlbasicstate[x=0, y=0, name=menuDisconnected, do={bekannte Server\\ anpingen}, align=center]{Hauptmenü (disconnected)}
		\umlbasicstate[x=0, y=-4, name=menuConnected, do={Lobbies abfragen}]{Hauptmenü (connected)}
	\end{umlstate}
	
	\node[singlestate] at (0, -2.7) (help){Hilfe};
	\node[singlestate] at(16, -4.85) (settings){Einstellungen};
	\umlstatefinal[x=16, y=-0, name=final]
	\node[singlestate] at (7, -8) (lobbyView){Lobbyübersicht};
	
	\begin{umlstate}[x=7, y=-13, name=lobby]{Lobby}
		\umlstatetext{}
		\umlbasicstate[x=0, y=0, name=spieler, do={}, align=center]{in Lobby (Rolle: Spieler)}
		\umlbasicstate[x=0, y=-4, name=zuschauer, do={}]{in Lobby (Rolle: Zuschauer)}
	\end{umlstate}
	
	\node[singlestate] at (7, -20) (inGame){im Spiel};
	\node[singlestate] at (7, -22) (gameEnded){Spiel beendet};
	
	\umlVHtrans{initial}{menuDisconnected}
	
	\umltrans[align=center, arg={verbinde/}, pos=0.5, align=right, anchor1=-120, anchor2=125]{menuDisconnected}{menuConnected}
	\umltrans[arg={trennen/}, pos=0.5, align=left, anchor1=55, anchor2=-60]{menuConnected}{menuDisconnected}
	
	\umlHVtrans[arg={Hilfe aufrufen/}, pos=0.5, align=right, anchor1=160]{menu}{help}
	\umlVHtrans[arg={Hilfe schließen/}, pos=0.5, anchor2=200]{help}{menu}
	
	\umlHVtrans[arg={Einstellungen aufrufen/}, pos=0.8, align=right, anchor1=-30]{menu}{settings}
	\umlVHtrans[arg={Einstellungen schließen/}, pos=1.2, anchor1=-90, anchor2=-48, align=right]{settings}{menu}
	
	\umlHVtrans[arg={Client schließen/}, pos=0.7, align=right, anchor1=30]{menu}{final}
	
	\umltrans[arg={Lobbyübersicht aufrufen/}, pos=0.7, align=right, anchor1=-100, anchor2=158]{menu}{lobbyView}
	\umltrans[arg={Zurück zum Menü/}, pos=0.3, align=left, anchor1=23, anchor2=-81]{lobbyView}{menu}
	
	\umltrans[arg={verlasse Lobby/}, pos=0.4, align=left, anchor1=90, anchor2=-90]{lobby}{lobbyView}
	
	\umlHVHtrans[arg={betrete Lobby[Spielerslots belegt]/}, pos=1.5, align=left, anchor1=0, anchor2=0, arm1=2cm]{lobbyView}{zuschauer}
	\umlHVHtrans[arg={betrete Lobby[Spielerslot frei]/}, pos=1.5, align=right, anchor1=180, anchor2=180, arm1=-2cm]{lobbyView}{spieler}
	
	\umltrans[arg={Rollenwechsel/}, pos=0.5, align=right, anchor1=-110, anchor2=110]{spieler}{zuschauer}
	\umltrans[arg={Rollenwechsel/}, pos=0.5, align=left, anchor1=70, anchor2=-70]{zuschauer}{spieler}
	
	\umlHVHtrans[arg={Zurück zum Menü/}, pos=0.2, align=left, arm1=8.5cm, anchor2=-15]{gameEnded}{menuConnected}
	
	\umltrans[arg={Spielende erreicht/}, pos=0.6, align=left]{inGame}{gameEnded}
	
	\umltrans[arg={Spiel starten[alle Spieler bereit]/}, pos=0.6, align=left]{lobby}{inGame}
%
\end{tikzpicture}

\subsection{Zustände des KI-Clients}
Das untenstehende Zustandsdiagramm gibt einen kleinen Überblick über die grobe Funktionsweise des KI-Clients. 

\begin{tikzpicture} 
    \umlstateinitial[y=4, name=initial]
    \begin{umlstate}[name=running]{running}
        \begin{umlstate}[name=waiting, y=0]{Warte}
        \end{umlstate}

        \begin{umlstate}[name=analyzing, y=-4]{Analysiere Spielzustand}
        \end{umlstate}

        \begin{umlstate}[name=planning, y=-8]{Plane nächste Aktion}
        \end{umlstate}

        \umlHVHtrans[arm1=5, arg={Aktion Senden}, pos=0.5]{planning}{waiting}
        \umltrans[arg={Aktion Empfangen}, pos=0.5]{waiting}{analyzing}
        \umltrans{analyzing}{planning}

    \end{umlstate}

    \umltrans[arg={verbinde/}, pos=0.3]{initial}{waiting}
    \umlstatefinal[name=final, x=-6, y=-1]
    \umlHVtrans[arg={Spielende erreicht/}, pos=0.6]{waiting}{final}
\end{tikzpicture}



\end{document}
