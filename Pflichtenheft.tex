\documentclass{uulm-assignment}

\usepackage{import}
\usepackage{tabularx}
\usepackage{listings}
\usepackage{todonotes}
\usepackage{graphicx}
\usepackage{siunitx}
\usepackage{placeins}
\usepackage{tikz}
\usepackage{tikz-uml}
\setboolean{showsolutions}{false}

\ifthenelse{\boolean{showsolutions}}{
\newcommand\mitloesung{1}%
}{
\newcommand\mitloesung{0}%
}


% Für Korrektur-Kommentare in roten Boxen:
\newcommand{\flo}[1]{
    \fcolorbox{purple}{pink}{\sffamily\scriptsize\bfseries\textcolor{black}{Flo:}} {\sffamily\bfseries\textcolor{purple}{#1}}
}

% Entnommen und angepasst von https://github.com/SoPra-Team-10
\newcommand{\Fachwissen}[7] {
		\begin{tabular}{|p{2,25cm}|p{12cm}|}
			\hline
			\textbf{Begriff} & \textbf{#1} \\
			\hline
			\textbf{Beschreibung} & #2 \\ 
			\hline
			\textbf{Ist-ein} & #3 \\
			\hline
			\textbf{Kann-sein} & #4 \\ 
			\hline
			\textbf{Aspekt} & #5 \\ 
			\hline
			\textbf{Bemerkung} & #6 \\ 
			\hline
			\textbf{Beispiel} & #7 \\
			\hline
		\end{tabular}
}

\newcommand{\FachwissenSyn}[8] {
		\begin{tabular}{|p{2,25cm}|p{12cm}|}
			\hline
			\textbf{Begriff} & \textbf{#1} \\
			\hline
			\textbf{Beschreibung} & #2 \\ 
			\hline
			\textbf{Ist-ein} & #3 \\
			\hline
			\textbf{Kann-sein} & #4 \\ 
			\hline
			\textbf{Aspekt} & #5 \\ 
			\hline
			\textbf{Bemerkung} & #6 \\ 
			\hline
			\textbf{Beispiel} & #7 \\
			\hline
			\textbf{Synonym} & #8 \\
			\hline
		\end{tabular}
}



\hypersetup{colorlinks=false,urlcolor=uulm-in}

\faculty{Institut für Softwaretechnik und Programmiersprachen\hspace{0.05cm}}
\course{Softwaregrundprojekt}
\semester{\hspace{0.05cm}WiSe 2019/20}
\supervisor{\textbf{} \hspace{7.9cm} Prof. Dr. Matthias Tichy, Florian Ege, Dennis Jehle}
 
%\assignmentdeadline{}         % Abgabedatum: XYZ
%\assignmentduration{15 Minuten} % Bearbeitungsdauer: XYZ
% \studentdata                      % Name & Matrikelnummer Feld

\assignmenttype{}
\assignmentno{}
\title{Pflichtenheft Team 17}

\begin{document}
\setcounter{secnumdepth}{4}

\maketitle

\tableofcontents

%\addtocontents{toc}{\protect\newpage}

\clearpage
\section{Überblick}
Der Zweck dieses Dokuments ist es, eine detaillierte Beschreibung der Anforderungen, sowie der Benutzerschnittstelle
für die Anwendung No time to spy bereitzustellen. Es wird abgegrenzt, welche Anforderungen erfüllt werden müssen,
damit die entwickelte Anwendung vom Kunden akzeptiert wird.


\subsection{Anwendungsbereiche}
Das Spiel No time to spy ist eine C++-Anwendung, die es dem Benutzer ermöglicht, ein 2D-Spiel zu spielen.


\clearpage
\section{Anforderungsanalyse}
\subsection{Fachwissen}
Diese Tabelle enthält Abkürzungen und domänenspezifische Begriffe, die im Dokument verwendet werden


\clearpage
\subsection{Anwendungskontext}
\subsubsection{Akteure}
%Hier kommen die Akteure rein

\begin{tabularx}{16cm}{l|X}
	\textbf{Akteur} & \textbf{Server} \\
	\hline
	Rolle & Zentrale Software-Einheit zur Kommunikation mit den Clients.\\ 
	\hline
	Aufgabe & Der Server ist Vermittler in der Kommunikation zwischen den Clients. Er geht auf Anfrage Verbindungen mit Clients ein und kann diese auch wieder beenden. Während ein Client mit dem Server verbunden ist, bearbeitet und validiert der Server alle Anfragen des Clients und sorgt dafür, dass der Client alle notwendigen Informationen zum aktuellen Stand erhält.\\ 
\end{tabularx}

\begin{tabularx}{16cm}{l|X}
	\textbf{Akteur} & \textbf{Client} \\
	\hline
	Rolle & Lokale Software-Einheit zur Kommunikation mit dem Server \\ 
	\hline
	Aufgabe & Da Clients nicht direkt miteinander kommunizieren können, kommuniziert der Client mit einem Server. Der Client kann eine Verbindung zum Server auf- und auch wieder abbauen. Ein verbundener Client kann Informationen weitergeben, anfragen und erhalten.\\ 
\end{tabularx}

\begin{tabularx}{16cm}{l|X}
	\textbf{Akteur} & \textbf{Benutzer-Client} \\
	\hline
	Rolle & Client, der von einem Menschen bedient wird.\\ 
	\hline
	Aufgabe & Der Benutzer-Client stellt eine graphische Oberfläche zur Bedienung zur Verfügung. Er ermöglicht die Bedienung des Editors und die Teilnahme an einer Spielpartie.\\ 
\end{tabularx}

\begin{tabularx}{16cm}{l|X}
	\textbf{Akteur} & \textbf{KI-Client} \\
	\hline
	Rolle & Client, der von einem Computer gesteuert wird.\\ 
	\hline
	Aufgabe & Der KI-Client kann von einem Benutzer-Client als Spieler zur Spielpartie hinzugefügt werden und benutzt keine graphische Oberfläche.\\ 
\end{tabularx}

\begin{tabularx}{16cm}{l|X}
	\textbf{Akteur} & \textbf{Teilnehmer} \\
	\hline
	Rolle & Alle Akteure, die an einer Spielpartie teilnehmen.\\ 
	\hline
	Aufgabe & Teilnehmer erhalten Informationen zur Spielpartie vom Server und können die Spielpartie jederzeit wieder verlassen.\\ 
\end{tabularx}

\begin{tabularx}{16cm}{l|X}
	\textbf{Akteur} & \textbf{Zuschauer} \\
	\hline
	Rolle & Benutzer-Client, der passiv an einer Spielpartie teilnimmt.\\ 
	\hline
	Aufgabe & Über die graphische Oberfläche des Benutzer-Clients, kann der Benutzer das laufende Spiel beobachten, aber nicht eingreifen. \\ 
\end{tabularx}

\begin{tabularx}{16cm}{l|X}
	\textbf{Akteur} & \textbf{Spieler} \\
	\hline
	Rolle & Client, der aktiv an einer Spielpartei teilnimmt.\\ 
	\hline
	Aufgabe & Wenn ein Spieler an der Reihe ist, führt er durch Steuerung seiner Charaktere valide Spielzüge aus und informiert den Server über seine Aktionen. Ziel des Spielers ist es, die Spielpartie zu gewinnen.\\ 
\end{tabularx}

\begin{tabularx}{16cm}{l|X}
	\textbf{Akteur} & \textbf{Benutzer-Spieler} \\
	\hline
	Rolle & Spieler, der ein Benutzer-Client ist.\\ 
	\hline
	Aufgabe & Ein Benutzer-Spieler spielt die Spielpartie über die graphische Oberfläche.\\ 
\end{tabularx}

\begin{tabularx}{16cm}{l|X}
	\textbf{Akteur} & \textbf{KI-Spieler} \\
	\hline
	Rolle & Spieler, der ein KI-Client ist.\\ 
	\hline
	Aufgabe & Ein KI-Spieler spielt die Spielpartie unter Verwendung von künstlicher Intelligenz.\\ 
\end{tabularx}

\begin{tabularx}{16cm}{l|X}
	\textbf{Akteur} & \textbf{Agent} \\
	\hline
	Rolle & Charakter in einer Spielpartie, der von einem Spieler gesteuert wird.\\ 
	\hline
	Aufgabe & Agenten sind Charaktere, die während der Spielpartie von Spielern gesteuert werden.\\ 
\end{tabularx}

\begin{tabularx}{16cm}{l|X}
	\textbf{Akteur} & \textbf{NPC} \\
	\hline
	Rolle & Charakter in einer Spielpartie, der von keinem Spieler gesteuert werden kann.\\ 
	\hline
	Aufgabe & NPCs sind Charaktere, die vom Server gesteuert werden. Der Server informiert die Teilnehmer über die Spielzüge der NPCs.\\ 
\end{tabularx}

\begin{tabularx}{16cm}{l|X}
	\textbf{Akteur} & \textbf{Auftraggeber} \\
	\hline
	Rolle & Florian Ege, stellvertretend für das Institut für Softwaretechnik und Programmiersprachen\\ 
	\hline
	Aufgabe & Gibt die Anforderungen für \glqq No Time To Spy\grqq vor und nimmt das fertige Produkt ab.\\ 
\end{tabularx}

\begin{tabularx}{16cm}{l|X}
	\textbf{Akteur} & \textbf{Vertreter des Auftraggebers} \\
	\hline
	Rolle & Tutor Ismail Temel\\ 
	\hline
	Aufgabe & Dient als Vermittler zwischen Entwicklern und Auftraggeber.\\ 
\end{tabularx}

\begin{tabularx}{16cm}{l|X}
\textbf{Akteur} & \textbf{Entwickler} \\
\hline
Rolle & SoPra-Team-17: Dominik Authaler, Lukas Bleile, Marco Deuscher, Jonas Otto, Carolin Schindler, Dominik Tabib Khoie\\ 
\hline
Aufgabe & Agile Entwicklung von \glqq No Time To Spy\grqq nach den Anforderungen des Auftraggebers.\\ 
\end{tabularx}


\clearpage
\subsubsection{Anwendungsfälle des Systems}
\paragraph{Verbindungsverwaltung zwischen Client und Server} \mbox{}\\
Der Client ist entweder ein KI-Client, der von Software gesteuert ist oder ein Benutzer-Client, der von einem Menschen gesteuert wird. Deshalb gibt es eine extend-Relation von den beiden Akteuren 'KI-Client' und 'Benutzer-Client' zum abstrakten Client. \\
Nachdem der Client die Verbindung aufgebaut hat, muss die Verbindung von Client und Server gehalten werden, deshalb ist der Anwendungsfall 'Verbindung halten' mit einer include-Relation an den Anwendungsfall 'Verbindung initiieren' gebunden. \\
Die Dateien des Spiels müssen im JSON-Format über die Verbindung übertragen werden, deshalb die include-Relation vom Anwendungsfall 'Verbindung halten' zum Anwendungsfall 'JSON-Codierung'. \\
Falls die Verbindung abbricht, muss die Verbindung wieder aufgebaut werden, damit die Verbindung gehalten ist. Deshalb gibt es eine include-Relation vom Anwendungsfall 'Verbindung halten' zum Anwendungsfall 'Verbindung wieder aufbauen'. \\
Wenn der Client eine Verbindung zum Server initiiert, muss der Client dem Server übermitteln, ob es sich um einen KI-Client oder um einen von Menschen gesteuerten Client handelt. Deshalb gibt es eine include-Relation vom Anwendungsfall 'Verbindung initiieren' zum Anwendungsfall 'Registrieren des Client-Typs'. \\
Falls der Client während dem Halten der Verbindung die Anwortfrist nicht einhält, so soll der Server die Verbindung zum Client trennen. Deshalb gibt es eine bedingte extend-Relation vom Anwendungsfall 'Verbindung trennen' zum Anwendungsfall 'Verbindung halten'. \\


Anforderungsabdeckungen:\\

Verbindung initiieren

FA-C \ref{c-session} %FA-C 33 Websocket Verbindung
FA-KI \ref{ki-session} %FA-KI 39 Websocket Verbindung

Verbindung halten

FA-S \ref{s-clientconnection} %FA-S 4 Client Verbindung
FA-S \ref{s-websockets} %FA-S 19 Websockets
FA-C \ref{c-session} %FA-C 33 Websocket Verbindung
FA-KI \ref{ki-session} %FA-KI 39 Websocket Verbindung

Verbindung wieder aufbauen

FA-S \ref{s-clientconnection} %FA-S 4 Client Verbindung
FA-C \ref{c-persistentsession} %FA-C 34 Persistente Session

Verbindung trennen

FA-S \ref{s-timeout} %FA-S 6 Client Timeout

JSON-Codierung

FA-S \ref{s-json-encoding} %FA-S 17 JSON Encoding
FA-S \ref{s-json-decoding} %FA-S 18 JSON Decoding
FA-C \ref{c-session} %FA-C 33 Websocket Verbindung
FA-KI \ref{ki-session} %FA-KI 39 Websocket Verbindung

Registrieren des Client-Typs

FA-C \ref{c-join-human} %FA-C 24 Registrieren als menschlicher Spieler
FA-KI \ref{ki-register} %FA-KI 40 Registrieren als KI

\begin{figure}
  \centering
  \includegraphics[width=\textwidth]{Meilenstein02/use_case_network.pdf}
  \caption{Anwendungsfälle Netzwerkverbindungsmanagement}
\end{figure}
\paragraph{Verwaltung der Spielpartien bei Client und Server} \mbox{}\\
Hauptmenü anzeigen

FA-C \ref{c-menu} %FA-C 21 Hauptmenü

Anwendung beenden

FA-C \ref{c-menu} %FA-C 21 Hauptmenü

Zur Lobby-Übersicht wechseln

FA-C \ref{c-menu} %FA-C 21 Hauptmenü
FA-C \ref{c-lobby-overview} %FA-C 22 Lobby-Übersicht

Lobby-Übersicht anzeigen

FA-C \ref{c-lobby-overview} %FA-C 22 Lobby-Übersicht

Lobby-Übersicht verlassen

FA-C \ref{c-menu} %FA-C 21 Hauptmenü
FA-C \ref{c-lobby-overview} %FA-C 22 Lobby-Übersicht

Nutzernamen festlegen

FA-C \ref{c-username} %FA-C 26 Nutzername

Lobby erstellen

Konfigurationsdateien festlegen

FA-S \ref{c-partieconfig} %FA-S 1 Partie Konfiguration
FA-S \ref{c-szenarioconfig} %FA-S 2 Szenario Konfiguration
FA-S \ref{c-charakterconfig} %FA-S 3 Charakter Konfiguration

Lobby beitreten

FA-C \ref{c-join} %FA-C 23 Beitreten als Mitspieler
FA-C \ref{c-join-spectator} %FA-C 24 Beitreten als Zuschauer
FA-S \ref{c-zuschauer} %FA-S 5 Zuschauer

Lobby anzeigen

Lobby verlassen

FA-C \ref{c-lobby-overview} %FA-C 22 Lobby-Übersicht

Rolle wechseln

FA-C \ref{c-join} %FA-C 23 Beitreten als Mitspieler
FA-C \ref{c-join-spectator} %FA-C 24 Beitreten als Zuschauer
FA-S \ref{c-zuschauer} %FA-S 5 Zuschauer

Spiel starten
FA-S \ref{c-partieinit} %FA-S 8 Spiel Start und Initialisierung


\begin{figure}
  \centering
  \includegraphics[width=\textwidth]{Meilenstein02/use_case_lobbymanagement.pdf}
  \caption{Anwendungsfälle Lobbymanagement}
\end{figure}
\paragraph{Anwendungsfälle innerhalb einer Spielpartie} \mbox{}\\
Partie-Vorbereitung

FA-G \ref{Partie-Vorbereitung} %FA-G 126 Partie-Vorbereitung

Wahlphase

FA-G \ref{Wahlphase} %FA-G 127 Wahlphase

Ausrüstungsphase

FA-G \ref{Ausruestungsphase} %FA-G 128 Ausrüstungsphase

Startplatzverteilung der Charaktere

FA-S \ref{s-partieinit} %FA-S 8 Spiel Start und Initialisierung

Punkt verwenden

Spielzug durchführen

Bewegung durchführen

FA-G \ref{Bewegung druchfuehren} %FA-G 115 Bewegung durchführen

Drängeln

FA-G \ref{Draengeln} %FA-G 116 Drängeln

Aktion durchführen

FA-G \ref{Aktion durchfuehren} %FA-G 117 Aktion durchführen

Roulette spielen

FA-G \ref{Roulette spielen} %FA-G 119 Roulette spielen

Gadget verwenden

FA-G \ref{Gadget verwenden} %FA-G 118 Gadget verwenden
FA-G \ref{Gadgets} %FA-G 94 Gadgets

Tresor spicken

FA-G \ref{Tresor-Spicken} %FA-G 124 Tresor-Spicken

Spionieren

FA-G \ref{Spionieren} %FA-G 123 Spionieren

Charaktereigenschaft anwenden

FA-G \ref{Eigenschaften} %FA-G 73 Eigenschaften
FA-G \ref{Bang and Burn} %FA-G 90 Bang and Burn
FA-G \ref{Observation} %FA-G 93 Observation

Cocktail-Aktion durchführen

Cocktail aufnehmen

FA-G \ref{Cocktail aufnehmen} %FA-G 120 Cocktail aufnehmen
FA-G \ref{haelt Cocktail} %FA-G 75 hält Cocktail

Jemanden mit einem Cocktail übergießen

FA-G \ref{Cocktail-Guss} %FA-G 121 Cocktail-Guss

Cocktail schlürfen

FA-G \ref{Cocktail schluerfen} %FA-G 122 Cocktail schlürfen



%\begin{figure}
%  \centering
%  \includegraphics[width=\textwidth]{Meilenstein02/use_case_lobbymanagement.pdf}
%  \caption{Anwendungsfälle %Lobbymanagement}
%\end{figure}
\paragraph{Editor} \mbox{}\\
Die drei Anwendungsfälle 'Charakterkonfiguration erstellen und editieren', Szenario-Konfiguration erstellen und editieren' und 'Partie-Konfiguration erstellen und editieren' beinhalten den Anwendungsfall 'GUI anzeigen', da es notwendig ist dem Benutzer die zu editierenden Daten graphisch anzuzeigen. Die drei erstellten Konfigurationsdateien müssen dann in das JSON-Format codiert werden. Dies wird durch den Anwendungsfall 'JSON-Datei bearbeiten' erweitert.

JSON-Datei speichern

FA-E \ref{e-json-encodeing} %FA-E 47 JSON Encoding

JSON-Datei bearbeiten

FA-E \ref{e-json-decodeing} %FA-E 48 JSON Decoding

GUI anzeigen

FA-E \ref{e-gui} %FA-E 49 GUI

Szenarion Konfiguration erstellen und editieren

FA-E \ref{e-szenarioedit} %FA-E 50 Szenario Editor

Partie Konfiguration erstellen und editieren

FA-E \ref{e-partieedit} %FA-E 51 Partie Editor

Charakter Konfiguration erstellen und editieren

FA-E \ref{e-charedit} %FA-E 52 Charakter Editor

\begin{figure}
  \centering
  \includegraphics[width=\textwidth]{Meilenstein02/use_case_editor.pdf}
  \caption{Anwendungsfälle Editor}
\end{figure}

\clearpage
\subsubsection{Abläufe im System}
\paragraph{Verbindungsaufbau zum Server}
\label{connect} \mbox{}\\
Beim Aufbau einer Verbindung mit dem Server, muss der Spieler seinen Anzeige-Namen angeben und ob er ein Mensch- oder ein KI-Spieler ist. \\
\begin{tikzpicture} 
    \begin{umlseqdiag} 
        \umlactor[class=Spieler]{P}
        \umlobject[no ddots]{Server}
        
        \begin{umlcall}[op={register}, return={Success}]{P}{Server}
            \begin{umlcall}[op={name(String)}]{P}{Server} 
            \end{umlcall}
            
            \begin{umlfragment}[type=alt, label=Mensch, inner xsep=7] 
                \begin{umlcall}[op={type("Mensch")}]{P}{Server} 
                \end{umlcall}

                \umlfpart[KI]

                \begin{umlcall}[op={type("KI")}]{P}{Server}
                \end{umlcall}
            \end{umlfragment}

        \end{umlcall}

    \end{umlseqdiag} 
\end{tikzpicture}

\paragraph{Beitreten einer Lobby und Spiel starten} \mbox{}\\
\label{Sequ_Lobby}
\begin{tikzpicture}
\begin{umlseqdiag} 
	\umlactor[class=Mensch-Spieler]{P1}
	\umlobject[no ddots]{Server}
	\umlactor[class=Spieler]{P2}

	
	\begin{umlcall}[op=newLobby(String), dt=5]{P1}{Server}
		\begin{umlcall}[type=return, op=lobbyCreated]{Server}{P1}
		\end{umlcall}
		
		\begin{umlcall}[op=joinLobby(String)]{P1}{Server}
			\begin{umlcall}[type=return, op=lobbyInfo]{Server}{P1}
			\end{umlcall}
		\end{umlcall}
	
	
		\begin{umlfragment}[type=alt, label=KI, inner xsep=7]
			\begin{umlcall}[op=addKiToLobby, dt = 10]{P1}{Server}
			\end{umlcall}
			\begin{umlcall}[op=lobbyInfo]{Server}{P2}
			\end{umlcall}
			
			\umlfpart[Mensch]
		
			\begin{umlcall}[op=joinLobby(String)]{P2}{Server}
				\begin{umlcall}[type=return, op=lobbyInfo]{Server}{P2}
				\end{umlcall}
			\end{umlcall}
		\end{umlfragment}
		
	\end{umlcall}
    
\end{umlseqdiag}
\end{tikzpicture}
\paragraph{Ablauf des Spiels} \mbox{}\\
\subparagraph{Wahl- und Ausrüstungsphase} \mbox{}\\
\subsubsection{Wahl- und Ausrüstungsphase}
In diesem Diagramm wird die Wahl- und Ausrüstungsphase durch ein Sequenzdiagramm beschrieben.

\begin{tikzpicture}
	\begin{umlseqdiag}
		\umlactor[class=Spieler]{P1}
		\umlactor[no ddots]{Server}
		\umlactor[class=Spieler]{P2}

		\begin{umlfragment}[type=Wahlphase,inner xsep=7]
			\begin{umlcall}[op={3 Char./ 3 Gadgets},dt=10,return={Gewähltes Gadget/Char.},padding=5]{Server}{P1}
			\end{umlcall}
			\begin{umlcall}[op={3 Char./ 3 Gadgets},dt=-6,return={Gewähltes Gadget/Char.},padding=5]{Server}{P2}
			\end{umlcall}
		\end{umlfragment}

		

		\begin{umlfragment}[type=Ausrüstungsphase,inner xsep=7]
			\begin{umlcall}[op={Gadget auf Char.},dt=10,return={Bestätigung},padding=5]{P1}{Server}
			\end{umlcall}
			\begin{umlcall}[op={Gadget auf Char.},dt=10,return={Bestätigung},padding=5]{P2}{Server}
			\end{umlcall}
		\end{umlfragment}
	\end{umlseqdiag}
\end{tikzpicture}

In der Wahlphase bietet der Server jedem Spieler drei Charaktere und drei Gadgets zur Auswahl an. 
Es wird dabei durch den Server sichergestellt, dass die Spieler nicht das gleiche Gadget oder den gleichen Charakter wählen können.\\
Eine vom Spieler zusammengestellte Fraktion besteht aus zwei bis vier Charakteren.
Insgesamt wird die im Diagramm beschriebene Wahlphase acht mal wiederholt.\\

In der Ausrüstungsphase muss jeder Spieler seine gewählten Gadgets einem seiner Agenten zuordnen. Die im Diagramm beschriebene Ausrüstungsphase wird solange wiederholt, bis alle Gadgets einem Charakter zugeordnet sind.

\subparagraph{Durchführen eines Zugs} \mbox{}\\
In jeder Runde legt der Server eine Zugreihenfolge für die Charaktere fest und berechnet deren Bewegungs- und Aktionspunkte. Hier ist gerade ein Charakter aus der Fraktion von P1 am Zug. Ein Zug kann entweder eine Bewegung oder eine Aktion sein. Die dargestellte Sequenz wiederholt sich solange bis der Charakter von P1 keine BP und AP mehr hat oder P1 den Zug beendet. Anschließend ist der nächste Charakter an der Reihe.\\

\begin{tikzpicture}
    \begin{umlseqdiag}
        \umlactor[class=Client]{P1}
        \umlobject[no ddots]{Server}
        \umlactor[class=Client]{P2}

        \begin{umlcall}[op=zug]{P1}{Server}

            \begin{umlcallself}[op=validate]{Server}
            \end{umlcallself}

            \begin{umlfragment}[type=alt, label=fail, inner xsep=8]

                \begin{umlcall}[type=return, op=fail]{Server}{P1}
                \end{umlcall}

                \umlfpart[success]

                \begin{umlcall}[op=zug(result)]{Server}{P2}
                \end{umlcall}

                \begin{umlcall}[type=return, op=zug(result)]{Server}{P1}
                \end{umlcall}

            \end{umlfragment}

        \end{umlcall}

    \end{umlseqdiag}
\end{tikzpicture}

\subparagraph{Aktualisieren des Spielfelds} \mbox{}\\
\begin{tikzpicture}
    \begin{umlseqdiag}
        \umlactor[class=Zuschauer]{W}
        \umlobject[no ddots]{Server}
        \umlactor[class=Spieler]{P}

        \begin{umlcall}[op=action(result)]{Server}{P}
	        \begin{umlcall}[op=action(result)]{Server}{W}
	        \end{umlcall}

            \begin{umlcallself}[op=updateView]{W}
            \end{umlcallself}
            
			\begin{umlfragment}[type=alt, label=Mensch, inner xsep=5]

				\begin{umlfragment}[type=alt, label=wasTurn, inner xsep=9]
					\begin{umlcallself}[op=validateView, dt=15]{P}
					\end{umlcallself}
					
					\umlfpart[wasNotTurn]
					
					\begin{umlcallself}[op=updateView]{P}
					\end{umlcallself}
				\end{umlfragment}
			
				\umlfpart[KI]
				
				\begin{umlcallself}[op=updateModel]{P}
				\end{umlcallself}
			\end{umlfragment}
            

        \end{umlcall}

    \end{umlseqdiag}
\end{tikzpicture}

\paragraph{Spielende und Abbau der Verbindung zum Server} \mbox{}\\
\label{Sequ_Winner}
\begin{tikzpicture} 
    \begin{umlseqdiag} 
        \umlactor[class=Spieler]{P1}
        \umlobject[no ddots]{Server}
        \umlactor[class=Spieler]{P2}
        
        \begin{umlcall}[op={action(result)}]{Server}{P2}
        \begin{umlcall}[type=return, op={action(result)}]{Server}{P1}
            
            \begin{umlfragment}[type=alt, label=Winner, inner xsep=8]
             
             	\begin{umlfragment}[type=alt, label=Mensch, inner xsep=4]
	             	\begin{umlcallself}[op=displayWinner]{P1}
	             	\end{umlcallself}
	             	
	             	\umlfpart[KI]
	             	
	             	\begin{umlcallself}[op=infoWinner]{P1}
	             	\end{umlcallself}
             	\end{umlfragment}
             	
             	\begin{umlfragment}[type=alt, label=Mensch, inner xsep=7]
	             	\begin{umlcallself}[op=displayWinner]{P2}
	             	\end{umlcallself}
	             	
	             	\umlfpart[KI]
	             	
	             	\begin{umlcallself}[op=infoWinner]{P2}
	             	\end{umlcallself}
             	\end{umlfragment}
             	
             	\begin{umlcall}[op={deregister}, dt=7]{P1}{Server}
             	\begin{umlcall}[op={deregister}, dt=10]{P2}{Server}
             	
				\end{umlcall}
				\end{umlcall}

            \end{umlfragment}
            
            
		\end{umlcall}
        \end{umlcall}

    \end{umlseqdiag} 
\end{tikzpicture}

\paragraph{Zustände im Spiel} \mbox{}\\
In diesem Zustandsdiagramm werden abstrakt die Zustände beschrieben, welche während eines Spiels auftreten können.
\tikzset{singlestate/.style={draw,fill=yellow!20, rounded corners, align=center}}

\begin{tikzpicture}
	\umlstateinitial[name=initial, x=0, y=0]
	\umlstatefinal[name=final,x=8,y=-14]
	\begin{umlstate}[x=4, y=-10, name=turnp1]{Spieler 1 ist am Zug}
	\end{umlstate}

	\begin{umlstate}[x=12,y=-10,name=turnp2]{Spieler 2 ist am Zug}
	\end{umlstate}

	\begin{umlstate}[x=8,y=-6,name=apbp]{Bestimmen der AP/BP}
		\umlstatetext{Do/Bestimmen der AP und BP \\Abhängig von den \\gewählten Charaktere}
	\end{umlstate}

	\node[singlestate] at (3.5,-16)(charAction){Charakter gewählt};
	\node[singlestate] at (12.5,-16)(charAction2){Charakter gewählt};

	\umlHVtrans[anchor2=90]{initial}{apbp}

	\umlHVtrans[arg=Zug beendet,align=right,pos=1.5,anchor1=0,anchor2=260]{turnp1}{apbp}
	\umlHVtrans[arg=Zug beendet,align=left,pos=1.5,anchor1=180,anchor2=280]{turnp2}{apbp}

	\umlHVtrans[arg={Nächster Spieler},anchor1=175,anchor2=130,pos=0.05,align=right]{apbp}{turnp1}
	\umlHVtrans[arg={Nächster Spieler},anchor2=65,pos=0.8]{apbp}{turnp2}

	\umlHVtrans[arg={Siegbedingung},pos=1.5,align=right,anchor1=-20,anchor2=135]{turnp1}{final}
	\umlHVtrans[arg={Siegbedingung},pos=1.5,anchor1=200,anchor2=45,align=left]{turnp2}{final}
	
	\umltrans[arg={Überlanges Spiel  }, recursive=170|190|1.5cm, pos=2.1,align=right, recursive direction=left to left]{turnp1}{turnp1}
	\umltrans[arg=Überlanges Spiel, recursive=10|-10|1.5cm, pos=1.5, recursive direction=left to left]{turnp2}{turnp2}


	\umlHVtrans[arg=Charakter ausgewählt,anchor1=248,anchor2=100,align=right,pos=1.5]{turnp1}{charAction}
	\umlHVtrans[arg=Aktion/Bewegung,pos=1.5,align=left]{charAction}{turnp1}

	\umlHVtrans[arg=Charakter ausgewählt,anchor1=298,anchor2=80 ,align=left,pos=1.5]{turnp2}{charAction2}
	\umlHVtrans[arg=Aktion/Bewegung,align=right,pos=1.5]{charAction2}{turnp2}

\end{tikzpicture}

\paragraph{Zustände des Clients} \mbox{}\\
%Found at: %https://tex.stackexchange.com/questions/420217/change-style-of-basic-states-in-tikz-uml-package
\tikzset{singlestate/.style={draw,fill=yellow!20, rounded corners, align=center}}

\begin{tikzpicture}
	\umlstateinitial[name=initial, x=0, y=0]
	\begin{umlstate}[x=7, y=-2, name=menu]{Hauptmenü}
		\umlstatetext{}
		\umlbasicstate[x=0, y=0, name=menuDisconnected, do={bekannte Server\\ anpingen}, align=center]{Hauptmenü (disconnected)}
		\umlbasicstate[x=0, y=-4, name=menuConnected, do={Lobbies abfragen}]{Hauptmenü (connected)}
	\end{umlstate}
	
	\node[singlestate] at (0, -2.7) (help){Hilfe};
	\node[singlestate] at(16, -4.85) (settings){Einstellungen};
	\umlstatefinal[x=16, y=-0, name=final]
	\node[singlestate] at (7, -8) (lobbyView){Lobbyübersicht};
	
	\begin{umlstate}[x=7, y=-13, name=lobby]{Lobby}
		\umlstatetext{}
		\umlbasicstate[x=0, y=0, name=spieler, do={}, align=center]{in Lobby (Rolle: Spieler)}
		\umlbasicstate[x=0, y=-4, name=zuschauer, do={}]{in Lobby (Rolle: Zuschauer)}
	\end{umlstate}
	
	\node[singlestate] at (7, -20) (inGame){im Spiel};
	\node[singlestate] at (7, -22) (gameEnded){Spiel beendet};
	
	\umlVHtrans{initial}{menuDisconnected}
	
	\umltrans[align=center, arg={verbinde/}, pos=0.5, align=right, anchor1=-120, anchor2=125]{menuDisconnected}{menuConnected}
	\umltrans[arg={trennen/}, pos=0.5, align=left, anchor1=55, anchor2=-60]{menuConnected}{menuDisconnected}
	
	\umlHVtrans[arg={Hilfe aufrufen/}, pos=0.5, align=right, anchor1=160]{menu}{help}
	\umlVHtrans[arg={Hilfe schließen/}, pos=0.5, anchor2=200]{help}{menu}
	
	\umlHVtrans[arg={Einstellungen aufrufen/}, pos=0.8, align=right, anchor1=-30]{menu}{settings}
	\umlVHtrans[arg={Einstellungen schließen/}, pos=1.2, anchor1=-90, anchor2=-48, align=right]{settings}{menu}
	
	\umlHVtrans[arg={Client schließen/}, pos=0.7, align=right, anchor1=30]{menu}{final}
	
	\umltrans[arg={Lobbyübersicht aufrufen/}, pos=0.7, align=right, anchor1=-100, anchor2=158]{menu}{lobbyView}
	\umltrans[arg={Zurück zum Menü/}, pos=0.3, align=left, anchor1=23, anchor2=-81]{lobbyView}{menu}
	
	\umltrans[arg={verlasse Lobby/}, pos=0.4, align=left, anchor1=90, anchor2=-90]{lobby}{lobbyView}
	
	\umlHVHtrans[arg={betrete Lobby[Spielerslots belegt]/}, pos=1.5, align=left, anchor1=0, anchor2=0, arm1=2cm]{lobbyView}{zuschauer}
	\umlHVHtrans[arg={betrete Lobby[Spielerslot frei]/}, pos=1.5, align=right, anchor1=180, anchor2=180, arm1=-2cm]{lobbyView}{spieler}
	
	\umltrans[arg={Rollenwechsel/}, pos=0.5, align=right, anchor1=-110, anchor2=110]{spieler}{zuschauer}
	\umltrans[arg={Rollenwechsel/}, pos=0.5, align=left, anchor1=70, anchor2=-70]{zuschauer}{spieler}
	
	\umlHVHtrans[arg={Zurück zum Menü/}, pos=0.2, align=left, arm1=8.5cm, anchor2=-15]{gameEnded}{menuConnected}
	
	\umltrans[arg={Spielende erreicht/}, pos=0.6, align=left]{inGame}{gameEnded}
	
	\umltrans[arg={Spiel starten[alle Spieler bereit]/}, pos=0.6, align=left]{lobby}{inGame}
%
\end{tikzpicture}
\paragraph{Zustände des KI-Clients} \mbox{}\\
Das untenstehende Zustandsdiagramm gibt einen kleinen Überblick über die grobe Funktionsweise des KI-Clients. 

\begin{tikzpicture} 
    \umlstateinitial[y=4, name=initial]
    \begin{umlstate}[name=running]{running}
        \begin{umlstate}[name=waiting, y=0]{Warte}
        \end{umlstate}

        \begin{umlstate}[name=analyzing, y=-4]{Analysiere Spielzustand}
        \end{umlstate}

        \begin{umlstate}[name=planning, y=-8]{Plane nächste Aktion}
        \end{umlstate}

        \umlHVHtrans[arm1=5, arg={Aktion Senden}, pos=0.5]{planning}{waiting}
        \umltrans[arg={Aktion Empfangen}, pos=0.5]{waiting}{analyzing}
        \umltrans{analyzing}{planning}

    \end{umlstate}

    \umltrans[arg={verbinde/}, pos=0.3]{initial}{waiting}
    \umlstatefinal[name=final, x=-6, y=-1]
    \umlHVtrans[arg={Spielende erreicht/}, pos=0.6]{waiting}{final}
\end{tikzpicture}

\paragraph{Zustände des Servers} \mbox{}\\
\begin{tikzpicture} 
    \umlstateinitial[y=4, name=initial]
    \umlstateenter[name=connect, x=-6, y=4]
    \umlstateend[name=end, x=-6, y=-6]
    \begin{umlstate}{running}
        \begin{umlstate}[y=0, name=lobby]{Lobby}
        \end{umlstate}
        \begin{umlstate}[y=-4, name=game]{Spiel}
        \end{umlstate}
    \end{umlstate}

    \umltrans{initial}{running}
    \umlVHtrans[arg={Verbinde Client}, pos=1.5]{connect}{lobby}
    \umlHVHtrans[arg={Spielstart[$\geq 2$ Spieler]}, arm1=2, pos=1.5]{lobby}{game}
    \umlHVtrans[arg={Spielende[Gewonnen]/NotifyClients}, pos=1.8]{game}{end}
    \umlVHtrans[arg={Spielende[Timeout]/NotifyClients}, pos=1.5]{game}{end}
\end{tikzpicture}


\clearpage
\subsubsection{Funktionale Systemanforderungen}
Dieser Abschnitt enthält alle Anforderungen, die die grundlegenden Aktionen des Softwaresystems spezifizieren.
\paragraph{Server} \mbox{}\\
\begin{tabularx}{16cm}{l|X}
\textbf{ID} & \textbf{FA-S} \\
\hline
TITEL: & Partie Konfiguration \\
\hline
BESCHREIBUNG: & Der Server muss beim Start die Partie Konfiguration aus einer Datei lesen. \\
\hline
BEGRÜNDUNG: & Die Konfiguration soll für das Balancing des Spiels einfach angepasst werden können. \\
\hline
ABHÄNGIGKEITEN: & \\
\end{tabularx}

\begin{tabularx}{16cm}{l|X}
\textbf{ID} & \textbf{FA-S} \\
\hline
TITEL: & Szenario Konfiguration \\
\hline
BESCHREIBUNG: & Der Server muss beim Start die Szenario Konfiguration (Spielfeld) aus einer Datei lesen. \\
\hline
BEGRÜNDUNG: & Das Spielfeld soll veräderbar sein. \\
\hline
ABHÄNGIGKEITEN: & \\
\end{tabularx}

\begin{tabularx}{16cm}{l|X}
\textbf{ID} & \textbf{FA-S} \\
\hline
TITEL: & Charakter Konfiguration \\
\hline
BESCHREIBUNG: & Der Server muss beim Start die Konfiguration der Charaktere aus einer Datei lesen. \\
\hline
BEGRÜNDUNG: & \\
\hline
ABHÄNGIGKEITEN: & \\
\end{tabularx}

\begin{tabularx}{16cm}{l|X}
\textbf{ID} & \textbf{FA-S} \\
\hline
TITEL: & Client Verbindung \\
\hline
BESCHREIBUNG: & Der Server muss mittels Websockets über TCP eine Verbindung mit dem Client halten.
Die Verbindung wird vom Client initiiert. Bei Verbindungsabbruch muss die Verbindung wiederhergestellt werden. \\
\hline
BEGRÜNDUNG: & Integrale Funktionalität des Servers \\
\hline
ABHÄNGIGKEITEN: & \\
\end{tabularx}

\begin{tabularx}{16cm}{l|X}
\textbf{ID} & \textbf{FA-S} \\
\hline
TITEL: & Zuschauer \\
\hline
BESCHREIBUNG: & Ein Client kann sich als Zuschauer verbinden, bekommt also Updates über den Spielzustand, kann aber nicht aktiv am Spiel teilnehmen. \\
\hline
BEGRÜNDUNG: & \\
\hline
ABHÄNGIGKEITEN: & \\
\end{tabularx}

\begin{tabularx}{16cm}{l|X}
\textbf{ID} & \textbf{FA-S} \\
\hline
TITEL: & Client Timeout \\
\hline
BESCHREIBUNG: & \\
\hline
BEGRÜNDUNG: & \\
\hline
ABHÄNGIGKEITEN: & \\
\end{tabularx}

\begin{tabularx}{16cm}{l|X}
\textbf{ID} & \textbf{FA-S} \\
\hline
TITEL: & Verspätete Nachrichten \\
\hline
BESCHREIBUNG: & \\
\hline
BEGRÜNDUNG: & \\
\hline
ABHÄNGIGKEITEN: & \\
\end{tabularx}

\begin{tabularx}{16cm}{l|X}
\textbf{ID} & \textbf{FA-S} \\
\hline
TITEL: & Spiel Durchführung \\
\hline
BESCHREIBUNG: & \\
\hline
BEGRÜNDUNG: & \\
\hline
ABHÄNGIGKEITEN: & \\
\end{tabularx}

\begin{tabularx}{16cm}{l|X}
\textbf{ID} & \textbf{FA-S} \\
\hline
TITEL: & Spielpause \\
\hline
BESCHREIBUNG: & \\
\hline
BEGRÜNDUNG: & \\
\hline
ABHÄNGIGKEITEN: & \\
\end{tabularx}

\begin{tabularx}{16cm}{l|X}
\textbf{ID} & \textbf{FA-S} \\
\hline
TITEL: & Erkennung von Regelverstößen \\
\hline
BESCHREIBUNG: & \\
\hline
BEGRÜNDUNG: & \\
\hline
ABHÄNGIGKEITEN: & \\
\end{tabularx}

\begin{tabularx}{16cm}{l|X}
\textbf{ID} & \textbf{FA-S} \\
\hline
TITEL: & Senden des Spielzustands \\
\hline
BESCHREIBUNG: & \\
\hline
BEGRÜNDUNG: & \\
\hline
ABHÄNGIGKEITEN: & \\
\end{tabularx}

\begin{tabularx}{16cm}{l|X}
\textbf{ID} & \textbf{FA-S} \\
\hline
TITEL: & Gewinner Erkennung \\
\hline
BESCHREIBUNG: & \\
\hline
BEGRÜNDUNG: & \\
\hline
ABHÄNGIGKEITEN: & \\
\end{tabularx}

\begin{tabularx}{16cm}{l|X}
\textbf{ID} & \textbf{FA-S} \\
\hline
TITEL: & Erstellen von Replay Files \\
\hline
BESCHREIBUNG: & \\
\hline
BEGRÜNDUNG: & \\
\hline
ABHÄNGIGKEITEN: & \\
\end{tabularx}

\begin{tabularx}{16cm}{l|X}
\textbf{ID} & \textbf{FA-S} \\
\hline
TITEL: & JSON Encoding \\
\hline
BESCHREIBUNG: & \\
\hline
BEGRÜNDUNG: & \\
\hline
ABHÄNGIGKEITEN: & \\
\end{tabularx}

\begin{tabularx}{16cm}{l|X}
\textbf{ID} & \textbf{FA-S} \\
\hline
TITEL: & JSON Decoding \\
\hline
BESCHREIBUNG: & \\
\hline
BEGRÜNDUNG: & \\
\hline
ABHÄNGIGKEITEN: & \\
\end{tabularx}

\begin{tabularx}{16cm}{l|X}
\textbf{ID} & \textbf{FA-S} \\
\hline
TITEL: & Websockets \\
\hline
BESCHREIBUNG: & \\
\hline
BEGRÜNDUNG: & \\
\hline
ABHÄNGIGKEITEN: & \\
\end{tabularx}


\paragraph{Client} \mbox{}\\
\begin{tabularx}{16cm}{l|X}
\textbf{ID} & \textbf{FA-C} \\
\hline
TITEL: & Optional: Intro \\
\hline 
BESCHREIBUNG: & Beim Starten des Programms soll ein Film ablaufen, der den Titel des Spiels präsentiert und die Teammitglieder nennt. Dieser kann jederzeit mittels Tastendruck abgebrochen werden. Nach Ende des Films oder Abbruch gelangt der Nutzer ins Hauptmenü. \\
\hline
BEGRÜNDUNG: & Der Film versteckt etwaige Ladezeiten beim Starten und informiert den Spieler über das Spiel und die Autoren.\\
\hline
ABHÄNGIGKEITEN: & \\
\end{tabularx}

\begin{tabularx}{16cm}{l|X}
\textbf{ID} & \textbf{FA-C} \\
\hline
TITEL: & Hauptmenü \\
\hline 
BESCHREIBUNG: & Im Hauptmenü hat der Nutzer die Möglichkeit, seinen Namen festzulegen, einem Server durch Eingabe der Adresse beizutreten, oder das Spiel zu beenden.\\ 
\hline
BEGRÜNDUNG: & Vor dem Spielen muss der Spieler eine Verbindung zum Server aufbauen. \\
\hline
ABHÄNGIGKEITEN: & \\
\end{tabularx}

\begin{tabularx}{16cm}{l|X}
\textbf{ID} & \textbf{FA-C} \\
\hline
TITEL: & Beitreten als Mitspieler \\
\hline 
BESCHREIBUNG: & Nach Verbinden mit dem Server kann der Spieler die Rolle als aktiver Spieler wählen. Dies wird dem Server mitgeteilt. \\
\hline
BEGRÜNDUNG: & Es muss zwischen aktiven Spielern und passiven Zuschauern unterschieden werden. \\
\hline
ABHÄNGIGKEITEN: & \\
\end{tabularx}

\begin{tabularx}{16cm}{l|X}
\textbf{ID} & \textbf{FA-C} \\
\hline
TITEL: & Beitreten als Zuschauer \\
\hline 
BESCHREIBUNG: & Nach Verbinden mit dem Server kann der Spieler die Rolle als passiver Zuschauer wählen. Dies wird dem Server mitgeteilt. \\
\hline
BEGRÜNDUNG: & Es muss zwischen aktiven Spielern und passiven Zuschauern unterschieden werden. \\
\hline
ABHÄNGIGKEITEN: & \\
\end{tabularx}

\begin{tabularx}{16cm}{l|X}
\textbf{ID} & \textbf{FA-C} \\
\hline
TITEL: & Registrieren als menschlicher Spieler \\
\hline 
BESCHREIBUNG: & Beim Verbinden mit dem Server muss diesem mitgeteilt werden, dass es sich um einen menschlichen Spieler handelt. \\
\hline
BEGRÜNDUNG: & Es muss zwischen menschlichen Spielern und KI Spielern unterschieden werden. \\
\hline
ABHÄNGIGKEITEN: & \\
\end{tabularx}

\begin{tabularx}{16cm}{l|X}
\textbf{ID} & \textbf{FA-C} \\
\hline
TITEL: & Nutzername \\
\hline 
BESCHREIBUNG: & Der Client muss es dem Nutzer ermöglichen einen Nutzernamen zu wählen. Dieser soll vom Server validiert werden und wird eventuell nicht akzeptiert. \\
\hline
BEGRÜNDUNG: & Beide Mitspieler werden durch Nutzernamen identifiziert. \\
\hline
ABHÄNGIGKEITEN: & \\
\end{tabularx}

\begin{tabularx}{16cm}{l|X}
\textbf{ID} & \textbf{FA-C} \\
\hline
TITEL: & Spiel Anzeige \\
\hline 
BESCHREIBUNG: & Der Client muss dem Benutzer den Aktuellen Zustand des Spiels jederzeit darstellen. Dazu gehört:
\begin{itemize}
    \item Charaktere der eigenen Fraktion
    \item Eigenschaften der Charaktere
    \item Gadgets der Charaktere
    \item Spielfeld mit Spielern und Gadgets
\end{itemize} \\
\hline
BEGRÜNDUNG: & Integraler Teil des Spiels \\
\hline
ABHÄNGIGKEITEN: & \\
\end{tabularx}

\begin{tabularx}{16cm}{l|X}
\textbf{ID} & \textbf{FA-C} \\
\hline
TITEL: & Interaktion mit dem Spiel \\
\hline 
BESCHREIBUNG: & Der Client muss dem Spieler ermöglichen, Aktionen auf dem Spielfeld durchzuführen, wenn der Spieler am Zug ist. \\
\hline
BEGRÜNDUNG: & Integraler Bestandteil des Spiels \\
\hline
ABHÄNGIGKEITEN: & \\
\end{tabularx}

\begin{tabularx}{16cm}{l|X}
\textbf{ID} & \textbf{FA-C} \\
\hline
TITEL: & Optional: Hilfefunktion \\
\hline 
BESCHREIBUNG: & Der Client soll den Spieler durch Vorschläge für Aktionen unterstützen. \\
\hline
BEGRÜNDUNG: & Dies vereinfacht das Erlernen des Spiels und das Kennenlernen der verschiedenen Aktionen. \\
\hline
ABHÄNGIGKEITEN: & \\
\end{tabularx}

\begin{tabularx}{16cm}{l|X}
\textbf{ID} & \textbf{FA-C} \\
\hline
TITEL: & Optional: Hotkeys \\
\hline 
BESCHREIBUNG: & Die Aktionen während dem Spiel sollen durch Hotkeys ausführbar sein. \\
\hline
BEGRÜNDUNG: & Dies dient einer komfortableren Bedienung des Spiels. \\
\hline
ABHÄNGIGKEITEN: & \\
\end{tabularx}

\begin{tabularx}{16cm}{l|X}
\textbf{ID} & \textbf{FA-C} \\
\hline
TITEL: & Animation der Aktionen \\
\hline 
BESCHREIBUNG: & Die Aktionen einer Runde sollen animiert dargestellt werden. \\
\hline
BEGRÜNDUNG: & Einfacheres Nachvollziehen der eigenen und gegnerischen Aktionen \\
\hline
ABHÄNGIGKEITEN: & \\
\end{tabularx}

\begin{tabularx}{16cm}{l|X}
\textbf{ID} & \textbf{FA-C} \\
\hline
TITEL: & Wunsch auf Pausieren \\
\hline 
BESCHREIBUNG: & Der Client ermöglicht es dem Spieler, ein Pausieren der Runde zu beantragen. \\
\hline
BEGRÜNDUNG: & Für bessere Immersion kann der Benutzer während dem Spielen selbst Cocktails trinken, was eventuelle Pausen fürs Nachfüllen bedingt. \\
\hline
ABHÄNGIGKEITEN: & \\
\end{tabularx}

\begin{tabularx}{16cm}{l|X}
\textbf{ID} & \textbf{FA-C} \\
\hline
TITEL: & Wunsch auf Wiederaufnahme des Spiels \\
\hline 
BESCHREIBUNG: & Der Client ermöglicht es dem Spieler, die Wiederaufnahme des Spiels (Beenden der Pause) zu beantragen. \\
\hline
BEGRÜNDUNG: & Ermöglicht Beenden der Pause \\
\hline
ABHÄNGIGKEITEN: & \\
\end{tabularx}

\begin{tabularx}{16cm}{l|X}
\textbf{ID} & \textbf{FA-C} \\
\hline
TITEL: & Persistente Session \\
\hline 
BESCHREIBUNG: & Ein Abbruch der Verbindung zum Server darf nicht zum Beenden des Spiels führen. Der Client muss versuchen, die Verbindung wiederherzustellen und im Erfolgsfall das Spiel fortsetzen. \\
\hline
BEGRÜNDUNG: & Mit Verbindungsabbrüchen ist z.B bei der Verwendung von WiFi zu rechnen, es soll trotzdem möglich sein zu spielen. \\
\hline
ABHÄNGIGKEITEN: & \\
\end{tabularx}

\begin{tabularx}{16cm}{l|X}
\textbf{ID} & \textbf{FA-C} \\
\hline
TITEL: & Gewinneranzeige \\
\hline 
BESCHREIBUNG: & Wenn durch den Server ein Gewinner festgestellt wird, muss dies im Client dem Spieler präsentiert werden. \\
\hline
BEGRÜNDUNG: & Integraler Bestandteil des Spiels \\
\hline
ABHÄNGIGKEITEN: & \\
\end{tabularx}

\begin{tabularx}{16cm}{l|X}
\textbf{ID} & \textbf{FA-C} \\
\hline
TITEL: & Optional: Statistik \\
\hline 
BESCHREIBUNG: & Neben der Anzeige des Gewinners sollen zusätzlich Statistiken zum Spielverlauf angezeigt werden. \\
\hline
BEGRÜNDUNG: & Information des Spielers \\
\hline
ABHÄNGIGKEITEN: & \\
\end{tabularx}

\begin{tabularx}{16cm}{l|X}
\textbf{ID} & \textbf{FA-C} \\
\hline
TITEL: & Optional: Replay \\
\hline 
BESCHREIBUNG: & Der Client kann statt eines aktiven Spiels auch eine vom Server erstellte Aufzeichnung eines vergangenen Spiels anzeigen. Hierbei ist keinerlei Interaktion möglich. \\
\hline
BEGRÜNDUNG: & Erneutes Anschauen besonders interessanter Partien \\
\hline
ABHÄNGIGKEITEN: & \\
\end{tabularx}

\paragraph{KI-Client} \mbox{}\\
\begin{tabularx}{16cm}{l|X}
\refstepcounter{table}\label{ki-session}
\textbf{ID} & \textbf{FA-KI \arabic{table}} \\
\hline
TITEL: & Websocket Verbindung \\
\hline 
BESCHREIBUNG: & Der KI-Client muss eine Websocket Verbindung zum Server aufbauen können. Darüber können JSON encodierte Nachrichten geschickt werden. \\
\hline
BEGRÜNDUNG: & Kommunikation mit dem Server \\
\hline
ABHÄNGIGKEITEN: & \\
\end{tabularx}

\begin{tabularx}{16cm}{l|X}
\refstepcounter{table}\label{ki-register}
\textbf{ID} & \textbf{FA-KI \arabic{table}} \\
\hline
TITEL: & Registrieren als KI \\
\hline 
BESCHREIBUNG: & Der KI-Client muss sich nach der Verbindung mit dem Server als KI identifizieren. \\
\hline
BEGRÜNDUNG: & Anzeige für den Spielers, ob er gegen eine KI oder einen menschlichen Gegner spielt. \\
\hline
ABHÄNGIGKEITEN: & FA-KI \ref{ki-session} \\
\end{tabularx}

\begin{tabularx}{16cm}{l|X}
\refstepcounter{table}\label{ki-cli}
\textbf{ID} & \textbf{FA-KI \arabic{table}} \\
\hline
TITEL: & Commandline Argumente \\
\hline 
BESCHREIBUNG: & Der KI-Client muss Kommandozeilen Argumente unterstützen um Parameter zu setzen. Die Argumente sind vom Standardisierungskomitee definiert. \\
\hline
BEGRÜNDUNG: & Nicht-interaktive Benutzung, Konfiguration des Clients. \\
\hline
ABHÄNGIGKEITEN: & \\
\end{tabularx}

\begin{tabularx}{16cm}{l|X}
\refstepcounter{table}\label{ki-config}
\textbf{ID} & \textbf{FA-KI \arabic{table}} \\
\hline
TITEL: & Konfigurationsdatei \\
\hline 
BESCHREIBUNG: & Die Parameter des Commandline Interface müssen auch über eine Konfigurationsdatei konfigurierbar sein. \\
\hline
BEGRÜNDUNG: & Persistentes Speichern der Parameter \\
\hline
ABHÄNGIGKEITEN: & FA-KI \ref{ki-cli}\\
\end{tabularx}

\begin{tabularx}{16cm}{l|X}
\refstepcounter{table}\label{ki-intelligenz}
\textbf{ID} & \textbf{FA-KI \arabic{table}} \\
\hline
TITEL: & Intelligenzstufen \\
\hline 
BESCHREIBUNG: & Der KI-Client muss die Möglichkeit besitzen, zwischen mehreren Strategien zu wechseln abhängig von der eingestellten Intelligenzstufe. Die Intelligenzstufe kann durch ein Commandline Argument oder in der Konfigurationsdatei festgelegt werden. \\
\hline
BEGRÜNDUNG: & Unterschiedlicher Schwierigkeitsgrad für Anfänger vs. fortgeschrittene Spieler \\
\hline
ABHÄNGIGKEITEN: & FA-KI \ref{ki-cli}, \ref{ki-config}\\
\end{tabularx}

\begin{tabularx}{16cm}{l|X}
\refstepcounter{table}\label{ki-actions}
\textbf{ID} & \textbf{FA-KI \arabic{table}} \\
\hline
TITEL: & Aktionen \\
\hline 
BESCHREIBUNG: & Die KI muss in jeder Runde regelkonforme und sinnvolle Aktionen bestimmen, und diese dem Server senden. \\
\hline
BEGRÜNDUNG: & Grundlegende Funktionalität der KI \\
\hline
ABHÄNGIGKEITEN: & FA-KI \ref{ki-session}\\
\end{tabularx}

\begin{tabularx}{16cm}{l|X}
\refstepcounter{table}\label{ki-api}
\textbf{ID} & \textbf{FA-KI \arabic{table}} \\
\hline
TITEL: & Optional: API \\
\hline 
BESCHREIBUNG: & Die KI soll eine Programmierschnittstelle zur Anbindung an den Benutzer Client besitzen. \\
\hline
BEGRÜNDUNG: & Verwenden der KI zum Vorschlagen von Aktionen für den Spieler \\
\hline
ABHÄNGIGKEITEN: & FA-KI \ref{ki-actions}\\
\end{tabularx}

\paragraph{Editor} \mbox{}\\
\begin{tabularx}{16cm}{l|X}
\refstepcounter{table}\label{e-json-encoding}
\textbf{ID} & \textbf{FA-E \arabic{table}} \\
\hline
TITEL: & JSON Encoding \\
\hline
BESCHREIBUNG: & Der Editor muss die Funktionalität besitzen, Konfigurationsdateien für Szenarios, Partien und Charaktere im JSON Format zu encodieren. \\
\hline
BEGRÜNDUNG: & Notwendig zum Speichern der Konfiguration. \\
\hline
PRIORITÄT: & ++\\ 
\hline
ABHÄNGIGKEITEN: & \\
\end{tabularx}

\begin{tabularx}{16cm}{l|X}
\refstepcounter{table}\label{e-json-decoding}
\textbf{ID} & \textbf{FA-E \arabic{table}} \\
\hline
TITEL: & JSON Decoding \\
\hline
BESCHREIBUNG: & Der Editor muss die Funktionalität besitzen, Konfigurationsdateien für Szenarios, Partien und Charaktere aus dem JSON Format zu decodieren. \\
\hline
BEGRÜNDUNG: & Notwendig zum laden der Konfiguration. \\
\hline
PRIORITÄT: & ++\\
\hline
ABHÄNGIGKEITEN: & \\
\end{tabularx}

\begin{tabularx}{16cm}{l|X}
\refstepcounter{table}\label{e-gui}
\textbf{ID} & \textbf{FA-E \arabic{table}} \\
\hline
TITEL: & GUI \\
\hline
BESCHREIBUNG: & Der Editor muss über eine grafische Benutzeroberfläche zum Editieren der Konfiguration besitzen. \\
\hline
BEGRÜNDUNG: & Grundlegende Funktionalität \\
\hline
PRIORITÄT: & +\\
\hline
ABHÄNGIGKEITEN: & \\
\end{tabularx}

\begin{tabularx}{16cm}{l|X}
\refstepcounter{table}\label{e-szenarioedit}
\textbf{ID} & \textbf{FA-E \arabic{table}} \\
\hline
TITEL: & Szenario Editor \\
\hline
BESCHREIBUNG: & Benutzeroberfläche zum Editieren des Szenarios mit Darstellung äquivalent zum Benutzer Client des Spiels \\
\hline
BEGRÜNDUNG: & Grundlegende Funktionalität \\
\hline
PRIORITÄT: & 0\\
\hline
ABHÄNGIGKEITEN: & FA-E \ref{e-gui}\\
\end{tabularx}

\begin{tabularx}{16cm}{l|X}
\refstepcounter{table}\label{e-partieedit}
\textbf{ID} & \textbf{FA-E \arabic{table}} \\
\hline
TITEL: & Partie Editor \\
\hline
BESCHREIBUNG: & Im Partie Editor werden die Partie Konfigurationen (Zeit für Phasen, Wahrscheinlichkeiten) in tabellarischer Form editiert. \\
\hline
BEGRÜNDUNG: & Grundlegende Funktionalität \\
\hline
PRIORITÄT: & +\\
\hline
ABHÄNGIGKEITEN: & FA-E \ref{e-gui}\\
\end{tabularx}

\begin{tabularx}{16cm}{l|X}
\refstepcounter{table}\label{e-charedit}
\textbf{ID} & \textbf{FA-E \arabic{table}} \\
\hline
TITEL: & Charakter Editor \\
\hline
BESCHREIBUNG: & Der Charakter Editor zeigt die existierenden Charaktere in der Konfiguration an und erlaubt es, bestehende und neue Charaktere zu editieren. \\
\hline
BEGRÜNDUNG: & Grundlegende Funktionalität \\
\hline
PRIORITÄT: & -\\
\hline
ABHÄNGIGKEITEN: & FA-E \ref{e-gui}\\
\end{tabularx}

\paragraph{Allgemeine funktionale Anforderungen} \mbox{}\\
%Hier kommen die generellen FA rein

\begin{tabularx}{16cm}{l|X}
	\refstepcounter{table}\label{Felder}
	\textbf{ID} & \textbf{FA-G \arabic{table}} \\
	\hline
	TITEL: & Felder \\
	\hline
	BESCHREIBUNG: & Die schachbrettartigen Felder, aus denen das Spielbrett aufgebaut ist, können freier Raum sein, oder mit Hindernissen oder Objekten besetzt sein. \\
	\hline
	BEGRÜNDUNG: & Unterschiedliche Arten von Feldern erlauben den Charakteren das Fortbewegen und interagieren auf dem Spielbrett. \\
	\hline
	ABHÄNGIGKEITEN: & FA-G \ref{Spielbrett} \\
\end{tabularx}

\begin{tabularx}{16cm}{l|X}
	\refstepcounter{table}\label{Spielbrett}
	\textbf{ID} & \textbf{FA-G \arabic{table}} \\
	\hline
	TITEL: & Spielbrett \\
	\hline
	BESCHREIBUNG: & Das Spiel findet auf einem Spielbrett statt, welches aus den Feldern eines kartesischen Gitters aufgebaut ist. \\
	\hline
	BEGRÜNDUNG: & Der Aufbau aus Feldern in einem kartesischen Gitter ermöglicht die Fortbewegung der Charaktere durch Bewegungspunkte. \\
	\hline
	ABHÄNGIGKEITEN: & FA-G \ref{Felder} \\
\end{tabularx}

\begin{tabularx}{16cm}{l|X}
	\refstepcounter{table}\label{Entfernung}
	\textbf{ID} & \textbf{FA-G \arabic{table}} \\
	\hline
	TITEL: & Entfernung \\
	\hline
	BESCHREIBUNG: & Die Entfernung zwischen zwei Feldern A und B ist die minimale Anzahl an Schritten auf Nachbarfelder (in alle acht Richtungen), die benötigt wird um von A nach B zu gelangen. \\
	\hline
	BEGRÜNDUNG: & Mit der Entfernung zwischen zwei Feldern A und B wird berechnet, wieviele Bewegungspunkte ein Charakter benötigt, um von Feld A zu Feld B zu gelangen. \\
	\hline
	ABHÄNGIGKEITEN: & FA-G \ref{Felder}, FA-G \ref{Spielbrett} \\
\end{tabularx}

\begin{tabularx}{16cm}{l|X}
	\refstepcounter{table}\label{Sichtlinie}
	\textbf{ID} & \textbf{FA-G \arabic{table}} \\
	\hline
	TITEL: & Sichtlinie \\
	\hline
	BESCHREIBUNG: & Es existiert eine Sichtlinie zwischen zwei Feldern A und B, wenn gilt: Betrachtet man die Verbindungslinie vom Mittelpunkt von A zum Mittelpunkt von B, so müssen alle Felder, die von dieser Verbindunglinie geschnitten werden Felder sein, die die Sichtlinie nicht blockieren. Dann ist B von A aus sichtbar. Felder, die von dieser Verbindungslinie nur tangiert werden blockieren die Sicht nicht. \\
	\hline
	BEGRÜNDUNG: & Dadurch wird festgestellt, ob ein Feld B von einem Feld A aus sichtbar ist, also ob ein Charakter auf Feld A sehen kann, was sich auf Feld B befindet. \\
	\hline
	ABHÄNGIGKEITEN: & FA-G \ref{Felder}, FA-G \ref{Spielbrett} \\
\end{tabularx}

\begin{tabularx}{16cm}{l|X}
	\refstepcounter{table}\label{Freies Feld}
	\textbf{ID} & \textbf{FA-G \arabic{table}} \\
	\hline
	TITEL: & Freies Feld \\
	\hline
	BESCHREIBUNG: & Charaktere können auf einem Freien Feld stehen oder darüber hinweglaufen. Es blockiert die Sichtlinie nicht. \\
	\hline
	BEGRÜNDUNG: & Freie Felder dienen der Fortbewegung der Charaktere. \\
	\hline
	ABHÄNGIGKEITEN: & FA-G \ref{Felder}, FA-G \ref{Sichtlinie} \\
\end{tabularx}

\begin{tabularx}{16cm}{l|X}
	\refstepcounter{table}\label{Wand}
	\textbf{ID} & \textbf{FA-G \arabic{table}} \\
	\hline
	TITEL: & Wand \\
	\hline
	BESCHREIBUNG: & Ein Feld, auf welchem sich eine Wand befindet ist nicht betretbar und es blockiert die Sichtlinie. \\
	\hline
	BEGRÜNDUNG: & Durch Wände können große Räume getrennt werden oder neue Räume entstehen. \\
	\hline
	ABHÄNGIGKEITEN: & FA-G \ref{Felder}, FA-G \ref{Sichtlinie} \\
\end{tabularx}

\begin{tabularx}{16cm}{l|X}
	\refstepcounter{table}\label{Kamin-Feld}
	\textbf{ID} & \textbf{FA-G \arabic{table}} \\
	\hline
	TITEL: & Kamin-Feld \\
	\hline
	BESCHREIBUNG: & Kamin-Felder sind nicht betretbar und sie blockieren die Sichtlinie. Befindet sich ein Charakter mit klammen Klamotten am Anfang einer Runde auf einem Nachbarfeld eines Kamin-Feldes, so wird die Klamme Klamotten-Eigenschaft am Ende der Runde entfernt. \\
	\hline
	BEGRÜNDUNG: & Der Kamin trocknet die klammen Klamotten, dadurch kann ein Charakter diese Eigenschaft los werden. \\
	\hline
	ABHÄNGIGKEITEN: & FA-G \ref{Felder}, FA-G \ref{Sichtlinie} \\
\end{tabularx}

\begin{tabularx}{16cm}{l|X}
	\refstepcounter{table}\label{Sitzplatz}
	\textbf{ID} & \textbf{FA-G \arabic{table}} \\
	\hline
	TITEL: & Sitzplatz \\
	\hline
	BESCHREIBUNG: & Auf einem solchen Feld befindet sich eine Sitzgelegenheit, bspw. ein Sessel, Barhocker, etc. Das Feld ist betretbar und blockiert die Sichtlinie nicht. Befindet sich ein Charakter am Anfang einer Runde auf einem Sitzplatz-Feld, so werden seine Health Points am Ende der Runde auf den Maximalwert aufgefüllt. \\
	\hline
	BEGRÜNDUNG: & Dadurch können Charaktere mit wenigen Health Points diese wieder füllen. \\
	\hline
	ABHÄNGIGKEITEN: & FA-G \ref{Felder}, FA-G \ref{Sichtlinie} \\
\end{tabularx}

\begin{tabularx}{16cm}{l|X}
	\refstepcounter{table}\label{Spielchips}
	\textbf{ID} & \textbf{FA-G \arabic{table}} \\
	\hline
	TITEL: & Spielchips \\
	\hline
	BESCHREIBUNG: & Charaktere haben die Möglichkeit an Roulette-Tischen Spielchips zu gewinnen. Alle gesammelten Spielchips einer Fraktion werden am Ende einer Partie in Intelligence Points umgerechnet. \\
	\hline
	BEGRÜNDUNG: & Dadurch wird das Spiel spannender, da die Spielchips den Spielausgang beeinflussen. \\
	\hline
	ABHÄNGIGKEITEN: & FA-G \ref{Roulette-Tisch} \\
\end{tabularx}

\begin{tabularx}{16cm}{l|X}
	\refstepcounter{table}\label{Roulette-Tisch}
	\textbf{ID} & \textbf{FA-G \arabic{table}} \\
	\hline
	TITEL: & Roulette-Tisch \\
	\hline
	BESCHREIBUNG: & Ein Feld, auf welchem ein Roulette-Tisch steht ist nicht betretbar und es blockiert die Sichtlinie nicht. Wenn ein Charakter auf einem Nachbarfeld eines Roulette-Tisches steht, kann er als Aktion einmal Roulette spielen. Jeder Roulette-Tisch verfügt zu Beginn einer Partie über eine im Szenario festgelegte individuelle Anzahl an Chips. \\
	\hline
	BEGRÜNDUNG: & An Roulette-Tischen haben Charaktere die Chance Chips zu gewinnen bzw. zu verlieren. \\
	\hline
	ABHÄNGIGKEITEN: & FA-G \ref{Felder}, FA-G \ref{Sichtlinie} \\
\end{tabularx}

\begin{tabularx}{16cm}{l|X}
	\refstepcounter{table}\label{Bar-Tisch}
	\textbf{ID} & \textbf{FA-G \arabic{table}} \\
	\hline
	TITEL: & Bar-Tisch \\
	\hline
	BESCHREIBUNG: & Felder, auf denen sich ein Bar-Tisch befindet sind nicht betretbar und sie blockieren die Sichtlinie nicht. Zu Beginn jeder Runde erscheint auf jedem leeren Bar-Tisch ein Cocktail. \\
	\hline
	BEGRÜNDUNG: & An Bar-Tischen können die Charaktere Cocktails aufnehmen. \\
	\hline
	ABHÄNGIGKEITEN: & FA-G \ref{Felder}, FA-G \ref{Cocktail aufnehmen}, FA-G \ref{Sichtlinie} \\
\end{tabularx}

\begin{tabularx}{16cm}{l|X}
	\refstepcounter{table}\label{Tresor}
	\textbf{ID} & \textbf{FA-G \arabic{table}} \\
	\hline
	TITEL: & Tresor \\
	\hline
	BESCHREIBUNG: & Felder, auf denen sich ein Tresor befindet, sind nicht betretbar und sie blockieren die Sichtlinie nicht. Tresore können Geheiminformationen oder Gadgets enthalten. Die Tresore sind eindeutig und sichtbar durchnummeriert (1, 2, ...). \\
	\hline
	BEGRÜNDUNG: & An Tresoren können Charaktere Geheimnisse oder Gadgets erhalten. \\
	\hline
	ABHÄNGIGKEITEN: & FA-G \ref{Felder}, FA-G \ref{Sichtlinie},  \todo[inline]{2.4 Gadgets, 2.7 Geheimnisse} \\
\end{tabularx}

\begin{tabularx}{16cm}{l|X}
	\refstepcounter{table}\label{Charakter}
	\textbf{ID} & \textbf{FA-G \arabic{table}} \\
	\hline
	TITEL: & Charakter \\
	\hline
	BESCHREIBUNG: & Ein Charakter besitzt folgende Werte, die normalerweise für die gegnerische Fraktion nicht sichtbar sind: 
	\begin{itemize}
		\item Name
		\item Beschreibung
		\item Position
		\item Fraktion
		\item Bewegungspunkte (BP) und Aktionspunkte (AP)
		\item Health Points (HP)
		\item Intelligence Points (IP)
		\item Eigenschaften
		\item Inventar
		\item hält Cocktail
		\item Spielchips
	\end{itemize}
	\\
	\hline
	BEGRÜNDUNG: & Jeder Charakter soll individuell sein und sich über das Spiel hinweg verändern können.\\
	\hline
	ABHÄNGIGKEITEN: & FA-G \ref{Charakter-NameBeschreibung}, FA-G \ref{Charakter-Position}, FA-G \ref{Charakter-Fraktion},  FA-G \ref{BP und AP}, FA-G \ref{HP}, FA-G \ref{IP}, FA-G \ref{Eigenschaften}, FA-G \ref{Inventar}, FA-G \ref{haelt Cocktail}, FA-G \ref{Spielchips} \\
\end{tabularx}

\begin{tabularx}{16cm}{l|X}
	\refstepcounter{table}\label{Charakter-NameBeschreibung}
	\textbf{ID} & \textbf{FA-G \arabic{table}} \\
	\hline
	TITEL: & Charakter-Name und -Beschreibung \\
	\hline
	BESCHREIBUNG: & Jeder Charakter hat einen eindeutigen Namen und die Charakter-Beschreibung beschreibt den Charakter aus \glqq James Bond\grqq in wenigen Sätzen.\\
	\hline
	BEGRÜNDUNG: & Zusätzliche Informationen für den Spieler. \\
	\hline
	ABHÄNGIGKEITEN: & \\
\end{tabularx}

\begin{tabularx}{16cm}{l|X}
	\refstepcounter{table}\label{Charakter-Position}
	\textbf{ID} & \textbf{FA-G \arabic{table}} \\
	\hline
	TITEL: & Charakter-Position \\
	\hline
	BESCHREIBUNG: & Gibt die Position des Charakters auf dem Spielfeld in x- und y-Koordinaten an.\\
	\hline
	BEGRÜNDUNG: & Hält fest, wo sich der Charakter auf dem Spielfeld befindet.\\
	\hline
	ABHÄNGIGKEITEN: & FA-G \ref{Exfiltratation},  \todo[inline]{Spielfeld, 2.9. Beginn der Partie, 2.5 Bewegung, }\\
\end{tabularx}

\begin{tabularx}{16cm}{l|X}
	\refstepcounter{table}\label{Charakter-Fraktion}
	\textbf{ID} & \textbf{FA-G \arabic{table}} \\
	\hline
	TITEL: & Charakter-Fraktion \\
	\hline
	BESCHREIBUNG: & Gibt an, zu welcher Fraktion der Charakter gehört. Mögliche Fraktionen sind Spieler1, Spieler2 oder NPC.\\
	\hline
	BEGRÜNDUNG: & Hält fest, von wem der Charakter zu steuern ist.\\
	\hline
	ABHÄNGIGKEITEN: & \todo[inline]{2.8.1 Wahlphase}\\
\end{tabularx}

\begin{tabularx}{16cm}{l|X}
	\refstepcounter{table}\label{BP und AP}
	\textbf{ID} & \textbf{FA-G \arabic{table}} \\
	\hline
	TITEL: & Bewegungspunkte (BP) und Aktionspunkte (AP) \\
	\hline
	BESCHREIBUNG: & Während eine Charakter am Zug ist, kann er BP für Bewegungen auf dem Spielfeld und AP für Aktionen einsetzen.
	Zu Beginn eines Zuges erhält er jeweils Punkte. Nach Beenden eines Zuges verfallen übrig gebliebene Punkte. BP und AP können nicht negativ sein.\\
	\hline
	BEGRÜNDUNG: & Hält fest, wie viele Bewegungen und Aktionen der Charakter in diesem Spielzug noch ausführen kann. \\
	\hline
	ABHÄNGIGKEITEN: & FA-G \ref{Flinkheit}, FA-G \ref{Schwerfaelligkeit}, FA-G \ref{Behaendigkeit}, FA-G \ref{Behaebigkeit}, FA-G \ref{Agilitaet} \todo[inline]{2.10.1 Züge, 2.5 Bewegung, 2.6 Aktionen}\\
\end{tabularx}

\begin{tabularx}{16cm}{l|X}
	\refstepcounter{table}\label{HP}
	\textbf{ID} & \textbf{FA-G \arabic{table}} \\
	\hline
	TITEL: & Health Points (HP) \\
	\hline
	BESCHREIBUNG: & HP geben Auskunft über den Gesundheitszustand des Charakters. Zu Spielbeginn werden die HP auf 100 gesetzt. Während des Spiels sorgen verschiedene Aktionen dafür, dass HP hinzugefügt oder abgezogen werden. Werden HP abgezogen, spricht man von Schaden. Die Punkte nehmen Werte zwischen 0 und 100 an.\\
	\hline
	BEGRÜNDUNG: & Zeigt Auswirkung verschiedener Aktionen auf den Charakter. \\
	\hline
	ABHÄNGIGKEITEN: & FA-G \ref{Robuster Magen}, FA-G \ref{Zaehigkeit} \todo[inline]{2.6 Aktionen, 2.4 Gadgets}\\
\end{tabularx}

\begin{tabularx}{16cm}{l|X}
	\refstepcounter{table}\label{Exfiltratation}
	\textbf{ID} & \textbf{FA-G \arabic{table}} \\
	\hline
	TITEL: & Exfiltration \\
	\hline
	BESCHREIBUNG: & Sinken die HP eines Charakter auf 0, so wird eine Exfiltration durchgeführt. Dabei wird der Charakter auf ein zufällig gewähltes freies Sitzplatz-Feld versetzt und seine HP auf 1 gesetzt. Ist kein freier Sitzplatz vorhanden, so wird ein Sitzplatz zufällig ausgewählt und die Person, die dort saß, wird auf ein zufälliges freies Nachbarfeld des Sitzplatzes platziert.\\
	\hline
	BEGRÜNDUNG: & HP sollen nicht 0 sein.\\
	\hline
	ABHÄNGIGKEITEN: & FA-G \ref{HP} \todo[inline]{2.12 Zufall und Alternativen}\\
\end{tabularx}

\begin{tabularx}{16cm}{l|X}
	\refstepcounter{table}\label{IP}
	\textbf{ID} & \textbf{FA-G \arabic{table}} \\
	\hline
	TITEL: & Intelligence Points (IP) \\
	\hline
	BESCHREIBUNG: & IP geben Auskunft, über die Spionagefähigkeiten des Charakter. Zu Beginn besitzen die IP den Wert 0.\\
	\hline
	BEGRÜNDUNG: & Punkte, die später für Sieg relevant sind.\\
	\hline
	ABHÄNGIGKEITEN: & \todo[inline]{2.4. Gaspatronen-Lippenstift, 2.4 Wanze und Ohrstöpsel, 2.4. Chicken Feed, 2.6 Roulette?, 2.7 Geheimnisse, Abhängigkeit von 2.11}\\
\end{tabularx}


\begin{tabularx}{16cm}{l|X}
	\refstepcounter{table}\label{Eigenschaften}
	\textbf{ID} & \textbf{FA-G \arabic{table}} \\
	\hline
	TITEL: & Eigenschaften \\
	\hline
	BESCHREIBUNG: & Sind entweder dauerhafte Fähigkeiten eines Charakters oder vorübergehende Zustände. Fähigkeiten kommen passiv zum Tragen oder ermöglichen dem Charakter bestimmte Aktionen. Zustände werden durch Aktionen erworben oder verloren.\\
	\hline
	BEGRÜNDUNG: & Fähigkeiten sorgen für individuelle Charaktere und Eigenschaften halten den aktuellen Zustand des Charakters fest. \\
	\hline
	ABHÄNGIGKEITEN: & FA-G \ref{Flinkheit}, FA-G \ref{Schwerfaelligkeit}, FA-G \ref{Behaendigkeit}, FA-G \ref{Behaebigkeit}, FA-G \ref{Agilitaet}, FA-G \ref{Glueckspilz}, FA-G \ref{Pechvogel}, FA-G \ref{Klamme Klamotten}, FA-G \ref{Konstant Klamme Klamotten}, FA-G \ref{Robuster Magen}, FA-G \ref{Zaehigkeit}, FA-G \ref{Babysitter}, FA-G \ref{Honey Trap}, FA-G \ref{Bang and Burn}, FA-G \ref{Flaps and Seals}, FA-G \ref{Tradecraft}, FA-G \ref{Observation} \\
\end{tabularx}

\begin{tabularx}{16cm}{l|X}
	\refstepcounter{table}\label{Inventar}
	\textbf{ID} & \textbf{FA-G \arabic{table}} \\
	\hline
	TITEL: & Inventar \\
	\hline
	BESCHREIBUNG: & Im Inventar sind alle Gadgets aufgelistet, die der Charakter aktuell bei sich trägt.\\
	\hline
	BEGRÜNDUNG: & Hält fest, welche Gadgets der Charakter nutzen kann.\\
	\hline
	ABHÄNGIGKEITEN: & \todo[inline]{2.4 Gadgets}\\
\end{tabularx}

\begin{tabularx}{16cm}{l|X}
	\refstepcounter{table}\label{haelt Cocktail}
	\textbf{ID} & \textbf{FA-G \arabic{table}} \\
	\hline
	TITEL: & hält Cocktail \\
	\hline
	BESCHREIBUNG: & Gibt an, ob der Charakter einen Cocktail in seiner Hand hält oder nicht. Ein Charakter kann maximal einen Cocktail in der Hand halten. Wird ein Cocktail in der Hand gehalten, so sind Aktionen mit diesem möglich. \\
	\hline
	BEGRÜNDUNG: & Hält fest, ob Interaktion mit einem Cocktail möglich ist oder nicht.\\
	\hline
	ABHÄNGIGKEITEN: & \todo[inline]{2.6 Cocktail}\\
\end{tabularx}

\begin{tabularx}{16cm}{l|X}
	\refstepcounter{table}\label{Spielchips}
	\textbf{ID} & \textbf{FA-G \arabic{table}} \\
	\hline
	TITEL: & Spielchips \\
	\hline
	BESCHREIBUNG: & Zu Beginn des Spiels besitzt jeder Charakter 10 Spielchips. Mit diesen kann er an Roulette-Tischen spielen und dadurch seine Anzahl an Spielchips erhöhen oder verringern.\\
	\hline
	BEGRÜNDUNG: & Werden benötigt, um Roulette zu spielen.\\
	\hline
	ABHÄNGIGKEITEN: & \todo[inline]{2.6 Roulette}\\
\end{tabularx}

\begin{tabularx}{16cm}{l|X}
	\refstepcounter{table}\label{Flinkheit}
	\textbf{ID} & \textbf{FA-G \arabic{table}} \\
	\hline
	TITEL: & Flinkheit \\
	\hline
	BESCHREIBUNG: & Besitzt ein Charakter die Fähigkeit Flinkheit, so erhält er in jeder Runde drei Bewegungspunkte.\\
	\hline
	BEGRÜNDUNG: & Unterschiedliche Anzahl an BP und AP je nach Charakter.\\
	\hline
	ABHÄNGIGKEITEN: & \\
\end{tabularx}

\begin{tabularx}{16cm}{l|X}
	\refstepcounter{table}\label{Schwerfaelligkeit}
	\textbf{ID} & \textbf{FA-G \arabic{table}} \\
	\hline
	TITEL: & Schwerfälligkeit \\
	\hline
	BESCHREIBUNG: & Besitzt ein Charakter die Fähigkeit Schwerfälligkeit, so erhält er in jeder Runde einen Bewegungspunkt.\\
	\hline
	BEGRÜNDUNG: & Unterschiedliche Anzahl an BP und AP je nach Charakter.\\
	\hline
	ABHÄNGIGKEITEN: & \\
\end{tabularx}

\begin{tabularx}{16cm}{l|X}
	\refstepcounter{table}\label{Behaendigkeit}
	\textbf{ID} & \textbf{FA-G \arabic{table}} \\
	\hline
	TITEL: & Behändigkeit \\
	\hline
	BESCHREIBUNG: & Besitzt ein Charakter die Fähigkeit Behändigkeit, so erhält er in jeder Runde zwei Aktionspunkte.\\
	\hline
	BEGRÜNDUNG: & Unterschiedliche Anzahl an BP und AP je nach Charakter.\\
	\hline
	ABHÄNGIGKEITEN: & \\
\end{tabularx}

\begin{tabularx}{16cm}{l|X}
	\refstepcounter{table}\label{Behaebigkeit}
	\textbf{ID} & \textbf{FA-G \arabic{table}} \\
	\hline
	TITEL: & Behäbigkeit \\
	\hline
	BESCHREIBUNG: & Besitzt ein Charakter die Fähigkeit Behäbigkeit, so wird ihm zu Beginn jeder Runde zufällig entweder ein Bewegungspunkt oder einer Aktionspunkt abgezogen.\\
	\hline
	BEGRÜNDUNG: & Unterschiedliche Anzahl an BP und AP je nach Charakter.\\
	\hline
	ABHÄNGIGKEITEN: & \todo[inline]{2.12 Zufall}\\
\end{tabularx}

\begin{tabularx}{16cm}{l|X}
	\refstepcounter{table}\label{Agilitaet}
	\textbf{ID} & \textbf{FA-G \arabic{table}} \\
	\hline
	TITEL: & Agilität \\
	\hline
	BESCHREIBUNG: & Besitzt ein Charakter die Fähigkeit Agilität, so wird ihm zu Beginn jeder Runde zufällig entweder ein Bewegungspunkt oder einer Aktionspunkt hinzugefügt.\\
	\hline
	BEGRÜNDUNG: & Unterschiedliche Anzahl an BP und AP je nach Charakter.\\
	\hline
	ABHÄNGIGKEITEN: & \todo[inline]{2.12 Zufall}\\
\end{tabularx}

\begin{tabularx}{16cm}{l|X}
	\refstepcounter{table}\label{Glueckspilz}
	\textbf{ID} & \textbf{FA-G \arabic{table}} \\
	\hline
	TITEL: & Glückspilz \\
	\hline
	BESCHREIBUNG: & Besitzt ein Charakter die Fähigkeit Glückspilz, so beträgt seine Gewinnchance beim Roulette $\frac{23}{37}$.\\
	\hline
	BEGRÜNDUNG: & Unterschiedliche Gewinnchancen beim Roulette je nach Charakter.\\
	\hline
	ABHÄNGIGKEITEN: & \todo[inline]{2.12 Zufall, Abhängigkeit von 2.6 Roulette}\\
\end{tabularx}

\begin{tabularx}{16cm}{l|X}
	\refstepcounter{table}\label{Pechvogel}
	\textbf{ID} & \textbf{FA-G \arabic{table}} \\
	\hline
	TITEL: & Pechvogel \\
	\hline
	BESCHREIBUNG: & Besitzt ein Charakter die Fähigkeit Pechvogel, so beträgt seine Gewinnchance beim Roulette $\frac{13}{37}$.\\
	\hline
	BEGRÜNDUNG: & Unterschiedliche Gewinnchancen beim Roulette je nach Charakter. \\
	\hline
	ABHÄNGIGKEITEN: & \todo[inline]{2.12 Zufall, Abhängigkeit von 2.6 Roulette}\\
\end{tabularx}

\begin{tabularx}{16cm}{l|X}
	\refstepcounter{table}\label{Klamme Klamotten}
	\textbf{ID} & \textbf{FA-G \arabic{table}} \\
	\hline
	TITEL: & Klamme Klamotten \\
	\hline
	BESCHREIBUNG: & Besitzt ein Charakter den Zustand Klamme Klamotten, so halbiert sich seine Erfolgswahrscheinlichkeit bei einer Wahrscheinlichkeitsprobe. \\
	\hline
	BEGRÜNDUNG: & Unterschiedliche Erfolgswahrscheinlichkeiten je nach Zustand des Charakters.\\
	\hline
	ABHÄNGIGKEITEN: & FA-G \ref{Konstant Klamme Klamotten} \todo[inline]{2.1 Kamin-Feld, 2.4 Föhn, 2.6 Cocktail, 2.12 Zufall}\\
\end{tabularx}

\begin{tabularx}{16cm}{l|X}
	\refstepcounter{table}\label{Konstant Klamme Klamotten}
	\textbf{ID} & \textbf{FA-G \arabic{table}} \\
	\hline
	TITEL: & Konstant Klamme Klamotten \\
	\hline
	BESCHREIBUNG: & Besitzt ein Charakter die Fähigkeit Konstant Klamme Klamotten, so hat er dauerhaft den Zustand Klamme Klamotten.\\
	\hline
	BEGRÜNDUNG: & Unterschiedliche Erfolgswahrscheinlichkeiten je nach Charakter.\\
	\hline
	ABHÄNGIGKEITEN: & \\
\end{tabularx}

\begin{tabularx}{16cm}{l|X}
	\refstepcounter{table}\label{Robuster Magen}
	\textbf{ID} & \textbf{FA-G \arabic{table}} \\
	\hline
	TITEL: & Robuster Magen \\
	\hline
	BESCHREIBUNG: & Besitzt ein Charakter die Fähigkeit Robuster Magen, so erhält er die doppelte Anzahl Cocktail-HP und nur den halben Gift-Cocktail-Schaden durch vergiftete Cocktails.\\
	\hline
	BEGRÜNDUNG: & Unterschiedliche Health Points je nach Charakter.\\
	\hline
	ABHÄNGIGKEITEN: & \todo[inline]{Cocktail-HP, Gift-Cocktail-Schaden}\\
\end{tabularx}

\begin{tabularx}{16cm}{l|X}
	\refstepcounter{table}\label{Zaehigkeit}
	\textbf{ID} & \textbf{FA-G \arabic{table}} \\
	\hline
	TITEL: & Zähigkeit \\
	\hline
	BESCHREIBUNG: & Besitzt ein Charakter die Fähigkeit Zähigkeit, so wird jeder Schaden, der nicht durch vergiftete Cocktails entsteht, um die Hälfte reduziert.\\
	\hline
	BEGRÜNDUNG: & Unterschiedliche Health Points je nach Charakter.\\
	\hline
	ABHÄNGIGKEITEN: & \\
\end{tabularx}

\begin{tabularx}{16cm}{l|X}
	\refstepcounter{table}\label{Babysitter}
	\textbf{ID} & \textbf{FA-G \arabic{table}} \\
	\hline
	TITEL: & Babysitter \\
	\hline
	BESCHREIBUNG: & Besitzt ein Charakter die Fähigkeit Babysitter, so wehrt er Angriffe auf benachbarte Charaktere der eigenen Fraktion mit der vorgegebenen Babysitter-Wahrscheinlichkeit ab. Bei einem Angriff wird zuerst die Wahrscheinlichkeitsprobe durch den Angreifer gemacht und wenn diese erfolgreich ist wird die Babysitter Fähigkeit eingesetzt. Ist der Charakter mit der Fähigkeit Babysitter erfolgreich, so misslingt der Angriff, ohne dass die gegnerische Fraktion etwas von der Fähigkeit erfährt, ansonsten wird der Angriff durchgeführt.\\
	\hline
	BEGRÜNDUNG: & Ermöglicht es Angriffen zu entgehen.\\
	\hline
	ABHÄNGIGKEITEN: & \todo[inline]{Babysitter-Wahrscheinlichkeit, Wahrscheinlichkeitsprobe, Abhängigkeit von 2.6 Aktion, 2.4 Gadgets}\\
\end{tabularx}

\begin{tabularx}{16cm}{l|X}
	\refstepcounter{table}\label{Honey Trap}
	\textbf{ID} & \textbf{FA-G \arabic{table}} \\
	\hline
	TITEL: & Honey Trap \\
	\hline
	BESCHREIBUNG: & Besitzt ein Charakter die Fähigkeit Honey Trap, so werden mit der vorgegebenen Honey-Trap-Wahrscheinlichkeit Angriffe anstatt auf diesen Charakter auf einen zufälligen anderen Charakter ausgeübt, insofern dieser andere Charakter ebenfalls Ziel der Aktion hätte sein können.\\
	\hline
	BEGRÜNDUNG: & Ermöglicht es Angriffen zu entgehen.\\
	\hline
	ABHÄNGIGKEITEN: & \todo[inline]{Honey-Trap-Wahrscheinlichkeit}\\
\end{tabularx}

\begin{tabularx}{16cm}{l|X}
	\refstepcounter{table}\label{Bang and Burn}
	\textbf{ID} & \textbf{FA-G \arabic{table}} \\
	\hline
	TITEL: & Bang and Burn \\
	\hline
	BESCHREIBUNG: & Besitzt ein Charakter die Fähigkeit Bang and Burn, so kann er einen benachbarten Roulette-Tisch unbrauchbar machen. Auf diesem kann dann nicht mehr gespielt werden.\\
	\hline
	BEGRÜNDUNG: & Verhindert das Sammeln von Spielchips, die später über den Sieg mitentscheiden.\\
	\hline
	ABHÄNGIGKEITEN: & \todo[inline]{Abhängigkeit von 2.1 Roulette-Tisch}\\
\end{tabularx}

\begin{tabularx}{16cm}{l|X}
\refstepcounter{table}\label{Flaps and Seals}
\textbf{ID} & \textbf{FA-G \arabic{table}} \\
\hline
TITEL: & Flaps and Seals \\
\hline
BESCHREIBUNG: & Besitzt ein Charakter die Fähigkeit Flaps and Seals, so kann er in einen Tresor spicken, der zwei Felder von ihm entfernt und somit nicht auf einem Nachbarfeld steht.\\
\hline
BEGRÜNDUNG: & Weniger Bewegungspunkte notwendig, um zu Tresor zu gelangen.\\
\hline
ABHÄNGIGKEITEN: & \todo[inline]{2.4 Maulwürfel, Abhängigkeit von 2.6 Tresor-Spicken}\\
\end{tabularx}


\begin{tabularx}{16cm}{l|X}
\refstepcounter{table}\label{Tradecraft}
\textbf{ID} & \textbf{FA-G \arabic{table}} \\
\hline
TITEL: & Tradecraft \\
\hline
BESCHREIBUNG: & Besitzt ein Charakter die Fähigkeit Tradecraft, so wiederholt er eine fehlgeschlagene Wahrscheinlichkeitsprobe einer Aktion einmal.\\
\hline
BEGRÜNDUNG: & Höhere Chance für eine erfolgreiche Aktion.\\
\hline
ABHÄNGIGKEITEN: & \todo[inline]{Wahrscheinlichkeitsprobe, 2.4 Maulwuerfel}\\
\end{tabularx}

\begin{tabularx}{16cm}{l|X}
\refstepcounter{table}\label{Observation}
\textbf{ID} & \textbf{FA-G \arabic{table}} \\
\hline
TITEL: & Observation \\
\hline
BESCHREIBUNG: & Besitzt ein Charakter die Fähigkeit Observation, so kann er diese als Aktion gegen einen anderen Charakter in Sichtlinie ausführen. Dabei wird mit der vorgegebenen Observation-Erfolg-Wahrscheinlichkeit aufgedeckt, ob der Charakter zur gegnerischen Fraktion gehört oder nicht. Die Aktion bleibt vom observierten Charakter unbemerkt. \\
\hline
BEGRÜNDUNG: & Möglichkeit um herauszufinden, welche Charaktere zur gegnerischen Fraktion gehören und welche NPCs sind.\\
\hline
ABHÄNGIGKEITEN: & \todo[inline]{Observation-Erfolg-Wahrscheinlichkeit, 2.4 Maulwuerfel}\\
\end{tabularx}

\begin{tabularx}{16cm}{l|X}
	\refstepcounter{table}\label{Gadgets}
	\textbf{ID} & \textbf{FA-G \arabic{table}} \\
	\hline
	TITEL: & Gadgets \\
	\hline
	BESCHREIBUNG: & Charaktere können Gadgets in ihrem Inventar bei sich tragen. Sie verschaffen ihrem Besitzer bestimmte Eigenschaften oder ermöglichen bestimmte Aktionen. Jedes Gadget kommt höchstens einmal im Spiel vor und kann, falls nicht anders festgelegt, mehrmals verwendet werden. \\
	\hline
	BEGRÜNDUNG: & Dadurch wird das Spiel spannender, denn die Gadgets beeinflussen den Spielverlauf. \\
	\hline
	ABHÄNGIGKEITEN: & FA-G \ref{Charakter}, FA-G \ref{Inventar}, FA-G \ref{Eigenschaften}, FA-G \ref{Aktion durchführen} \\
\end{tabularx}

\begin{tabularx}{16cm}{l|X}
	\refstepcounter{table}\label{Akku-Foehn}
	\textbf{ID} & \textbf{FA-G \arabic{table}} \\
	\hline
	TITEL: & Akku-Föhn \\
	\hline
	BESCHREIBUNG: & Wird der Akku-Föhn von einem Charakter als Aktion an sich selbst oder einem benachbarten Charakter angewendet, so verliert dieser Ziel-Charakter die Klamme Klamotten-Eigenschaft. Der Akku-Föhn kann beliebig oft eingesetzt werden. \\
	\hline
	BEGRÜNDUNG: & Das ermöglicht es den Charakteren die Klamme Klamotten-Eigenschaft loszuwerden.  \\
	\hline
	ABHÄNGIGKEITEN: & FA-G \ref{Gadgets}, FA-G \ref{Charakter}, FA-G \ref{Aktion durchführen} \\
\end{tabularx}

\begin{tabularx}{16cm}{l|X}
	\refstepcounter{table}\label{Maulwuerfel}
	\textbf{ID} & \textbf{FA-G \arabic{table}} \\
	\hline
	TITEL: & Maulwürfel \\
	\hline
	BESCHREIBUNG: & Hat ein Charakter den Maulwürfel im Inventar, so werden die Eigenschaften Tradecraft, Flaps and Seals und Observation deaktiviert, falls er sie hat. Er verliert diese Eigenschaften vorübergehend und bekommt sie wieder, wenn er den Maulwürfel losgeworden ist. Mit einer Aktion kann ein Charakter den Maulwürfel auf ein beliebiges Nicht-Wand-Feld in Sichtweite und Maulwürfel-Wurfweite werfen. Befindet sich auf diesem Zielfeld ein Charakter, nimmt dieser den Maulwürfel in sein Inventar auf. Ansonsten prallt der Maulwürfel von dem Zielfeld ab und landet im Inventar des sich am nächsten befindenden Charakters. Falls der Maulwürfel im Inventar eines NPC ist, macht dieser bei seinem nächsten Zug einen Maulwürfelwurf auf ein zufälliges Zielfeld. \\
	\hline
	BEGRÜNDUNG: &  Der Maulwürfel deaktiviert bei dem Charakter, der ihn im Inventar hat positive Eigenschaften und bringt dadurch Spannung in das Spiel. \\
	\hline
	ABHÄNGIGKEITEN: & FA-G \ref{Gadgets}, FA-G \ref{Charakter}, FA-G \ref{Inventar},FA-G \ref{Eigenschaften}, \todo[inline]{Maulwürfel-Wurfweite} \\
\end{tabularx}

\begin{tabularx}{16cm}{l|X}
	\refstepcounter{table}\label{Technicolor-Prisma}
	\textbf{ID} & \textbf{FA-G \arabic{table}} \\
	\hline
	TITEL: & Technicolor-Prisma \\
	\hline
	BESCHREIBUNG: & Durch eine Aktion kann das Technicolor-Prisma an einem Roulette-Tisch installiert werden und vertauscht dann die Farben rot und schwarz. Dadurch wird das Resultat beim Roulette-Spielen negiert, d.h. ein Charakter verliert, wenn er normalerweise gewonnen hätte und umgekehrt. Damit sind die Erfolgswahrscheinlichkeiten für Glückspilze und Pechvögel beim Roulette-Spielen vertauscht. Ein Technicolor-Prisma kann nur einmal verwendet werden. \\
	\hline
	BEGRÜNDUNG: &  Durch das Technicolor-Prisma wird das Roulette-Spielen beeinflusst. \\
	\hline
	ABHÄNGIGKEITEN: & FA-G \ref{Gadgets}, FA-G \ref{Roulette spielen}, \\
\end{tabularx}

\begin{tabularx}{16cm}{l|X}
	\refstepcounter{table}\label{Charakter Liste}
	\textbf{ID} & \textbf{FA-G \arabic{table}} \\
	\hline
	TITEL: & Charakter Liste \\
	\hline
	BESCHREIBUNG: & Eine Liste, die alle spielbaren Charaktere mit ihren Fähigkeiten beinhaltet. Die Charaktere, die im Lastenheft unter \glqq A Einige Vorschläge für Charaktere\grqq aufgeführt sind, sollen mindestens in der Liste enthalten sein.\\
	\hline
	BEGRÜNDUNG: & Aus dieser Liste können Charaktere zum Spielen gewählt werden.\\
	\hline
	ABHÄNGIGKEITEN: & FA-G \ref{Charakter} \todo[inline]{2.8.1 Wahlphase}\\
\end{tabularx}

\begin{tabularx}{16cm}{l|X}
	\refstepcounter{table}\label{Bewegung durchfuehren}
	\textbf{ID} & \textbf{FA-G\arabic{table}} \\
	\hline
	TITEL: & Bewegung durchführen \\
	\hline
	BESCHREIBUNG: & Innerhalb eines Zuges kann ein Charakter, der noch mehr als einen Bewegungspunkt besitzt, eine Bewegung durchführen. Das bedeutet, dass der Charakter sich von dem Spielfeld seiner aktuellen Position auf ein angrenzendes betretbares Spielfeld bewegen kann.
	Jedes Spielfeld, welches mit einer Seite oder einer Ecke das Spielfeld der aktuellen Position berührt, ist ein angrenzendes Spielfeld. Dementsprechend sind Bewegungen in horizontaler, vertikaler und diagonaler Linie möglich.
	Wenn der Client die Anweisung gibt, den Charakter auf ein angrenzendes Feld zu bewegen, welches nicht betretbar ist, so darf der Charakter sich danach nicht auf diesem Feld befinden, sondern muss auf dem Spielfeld der aktuellen Position bleiben. In diesem Fall wurde die  Bewegung nicht erfolgreich durchgeführt. In der graphischen Darstellung kann dem Benutzer mit einer Animation angezeigt werden, dass die Bewegung nicht erfolgreich war bzw. dass es nicht möglich ist, das gewählte Spielfeld zu betreten.
 Bei erfolgreicher Durchführung einer Bewegung muss dem Charakter am Zug die Anzahl der Bewegungspunkte um 1 reduziert werden. Bei nicht erfolgreicher Durchführung darf sich die Anzahl der Bewegungspunkte  nicht verändern.\\
	\hline
	BEGRÜNDUNG: & Die Bedingungen für das Bewegen eines Charakters müssen eindeutig definiert sein.\\
	\hline
	ABHÄNGIGKEITEN: & FA-G\ref{Charakter} \todo[inline]{2.8.1 Wahlphase}\\
\end{tabularx}

\begin{tabularx}{16cm}{l|X}
	\refstepcounter{table}\label{Draengeln}
	\textbf{ID} & \textbf{FA-G\arabic{table}} \\
	\hline
	TITEL: & Drängeln \\
	\hline
	BESCHREIBUNG: & Wenn ein Charakter sich auf ein Feld bewegt, auf dem bereits ein anderer Charakter 		befindet, so tauschen die beiden Charaktere Plätze. Das heißt, der Charakter der die Bewegung durchgeführt hat steht auf dem Feld, auf das er sich bewegen wollte. Dies ist das Feld, auf dem der andere Charakter vor der Bewegung stand. Der andere Charakter befindet sich, nach dem Durchführen der Bewegung, auf dem Feld, auf dem der aktive Charakter ursprünglich stand. \\
	\hline
	BEGRÜNDUNG: & Um das Verhalten im Fall der Bewegung auf ein Spielfeld, welches mit einem anderen 			Charakter besetzt ist, eindeutig zu definieren.\\
	\hline
	ABHÄNGIGKEITEN: & FA-G\ref{Bewegung durchfuehren} \todo[inline]{2.8.1 Wahlphase}\\
\end{tabularx}

\begin{tabularx}{16cm}{l|X}
	\refstepcounter{table}\label{Aktion durchfuehren}
	\textbf{ID} & \textbf{FA-G\arabic{table}} \\
	\hline
	TITEL: & Aktion durchführen \\
	\hline
	BESCHREIBUNG: & Innerhalb eines Zuges kann ein Charakter, der noch mehr als einen Aktionspunkt hat und die Ausführungsbedingungen einer spezifischen Aktion erfüllt, diese spezifische Aktion ausführen.
	 Wenn der Client eine gültigen Aktionsbefehl eingibt, so gilt die Ausführung der Aktion als erfolgreich. Bei der Eingabe eines ungültigen Aktionsbefehls gilt die Ausführung der Aktions als nicht erfolgreich.
	 Bei erfolgreicher Ausführung der Aktion muss dem Charakter am Zug die Anzahl der Aktionspunkte um 1 reduziert werden und die Konsequenzen der spezifischen Aktion müssen den Zustand des Spiels entsprechend ändern. Bei nicht erfolgreicher Ausführung darf sich die Anzahl der Aktionspunkte und der Zustand des Spiels nicht ändern.  \\
	\hline
	BEGRÜNDUNG: & Es muss genau definiert sein, unter welchen Bedingungen das Ausführen einer Aktion möglich ist und welche Konsequenzen die Aktion hat.\\
	\hline
	ABHÄNGIGKEITEN: & FA-G\ref{Bewegung durchfuehren} \todo[inline]{2.8.1 Wahlphase}\\
\end{tabularx}


\begin{tabularx}{16cm}{l|X}
	\refstepcounter{table}\label{Gadget verwenden}
	\textbf{ID} & \textbf{FA-G\arabic{table}} \\
	\hline
	TITEL: & Gadget verwenden \\
	\hline
	BESCHREIBUNG: & Wenn sich ein verwendbares Gadget im Inventar eines Charakters befindet, dieser Charakter am Zug ist und die Ausführungsbedingungen des Gadgets für diesen Charakter erfüllt sind, so kann der Charakter als Aktion das Gadget verwenden. \\
	 
	\hline
	BEGRÜNDUNG: & Das Verwenden von Gadgets stellt eine Aktion dar.\\
	\hline
	ABHÄNGIGKEITEN: & FA-G\ref{Aktion durchfuehren}  \todo[inline]{2.8.1 Wahlphase}\\
\end{tabularx}

\begin{tabularx}{16cm}{l|X}
	\refstepcounter{table}\label{Roulette spielen}
	\textbf{ID} & \textbf{FA-G\arabic{table}} \\
	\hline
	TITEL: & Roulette spielen \\
	\hline
	BESCHREIBUNG: & Ein Charakter, der sich innerhalb eines Zuges auf einem Feld befindet, welches an ein Feld mit Roulette-Tisch angrenzt, der kann als Aktion einmal Roulette spielen. Der Client muss dazu einen Betrag an Spielchips eingeben. Dieser Betrag hat den Mindestwert 1 und einen Höchstwert. Der Höchstwert ist der kleinere der beiden folgenden Werte: die Anzahl der Spielchips, welche der Charakter im Moment besitzt; die Anzahl der Spielchips, die am Tisch verfügbar sind. Die eingegebene Betrag stellt den Einsatz des Charakters dar. Wenn der Charakter gewinnt, dann wird der Spielchips-Wert des Charakters um den Einsatz-Wert erhöht und der Spielchips-Wert des Roulette-Tisches um den Einsatz-Wert verringert. Wenn der Charakter verliert, dann wird der Spielchips-Wert des Charakters um den Einsatz-Wert reduziert und der Spielchipswert des Roulette-Tisches um den Einsatz-Wert erhöht. Ob ein Charakter gewinnt wird durch die Roulette-Gewinn-Wahrscheinlichkeit bestimmt.\\
	 
	\hline
	BEGRÜNDUNG: & Aufgrund der Spielmechanik ist Roulette spielen eine Aktion. Das Verhalten beim Ausführen dieser Aktion muss eindeutig definiert sein.\\
	\hline
	ABHÄNGIGKEITEN: & FA-G\ref{Aktion durchfuehren} \todo[inline]{2.8.1 Wahlphase}\\
\end{tabularx}

\begin{tabularx}{16cm}{l|X}
	\refstepcounter{table}\label{Cocktail aufnehmen}
	\textbf{ID} & \textbf{FA-G\arabic{table}} \\
	\hline
	TITEL: & Cocktail aufnehmen \\
	\hline
	BESCHREIBUNG: & Wenn ein Charakter am Zug ist, noch mehr als 0 Aktionpunkte hat, sich auf einem Feld befindet, welches an einen Bar-Tisch angrenzt, sich auf diesem Bartisch zu diesem Zeitpunkt ein Cocktail befindet und er selbst im Moment keinen Cocktail in der Hand hält, so kann der Charakter als Aktion diesen Cocktail aufnehmen. Danach befindet sich der Cocktail in der Hand des Charakters. Diese Information muss für alle Clients sichtbar sein.\\
	 
	\hline
	BEGRÜNDUNG: & Aufgrund der Spielmechanik ist das Aufnehmen eines Cocktails eine Aktion. Das Verhalten beim Ausführen dieser Aktion muss eindeutig definiert sein.\\
	\hline
	ABHÄNGIGKEITEN: & FA-G\ref{Aktion durchfuehren}  \todo[inline]{2.8.1 Wahlphase}\\
\end{tabularx}


\clearpage
\section{Softwarespezifikation}
\subsection{Schnittstellenarten und Dialogstruktur}
\subsubsection{Schnittstellenbeschreibung des Clients}
Für den Client wird eine graphische Benutzerschnittstelle gewählt.
Eine solche bietet dem Benutzer eine einfache Bedienung und ermöglicht es komplexe Zusammenhänge einfach für den Spieler darzustellen.
Des weiteren bietet eine graphische Benutzerschnittstelle mehr Möglichkeiten, wie der Spieler mit dem Spiel interagieren kann als eine Kommanozeilenanwendung.\\

Im folgenden werden die einzelnen Dialoge und Popups aufgelistet und welche Anwendungsfälle diese jeweils abdecken

\begin{itemize}
	\item Dialog Hauptmenü: Hauptmenü anzeigen, Anwendung beenden, zur Lobbyübersicht wechseln, Einstellungen, Hilfe anzeigen
	\item Dialog Einstellungen: Vornehmen von Einstellungen
	\item Popup Hilfe: Anzeigen von Hilfsfunktionen
	\item Popup Fehler bei der Verbindung zum Server: Übergang zur Lobbyübersicht, Fehlerfall
	\item Dialog Lobbyübersicht: Lobbyübersicht anzeigen, Lobbyübersicht verlassen, Lobby erstellen, Lobby beitreten, Nutzernamen festlegen
	\item Popup Nutzernamen ändern: Nutzernamen festlegen
	\item Popup Lobbyname eingeben und Konfig. erstellen: Konfigurationsdaten festlegen
	\item Dialog Editor: erstellen von Konfigurationsdaten
	\item Dialog Lobby: Lobby anzeigen
	\item Popup Rolle ändern: Rolle wechseln
	\item Popup Konfiguration anzeigen: anzeigen der Konfiguration
	\item Dialog Wahlphase: Wahlphase anzeigen, Character/Gadget wählen
	\item Dialog Ausrüstungsphase: Ausrüstungsphase anzeigen, Gadget zuweisen
	\item Dialog Spielbildschirm: Spielbildschirm anzeigen, Spielstand darstellen, Spielaktion durchführen, Spiel verlassen
	\item Popup Optionen: Einstellungen, Spiel pausieren
	\item Popup Einstellungen: Einstellungen vornehmen
	\item Popup Pausieren: Spiel pausieren und entpausieren
	\item Dialog Gewinnerbildschirm: Gewinnerbildschirm anzeigen, Gewinner zeigen, Statistiken anzeigen
\end{itemize}

Im Spielbildschirm wurden alle möglichen Aktionen die ein Spieler durchführen kann aus Übersichtlichkeitsgründen zu \textit{Spielaktion durchführen} zusammengefasst. Dazu gehören Charakter anzeigen, Feld anzeigen, Charakter bewegen, Gadget verwenden, etc.
\begin{figure}
  \centering
  \includegraphics[width=\textwidth]{Meilenstein03/client_dialog.pdf}
  \caption{Dialog Diagram für den Client}
\end{figure}

\subsubsection{Schnittstellenbeschreibung des Servers}
Der Server wird über die Kommandozeile in einem Docker-Container gestartet (QA2).\\
Mit den Clients (jeglicher Art) kommuniziert er über eine Websocket-Verbindung (FA-S \ref{s-websockets}), über welche Nachrichten im JSON-Format ausgetauscht werden (FA-S \ref{s-json-encoding}, FA-S \ref{s-json-decoding}).
\subsubsection{Schnittstellenbeschreibung des Editors}
Erstellt ein Mensch-Client über die Lobbyübersicht eine neue Lobby, so muss er die Konfiguration des Editors über eine graphische Oberfläche durchführen (FA-E \ref{e-gui}, FA-E \ref{e-szenarioedit}, FA-E \ref{e-partieedit}, FA-E \ref{e-charedit}).

TODO: Dialogstrukturdiagramm (new Lobby -> Dialog Editor -> Szenario -> Partiekonfiguration(Zeit für Phasen, Wahrscheinlichkeiten) -> CharakterBeschreibung -> back / new Lobby) einfache Benutzung QA10

\mbox{}\\
Um die Konfiguration, die im Editor vorgenommen wurde, über die Websocket-Schnittstelle des Mensch-Clients mit dem Server zu kommunizieren, verfügt der Editor über eine JSON-Schnittstelle (FA-E \ref{e-json-encoding}, FA-E \ref{e-json-decoding}).
\subsubsection{Schnittstellenbeschreibung des KI-Clients}
Die KI kommuniziert mit dem Server über eine Websocket Verbindung (FA-KI \ref{ki-session}, FA-KI \ref{ki-register}).\\
Mit dem Mensch-Client interagiert die KI über eine graphische Oberfläche, die der Mensch-Client implementiert. Dabei hat der Mensch-Client in der Lobby die Möglichkeit über einen Button mit der Aufschrift "add KI" eine KI als gegnerischen Spieler hinzuzufügen (FA-KI \ref{ki-config}, FA-KI \ref{ki-intelligenz}).

TODO: Dialogstrukturdiagramm (add KI -> Dialog KonfigKI -> back / add KI) einfache Benutzung QA10

Während des Spiels können über die API der KI Tipps für den nächsten Spielzug angezeigt werden (FA-KI \ref{ki-api}).\\
Zudem besitzt die KI ein Kommandozeileninterface, um diese unabhängig von einem Mensch-Client konfigurieren und einer Lobby hinzufügen zu können (FA-KI \ref{ki-cli}).

\subsection{Graphische Gestaltung und Nutzungskonzept}
Dialog \glqq{}Hauptmenü\grqq{}

Im Hauptmenü kann der Benutzer über den Spielen-Button in die Lobby-Übersicht gelangen, wenn die Verbindung erfolgreich aufgebaut werden konnte. Falls nicht, erscheint eine Fehlermeldung als Popup. Mit dem Button Einstellungen kommt der Benutzer zu dem Einstellungen-Dialog. Durch Klick auf den Hilfe-Button öffnet sich ein Popup, in dem dem Nutzer hilfreiche Informationen zur Verfügung gestellt werden. Mit dem Beenden-Button kann der Benutzer die Anwendung verlassen und schließen.

\begin{figure}
  \centering
  \includegraphics[width=\textwidth]{Meilenstein03/Hauptmenue_Mockup.png}
  \caption{Mockup für das Hauptmenü}
\end{figure}

Dialog \glqq{}Einstellungen\grqq{}

In den Einstellungen kann man diverse Einstellungen vornehmen. Über den Button Hauptmenü gelangt man zurück ins Hauptmenü.

\begin{figure}
  \centering
  \includegraphics[width=\textwidth]{Meilenstein03/Einstellungen_Mockup.png}
  \caption{Mockup für die Einstellungen}
\end{figure}

Popup \glqq{}Hilfe-Hauptmenü\grqq{}

In dem Hilfe-Popup werden dem Benutzer alle möglichen Interaktionen mit dem entsprechenden Dialog aufgezeigt.

\begin{figure}
  \centering
  \includegraphics[width=\textwidth]{Meilenstein03/Hilfe-Hauptmenue_Mockup.png}
  \caption{Mockup für das Hilfe-Hauptmenü-Popup}
\end{figure}

Popup \glqq{}Fehler bei der Verbindung zum Server\grqq{}

Falls die Verbindung zum Server fehlgeschlagen ist, wird dem Benutzer ein Popup angezeigt, das ihm diese Information mitteilt.

\begin{figure}
  \centering
  \includegraphics[width=\textwidth]{Meilenstein03/FehlerBeiDerVerbindungZumServer_Mockup.png}
  \caption{Mockup für das Fehler bei der Verbindung zum Server-Popup}
\end{figure}

Dialog \glqq{}Lobby-Übersicht\grqq{}

In der Lobby-Übersicht werden dem Benutzer alle vorhandenen Lobbys und die Anzahl der Spieler und Zuschauer, die sich in ihr befinden angezeigt. Durch Klicken auf Beitreten kann der Benutzer einer bestehenden Lobby beitreten. Durch Klicken auf den Verlassen-Button gelangt man zum Hauptmenü. Der Aktualisieren-Button aktualisiert die Lobby-Übersicht. Mit dem Button Namen ändern, kann man den Namen einer Lobby verändern. Mit dem Button Lobby erstellen kann man eine neue Lobby erstellen.

\begin{figure}
  \centering
  \includegraphics[width=\textwidth]{Meilenstein03/Lobby-Uebersicht_Mockup.png}
  \caption{Mockup für die Lobby-Übersicht}
\end{figure}

Popup \glqq{}Namen ändern\grqq{}

In diesem Popup kann der Benutzer den Namen einer Lobby nachträglich verändern. Durch den Schließen-Button verschwindet das Popup-Fenster und der benutzer gelangt zurück in die Lobby-Übersicht.

\begin{figure}
  \centering
  \includegraphics[width=\textwidth]{Meilenstein03/NamenAendern_Mockup.png}
  \caption{Mockup für das Namen ändern-Popup}
\end{figure}

Popup \glqq{}Lobbynamen eingeben und Konfig. erstellen\grqq{}

In diesem Popup kann der Benutzer eine neue Lobby erstellen. Dazu muss er den Lobbynamen und eine Konfiguration festlegen. Durch Klicken auf den Button Neue Konfig. kann der Benutzer eine Konfiguration auswählen oder eine neue erstellen. Durch Klicken auf den Bestätigen-Button, kommt der Benutzer zurück in die Lobbyübersicht.

\begin{figure}
  \centering
  \includegraphics[width=\textwidth]{Meilenstein03/LobbynamenEingebenUndKonfigErstellen_Mockup.png}
  \caption{Mockup für das Lobbynamen eingeben und Konfig. erstellen-Popup}
\end{figure}

Dialog \glqq{}Editor\grqq{}

Mit der Schaltfläche \glqq{}Schließen\grqq{} wechselt man zum Dialog \glqq{}Lobby-Übersicht\grqq{}.
Mit der Schaltfläche \glqq{}Hilfe\grqq{} öffnet sich ein Hilfe-Popup, in dem dem Benutzer Informationen zu möglichen Interaktionen angezeigt werden.
In den drei Dropdown-Menüs kann der Benutzer das gewünschte Szenario, die gewünschte Charakterliste und die gewünschte Partie-Konfiguration auswählen. Wenn man in einem Dropdown-Menü den Eintrag 'Neue(s) Szenario/Charakterliste/Partie-Konfiguration erstellen' auswählt, dann wechselt die Ansicht zum Dialog \glqq{}Szenario-/Charakter-/Partie-Editor\grqq{} und eine neue Datei des entsprechenden Typs wird erstellt und zum editieren geöffnet.
Wenn man in einem der Dropdown-Menüs einen Eintrag auswählt, dann wird eine Schaltfläche mit der Aufschrift \glqq{}bearbeiten\grqq{} neben dem Eintrag angezeigt. Mit dieser Schaltfläche wechselt man ebenfalls in den entsprechenden Editor und die Datei des Eintrags wird geöffnet zum editieren.

\begin{figure}
  \centering
  \includegraphics[width=\textwidth]{Meilenstein03/Editor_Mockup.png}
  \caption{Mockup für den Editor}
\end{figure}

Popup \glqq{}Hilfe-Editor\grqq{}

In diesem Popup wird dem Benutzer mitgeteilt, welche Interaktionen er mit dem Editor-View eingehen kann. Über den Schließen-Button wird das Hilfe-Popup-Fenster geschlossen und der benutzer kehrt zum Editor zurück.

\begin{figure}
  \centering
  \includegraphics[width=\textwidth]{Meilenstein03/Hilfe-Editor_Mockup.png}
  \caption{Mockup für das Hilfe-Editor-Popup}
\end{figure}

Dialog \glqq{}Lobby\grqq{}

Die Liste mit den verbundenen Clients wird kontinuierlich sortiert, sodass die Spieler immer an oberster Stelle stehen. Die Spieler und die Zuschauer werden untereinander chronologisch nach Beitritt zur Lobby bzw. dem letzten Rollenwechsel sortiert, sodass der Client, der als letzter als Zuschauer der Lobby beigetreten ist bzw. als letzter innerhalb der Lobby die Rolle zu Zuschauer geändert hat, den untersten Eintrag in der Liste hat. Die KI-Clients sind am Benutzernamen erkennbar. 
Mit der Schaltfläche \glqq{}Rolle wechseln\grqq{} ändert sich die eigene Rolle von 'Spieler' zu 'Zuschauer' und umgekehrt. 
Mit der Schaltfläche \glqq{}KI hinzufügen\grqq{} wechselt der Client in den Dialog \glqq{}KI-Konfiguration\grqq{}.
Mit der Schaltfläche \glqq{}Lobby verlassen\grqq{} wechselt der Client in den Dialog \glqq{}Lobby-Übersicht\grqq{}.
Die Schaltfläche \glqq{}Spiel starten\grqq{} kann nur dann von einem Client gedrückt werden, wenn dieser Client die Rolle 'Spieler' hat. Sobald sie gedrückt wurde, wird eine Spielpartie gestartet und die Ansicht wird zum Dialog \glqq{}Spielfeld\grqq{} gewechselt.

\begin{figure}
  \centering
  \includegraphics[width=\textwidth]{Meilenstein03/Lobby_Mockup.png}
  \caption{Mockup für das Lobby-View}
\end{figure}

Popup \glqq{}Rolle wechseln Lobby\grqq{}

In diesem Popup wird der Benutzer gefragt, ob er sich sicher ist, dass er seine Rolle wechseln will. Klickt er auf Ja, wechselt er seine Rolle und kehrt zur Lobby zurück, klickt er auf Abbrechen, wechselt er seine Rolle nicht und kehrt auch zur Lobby zurück.

\begin{figure}
  \centering
  \includegraphics[width=\textwidth]{Meilenstein03/RolleWechseln-Lobby_Mockup.png}
  \caption{Mockup für das Rolle wechseln-Lobby-Popup}
\end{figure}

Popup \glqq{}Konfiguration anzeigen Lobby\grqq{}

In diesem Popup werden dem Benutzer die ausgewählten und für die Partie gültigen Konfigurationsdateien angezeigt. Über den Schließen-Button wird das Popup-Fenster geschlossen und der Benutzer kehrt zum Lobby-View zurück.

\begin{figure}
  \centering
  \includegraphics[width=\textwidth]{Meilenstein03/Konfiguration-Lobby_Mockup.png}
  \caption{Mockup für das Konfiguration-Lobby-Popup}
\end{figure}

Dialog \glqq{}Szenario-Editor\grqq{}

Alle quadratischen Felder des Spielbretts sind mit Mausklick auswählbar. Die Schaltflächen in Form eines Pluszeichens, die den Rand des Spielbretts säumen erweitern das Spielbrett. Wenn eine Plus-Schaltfläche geklickt wird, dann wird sie ersetzt mit einem neuen Feld und an dessen äußeren Rändern, dem neuen äußeren Rand des Spielbretts erscheinen neue Plus-Schaltflächen. Alle Felder des Spielbretts sind zu Beginn freie Felder. Oberhalb des Spielbretts befinden sich die Icons der verschiedenen Feldarten. Diese lassen sich per Drag-and-Drop auf freie Felder des Spielbretts ziehen. Ist die Szenario-Konfiguration abgeschlossen, kehrt der Benutzer über den Speichern-Button zum Editor zurück. Bei Klicken auf den Speichern-Button erscheint ein Popup-Fenster, in dem der Benutzer einen Namen für die Konfiguration festlegen muss. Über den Abbrechen-Button kehrt der Benutzer zurück zum Editor-View. Klickt der Benutzer auf den Abbrechen-Button, erscheint eine Warnmeldung, die den Benutzer darauf hinweist, dass nicht gespeicherte Änderungen verworfen werden.

\begin{figure}
  \centering
  \includegraphics[width=\textwidth]{Meilenstein03/Szenario-Editor_Mockup.png}
  \caption{Mockup für den Szenario-Editor}
\end{figure}

Popup \glqq{}Speichern-Szenario-Editor\grqq{}

Wenn der Benutzer seine Konfigurationsdatei speichern möchte, muss er hier einen Dateinamen festlegen. Über Abbrechen gelangt er zurück in den Szenario-Editor und über Speichern gelangt der Benutzer zurück zum Editor.

\begin{figure}
  \centering
  \includegraphics[width=\textwidth]{Meilenstein03/Speichern-Szenario-Editor_Mockup.png}
  \caption{Mockup für das Speichern-Szenario-Editor-Popup}
\end{figure}

Popup \glqq{}Änderungen verwerfen-Szenario-Editor\grqq{}

Wenn der Benutzer die Szenario-Konfiguration abbrechen möchte, muss er in diesem Popup bestätigen, dass er sich sicher ist, alle Änderungen zu verwerfen.

\begin{figure}
  \centering
  \includegraphics[width=\textwidth]{Meilenstein03/AenderungenVerwerfen-Szenario-Editor_Mockup.png}
  \caption{Mockup für das Änderungen verwerfen-Szenario-Editor-Popup}
\end{figure}

Dialog \glqq{}Partie-Editor\grqq{}

Mit der Schaltfläche \glqq{}Abbrechen\grqq{} wechselt man zum Editor zurück. Mit der Schaltfläche \glqq{}Speichern\grqq{} wird die Datei gespeichert und die Ansicht wechselt zum Editor. Dabei öffnet sich ein Popup-Fenster, in welchem ein Name für die datei festgelegt werden muss. Über Abbrechen gelangt der Benutzer wieder zum Editor und alle Änderungen werden verworfen. Dabei öffnet sich ein Popup-Fenster, in welchem der benutzer bestätigen muss, dass er die datei nicht speichern möchte. Mit direkter Eingabe in das Feld der Anzeige des Werts einer Wahrscheinlichkeit oder eines anderen Werts, dem Schieberegler daneben oder den Inkrement- und Dekrementschaltflächen kann der Wert einer Wahrscheinlichkeit in Prozent oder ein Schadens-, Reichweiten-, etc. -wert festgelegt und angepasst werden.

\begin{figure}
  \centering
  \includegraphics[width=\textwidth]{Meilenstein03/Partie-Editor_Mockup.png}
  \caption{Mockup für den Partie-Editor}
\end{figure}

Popup \glqq{}Speichern-Partie-Editor\grqq{}

In diesem Popup muss der Benutzer einen Dateinamen für seine Konfiguration festlegen. Mit Speichern kommt er zurück zum Editor und über Abbrechen wird das Popup-Fenster geschlossen und er gelangt wieder zum Partie-Editor.

\begin{figure}
  \centering
  \includegraphics[width=\textwidth]{Meilenstein03/Speichern-Partie-Editor_Mockup.png}
  \caption{Mockup für das Speichern-Partie-Editor-Popup}
\end{figure}

Popup \glqq{}Änderungen verwerfen-Partie-Editor\grqq{}

Wenn der Benutzer die Partie-Konfiguration abbrechen möchte, muss er in diesem Popup bestätigen, dass er sich sicher ist, alle Änderungen zu verwerfen.

\begin{figure}
  \centering
  \includegraphics[width=\textwidth]{Meilenstein03/AenderungenVerwerfen-Partie-Editor_Mockup.png}
  \caption{Mockup für das Änderungen verwerfen-Partie-Editor-Popup}
\end{figure}

Dialog \glqq{}Charakter-Editor\grqq{}

Vorlage muss noch bearbeitet werden.

\begin{figure}
  \centering
  \includegraphics[width=\textwidth]{Meilenstein03/Charakter-Editor_Mockup.png}
  \caption{Mockup für den Charakter-Editor}
\end{figure}

Dialog \glqq{}KI-Konfiguration\grqq{}

Mit der Schaltfläche \glqq{}Schließen\grqq{} wechselt man zum Dialog \glqq{}Lobby\grqq{}, ohne dass ein KI-Client hinzugefügt wurde, mit der Schaltfläche \glqq{}KI hinzufügen\grqq{} wechselt man zum Dialog \glqq{}Lobby\grqq{} und fügt einen KI-Client mit den ausgewählten Parametern hinzu.
Die Intelligenzstufe der KI kann über die Radiobuttons mit den Bezeichnungen 'dumm', 'normal' und 'schlau' eingestellt werden.
Wie in FA-KI 43 beschrieben, muss es möglich sein, die KI mithilfe einer Konfigurationsdatei zu konfigurieren. Deswegen wird in diesem Dialog eine Liste mit Konfigurationsdateien dargestellt, aus denen man eine gespeicherte Konfiguration auswählen kann. Nach der Auswahl einer Konfiguration kann man mit der Schaltfläche \glqq{}Konfiguration laden\grqq{} diese Konfiguration automatisch einstellen.
In das Eingabefeld über der Schaltfläche \glqq{}Konfiguration speichern\grqq{} wird der Name der Konfigurationsdatei vom Benutzer eingetragen. Wenn dort ein valider Dateiname eingegeben wurde, so kann die aktuell eingestellte Konfiguration mit der Schaltfläche \glqq{}Konfiguration speichern\grqq{} in das dafür vorgesehene Verzeichnis gespeichert werden und wird an die Liste der Konfigurationsdateien angefügt.

\begin{figure}
  \centering
  \includegraphics[width=\textwidth]{Meilenstein03/KI-Konfiguration_Mockup.png}
  \caption{Mockup für die KI-Konfiguration}
\end{figure}

Popup \glqq{}Hilfe-KI-Konfiguration\grqq{}

In diesem Popup wird die KI-Konfiguration erklärt. Durch Klicken auf den Schließen-Button wird das Popup wieder geschlossen und der benutzer gelangt zurück zur KI-Konfiguration.

\begin{figure}
  \centering
  \includegraphics[width=\textwidth]{Meilenstein03/Hilfe-KI-Konfiguration_Mockup.png}
  \caption{Mockup für das Hilfe-KI-Konfiguration-Popup}
\end{figure}

Dialog \glqq{}Wahlphase\grqq{}

Vorlage muss noch bearbeitet werden.

\begin{figure}
  \centering
  \includegraphics[width=\textwidth]{Meilenstein03/Wahlphase_Mockup.png}
  \caption{Mockup für die Wahlphase}
\end{figure}

Dialog \glqq{}Ausrüstungsphase\grqq{}

Vorlage muss noch bearbeitet werden.

\begin{figure}
  \centering
  \includegraphics[width=\textwidth]{Meilenstein03/Ausruestungsphase_Mockup.png}
  \caption{Mockup für die Ausrüstungsphase}
\end{figure}

Dialog \glqq{}Spielbildschirm\grqq{}

Durch Klicken auf die eigenen Agenten werden die möglichen Aktionen angezeigt. An der Seite öffnet sich dann eine Übersicht des Inventars des Agenten. In der unteren Leiste werden die Punkte des jeweiligen Agenten angezeigt. Durch Klicken auf Optionen öffnet sich ein Popup-Fenster mit allen Optionen.

\begin{figure}
  \centering
  \includegraphics[width=\textwidth]{Meilenstein03/Spielbildschirm_Mockup.png}
  \caption{Mockup für den Spielbildschirm}
\end{figure}

Popup \glqq{}Optionen\grqq{}

In diesem Popup kann der Benutzer zu Spiel pausieren wechseln, indem er auf Pausieren klickt, er kann zu den Einstellungen wechseln, indem er auf Einstellungen klickt und er kann das Optionenfenster wieder schließen, indem er auf Schließen klickt. Dann wechselt er wieder zum Spielbildschirm.

\begin{figure}
  \centering
  \includegraphics[width=\textwidth]{Meilenstein03/Optionen-Spielbildschirm_Mockup.png}
  \caption{Mockup für das Optionen-Spielbildschirm-Popup}
\end{figure}

Popup \glqq{}Pausieren\grqq{}
Fehlt noch.

Dialog \glqq{}Gewinnerbildschirm\grqq{}

Vorlage muss noch bearbeitet werden.

\begin{figure}
  \centering
  \includegraphics[width=\textwidth]{Meilenstein03/Gewinnerbildschirm_Mockup.png}
  \caption{Mockup für den Gewinnerbildschirm}
\end{figure}


\clearpage
\subsection{Domänenmodell}
\FloatBarrier
\begin{figure}[h]
    \centering
    \includegraphics[width=\textwidth]{domainmodel.pdf}
    \caption{Domänenmodell}
\end{figure}

\FloatBarrier

\begin{figure}[h]
    \centering
    \includegraphics[width=\textwidth]{fields.pdf}
    \caption{Domänenmodell: Feld}
\end{figure}
\FloatBarrier


\begin{figure}[h]
    \centering
    \includegraphics[width=\textwidth]{gadgets.pdf}
    \caption{Domänenmodell: Gadget}
\end{figure}
\FloatBarrier



\clearpage
\section{Randbedingungen}
\subsection{Nichtfunktionale Anforderungen}
Dieser Abschnitt spezifiziert die Qualitätsanforderungen (QA) an das Softwaresystem.

\subsection{Betriebskonzept}
Das fertige Spiel bietet die Möglichkeit über das Netzwerk oder lokal ein rundenbasiertes
Strategiespiel zu spielen.\\
Das Produkt besteht aus einem Client, welcher die graphische Nutzerschnittstelle zum Spiel
darstellt. In den Client inkludiert ist ein Editor für die Konfiguration von Charakteren und
Szenarien. Weiter enthält das Produkt einen Server mit welchem sich die Clients verbinden.\\
Weiter gibt es einen KI-Spieler, welche aus dem Client gestartet werden kann.\\
Der Client kann auf sämtlichen Desktop Plattformen ausgeführt werden. Eine Anleitung für die Nutzung des Client, wird in digitaler Form bereit gestellt. 
Außerdem hat gibt es Hilfe-Funktionen im Client.\\
In der Betriebsanleitung wird ebenfall spezifiziert wie der Server zu verwenden ist.\\

Das Produkt stellt keine Updateversion zur Verfügung. Nach jedem Update muss das gesamte Projekt
neu gebaut werden. Eine Weiterentwicklung, über das Praktikum hinaus, ist nicht nicht vorgesehen.

\subsection{Entwicklungsvorgaben}
Der Server und der KI-Client werden in C++ implementiert. Der Client wird in Python implementiert.\\
Die Dokumentation des Systems umfasst ausführliche, englische Dokumentation des Quellcodes, inklusive Doxygen, ein Entwicklerhandbuch, welches die Entscheidungen der Archiktektur beinhaltet und ein Benutzerhandbuch, welches die Verwendung der Software darstellt.\\
\\
Folgende Software wird bei der Entwicklung verwendet:
\begin{itemize}
\item Git als Versionierungssoftware
	
\item SonarQube bzw. Pylint als Qualitätssicherungssoftware
	
\item Docker als Containervirtualisierungssoftware 
\end{itemize}
Die Software wird von den sechs beteiligten Entwicklern mit Scrum als Umsetzung von agilen Methoden entwickelt. Die Tutorien erfüllen die Rolle von Sprint Meetings, der Tutor nimmt hierbei die Rolle des Product-Owners ein. Die Issue-Funktion von Git stellt das Scrum-Board dar.\\
SonarQube bzw. Pylint wird unterstützend verwendet um die Qualität vom Source-Code bei den Begutachtungen in den Sprint-Reviews zu angemessen zu analysieren.
Außerdem müssen die Kriterien der Codeanalyse-Tools erfüllt werden, bevor ein Pull-Request gemerged werden kann.\\
Die Komponenten Benutzer-Client, KI-Client und Server werden im Team implementiert, die Komponente Editor wird auf der Messe erworben.\\
Die Entwicklung der Netzwerkkommunkation wird auf dem Netzwerkprotokoll basieren, das vom Standardisierungskomitee erstellt wird.\\
Um die Übertragbarkeit des Systems im Bezug auf Hardware zu gewährleisten, wird das fertige Produkt in einem Docker-Container übergeben.\\


\subsection{Abnahmekriterien}
Es muss eine lauffähige Version der Software vorgelegt werden, die alle verbindlichen Anforderungen erfüllt. Die Software muss eine gute Qualität haben, ausreichend und sinnvoll dokumentiert und getestet sein. Die Qualität der Software muss nachgewiesen werden durch umfangreiche dokumentierte Qualitätssicherungsmethoden. Die Abnahme ist erfolgreich, wenn der Kunde mit dem Produkt zufrieden ist.



\end{document}
