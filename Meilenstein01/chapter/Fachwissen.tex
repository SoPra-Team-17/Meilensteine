Im Folgenden werden Abkürzungen und domänenspezifische Begriffe, welche für das Verständnis dieses Dokuments wichtig sind, definiert.

\todo[inline] {Fachwissen vervollständigen}

\Fachwissen{Angriff}
{TODO}
{-}
{-}
{-}
{-}
{-}

\Fachwissen{Balancing}
{Unter Balancing versteht man das Verändern von Spielparametern mit dem Ziel mehr Chancengleichheit zwischen den Spielern herzustellen.}
{Instrument der Spielgestaltung.}
{Stärken bzw. Schwächen von Spielfiguren und Items.}
{Die Veränderung der Spielparameter erfolgt über den Editor.}
{-}
{Erhöhung der Health-Points des Charakters \glqq James Bond \grqq.}

\FachwissenSyn{Charakter}
{Eine Figur aus den James-Bond Filmen oder Romanen, die durch eine Einheit auf dem Raster der Spielfelder repräsentiert wird.}
{Bewegliche Einheit.}
{Player-Character (PC), Non-Player-Charakter (NPC).}
{Besitzt einen Namen, Eigenschaften, Gadgets im Inventar, Vorrat an Casino-Spielchips, Health Points, Intelligence Points, eine Position auf dem Spielfeld und ist einer Fraktion zugehörig oder ein NPC.}
{Der Begriff \glqq Charakter\grqq bezeichnet eine Person, die in der Welt des Spiels im Casino herumläuft.}
{Dr. Madelein Swann ist ein zur Auswahl stehender Charakter.}
{Spielfigur}

\Fachwissen{Client}
{Der Client ist diejenige Spielkomponente, welche dem Nutzer eine grafisches Oberfläche bereitstellt. Mit dieser hat der Nutzer die Möglichkeit sich mit einem Server zu verbinden und einer Partie als Spieler oder Zuschauer beizutreten. Über die grafische Oberfläche wird während einer Partie das Spiel visualisiert und mit dem Spieler interagiert. }
{-}
{-}
{Notwendig zum Spielen von \glqq No Time To Spy\grqq.}
{Der Begriff Client bezeichnet nicht die Person welches die Anwendung bedient, sondern auf die Softwarekomponente selbst.}
{-}

\FachwissenSyn{Continous Integration}
{Continous Integration beschreibt den Vorgang des permanenten Zusammenfügens aller Komponenten während des Entwicklungsprozesses. Insbesondere bei der Entwicklung in Teams bedeutet dies, dass die Änderungen der einzelnen Personen sehr früh wieder zusammengeführt werden, sodass eventuelle Probleme in der Abstimmung frühzeitig erkannt und beseitigt werden können.}
{Prozess der Softwareentwicklung}
{-}
{-}
{-}
{-}
{Kontinuierliche Integration, Fortlaufende Integration}

\Fachwissen{C++}
{C++ ist eine von der \glqq International Organization for Standardization\grqq in verschiedenen Standards genormte Programmiersprache.}
{Programmiersprache}
{-}
{C++ ermöglicht imperative und objektorientierte Programmierung.}
{-}
{C++17 ist ein möglicher Standardisierung von C++.}

\Fachwissen{Docker-Container}
{Ein Docker-Container ist eine Form der Virtualisierung und enthält eine Anwendung sowie die zur Laufzeit notwendigen Ressourcen. Um die Anwendung auf einem System auszuführen wird also lediglich die Container-Engine von Docker benötigt, sämtliche anderen Abhängigkeiten werden vom Container aufgelöst.}
{Virtualisierungstool}
{-}
{-}
{Die Serverkomponente des Spiels wird in einem Docker-Container realisiert um plattformunabhängiger entwickeln zu können.}
{-}

\FachwissenSyn{Editor}
{Stellt dem Nutzer eine grafische Oberfläche bereit, mit welcher die Spielkonfiguration angepasst werden kann. }
{-}
{-}
{Ermöglicht dem Nutzer eigene Level zu erstellen.}
{-}
{-}
{Konfigurator, Level-Editor}

\Fachwissen{Fraktion}
{TODO}
{-}
{-}
{-}
{-}
{-}

\FachwissenSyn{Geheiminformation}
{TODO}
{-}
{-}
{-}
{-}
{-}
{Geheimnis}

\FachwissenSyn{Health-Points (HP)}
{Unter Health-Points versteht man die Anzahl der Lebenspunkte, die eine Spielfigur besitzt. Diese wird in der Spielkonfiguration festgelegt. Während einer Partie kann die Spielfigur von anderen Spielfiguren angegriffen werden und Schaden nehmen. Der erlittene Schaden wird dann von den noch vorhanden Lebenspunkten abgezogen.}
{-}
{-}
{Sinken die Health-Points unter einen Punkt, so muss die Spielfigur exfiltriert werden.}
{-}
{Die Spielfigur \glqq James Bond\grqq besitzt 100 HP.}
{Lebenspunkte}

\Fachwissen{Intelligence Points (IP)}
{Intelligence Points repräsentieren, wie gut ein Charakter beim Spionieren war. Für jedes in Erfahrung gebrachte Geheimnis bekommt der Charakter eine entsprechende Anzahl an Intelligence Points.}
{Elementarer Spielbestandteil.}
{-}
{Intelligence Points sind die primäre Metrik zur Bestimmung des Siegers.}
{-}
{-}

\Fachwissen{Künstliche Intelligenz (KI)}
{Die Künstliche Intelligenz definiert den Algorithmus bzw. die Regeln, nach welcher der KI-Client seine Aktionen im Spielverlauf auswählt.}
{-}
{-}
{Notwendig für Spiele gegen einen Computergegner.}
{Der Begriff \glqq Künstliche Intelligenz\grqq beschreibt lediglich die Logik, nach welcher die Aktionen ausgewählt werden. Die Implementierung erfolgt im KI-Client.}
{-}

\FachwissenSyn{KI-Client}
{Der KI-Client interagiert mit dem Server wie ein normaler Client, mit der Besonderheit, dass er das menschliche Verhalten lediglich simuliert und nicht von einem Menschen gesteuert wird. In Folge dessen besitzt der KI-Client keine graphische Benutzeroberfläche. }
{-}
{-}
{Notwendig für Spiele gegen einen Computergegner.}
{Der Begriff KI-Client beschreibt die Implementierung der Logik einer künstlichen Intelligenz.}
{-}
{Computergegner, Bot}

\Fachwissen{Level}
{TODO}
{-}
{-}
{-}
{-}
{-}

\Fachwissen{Merge-Request}
{Unter einem Merge-Request wird eine Anfrage zum \glqq mergen \grqq des Standes eines Entwicklungszweiges in den Hauptzweig verstanden. Ziel ist es dabei, dass neue Features auf eigenen Entwicklungszweigen implementiert werden und erst nach eingehender Prüfung in den Hauptzweig übernommen.}
{Feature der Versionsverwaltungssoftware Git.}
{-}
{-}
{Zentraler Bestandteil des Feature-Branch-Workflows.}
{-}

\Fachwissen{Monkey-Testing}
{Unter Monkey-Testing wird eine spezielle Form des Softwaretestens verstanden, bei welcher zufällige Nutzereingaben simuliert werden. Ziel ist es dabei sicherzustellen, dass die Anwendung in keinem Fall abstürzt.}
{Negativtest.}
{-}
{-}
{Wird in Ergänzung zu den Unit-Tests verwendet.}
{-}

\Fachwissen{Nicht-Spieler Charakter (NPC)}
{Unter einem NPC wird eine Spielfigur verstanden, welche von keinem der beiden Spieler gesteuert wird. Für die Spieler ist es dabei lediglich ersichtlich, dass der Charakter nicht zu ihrer Fraktion gehört, allerdings nicht, ob er zur gegnerischen Fraktion gehört. Die Aktionen der Spielfiguren werden dabei vom Server gesteuert, allerdings ohne auf ein Spielziel hinzuwirken.}
{Spielfigur}
{-}
{-}
{Die Spielfigur \glqq James Bond\grqq wird vom Server gesteuert und ist somit ein NPC.}

\Fachwissen{Nutzer}
{Ein Nutzer ist eine Person, welche über den Client mit dem Spiel  \glqq No Time To Spy\grqq interagiert. Diese Interaktion kann durch Bedienen des Konfigurators, Spielen des Spiels oder Beobachten einer laufenden Partie erfolgen.}
{-}
{Spieler, Zuschauer}
{-}
{-}
{Der Nutzer \glqq Max Mustermann\grqq interagiert mit dem Client.}

\Fachwissen{Partie}
{Unter einer Partie versteht man die Gesamtheit eines Spiels. Diese beginnt mit der Drafting-Phase und endet mit der Feststellung des Gewinners.}
{-}
{-}
{Zu Beginn der Partie wird die Partiekonfiguration vom Server geladen.}
{-}
{-}

\Fachwissen{Partie-Konfiguration}
{Die Einstellungen für eine Partie werden in einer separaten Konfigurationsdatei im JSON-Format gespeichert. }
{-}
{-}
{Enthält Zeitlimits für die Bewegungs- und Aktionsphase, Wahrscheinlichkeiten von Spielereignissen, Anzahl der Runden bis zum Wechsel in den Modus für überlange Partien und das zu spielende Szenario.}
{-}
{-}

\Fachwissen{Runde}
{TODO}
{-}
{-}
{-}
{-}
{-}

\Fachwissen{Server}
{Der Server ist die zentrale Komponente des Spiels, in welcher die gesamte Spiellogik implementiert ist und über welche die Clients miteinander kommunizieren. Der Server stellt eine Kommandozeilenanwendung dar und besitzt keine grafische Oberfläche. Für die Kommunikation zwischen Server und Clients werden Nachrichten im JSON-Format über das WebSocket-Protokoll ausgetauscht.}
{-}
{-}
{Zuständig für die Kommunikation mit Clients und die Verwaltung der Spielpartien mitsamt der gesamten Spiellogik.}
{Der Server besitzt keine grafische Oberfläche, sondern lediglich ein Kommandozeileninterface.}
{-}

\Fachwissen{Spieler}
{Ein Spieler ist ein Nutzer, welcher sich in einer Partie des Spiels \glqq No Time To Spy\grqq gegen einen anderen Spieler befindet.}
{Nutzer}
{-}
{Menschlicher Spieler, KI-Client}
{-}
{-}

\Fachwissen{Spielfeld}
{Das Spielfeld beschreibt die festgelegte Fläche, auf welcher sich die Spielfiguren bewegen.}
{-}
{-}
{-}
{-}
{-}

\FachwissenSyn{Unit-Test}
{Unit-Tests werden in der Softwareentwicklung verwendet, um einzelne Module gezielt und automatisiert Funktionstests unterziehen zu können.}
{Testvariante in der Softwareentwicklung}
{-}
{-}
{Werden eingesetzt um die Funktionalität der einzelnen Module des Spiels zu überprüfen.}
{Google Test ist ein Framework für Unit-Tests mit C++.}
{Modultest, Komponententest}

\Fachwissen{WebSocket-Protokoll}
{Das WebSocket-Protokoll ist ein Netzwerkprotokoll, welches auf TCP basiert und für eine bidirektionale Verbindung genutzt werden kann.}
{Netzwerkprotokoll}
{-}
{-}
{-}
{-}

\Fachwissen{Zufall}
{Unter Zufall wird in diesem Dokument das Ziehen aus einer Normalverteilung verstanden.}
{-}
{-}
{Der Eintritt von Ereignissen innerhalb einer Runde wird vom Zufall bestimmt.}
{-}
{-}

\Fachwissen{Zug}
{TODO}
{-}
{-}
{-}
{-}
{-}

\Fachwissen{Zuschauer}
{Ein Zuschauer ist ein Nutzer, welcher eine Partie des Spiels \glqq{}No Time To Spy\grqq{} beobachtet. Dabei kann er das gesamte Spielfeld einsehen, allerdings im Gegensatz zu den Spielern keinen Einfluss auf das Spielgeschehen nehmen. Der Zuschauer besitzt dabei die Möglichkeit im Gegensatz zu den Spielern das gesamte Spielfeld einsehen zu können, kann allerdings auch auswählen, das Spiel aus Sicht einer der beiden Spieler zu verfolgen.}
{Nutzer}
{-}
{-}
{-}
{Der Nutzer \glqq Max Mustermann\grqq beobachtet die laufende Spielpartie als Zuschauer.}

