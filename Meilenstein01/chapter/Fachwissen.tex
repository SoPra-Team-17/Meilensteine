Im Folgenden werden Abkürzungen und domänenspezifische Begriffe, welche für das Verständnis dieses Dokuments wichtig sind, definiert.

\Fachwissen{Aktion}
{}
{Elementarer Spielbestandteil.}
{Benutzung eines Gadgets, Roulette spielen, Spionieren, in einen Tresor spicken, einen Cocktail schlürfen oder über einen anderen Charakter schütten.}
{-}
{-}
{Der Charakter \glqq Eve Moneypenny\grqq führt eine Aktion aus und spielt Roulette.}

\Fachwissen{Aktionspunkt (AP)}
{Aktionspunkte werden zu Beginn eines jeden Zuges vom Server an die Charaktere entsprechend ihrer Eigenschaften vergeben. Für jeden Aktionspunkt kann der Charakter eine Aktion ausführen.}
{Elementarer Spielbestandteil.}
{-}
{Nicht genutzte Aktionspunkte verfallen am Ende eines Zuges.}
{-}
{Der Charakter \glqq Jaws\grqq nutzt einen seiner verfügbaren Aktionspunkte um sich selbst mit einem \glqq Akku-Föhn\grqq trocken zu föhnen.}

\Fachwissen{Angriff}
{Unter einem Angriff versteht man den Einsatz eines Gadgets gegen einen anderen Charakter, mit dem Ziel, diesem Schaden zuzufügen.}
{Aktion.}
{-}
{-}
{-}
{Der Charakter \glqq James Bond\grqq setzt den \glqq Klingenhut\grqq gegen den Charakter \glqq Quarrel\grqq ein und fügt diesem dadurch Schaden zu.}

\Fachwissen{Ausrüstungsphase}
{Die Ausrüstungsphase wird nach der Wahlphase durchlaufen. Dabei entscheidet der Spieler nun, welcher Agent seiner Fraktion welche Gadgets erhalten soll. }
{Elementarer Bestandteil der Spielvorbereitung.}
{-}
{-}
{-}
{Der Spieler \glqq Max Mustermann\grqq rüstet seinen Agenten \glqq James Bond\grqq mit dem Gadget \glqq Klingen-Hut\grqq aus.}

\Fachwissen{Balancing}
{Unter Balancing versteht man das Verändern von Spielparametern mit dem Ziel mehr Chancengleichheit zwischen den Spielern herzustellen.}
{Instrument der Spielgestaltung.}
{Stärken bzw. Schwächen von Charakteren und Items.}
{Die Veränderung der Spielparameter erfolgt über den Editor.}
{-}
{Erhöhung der Health-Points des Charakters \glqq James Bond \grqq.}

\Fachwissen{Bewegungspunkt (BP)}
{Bewegungspunkte werden zu Beginn eines jedes Zuges vom Server an die Charaktere entsprechend ihrer Eigenschaften vergeben. Für jeden Bewegungspunkt kann sich der Spieler ein Feld weiter bewegen, sofern dieses nicht blockiert ist.}
{Elementarer Spielbestandteil.}
{-}
{Nicht genutzte Bewegungspunkte verfallen am Ende eines Zuges.}
{-}
{Der Charakter \glqq Q\grqq nutzt zwei seiner fünf verfügbaren Bewegungspunkte um sich auf ein Kamin-Feld zu bewegen.}

\FachwissenSyn{Charakter}
{Eine Figur aus den James-Bond Filmen oder Romanen, die durch eine Einheit auf dem Raster der Spielfelder repräsentiert wird.}
{Bewegliche Einheit.}
{Player-Character (PC), Non-Player-Charakter (NPC).}
{Besitzt einen Namen, Eigenschaften, Gadgets im Inventar, Vorrat an Casino-Spielchips, Health Points, Intelligence Points, eine Position auf dem Spielfeld und ist einer Fraktion zugehörig oder ein NPC.}
{Der Begriff \glqq Charakter\grqq bezeichnet eine Person, die in der Welt des Spiels im Casino herumläuft.}
{Dr. Madelein Swann ist ein zur Auswahl stehender Charakter.}
{Spielfigur}

\Fachwissen{Client}
{Der Client ist diejenige Spielkomponente, welche dem Nutzer eine grafisches Oberfläche bereitstellt. Mit dieser hat der Nutzer die Möglichkeit sich mit einem Server zu verbinden und einer Partie als Spieler oder Zuschauer beizutreten. Über die grafische Oberfläche wird während einer Partie das Spiel visualisiert und mit dem Spieler interagiert. }
{-}
{-}
{Notwendig zum Spielen von \glqq No Time To Spy\grqq.}
{Der Begriff Client bezeichnet nicht die Person welches die Anwendung bedient, sondern auf die Softwarekomponente selbst.}
{-}

\FachwissenSyn{Continous Integration}
{Continous Integration beschreibt den Vorgang des permanenten Zusammenfügens aller Komponenten während des Entwicklungsprozesses. Insbesondere bei der Entwicklung in Teams bedeutet dies, dass die Änderungen der einzelnen Personen sehr früh wieder zusammengeführt werden, sodass eventuelle Probleme in der Abstimmung frühzeitig erkannt und beseitigt werden können.}
{Prozess der Softwareentwicklung}
{-}
{-}
{-}
{-}
{Kontinuierliche Integration, Fortlaufende Integration}

\Fachwissen{C++}
{C++ ist eine von der \glqq International Organization for Standardization\grqq in verschiedenen Standards genormte Programmiersprache.}
{Programmiersprache}
{-}
{C++ ermöglicht imperative und objektorientierte Programmierung.}
{-}
{C++17 ist ein möglicher Standardisierung von C++.}

\Fachwissen{Docker-Container}
{Ein Docker-Container ist eine Form der Virtualisierung und enthält eine Anwendung sowie die zur Laufzeit notwendigen Ressourcen. Um die Anwendung auf einem System auszuführen wird also lediglich die Container-Engine von Docker benötigt, sämtliche anderen Abhängigkeiten werden vom Container aufgelöst.}
{Virtualisierungstool}
{-}
{-}
{Die Serverkomponente des Spiels wird in einem Docker-Container realisiert um plattformunabhängiger entwickeln zu können.}
{-}

\Fachwissen{Drafting Phase}
{Bevor eine Partie startet wird die Zusammensetzung der Fraktionen der jeweiligen Spieler in der Drafting-Phase ermittelt. Diese ist dabei in Wahl- und Ausrüstungsphase unterteilt. }
{Elementarer Spielbestandteil.}
{-}
{-}
{-}
{Der Spieler \glqq Max Mustermann \grqq hat sich in der Drafting Phase für die Agenten \glqq James Bond\grqq und \glqq Le Chiffre\grqq sowie die Gadgets \glqq Akku-Föhn\grqq und \glqq Maulwürfel \grqq entschieden.}

\FachwissenSyn{Editor}
{Stellt dem Nutzer eine grafische Oberfläche bereit, mit welcher die Spielkonfiguration, Szenarien und Charakterkonfigurationen verändert werden können.}
{-}
{-}
{Ermöglicht dem Nutzer bestimmte Elemente des Spiels nach seinen eigenen Vorstellungen zu konfigurieren.}
{-}
{Um das Spiel zu balancen erhöht der Nutzer \glqq Max Mustermann\grqq die Health-Points des Charakters \glqq James Bond\grqq im Editor.}
{Konfigurator}

\Fachwissen{Fraktion}
{Der Begriff Fraktion beschreibt ein Team aus Agenten, welches sich die Spieler zu Beginn einer Partie zusammenstellen. }
{-}
{-}
{Da sich die Eigenschaften der Charaktere unterscheiden kann die Wahl der Fraktion einen großen Einfluss auf das Spielergebnis haben.}
{-}
{Spieler 1 besitzt nach der Wahlphase in seiner Fraktion die Agenten \glqq James Bond\grqq, \glqq Vesper Lynd\grqq und \glqq Wai Lin\grqq.}

\Fachwissen{Gadget}
{Gadgets sind technische Hilfsmittel, welche Agenten bei ihren Missionen unterstützen. Ein Agent kann mehrere Gadgets in seinem Inventar aufbewahren und diese dann durch die Nutzung von Aktionspunkten einsetzen. }
{Elementarer Spielbestandteil.}
{Akku-Föhn, Maulwürfel, Technicolor-Prisma, Klingen-Hut, Magnetfeld-Armbanduhr, Giftpillen-Flasche, Laser-Puderdose, Raketenwerfen-Füllfederhalter, Gaspatronen-Lippenstift, Mottenkugel-Beutel, Nebeldose, Wurfhaken, Jetpack, Wanze und Ohrstöpsel, Chicken Feed, Nugget, Mirror of Wilderness, Pocket Litter, Diamanthalsband der weißen Katze.}
{Gadgets ermöglichen den Agenten dabei spezielle Aktionen, welche ohne diese nicht möglich wären. }
{-}
{Das Gadget \glqq Wanze und Ohrstöpsel\grqq ermöglicht es einem Agenten eine Zielperson abzuhören.}

\FachwissenSyn{Geheiminformation}
{Geheiminformationen können in Tresoren gefunden oder von der weißen Katze im Tausch gegen ihr Diamanthalsband erhalten werden. Da mit dem erstmaligen in Erfahrung bringen einer Geheiminformation für jede Fraktion eine bestimmte Anzahl an Intelligence Points verbunden ist, sind sie ein elementarer Bestandteil des Spiels. }
{-}
{-}
{-}
{-}
{-}
{Geheimnis}

\FachwissenSyn{Health-Points (HP)}
{Unter Health-Points versteht man die Anzahl der Lebenspunkte, die ein Charakter besitzt. Diese wird in der Spielkonfiguration festgelegt. Während einer Partie kann die Charakter von anderen Charakteren angegriffen werden und Schaden nehmen. Der erlittene Schaden wird dann von den noch vorhanden Lebenspunkten abgezogen.}
{-}
{-}
{Sinken die Health-Points unter einen Punkt, so muss der Charakter exfiltriert werden.}
{-}
{Der Charakter \glqq James Bond\grqq besitzt 100 HP.}
{Lebenspunkte}

\Fachwissen{Intelligence Points (IP)}
{Intelligence Points repräsentieren, wie gut ein Charakter beim Spionieren war. Für jedes in Erfahrung gebrachte Geheimnis bekommt der Charakter eine entsprechende Anzahl an Intelligence Points.}
{Elementarer Spielbestandteil.}
{-}
{Intelligence Points sind die primäre Metrik zur Bestimmung des Siegers.}
{-}
{-}

\Fachwissen{Künstliche Intelligenz (KI)}
{Die Künstliche Intelligenz definiert den Algorithmus bzw. die Regeln, nach welcher der KI-Client seine Aktionen im Spielverlauf auswählt.}
{-}
{-}
{Notwendig für Spiele gegen einen Computergegner.}
{Der Begriff \glqq Künstliche Intelligenz\grqq beschreibt lediglich die Logik, nach welcher die Aktionen ausgewählt werden. Die Implementierung erfolgt im KI-Client.}
{-}

\FachwissenSyn{KI-Client}
{Der KI-Client interagiert mit dem Server wie ein normaler Client, mit der Besonderheit, dass er das menschliche Verhalten lediglich simuliert und nicht von einem Menschen gesteuert wird. In Folge dessen besitzt der KI-Client keine graphische Benutzeroberfläche. }
{-}
{-}
{Notwendig für Spiele gegen einen Computergegner.}
{Der Begriff KI-Client beschreibt die Implementierung der Logik einer künstlichen Intelligenz.}
{-}
{Computergegner, Bot}

\Fachwissen{Merge-Request}
{Unter einem Merge-Request wird eine Anfrage zum \glqq mergen \grqq des Standes eines Entwicklungszweiges in den Hauptzweig verstanden. Ziel ist es dabei, dass neue Features auf eigenen Entwicklungszweigen implementiert werden und erst nach eingehender Prüfung in den Hauptzweig übernommen.}
{Feature der Versionsverwaltungssoftware Git.}
{-}
{-}
{Zentraler Bestandteil des Feature-Branch-Workflows.}
{-}

\Fachwissen
{Lobby}
{Eine Lobby ist ein virtueller Raum, in welchem sich zwei Nutzer zu einem Spiel zusammenfinden können.}
{Elementarer Spielbestandteil.}
{-}
{-}
{-}
{-}

\Fachwissen{Monkey-Testing}
{Unter Monkey-Testing wird eine spezielle Form des Softwaretestens verstanden, bei welcher zufällige Nutzereingaben simuliert werden. Ziel ist es dabei sicherzustellen, dass die Anwendung in keinem Fall abstürzt.}
{Negativtest.}
{-}
{-}
{Wird in Ergänzung zu den Unit-Tests verwendet.}
{-}

\Fachwissen{Nichtspieler Charakter (NPC)}
{Unter einem NPC wird ein Charakter verstanden, welcher von keinem der beiden Spieler gesteuert wird. Für die Spieler ist es dabei lediglich ersichtlich, dass der Charakter nicht zu ihrer Fraktion gehört, allerdings nicht, ob er zur gegnerischen Fraktion gehört. Die Aktionen des Charakters werden dabei vom Server gesteuert, allerdings ohne auf ein Spielziel hinzuwirken.}
{Charakter}
{-}
{-}
{Der Charakter \glqq James Bond\grqq wird vom Server gesteuert und ist somit ein NPC.}

\Fachwissen{Nutzer}
{Ein Nutzer ist eine Person, welche über den Client mit dem Spiel  \glqq No Time To Spy\grqq interagiert. Diese Interaktion kann durch Bedienen des Konfigurators, Spielen des Spiels oder Beobachten einer laufenden Partie erfolgen.}
{-}
{Spieler, Zuschauer}
{-}
{-}
{Der Nutzer \glqq Max Mustermann\grqq  interagiert mit dem Client.}

\Fachwissen{Partie}
{Unter einer Partie versteht man die Gesamtheit eines Spiels. Diese beginnt mit der Drafting-Phase und endet mit der Feststellung des Gewinners.}
{-}
{-}
{Zu Beginn der Partie wird die Partiekonfiguration vom Server geladen.}
{-}
{-}

\Fachwissen{Partie-Konfiguration}
{Die Einstellungen für eine Partie werden in einer separaten Konfigurationsdatei im JSON-Format gespeichert. }
{-}
{-}
{Enthält Zeitlimits für die Bewegungs- und Aktionsphase, Wahrscheinlichkeiten von Spielereignissen, Anzahl der Runden bis zum Wechsel in den Modus für überlange Partien und das zu spielende Szenario.}
{-}
{-}

\Fachwissen{Runde}
{Das Spiel \glqq No Time To Spy\grqq ist ein rundenbasiertes Spiel. Zu Beginn einer jeden Runde handelt der Server Ereignisse wie das Platzieren von Cocktails auf leeren Bar-Tischen ab. Anschließend bestimmt er die Zugreihenfolge der Charaktere zufällig und lässt jeden der Reihe nach seinen Zug machen. }
{Elementarer Spielbestandteil.}
{-}
{-}
{-}
{-}

\Fachwissen{Server}
{Der Server ist die zentrale Komponente des Spiels, in welcher die gesamte Spiellogik implementiert ist und über welche die Clients miteinander kommunizieren. Der Server stellt eine Kommandozeilenanwendung dar und besitzt keine grafische Oberfläche. Für die Kommunikation zwischen Server und Clients werden Nachrichten im JSON-Format über das WebSocket-Protokoll ausgetauscht.}
{-}
{-}
{Zuständig für die Kommunikation mit Clients und die Verwaltung der Spielpartien mitsamt der gesamten Spiellogik.}
{Der Server besitzt keine grafische Oberfläche, sondern lediglich ein Kommandozeileninterface.}
{-}

\Fachwissen{Sichtlinie}
{Die Sichtlinie beschreibt die Verbindungslinie zwischen zwei beliebigen Kacheln auf dem Spielfeld. Wenn auf keinem Feld, dass von der Linie geschnitten wird eine Wand oder ein Kamin steht, so ist das eine Feld vom anderen aus sichtbar (und umgekehrt).}
{Elementarer Spielbestandteil.}
{-}
{-}
{Die Sichtlinie stellt sicher, dass Charaktere nicht durch Wände hindurch schauen können.}
{Zwischen den Feldern A und C liegt Feld B, auf welchem eine Wand platziert ist. Daher kann der Charakter \glqq James Bond\grqq, welcher sich auf Feld A befindet, das Feld C nicht sehen.}

\Fachwissen{Spieler}
{Ein Spieler ist ein Nutzer, welcher sich in einer Partie des Spiels \glqq No Time To Spy\grqq gegen einen anderen Spieler befindet.}
{Nutzer}
{-}
{Menschlicher Spieler, KI-Client}
{-}
{-}

\Fachwissen{Spieler Charakter (PC)}
{Ein Spieler Charakter ist ein Charakter, welcher von einem Spieler gesteuert wird.}
{Charakter}
{-}
{-}
{Das Pendant zum Spieler Charakter ist der Nichtspieler Charakter (NPC), welcher vom Server gesteuert wird und keiner der beiden Fraktionen angehört.}
{Der Spieler \glqq Max Mustermann\grqq steuert den Charakter \glqq Q\grqq.}

\Fachwissen{Spielfeld}
{Das Spielfeld beschreibt die festgelegte Fläche, auf welcher sich die Charaktere bewegen.}
{-}
{-}
{Das Spielfeld besteht aus einzelnen Kacheln, welche in einem kartesischen Gitter angeordnet sind.}
{-}
{-}

\Fachwissen{Spielfeldkachel}
{Das Spielfeld setzt sich aus unterschiedlichen Spielfeldkacheln zusammen, }
{Elementarer Bestandteil des Spielfelds.}
{Freies Feld, Wand, Kamin-Feld, Sitzplatz, Roulette-Tisch, Bar-Tisch, Tresor.}
{Das Spielfeld besteht aus einzelnen Feldern, welche in einem kartesischen Gitter angeordnet sind.}
{Bestimmte Arten von Feldern besitzen eine besondere Wirkung.}
{-}
{-}

\FachwissenSyn{Szenario}
{Das Spiel \glqq No Time To Spy\grqq spielt in einem Casino. Das Szenario wird dabei unter anderem durch die Art und Anordnung der Spielfeldkacheln sowie die Anzahl von Chips an den Roulette-Tischen bestimmt. }
{-}
{-}
{Der Server bekommt das zu spielende Szenario übergeben.}
{-}
{-}
{Level}

\FachwissenSyn{Unit-Test}
{Unit-Tests werden in der Softwareentwicklung verwendet, um einzelne Module gezielt und automatisiert Funktionstests unterziehen zu können.}
{Testvariante in der Softwareentwicklung}
{-}
{-}
{Werden eingesetzt um die Funktionalität der einzelnen Module des Spiels zu überprüfen.}
{Google Test ist ein Framework für Unit-Tests mit C++.}
{Modultest, Komponententest}

\Fachwissen{Wahlphase}
{Zum Partiebeginn stellen die Spieler sich ihre Fraktionen in einer Wahlphase zusammen. Dabei wählt der Server für jeden Spieler aus der Menge der Charaktere und Gadgets jeweils zufällig drei Charaktere und drei Gadgets aus, welche den Spielern angeboten werden. Jeder Spieler entscheidet sich dann für eines der angeboten Objekte. Der Server stellt dabei sicher, dass sich die Wahl der Spieler nicht überschneidet. Dieser Vorgang wird solange wiederholt, bis alle Slots belegt sind.}
{Elementarer Bestandteil der Spielvorbereitung.}
{-}
{-}
{-}
{Der Spieler \glqq Max Mustermann \grqq hat sich in der Wahlphase für die Agenten \glqq Tiger Tanaka\grqq und \glqq Kissy Suzuki\grqq sowie die Gadgets \glqq Klingen-Hut\grqq und \glqq Giftpillen-Flasche\grqq  entschieden.}

\Fachwissen{Wahrscheinlichkeit}
{Eine Wahrscheinlichkeit ist Kennzahl eines Zufallsprozesses, beschreibt diesen aber nicht vollständig.}
{-}
{-}
{Babysitter-Wahrscheinlichkeit, Honey-Trap-Wahrscheinlichkeit, Observations-Erfolg-Wahrscheinlichkeit, Hut-Treffer-Wahrscheinlichkeit, Laser-Treffer-Wahrscheinlichkeit, Wurfhaken-Treffer-Wahrscheinlichkeit, Wanzen-Ausfalls-Wahrscheinlichkeit, Mirror-of-Wilderness-Wahrscheinlichkeit, Cocktail-Ausweich-Wahrscheinlichkeit, Spionage-Erfolgs-Wahrscheinlichkeit}
{-}
{-}

\Fachwissen{Wahrscheinlichkeitsprobe}
{Wenn ein Charakter eine Aktion ausführt, welche von einer Wahrscheinlickeit abhängt führt der Server eine Wahrscheinlichkeitsprobe durch und stellt fest, ob die Aktion erfolgreich sein wird.}
{Elementarer Spielbestandteil.}
{-}
{-}
{-}
{-}

\Fachwissen{WebSocket-Protokoll}
{Das WebSocket-Protokoll ist ein Netzwerkprotokoll, welches auf TCP basiert und für eine bidirektionale Verbindung genutzt werden kann.}
{Netzwerkprotokoll}
{-}
{-}
{-}
{-}

\Fachwissen{Werte}
{Ein Wert ist im Folgenden Dokument eine reelle Zahl nach IEEE754.}
{-}
{Cocktail-HP, Gift-Cocktail-Schaden, Maulwürfel-Wurf-Weite, Hut-Wurf-Reichweite, Hut-Schaden, Rakten-Schaden, Reizgas-Schaden, Mottenkugel-Wurf-Reichweite, Mottenkugel-Schaden, Nebeldosen-Wurf-Reichweite, Wurfhaken-Reichweite, IPs-pro-Spielchip-Wert, Katzen-IPs, Überlange-Rundenzahl}
{Der Editor bietet dem Spieler die Möglichkeit die Werte zu konfigurieren.}
{}

\FachwissenSyn{Zufall}
{Unter Zufall wird in diesem Fall das gleichverteilte Ziehen aus einer Menge verstanden.}
{-}
{-}
{Der Eintritt von Ereignissen innerhalb einer Runde wird vom Zufall bestimmt.}
{-}
{Während der Drafting Phase werden vom Server zufällig pro Runde drei Charaktere ausgewählt.}
{Zufällig}

\Fachwissen{Zug}
{In jeder Runde darf jeder Charakter einen Zug machen, wobei dieser in verschiedene Zugphasen unterteilt ist. Zu Beginn eines Zuges wird vom Server bestimmt, wie viele Bewegungs- und Aktionspunkte dem Charakter zur Verfügung stehen. Eine Zugphase beschreibt nun den Verbrauch eines solches Punktes, wobei die Anordnung der Bewegungs- und Aktionsphasen dabei beliebig ist, solange genügend entsprechende Punkte vorhanden sind.}
{Elementarer Spielbestandteil.}
{-}
{Nicht genutzte Bewegungs- bzw. Aktionspunkte verfallen am Ende einer Runde.}
{-}
{Francisco Scaramanga ist am Zug und besitzt 2 Bewegungspunkte und einen Aktionspunkt. Mit diesen bewegt er sich erst ein Feld, setzt dann ein Gadget ein und lässt den zweiten Bewegungspunkt schließlich verfallen.}

\Fachwissen{Zuschauer}
{Ein Zuschauer ist ein Nutzer, welcher eine Partie des Spiels \glqq{}No Time To Spy\grqq{} beobachtet. Dabei kann er das gesamte Spielfeld einsehen, allerdings im Gegensatz zu den Spielern keinen Einfluss auf das Spielgeschehen nehmen. Der Zuschauer besitzt dabei die Möglichkeit im Gegensatz zu den Spielern das gesamte Spielfeld einsehen zu können, kann allerdings auch auswählen, das Spiel aus Sicht einer der beiden Spieler zu verfolgen.}
{Nutzer}
{-}
{-}
{-}
{Der Nutzer \glqq Max Mustermann\grqq beobachtet die laufende Spielpartie als Zuschauer.}

