\begin{tabularx}{16cm}{l|X}
\refstepcounter{table}\label{s-partieconfig}
\textbf{ID} & \textbf{FA-S \arabic{table}} \\
\hline
TITEL: & Partie Konfiguration \\
\hline
BESCHREIBUNG: & Der Server muss beim Start der Spielpartie die Partie-Konfiguration aus einer Datei lesen. \\
\hline
BEGRÜNDUNG: & Die Konfiguration soll für das Balancing des Spiels einfach angepasst werden können. \\
\hline
ABHÄNGIGKEITEN: & FA-S \ref{s-json-decoding}\\
\end{tabularx}

\begin{tabularx}{16cm}{l|X}
\refstepcounter{table}\label{s-szenarioconfig}
\textbf{ID} & \textbf{FA-S \arabic{table}} \\
\hline
TITEL: & Szenario Konfiguration \\
\hline
BESCHREIBUNG: & Der Server muss beim Start der Spielpartie die Szenario-Konfiguration (Spielfeld) aus einer Datei lesen. \\
\hline
BEGRÜNDUNG: & Das Spielfeld soll veränderbar sein. \\
\hline
ABHÄNGIGKEITEN: & FA-S \ref{s-json-decoding}\\
\end{tabularx}

\begin{tabularx}{16cm}{l|X}
\refstepcounter{table}\label{s-charakterconfig}
\textbf{ID} & \textbf{FA-S \arabic{table}} \\
\hline
TITEL: & Charakter Konfiguration \\
\hline
BESCHREIBUNG: & Der Server muss beim Start der Spielpartie die Konfiguration der Charaktere aus einer Datei lesen. \\
\hline
BEGRÜNDUNG: & Notwendig für Balancing des Spiels\\
\hline
ABHÄNGIGKEITEN: & FA-S \ref{s-json-decoding}\\
\end{tabularx}

\begin{tabularx}{16cm}{l|X}
\refstepcounter{table}\label{s-clientconnection}
\textbf{ID} & \textbf{FA-S \arabic{table}} \\
\hline
TITEL: & Client Verbindung \\
\hline
BESCHREIBUNG: & Der Server muss mittels Websockets über TCP eine Verbindung mit dem Client halten.
Die Verbindung wird vom Client initiiert. Bei Verbindungsabbruch muss die Verbindung wiederhergestellt werden. \\
\hline
BEGRÜNDUNG: & Integrale Funktionalität des Servers \\
\hline
ABHÄNGIGKEITEN: & \\
\end{tabularx}

\begin{tabularx}{16cm}{l|X}
\refstepcounter{table}\label{s-zuschauer}
\textbf{ID} & \textbf{FA-S \arabic{table}} \\
\hline
TITEL: & Zuschauer \\
\hline
BESCHREIBUNG: & Ein Client kann sich als Zuschauer verbinden, bekommt also Updates über den Spielzustand, kann aber nicht aktiv am Spiel teilnehmen. \\
\hline
BEGRÜNDUNG: & Ermöglicht anderen das Spiel zu beobachten.\\
\hline
ABHÄNGIGKEITEN: & FA-S \ref{s-clientconnection}\\
\end{tabularx}

\begin{tabularx}{16cm}{l|X}
\refstepcounter{table}\label{s-timeout}
\textbf{ID} & \textbf{FA-S \arabic{table}} \\
\hline
TITEL: & Client Timeout \\
\hline
BESCHREIBUNG: & Der Server muss die Verbindung mit Clients, die nicht innerhalb einer in der Partiekonfiguration festgelegten Zeit antworten, trennen. Im Falle eines aktiven Spielers wird dieser disqualifiziert und verliert somit das Spiel. \\
\hline
BEGRÜNDUNG: & Abbau von Verbindungen mit fehlerhaften und unresponsiven Clients\\
\hline
ABHÄNGIGKEITEN: & \\
\end{tabularx}

\begin{tabularx}{16cm}{l|X}
\refstepcounter{table}\label{s-latemessage}
\textbf{ID} & \textbf{FA-S \arabic{table}} \\
\hline
TITEL: & Verspätete Nachrichten \\
\hline
BESCHREIBUNG: & Der Server muss Nachrichten empfangen, die zum aktuellen Zeitpunkt ungültig sind, in einer vorherigen Phase aber korrekt waren. Diese Nachrichten werden verworfen und nicht als Protokollverletzung gewertet. \\
\hline
BEGRÜNDUNG: & Client ist nicht verantwortlich für verspätet eintreffende Pakete (z.B. aufgrund von Netzwerkproblemen)\\
\hline
ABHÄNGIGKEITEN: & FA-S \ref{s-clientconnection}, \ref{s-websockets}\\
\end{tabularx}

\begin{tabularx}{16cm}{l|X}
\refstepcounter{table}\label{s-partieinit}
\textbf{ID} & \textbf{FA-S \arabic{table}} \\
\hline
TITEL: & Spiel Start und Initialisierung \\
\hline
BESCHREIBUNG: & Zum Starten des Spiels verteilt der Server alle Charaktere auf gültigen Feldern auf dem Spielfeld. \\
\hline
BEGRÜNDUNG: & Integrale Funktionalität des Servers\\
\hline
ABHÄNGIGKEITEN: & FA-S \ref{s-regeln}\\
\end{tabularx}

\begin{tabularx}{16cm}{l|X}
\refstepcounter{table}\label{s-rundeinit}
\textbf{ID} & \textbf{FA-S \arabic{table}} \\
\hline
TITEL: & Rundenstart und Initialisierung \\
\hline
BESCHREIBUNG: & Zum Starten einer Runde legt der Server die Zugreihenfolge aller Agenten fest. \\
\hline
BEGRÜNDUNG: & Integrale Funktionalität des Servers\\
\hline
ABHÄNGIGKEITEN: & FA-S \ref{s-regeln}\\
\end{tabularx}

\begin{tabularx}{16cm}{l|X}
\refstepcounter{table}\label{s-spiel}
\textbf{ID} & \textbf{FA-S \arabic{table}} \\
\hline
TITEL: & Spiel Durchführung \\
\hline
BESCHREIBUNG: & Nach Starten eines Spiels muss der Server abwechselnd Spielzüge der Spieler akzeptieren und durchführen.\\
\hline
BEGRÜNDUNG: & Integrale Funktionalität des Servers\\
\hline
ABHÄNGIGKEITEN: & FA-S \ref{s-clientconnection}, \ref{s-websockets}, \ref{s-regeln}\\
\end{tabularx}

\begin{tabularx}{16cm}{l|X}
\refstepcounter{table}\label{s-alternatives}
\textbf{ID} & \textbf{FA-S \arabic{table}} \\
\hline
TITEL: & Finden freier Felder und Alternativen \\
\hline
BESCHREIBUNG: & Wenn für eine Aktion ein benachbartes freies Feld benötigt wird, und es dieses nicht gibt, muss ein anderes möglichst nahes freies Feld gefunden werden. \\
\hline
BEGRÜNDUNG: & Fortsetzen des Spiels auch wenn die Regeln (benachbarte Felder) nicht exakt erfüllt werden können.\\
\hline
ABHÄNGIGKEITEN: & FA-S \ref{s-regeln}\\
\end{tabularx}

\begin{tabularx}{16cm}{l|X}
\refstepcounter{table}\label{s-pause}
\textbf{ID} & \textbf{FA-S \arabic{table}} \\
\hline
TITEL: & Spielpause \\
\hline
BESCHREIBUNG: & Auf Anfrage eines (Nicht-KI-)Clients muss der Server das Spiel pausieren. \\
\hline
BEGRÜNDUNG: & Integrale Funktionalität des Spiels\\
\hline
ABHÄNGIGKEITEN: & \\
\end{tabularx}

\begin{tabularx}{16cm}{l|X}
\refstepcounter{table}\label{s-regeln}
\textbf{ID} & \textbf{FA-S \arabic{table}} \\
\hline
TITEL: & Erkennung von Regelverstößen \\
\hline
BESCHREIBUNG: & Während der Durchführung des Spiels muss der Server die Aktionen jedes Clients auf Regelkonformität prüfen.
Ungültige Aktionen werden vom Server abgelehnt. \\
\hline
BEGRÜNDUNG: & Um korrekten Spielablauf zu gewährleisten. \\
\hline
ABHÄNGIGKEITEN: & \\
\end{tabularx}

\begin{tabularx}{16cm}{l|X}
\refstepcounter{table}\label{s-state-senden}
\textbf{ID} & \textbf{FA-S \arabic{table}} \\
\hline
TITEL: & Senden des Spielzustands \\
\hline
BESCHREIBUNG: & Nachdem ein Spieler seinen Zug beendet hat, muss der Server den aktuellen Spielzustand an alle aktiven Spieler und Zuschauer senden. \\
\hline
BEGRÜNDUNG: & Notwendig für die Darstellung des Spielfelds beim Spieler. \\
\hline
ABHÄNGIGKEITEN: & FA-S \ref{s-websockets}\\
\end{tabularx}

\begin{tabularx}{16cm}{l|X}
\refstepcounter{table}\label{s-gewinner}
\textbf{ID} & \textbf{FA-S \arabic{table}} \\
\hline
TITEL: & Gewinner Erkennung \\
\hline
BESCHREIBUNG: & Nach jeder Aktion muss der Server überprüfen, ob eine der Bedingungen um das Spiel zu gewinnen erfüllt ist.
Wenn ein Spieler das Spiel gewinnt, wird dieses sofort beendet. \\
\hline
BEGRÜNDUNG: & Integrale Funktionalität des Servers\\
\hline
ABHÄNGIGKEITEN: & FA-S \ref{s-spiel}\\
\end{tabularx}

\begin{tabularx}{16cm}{l|X}
\refstepcounter{table}\label{s-replay}
\textbf{ID} & \textbf{FA-S \arabic{table}} \\
\hline
TITEL: & Erstellen von Replay Files \\
\hline
BESCHREIBUNG: & Der Server soll den Spielablauf in einer Log-Datei speichern, sodass dieser zu späterem Zeitpunkt vom Client angezeigt/abgespielt werden kann.\\
\hline
BEGRÜNDUNG: & Ermöglicht spätere Analyse von Spielabläufen\\
\hline
ABHÄNGIGKEITEN: & \\
\end{tabularx}

\begin{tabularx}{16cm}{l|X}
\refstepcounter{table}\label{s-json-encoding}
\textbf{ID} & \textbf{FA-S \arabic{table}} \\
\hline
TITEL: & JSON Encoding \\
\hline
BESCHREIBUNG: & Der Server muss die Funktionalität besitzen, Nachrichten, die Aktionen und Spielzustände darstellen, im JSON Format zu encodieren. \\
\hline
BEGRÜNDUNG: & Notwendig für Kommunikation mit Clients. \\
\hline
ABHÄNGIGKEITEN: & \\
\end{tabularx}

\begin{tabularx}{16cm}{l|X}
\refstepcounter{table}\label{s-json-decoding}
\textbf{ID} & \textbf{FA-S \arabic{table}} \\
\hline
TITEL: & JSON Decoding \\
\hline
BESCHREIBUNG: & Der Server muss die Funktionalität besitzen, Nachrichten, die Aktionen und Spielzustände darstellen, und Konfigurationsdateien aus dem JSON Format zu decodieren. \\
\hline
BEGRÜNDUNG: & Notwendig für Kommunikation mit Clients. \\
\hline
ABHÄNGIGKEITEN: & \\
\end{tabularx}

\begin{tabularx}{16cm}{l|X}
\refstepcounter{table}\label{s-websockets}
\textbf{ID} & \textbf{FA-S \arabic{table}} \\
\hline
TITEL: & Websockets \\
\hline
BESCHREIBUNG: & Der Server muss die Funktionalität besitzen, JSON encodierte Nachrichten über eine Websockets Verbindung mit mehreren Clients auszutauschen.\\
\hline
BEGRÜNDUNG: & Notwendig für Kommunikation mit Clients. \\
\hline
ABHÄNGIGKEITEN: & FA-S \ref{s-json-decoding}, \ref{s-json-encoding} \\
\end{tabularx}

