%Hier kommen die generellen FA rein
%Caro: gen, 2.2 in FA übersetzen with Appendix A
%Caro: gen, 2.3 in FA übersetzen

\begin{tabularx}{16cm}{l|X}
	\refstepcounter{table}\label{Charakter}
	\textbf{ID} & \textbf{FA-G\arabic{table}} \\
	\hline
	TITEL: & Charakter \\
	\hline
	BESCHREIBUNG: & Ein Charakter besitzt folgende Werte, die normalerweise für die gegnerische Fraktion nicht sichtbar sind: 
	\begin{itemize}
		\item Beschreibung
		\item Bewegungspunkte (BP) und Aktionspunkte (AP)
		\item Health Points (HP)
		\item Intelligence Points (IP)
		\item Eigenschaften
		\item Inventar
		\item hält Cocktail
		\item Spielchips
	\end{itemize}
	\\
	\hline
	BEGRÜNDUNG: & Jeder Charakter soll individuell sein und sich über das Spiel hinweg verändern können.\\
	\hline
	ABHÄNGIGKEITEN: & FA-G\ref{Charakter-Beschreibung}, FA-G\ref{BP und AP}, FA-G\ref{HP}, FA-G\ref{IP}, FA-G\ref{Eigenschaften}, FA-G\ref{Inventar}, FA-G\ref{haelt Cocktail}, FA-G\ref{Spielchips} \\
\end{tabularx}

\begin{tabularx}{16cm}{l|X}
	\refstepcounter{table}\label{Charakter-Beschreibung}
	\textbf{ID} & \textbf{FA-G\arabic{table}} \\
	\hline
	TITEL: & Charakter-Beschreibung \\
	\hline
	BESCHREIBUNG: & Die Charakter-Beschreibung beschreibt den Charakter aus \glqq James Bond\grqq in wenigen Sätzen.\\
	\hline
	BEGRÜNDUNG: & Zusätzliche Informationen für den Spieler. \\
	\hline
	ABHÄNGIGKEITEN: & \\
\end{tabularx}

\begin{tabularx}{16cm}{l|X}
	\refstepcounter{table}\label{BP und AP}
	\textbf{ID} & \textbf{FA-G\arabic{table}} \\
	\hline
	TITEL: & Bewegungspunkte (BP) und Aktionspunkte (AP) \\
	\hline
	BESCHREIBUNG: & Während eine Charakter am Zug ist, kann er BP für Bewegungen auf dem Spielfeld und AP für Aktionen einsetzen.
	Zu Beginn eines Zuges erhält er jeweils Punkte. Nach Beenden eines Zuges verfallen übrig gebliebene Punkte. BP und AP können nicht negativ sein.\\
	\hline
	BEGRÜNDUNG: & Hält fest, wie viele Bewegungen und Aktionen der Charakter in diesem Spielzug noch ausführen kann. \\
	\hline
	ABHÄNGIGKEITEN: & FA-G\ref{Flinkheit}, FA-G\ref{Schwerfaelligkeit}, FA-G\ref{Behaendigkeit}, FA-G\ref{Behaebigkeit}, FA-G\ref{Agilitaet} \todo[inline]{2.10.1 Züge, 2.5 Bewegung, 2.6 Aktionen}\\
\end{tabularx}

\begin{tabularx}{16cm}{l|X}
	\refstepcounter{table}\label{HP}
	\textbf{ID} & \textbf{FA-G\arabic{table}} \\
	\hline
	TITEL: & Health Points (HP) \\
	\hline
	BESCHREIBUNG: & HP geben Auskunft über den Gesundheitszustand des Charakters. Zu Spielbeginn werden die HP auf 100 gesetzt. Während des Spiels sorgen verschiedene Aktionen dafür, dass HP hinzugefügt oder abgezogen werden. Werden HP abgezogen, spricht man von Schaden. Die Punkte nehmen Werte zwischen 0 und 100 an.\\
	\hline
	BEGRÜNDUNG: & Zeigt Auswirkung verschiedener Aktionen auf den Charakter. \\
	\hline
	ABHÄNGIGKEITEN: & FA-G\ref{Robuster Magen}, FA-G\ref{Zaehigkeit} \todo[inline]{2.6 Aktionen, 2.4 Gadgets}\\
\end{tabularx}

\begin{tabularx}{16cm}{l|X}
	\refstepcounter{table}\label{Exfiltratation}
	\textbf{ID} & \textbf{FA-G\arabic{table}} \\
	\hline
	TITEL: & Exfiltration \\
	\hline
	BESCHREIBUNG: & Sinken die HP eines Charakter auf 0, so wird eine Exfiltration durchgeführt. Dabei wird der Charakter auf ein zufällig gewähltes freies Sitzplatz-Feld versetzt und seine HP auf 1 gesetzt. Ist kein freier Sitzplatz vorhanden, so wird ein Sitzplatz zufällig ausgewählt und die Person, die dort saß, wird auf ein zufälliges freies Nachbarfeld des Sitzplatzes platziert.\\
	\hline
	BEGRÜNDUNG: & XXX\\
	\hline
	ABHÄNGIGKEITEN: & FA-G\ref{HP} \todo[inline]{2.12 Zufall und Alternativen}\\
\end{tabularx}

\begin{tabularx}{16cm}{l|X}
	\refstepcounter{table}\label{Eigenschaften}
	\textbf{ID} & \textbf{FA-G\arabic{table}} \\
	\hline
	TITEL: & Eigenschaften \\
	\hline
	BESCHREIBUNG: & Sind entweder dauerhafte Fähigkeiten eines Charakters oder vorübergehende Zustände. Fähigkeiten kommen passiv zum Tragen oder ermöglichen dem Charakter bestimmte Aktionen. Zustände werden durch Aktionen erworben oder verloren.\\
	\hline
	BEGRÜNDUNG: & Fähigkeiten sorgen für individuelle Charaktere und Eigenschaften halten den aktuellen Zustand des Charakters fest. \\
	\hline
	ABHÄNGIGKEITEN: & FA-G\ref{Flinkheit}, FA-G\ref{Schwerfaelligkeit}, FA-G\ref{Behaendigkeit}, FA-G\ref{Behaebigkeit}, FA-G\ref{Agilitaet}, FA-G\ref{Glueckspilz}, FA-G\ref{Pechvogel}, FA-G\ref{Klamme Klamotten}, FA-G\ref{Konstant Klamme Klamotten}, FA-G\ref{Robuster Magen}, FA-G\ref{Zaehigkeit}, FA-G\ref{Babysitter}, FA-G\ref{Honey Trap}, FA-G\ref{Bang and Burn}, FA-G\ref{Flaps and Seals}, FA-G\ref{Tradecraft}, FA-G\ref{Observation} \\
\end{tabularx}

\begin{tabularx}{16cm}{l|X}
	\refstepcounter{table}\label{Inventar}
	\textbf{ID} & \textbf{FA-G\arabic{table}} \\
	\hline
	TITEL: & Inventar \\
	\hline
	BESCHREIBUNG: & Im Inventar sind alle Gadgets aufgelistet, die der Charakter aktuell bei sich trägt.\\
	\hline
	BEGRÜNDUNG: & Hält fest, welche Gadgets der Charakter nutzen kann.\\
	\hline
	ABHÄNGIGKEITEN: & \todo[inline]{2.4 Gadgets}\\
\end{tabularx}

\begin{tabularx}{16cm}{l|X}
	\refstepcounter{table}\label{haelt Cocktail}
	\textbf{ID} & \textbf{FA-G\arabic{table}} \\
	\hline
	TITEL: & hält Cocktail \\
	\hline
	BESCHREIBUNG: & Gibt an, ob der Charakter einen Cocktail in seiner Hand hält oder nicht. Ein Charakter kann maximal einen Cocktail in der Hand halten. Wird ein Cocktail in der Hand gehalten, so sind Aktionen mit diesem möglich. \\
	\hline
	BEGRÜNDUNG: & Hält fest, ob Interaktion mit einem Cocktail möglich ist oder nicht.\\
	\hline
	ABHÄNGIGKEITEN: & \todo[inline]{2.6 Cocktail}\\
\end{tabularx}

\begin{tabularx}{16cm}{l|X}
	\refstepcounter{table}\label{Spielchips}
	\textbf{ID} & \textbf{FA-G\arabic{table}} \\
	\hline
	TITEL: & Spielchips \\
	\hline
	BESCHREIBUNG: & Zu Beginn des Spiels besitzt jeder Charakter 10 Spielchips. Mit diesen kann er an Roulette-Tischen spielen und dadurch seine Anzahl an Spielchips erhöhen oder verringern.\\
	\hline
	BEGRÜNDUNG: & Werden benötigt, um Roulette zu spielen.\\
	\hline
	ABHÄNGIGKEITEN: & \todo[inline]{2.6 Roulette}\\
\end{tabularx}

\begin{tabularx}{16cm}{l|X}
	\refstepcounter{table}\label{Flinkheit}
	\textbf{ID} & \textbf{FA-G\arabic{table}} \\
	\hline
	TITEL: & Flinkheit \\
	\hline
	BESCHREIBUNG: & Besitzt ein Charakter die Fähigkeit Flinkheit, so erhält er in jeder Runde drei Bewegungspunkte.\\
	\hline
	BEGRÜNDUNG: & Unterschiedliche Anzahl an BP und AP je nach Charakter.\\
	\hline
	ABHÄNGIGKEITEN: & \\
\end{tabularx}

\begin{tabularx}{16cm}{l|X}
	\refstepcounter{table}\label{Schwerfaelligkeit}
	\textbf{ID} & \textbf{FA-G\arabic{table}} \\
	\hline
	TITEL: & Schwerfälligkeit \\
	\hline
	BESCHREIBUNG: & Besitzt ein Charakter die Fähigkeit Schwerfälligkeit, so erhält er in jeder Runde einen Bewegungspunkt.\\
	\hline
	BEGRÜNDUNG: & Unterschiedliche Anzahl an BP und AP je nach Charakter.\\
	\hline
	ABHÄNGIGKEITEN: & \\
\end{tabularx}

\begin{tabularx}{16cm}{l|X}
	\refstepcounter{table}\label{Behaendigkeit}
	\textbf{ID} & \textbf{FA-G\arabic{table}} \\
	\hline
	TITEL: & Behändigkeit \\
	\hline
	BESCHREIBUNG: & Besitzt ein Charakter die Fähigkeit Behändigkeit, so erhält er in jeder Runde zwei Aktionspunkte.\\
	\hline
	BEGRÜNDUNG: & Unterschiedliche Anzahl an BP und AP je nach Charakter.\\
	\hline
	ABHÄNGIGKEITEN: & \\
\end{tabularx}

\begin{tabularx}{16cm}{l|X}
	\refstepcounter{table}\label{Behaebigkeit}
	\textbf{ID} & \textbf{FA-G\arabic{table}} \\
	\hline
	TITEL: & Behäbigkeit \\
	\hline
	BESCHREIBUNG: & Besitzt ein Charakter die Fähigkeit Behäbigkeit, so wird ihm zu Beginn jeder Runde zufällig entweder ein Bewegungspunkt oder einer Aktionspunkt abgezogen.\\
	\hline
	BEGRÜNDUNG: & Unterschiedliche Anzahl an BP und AP je nach Charakter.\\
	\hline
	ABHÄNGIGKEITEN: & \todo[inline]{2.12 Zufall}\\
\end{tabularx}

\begin{tabularx}{16cm}{l|X}
	\refstepcounter{table}\label{Agilitaet}
	\textbf{ID} & \textbf{FA-G\arabic{table}} \\
	\hline
	TITEL: & Agilität \\
	\hline
	BESCHREIBUNG: & Besitzt ein Charakter die Fähigkeit Agilität, so wird ihm zu Beginn jeder Runde zufällig entweder ein Bewegungspunkt oder einer Aktionspunkt hinzugefügt.\\
	\hline
	BEGRÜNDUNG: & Unterschiedliche Anzahl an BP und AP je nach Charakter.\\
	\hline
	ABHÄNGIGKEITEN: & \todo[inline]{2.12 Zufall}\\
\end{tabularx}

\begin{tabularx}{16cm}{l|X}
	\refstepcounter{table}\label{Glueckspilz}
	\textbf{ID} & \textbf{FA-G\arabic{table}} \\
	\hline
	TITEL: & Glückspilz \\
	\hline
	BESCHREIBUNG: & Besitzt ein Charakter die Fähigkeit Glückspilz, so beträgt seine Gewinnchance beim Roulette $\frac{23}{37}$.\\
	\hline
	BEGRÜNDUNG: & Unterschiedliche Gewinnchancen beim Roulette je nach Charakter.\\
	\hline
	ABHÄNGIGKEITEN: & \todo[inline]{2.12 Zufall, Abhängigkeit von 2.6 Roulette}\\
\end{tabularx}

\begin{tabularx}{16cm}{l|X}
	\refstepcounter{table}\label{Pechvogel}
	\textbf{ID} & \textbf{FA-G\arabic{table}} \\
	\hline
	TITEL: & Pechvogel \\
	\hline
	BESCHREIBUNG: & Besitzt ein Charakter die Fähigkeit Pechvogel, so beträgt seine Gewinnchance beim Roulette $\frac{13}{37}$.\\
	\hline
	BEGRÜNDUNG: & Unterschiedliche Gewinnchancen beim Roulette je nach Charakter. \\
	\hline
	ABHÄNGIGKEITEN: & \todo[inline]{2.12 Zufall, Abhängigkeit von 2.6 Roulette}\\
\end{tabularx}

\begin{tabularx}{16cm}{l|X}
	\refstepcounter{table}\label{Klamme Klamotten}
	\textbf{ID} & \textbf{FA-G\arabic{table}} \\
	\hline
	TITEL: & Klamme Klamotten \\
	\hline
	BESCHREIBUNG: & Besitzt ein Charakter den Zustand Klamme Klamotten, so halbiert sich seine Erfolgswahrscheinlichkeit bei einer Wahrscheinlichkeitsprobe. \\
	\hline
	BEGRÜNDUNG: & Unterschiedliche Erfolgswahrscheinlichkeiten je nach Zustand des Charakters.\\
	\hline
	ABHÄNGIGKEITEN: & FA-G\ref{Konstant Klamme Klamotten} \todo[inline]{2.1 Kamin-Feld, 2.4 Föhn, 2.6 Cocktail, 2.12 Zufall}\\
\end{tabularx}

\begin{tabularx}{16cm}{l|X}
	\refstepcounter{table}\label{Konstant Klamme Klamotten}
	\textbf{ID} & \textbf{FA-G\arabic{table}} \\
	\hline
	TITEL: & Konstant Klamme Klamotten \\
	\hline
	BESCHREIBUNG: & Besitzt ein Charakter die Fähigkeit Konstant Klamme Klamotten, so hat er dauerhaft den Zustand Klamme Klamotten.\\
	\hline
	BEGRÜNDUNG: & Unterschiedliche Erfolgswahrscheinlichkeiten je nach Charakter.\\
	\hline
	ABHÄNGIGKEITEN: & \\
\end{tabularx}

\begin{tabularx}{16cm}{l|X}
	\refstepcounter{table}\label{Robuster Magen}
	\textbf{ID} & \textbf{FA-G\arabic{table}} \\
	\hline
	TITEL: & Robuster Magen \\
	\hline
	BESCHREIBUNG: & Besitzt ein Charakter die Fähigkeit Robuster Magen, so erhält er die doppelte Anzahl Cocktail-HP und nur den halben Gift-Cocktail-Schaden durch vergiftete Cocktails.\\
	\hline
	BEGRÜNDUNG: & Unterschiedliche Health Points je nach Charakter.\\
	\hline
	ABHÄNGIGKEITEN: & \todo[inline]{Cocktail-HP, Gift-Cocktail-Schaden}\\
\end{tabularx}

\begin{tabularx}{16cm}{l|X}
	\refstepcounter{table}\label{Zaehigkeit}
	\textbf{ID} & \textbf{FA-G\arabic{table}} \\
	\hline
	TITEL: & Zähigkeit \\
	\hline
	BESCHREIBUNG: & Besitzt ein Charakter die Fähigkeit Zähigkeit, so wird jeder Schaden, der nicht durch vergiftete Cocktails entsteht, um die Hälfte reduziert.\\
	\hline
	BEGRÜNDUNG: & Unterschiedliche Health Points je nach Charakter.\\
	\hline
	ABHÄNGIGKEITEN: & \\
\end{tabularx}

\begin{tabularx}{16cm}{l|X}
	\refstepcounter{table}\label{Babysitter}
	\textbf{ID} & \textbf{FA-G\arabic{table}} \\
	\hline
	TITEL: & Babysitter \\
	\hline
	BESCHREIBUNG: & Besitzt ein Charakter die Fähigkeit Babysitter, so wehrt er Angriffe auf benachbarte Charaktere der eigenen Fraktion mit der vorgegebenen Babysitter-Wahrscheinlichkeit ab. Bei einem Angriff wird zuerst die Wahrscheinlichkeitsprobe durch den Angreifer gemacht und wenn diese erfolgreich ist wird die Babysitter Fähigkeit eingesetzt. Ist der Charakter mit der Fähigkeit Babysitter erfolgreich, so misslingt der Angriff, ohne dass die gegnerische Fraktion etwas von der Fähigkeit erfährt, ansonsten wird der Angriff durchgeführt.\\
	\hline
	BEGRÜNDUNG: & Ermöglicht es Angriffen zu entgehen.\\
	\hline
	ABHÄNGIGKEITEN: & \todo[inline]{Babysitter-Wahrscheinlichkeit, Wahrscheinlichkeitsprobe, Abhängigkeit von 2.6 Aktion, 2.4 Gadgets}\\
\end{tabularx}

\begin{tabularx}{16cm}{l|X}
	\refstepcounter{table}\label{Honey Trap}
	\textbf{ID} & \textbf{FA-G\arabic{table}} \\
	\hline
	TITEL: & Honey Trap \\
	\hline
	BESCHREIBUNG: & Besitzt ein Charakter die Fähigkeit Honey Trap, so werden mit der vorgegebenen Honey-Trap-Wahrscheinlichkeit Angriffe anstatt auf diesen Charakter auf einen zufälligen anderen Charakter ausgeübt, insofern dieser andere Charakter ebenfalls Ziel der Aktion hätte sein können.\\
	\hline
	BEGRÜNDUNG: & Ermöglicht es Angriffen zu entgehen.\\
	\hline
	ABHÄNGIGKEITEN: & \todo[inline]{Honey-Trap-Wahrscheinlichkeit}\\
\end{tabularx}

\begin{tabularx}{16cm}{l|X}
	\refstepcounter{table}\label{Bang and Burn}
	\textbf{ID} & \textbf{FA-G\arabic{table}} \\
	\hline
	TITEL: & Bang and Burn \\
	\hline
	BESCHREIBUNG: & Besitzt ein Charakter die Fähigkeit Bang and Burn, so kann er einen benachbarten Roulette-Tisch unbrauchbar machen. Auf diesem kann dann nicht mehr gespielt werden.\\
	\hline
	BEGRÜNDUNG: & Verhindert das Sammeln von Spielchips, die später über den Sieg mitentscheiden.\\
	\hline
	ABHÄNGIGKEITEN: & \todo[inline]{Abhängigkeit von 2.1 Roulette-Tisch}\\
\end{tabularx}

\begin{tabularx}{16cm}{l|X}
\refstepcounter{table}\label{Flaps and Seals}
\textbf{ID} & \textbf{FA-G\arabic{table}} \\
\hline
TITEL: & Flaps and Seals \\
\hline
BESCHREIBUNG: & Besitzt ein Charakter die Fähigkeit Flaps and Seals, so kann er in einen Tresor spicken, der zwei Felder von ihm entfernt und somit nicht auf einem Nachbarfeld steht.\\
\hline
BEGRÜNDUNG: & Weniger Bewegungspunkte notwendig, um zu Tresor zu gelangen.\\
\hline
ABHÄNGIGKEITEN: & \todo[inline]{2.4 Maulwürfel, Abhängigkeit von 2.6 Tresor-Spicken}\\
\end{tabularx}

\begin{tabularx}{16cm}{l|X}
\refstepcounter{table}\label{Tradecraft}
\textbf{ID} & \textbf{FA-G\arabic{table}} \\
\hline
TITEL: & Tradecraft \\
\hline
BESCHREIBUNG: & Besitzt ein Charakter die Fähigkeit Tradecraft, so wiederholt er eine fehlgeschlagene Wahrscheinlichkeitsprobe einer Aktion einmal.\\
\hline
BEGRÜNDUNG: & Höhere Chance für eine erfolgreiche Aktion.\\
\hline
ABHÄNGIGKEITEN: & \todo[inline]{Wahrscheinlichkeitsprobe, 2.4 Maulwürfel}\\
\end{tabularx}

\begin{tabularx}{16cm}{l|X}
\refstepcounter{table}\label{Observation}
\textbf{ID} & \textbf{FA-G\arabic{table}} \\
\hline
TITEL: & Observation \\
\hline
BESCHREIBUNG: & Besitzt ein Charakter die Fähigkeit Observation, so kann er diese als Aktion gegen einen anderen Charakter in Sichtlinie ausführen. Dabei wird mit der vorgegebenen Observation-Erfolg-Wahrscheinlichkeit aufgedeckt, ob der Charakter zur gegnerischen Fraktion gehört oder nicht. Die Aktion bleibt vom observierten Charakter unbemerkt. \\
\hline
BEGRÜNDUNG: & Möglichkeit um herauszufinden, welche Charaktere zur gegnerischen Fraktion gehören und welche NPCs sind.\\
\hline
ABHÄNGIGKEITEN: & \todo[inline]{Observation-Erfolg-Wahrscheinlichkeit, 2.4 Maulwürfel}\\
\end{tabularx}

\begin{tabularx}{16cm}{l|X}
	\refstepcounter{table}\label{Charakter Liste}
	\textbf{ID} & \textbf{FA-G\arabic{table}} \\
	\hline
	TITEL: & Charakter Liste \\
	\hline
	BESCHREIBUNG: & Eine Liste, die alle spielbaren Charaktere mit ihren Fähigkeiten beinhaltet. Die Charaktere, die im Lastenheft unter \glqq A Einige Vorschläge für Charaktere\grqq aufgeführt sind, sollen mindestens in der Liste enthalten sein.\\
	\hline
	BEGRÜNDUNG: & Aus dieser Liste können Charaktere zum Spielen gewählt werden.\\
	\hline
	ABHÄNGIGKEITEN: & FA-G\ref{Charakter} \todo[inline]{2.8.1 Wahlphase}\\
\end{tabularx}

\begin{tabularx}{16cm}{l|X}
	\refstepcounter{table}\label{Bewegung durchführen}
	\textbf{ID} & \textbf{FA-G\arabic{table}} \\
	\hline
	TITEL: & Bewegung durchführen \\
	\hline
	BESCHREIBUNG: & Innerhalb eines Zuges kann ein Charakter, der noch mehr als einen Bewegungspunkt besitzt, eine Bewegung durchführen. Das bedeutet, dass der Charakter sich von dem Spielfeld seiner aktuellen Position auf ein angrenzendes betretbares Spielfeld bewegen kann. \\
	Jedes Spielfeld, welches mit einer Seite oder einer Ecke das Spielfeld der aktuellen Position berührt, ist ein angrenzendes Spielfeld. Dementsprechend sind Bewegungen in horizontaler, vertikaler und diagonaler Linie möglich. \\
	Wenn der Client die Anweisung gibt, den Charakter auf ein angrenzendes Feld zu bewegen, welches nicht betretbar ist, so darf der Charakter sich danach nicht auf diesem Feld befinden, sondern muss auf dem Spielfeld der aktuellen Position bleiben. In diesem Fall wurde die  Bewegung nicht erfolgreich durchgeführt. In der graphischen Darstellung kann dem Benutzer mit einer Animation angezeigt werden, dass die Bewegung nicht erfolgreich war bzw. dass es nicht möglich ist, das gewählte Spielfeld zu betreten. \\
 Bei erfolgreicher Durchführung einer Bewegung muss dem Charakter am Zug die Anzahl der Bewegungspunkte um 1 reduziert werden. Bei nicht erfolgreicher Durchführung darf sich die Anzahl der Bewegungspunkte  nicht verändern.\\
	\hline
	BEGRÜNDUNG: & Die Bedingungen für das Bewegen eines Charakters müssen eindeutig definiert sein.\\
	\hline
	ABHÄNGIGKEITEN: & FA-G\ref{Charakter} \todo[inline]{2.8.1 Wahlphase}\\
\end{tabularx}

\begin{tabularx}{16cm}{l|X}
	\refstepcounter{table}\label{Drängeln}
	\textbf{ID} & \textbf{FA-G\arabic{table}} \\
	\hline
	TITEL: & Drängeln \\
	\hline
	BESCHREIBUNG: & Wenn ein Charakter sich auf ein Feld bewegt, auf dem bereits ein anderer Charakter 		befindet, so tauschen die beiden Charaktere Plätze. Das heißt, der Charakter der die Bewegung durchgeführt hat steht auf dem Feld, auf das er sich bewegen wollte. Dies ist das Feld, auf dem der andere Charakter vor der Bewegung stand. Der andere Charakter befindet sich, nach dem Durchführen der Bewegung, auf dem Feld, auf dem der aktive Charakter ursprünglich stand. \\
	\hline
	BEGRÜNDUNG: & Um das Verhalten im Fall der Bewegung auf ein Spielfeld, welches mit einem anderen 			Charakter besetzt ist, eindeutig zu definieren.\\
	\hline
	ABHÄNGIGKEITEN: & FA-G\ref{Bewegung durchführen} \todo[inline]{2.8.1 Wahlphase}\\
\end{tabularx}

\begin{tabularx}{16cm}{l|X}
	\refstepcounter{table}\label{Aktion durchführen}
	\textbf{ID} & \textbf{FA-G\arabic{table}} \\
	\hline
	TITEL: & Aktion durchführen \\
	\hline
	BESCHREIBUNG: & Innerhalb eines Zuges kann ein Charakter, der noch mehr als einen Aktionspunkt hat und die Ausführungsbedingungen einer spezifischen Aktion erfüllt, diese spezifische Aktion ausführen. \\
	 Wenn der Client eine gültigen Aktionsbefehl eingibt, so gilt die Ausführung der Aktion als erfolgreich. Bei der Eingabe eines ungültigen Aktionsbefehls gilt die Ausführung der Aktions als nicht erfolgreich. \\
	 Bei erfolgreicher Ausführung der Aktion muss dem Charakter am Zug die Anzahl der Aktionspunkte um 1 reduziert werden und die Konsequenzen der spezifischen Aktion müssen den Zustand des Spiels entsprechend ändern. Bei nicht erfolgreicher Ausführung darf sich die Anzahl der Aktionspunkte und der Zustand des Spiels nicht ändern.  \\
	\hline
	BEGRÜNDUNG: & Es muss genau definiert sein, unter welchen Bedingungen das Ausführen einer Aktion möglich ist und welche Konsequenzen die Aktion hat.\\
	\hline
	ABHÄNGIGKEITEN: & FA-G\ref{Bewegung durchführen} \todo[inline]{2.8.1 Wahlphase}\\
\end{tabularx}


\begin{tabularx}{16cm}{l|X}
	\refstepcounter{table}\label{Gadget verwenden}
	\textbf{ID} & \textbf{FA-G\arabic{table}} \\
	\hline
	TITEL: & Gadget verwenden \\
	\hline
	BESCHREIBUNG: & Wenn sich ein verwendbares Gadget im Inventar eines Charakters befindet, dieser Charakter am Zug ist und die Ausführungsbedingungen des Gadgets für diesen Charakter erfüllt sind, so kann der Charakter als Aktion das Gadget verwenden. \\
	 
	\hline
	BEGRÜNDUNG: & Das Verwenden von Gadgets stellt eine Aktion dar.\\
	\hline
	ABHÄNGIGKEITEN: & FA-G\ref{Aktion durchführen}  \todo[inline]{2.8.1 Wahlphase}\\
\end{tabularx}

\begin{tabularx}{16cm}{l|X}
	\refstepcounter{table}\label{Roulette spielen}
	\textbf{ID} & \textbf{FA-G\arabic{table}} \\
	\hline
	TITEL: & Roulette spielen \\
	\hline
	BESCHREIBUNG: & Ein Charakter, der sich innerhalb eines Zuges auf einem Feld befindet, welches an ein Feld mit Roulette-Tisch angrenzt, der kann als Aktion einmal Roulette spielen. Der Client muss dazu einen Betrag an Spielchips eingeben. Dieser Betrag hat den Mindestwert 1 und einen Höchstwert. Der Höchstwert ist der kleinere der beiden folgenden Werte: die Anzahl der Spielchips, welche der Charakter im Moment besitzt; die Anzahl der Spielchips, die am Tisch verfügbar sind. Die eingegebene Betrag stellt den Einsatz des Charakters dar. Wenn der Charakter gewinnt, dann wird der Spielchips-Wert des Charakters um den Einsatz-Wert erhöht und der Spielchips-Wert des Roulette-Tisches um den Einsatz-Wert verringert. Wenn der Charakter verliert, dann wird der Spielchips-Wert des Charakters um den Einsatz-Wert reduziert und der Spielchipswert des Roulette-Tisches um den Einsatz-Wert erhöht. Ob ein Charakter gewinnt wird durch die Roulette-Gewinn-Wahrscheinlichkeit bestimmt.\\
	 
	\hline
	BEGRÜNDUNG: & Aufgrund der Spielmechanik ist Roulette spielen eine Aktion. Das Verhalten beim Ausführen dieser Aktion muss eindeutig definiert sein.\\
	\hline
	ABHÄNGIGKEITEN: & FA-G\ref{Aktion durchführen} \todo[inline]{2.8.1 Wahlphase}\\
\end{tabularx}

\begin{tabularx}{16cm}{l|X}
	\refstepcounter{table}\label{Cocktail aufnehmen}
	\textbf{ID} & \textbf{FA-G\arabic{table}} \\
	\hline
	TITEL: & Cocktail aufnehmen \\
	\hline
	BESCHREIBUNG: & Wenn ein Charakter am Zug ist, noch mehr als 0 Aktionpunkte hat, sich auf einem Feld befindet, welches an einen Bar-Tisch angrenzt, sich auf diesem Bartisch zu diesem Zeitpunkt ein Cocktail befindet und er selbst im Moment keinen Cocktail in der Hand hält, so kann der Charakter als Aktion diesen Cocktail aufnehmen. Danach befindet sich der Cocktail in der Hand des Charakters. Diese Information muss für alle Clients sichtbar sein.\\
	 
	\hline
	BEGRÜNDUNG: & Aufgrund der Spielmechanik ist das Aufnehmen eines Cocktails eine Aktion. Das Verhalten beim Ausführen dieser Aktion muss eindeutig definiert sein.\\
	\hline
	ABHÄNGIGKEITEN: & FA-G\ref{Aktion durchführen}  \todo[inline]{2.8.1 Wahlphase}\\
\end{tabularx}