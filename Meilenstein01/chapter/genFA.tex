%Hier kommen die generellen FA rein

\begin{tabularx}{16cm}{l|X}
	\refstepcounter{table}\label{Felder}
	\textbf{ID} & \textbf{FA-G \arabic{table}} \\
	\hline
	TITEL: & Felder \\
	\hline
	BESCHREIBUNG: & Die schachbrettartigen Felder, aus denen das Spielbrett aufgebaut ist, können freier Raum sein, oder mit Hindernissen oder Objekten besetzt sein. \\
	\hline
	BEGRÜNDUNG: & Unterschiedliche Arten von Feldern erlauben den Charakteren das Fortbewegen und interagieren auf dem Spielbrett. \\
	\hline
	ABHÄNGIGKEITEN: & FA-G \ref{Spielbrett} \\
\end{tabularx}

\begin{tabularx}{16cm}{l|X}
	\refstepcounter{table}\label{Spielbrett}
	\textbf{ID} & \textbf{FA-G \arabic{table}} \\
	\hline
	TITEL: & Spielbrett \\
	\hline
	BESCHREIBUNG: & Das Spiel findet auf einem Spielbrett statt, welches aus den Feldern eines kartesischen Gitters aufgebaut ist. \\
	\hline
	BEGRÜNDUNG: & Der Aufbau aus Feldern in einem kartesischen Gitter ermöglicht die Fortbewegung der Charaktere durch Bewegungspunkte. \\
	\hline
	ABHÄNGIGKEITEN: & FA-G \ref{Felder} \\
\end{tabularx}

\begin{tabularx}{16cm}{l|X}
	\refstepcounter{table}\label{Entfernung}
	\textbf{ID} & \textbf{FA-G \arabic{table}} \\
	\hline
	TITEL: & Entfernung \\
	\hline
	BESCHREIBUNG: & Die Entfernung zwischen zwei Feldern A und B ist die minimale Anzahl an Schritten auf Nachbarfelder (in alle acht Richtungen), die benötigt wird um von A nach B zu gelangen. \\
	\hline
	BEGRÜNDUNG: & Mit der Entfernung zwischen zwei Feldern A und B wird berechnet, wieviele Bewegungspunkte ein Charakter benötigt, um von Feld A zu Feld B zu gelangen. \\
	\hline
	ABHÄNGIGKEITEN: & FA-G \ref{Felder}, FA-G \ref{Spielbrett} \\
\end{tabularx}

\begin{tabularx}{16cm}{l|X}
	\refstepcounter{table}\label{Sichtlinie}
	\textbf{ID} & \textbf{FA-G \arabic{table}} \\
	\hline
	TITEL: & Sichtlinie \\
	\hline
	BESCHREIBUNG: & Es existiert eine Sichtlinie zwischen zwei Feldern A und B, wenn gilt: Betrachtet man die Verbindungslinie vom Mittelpunkt von A zum Mittelpunkt von B, so müssen alle Felder, die von dieser Verbindunglinie geschnitten werden Felder sein, die die Sichtlinie nicht blockieren. Dann ist B von A aus sichtbar. Felder, die von dieser Verbindungslinie nur tangiert werden blockieren die Sicht nicht. \\
	\hline
	BEGRÜNDUNG: & Dadurch wird festgestellt, ob ein Feld B von einem Feld A aus sichtbar ist, also ob ein Charakter auf Feld A sehen kann, was sich auf Feld B befindet. \\
	\hline
	ABHÄNGIGKEITEN: & FA-G \ref{Felder}, FA-G \ref{Spielbrett} \\
\end{tabularx}

\begin{tabularx}{16cm}{l|X}
	\refstepcounter{table}\label{Freies Feld}
	\textbf{ID} & \textbf{FA-G \arabic{table}} \\
	\hline
	TITEL: & Freies Feld \\
	\hline
	BESCHREIBUNG: & Charaktere können auf einem Freien Feld stehen oder darüber hinweglaufen. Es blockiert die Sichtlinie nicht. \\
	\hline
	BEGRÜNDUNG: & Freie Felder dienen der Fortbewegung der Charaktere. \\
	\hline
	ABHÄNGIGKEITEN: & FA-G \ref{Felder}, FA-G \ref{Sichtlinie} \\
\end{tabularx}

\begin{tabularx}{16cm}{l|X}
	\refstepcounter{table}\label{Wand}
	\textbf{ID} & \textbf{FA-G \arabic{table}} \\
	\hline
	TITEL: & Wand \\
	\hline
	BESCHREIBUNG: & Ein Feld, auf welchem sich eine Wand befindet ist nicht betretbar und es blockiert die Sichtlinie. \\
	\hline
	BEGRÜNDUNG: & Durch Wände können große Räume getrennt werden oder neue Räume entstehen. \\
	\hline
	ABHÄNGIGKEITEN: & FA-G \ref{Felder}, FA-G \ref{Sichtlinie} \\
\end{tabularx}

\begin{tabularx}{16cm}{l|X}
	\refstepcounter{table}\label{Kamin-Feld}
	\textbf{ID} & \textbf{FA-G \arabic{table}} \\
	\hline
	TITEL: & Kamin-Feld \\
	\hline
	BESCHREIBUNG: & Kamin-Felder sind nicht betretbar und sie blockieren die Sichtlinie. Befindet sich ein Charakter mit klammen Klamotten am Anfang einer Runde auf einem Nachbarfeld eines Kamin-Feldes, so wird die Klamme Klamotten-Eigenschaft am Ende der Runde entfernt. \\
	\hline
	BEGRÜNDUNG: & Der Kamin trocknet die klammen Klamotten, dadurch kann ein Charakter diese Eigenschaft los werden. \\
	\hline
	ABHÄNGIGKEITEN: & FA-G \ref{Felder}, FA-G \ref{Sichtlinie} \\
\end{tabularx}

\begin{tabularx}{16cm}{l|X}
	\refstepcounter{table}\label{Sitzplatz}
	\textbf{ID} & \textbf{FA-G \arabic{table}} \\
	\hline
	TITEL: & Sitzplatz \\
	\hline
	BESCHREIBUNG: & Auf einem solchen Feld befindet sich eine Sitzgelegenheit, bspw. ein Sessel, Barhocker, etc. Das Feld ist betretbar und blockiert die Sichtlinie nicht. Befindet sich ein Charakter am Anfang einer Runde auf einem Sitzplatz-Feld, so werden seine Health Points am Ende der Runde auf den Maximalwert aufgefüllt. \\
	\hline
	BEGRÜNDUNG: & Dadurch können Charaktere mit wenigen Health Points diese wieder füllen. \\
	\hline
	ABHÄNGIGKEITEN: & FA-G \ref{Felder}, FA-G \ref{Sichtlinie} \\
\end{tabularx}

\begin{tabularx}{16cm}{l|X}
	\refstepcounter{table}\label{Spielchips}
	\textbf{ID} & \textbf{FA-G \arabic{table}} \\
	\hline
	TITEL: & Spielchips \\
	\hline
	BESCHREIBUNG: & Charaktere haben die Möglichkeit an Roulette-Tischen Spielchips zu gewinnen. Alle gesammelten Spielchips einer Fraktion werden am Ende einer Partie in Intelligence Points umgerechnet. \\
	\hline
	BEGRÜNDUNG: & Dadurch wird das Spiel spannender, da die Spielchips den Spielausgang beeinflussen. \\
	\hline
	ABHÄNGIGKEITEN: & FA-G \ref{Roulette-Tisch} \\
\end{tabularx}

\begin{tabularx}{16cm}{l|X}
	\refstepcounter{table}\label{Roulette-Tisch}
	\textbf{ID} & \textbf{FA-G \arabic{table}} \\
	\hline
	TITEL: & Roulette-Tisch \\
	\hline
	BESCHREIBUNG: & Ein Feld, auf welchem ein Roulette-Tisch steht ist nicht betretbar und es blockiert die Sichtlinie nicht. Wenn ein Charakter auf einem Nachbarfeld eines Roulette-Tisches steht, kann er als Aktion einmal Roulette spielen. Jeder Roulette-Tisch verfügt zu Beginn einer Partie über eine im Szenario festgelegte individuelle Anzahl an Chips. \\
	\hline
	BEGRÜNDUNG: & An Roulette-Tischen haben Charaktere die Chance Chips zu gewinnen bzw. zu verlieren. \\
	\hline
	ABHÄNGIGKEITEN: & FA-G \ref{Felder}, FA-G \ref{Sichtlinie} \\
\end{tabularx}

\begin{tabularx}{16cm}{l|X}
	\refstepcounter{table}\label{Bar-Tisch}
	\textbf{ID} & \textbf{FA-G \arabic{table}} \\
	\hline
	TITEL: & Bar-Tisch \\
	\hline
	BESCHREIBUNG: & Felder, auf denen sich ein Bar-Tisch befindet sind nicht betretbar und sie blockieren die Sichtlinie nicht. Zu Beginn jeder Runde erscheint auf jedem leeren Bar-Tisch ein Cocktail. \\
	\hline
	BEGRÜNDUNG: & An Bar-Tischen können die Charaktere Cocktails aufnehmen. \\
	\hline
	ABHÄNGIGKEITEN: & FA-G \ref{Felder}, FA-G \ref{Cocktail aufnehmen}, FA-G \ref{Sichtlinie} \\
\end{tabularx}

\begin{tabularx}{16cm}{l|X}
	\refstepcounter{table}\label{Tresor}
	\textbf{ID} & \textbf{FA-G \arabic{table}} \\
	\hline
	TITEL: & Tresor \\
	\hline
	BESCHREIBUNG: & Felder, auf denen sich ein Tresor befindet, sind nicht betretbar und sie blockieren die Sichtlinie nicht. Tresore können Geheiminformationen oder Gadgets enthalten. Die Tresore sind eindeutig und sichtbar durchnummeriert (1, 2, ...). \\
	\hline
	BEGRÜNDUNG: & An Tresoren können Charaktere Geheimnisse oder Gadgets erhalten. \\
	\hline
	ABHÄNGIGKEITEN: & FA-G \ref{Felder}, FA-G \ref{Sichtlinie},  \todo[inline]{2.4 Gadgets, 2.7 Geheimnisse} \\
\end{tabularx}

\begin{tabularx}{16cm}{l|X}
	\refstepcounter{table}\label{Charakter}
	\textbf{ID} & \textbf{FA-G \arabic{table}} \\
	\hline
	TITEL: & Charakter \\
	\hline
	BESCHREIBUNG: & Ein Charakter besitzt folgende Werte, die normalerweise für die gegnerische Fraktion nicht sichtbar sind: 
	\begin{itemize}
		\item Name
		\item Beschreibung
		\item Position
		\item Fraktion
		\item Bewegungspunkte (BP) und Aktionspunkte (AP)
		\item Health Points (HP)
		\item Intelligence Points (IP)
		\item Eigenschaften
		\item Inventar
		\item hält Cocktail
		\item Spielchips
	\end{itemize}
	\\
	\hline
	BEGRÜNDUNG: & Jeder Charakter soll individuell sein und sich über das Spiel hinweg verändern können.\\
	\hline
	ABHÄNGIGKEITEN: & FA-G \ref{Charakter-NameBeschreibung}, FA-G \ref{Charakter-Position}, FA-G \ref{Charakter-Fraktion},  FA-G \ref{BP und AP}, FA-G \ref{HP}, FA-G \ref{IP}, FA-G \ref{Eigenschaften}, FA-G \ref{Inventar}, FA-G \ref{haelt Cocktail}, FA-G \ref{Spielchips} \\
\end{tabularx}

\begin{tabularx}{16cm}{l|X}
	\refstepcounter{table}\label{Charakter-NameBeschreibung}
	\textbf{ID} & \textbf{FA-G \arabic{table}} \\
	\hline
	TITEL: & Charakter-Name und -Beschreibung \\
	\hline
	BESCHREIBUNG: & Jeder Charakter hat einen eindeutigen Namen und die Charakter-Beschreibung beschreibt den Charakter aus \glqq James Bond\grqq in wenigen Sätzen.\\
	\hline
	BEGRÜNDUNG: & Zusätzliche Informationen für den Spieler. \\
	\hline
	ABHÄNGIGKEITEN: & \\
\end{tabularx}

\begin{tabularx}{16cm}{l|X}
	\refstepcounter{table}\label{Charakter-Position}
	\textbf{ID} & \textbf{FA-G \arabic{table}} \\
	\hline
	TITEL: & Charakter-Position \\
	\hline
	BESCHREIBUNG: & Gibt die Position des Charakters auf dem Spielfeld in x- und y-Koordinaten an.\\
	\hline
	BEGRÜNDUNG: & Hält fest, wo sich der Charakter auf dem Spielfeld befindet.\\
	\hline
	ABHÄNGIGKEITEN: & FA-G \ref{Exfiltratation},  \todo[inline]{Spielfeld, 2.9. Beginn der Partie, 2.5 Bewegung, }\\
\end{tabularx}

\begin{tabularx}{16cm}{l|X}
	\refstepcounter{table}\label{Charakter-Fraktion}
	\textbf{ID} & \textbf{FA-G \arabic{table}} \\
	\hline
	TITEL: & Charakter-Fraktion \\
	\hline
	BESCHREIBUNG: & Gibt an, zu welcher Fraktion der Charakter gehört. Mögliche Fraktionen sind Spieler1, Spieler2 oder NPC.\\
	\hline
	BEGRÜNDUNG: & Hält fest, von wem der Charakter zu steuern ist.\\
	\hline
	ABHÄNGIGKEITEN: & \todo[inline]{2.8.1 Wahlphase}\\
\end{tabularx}

\begin{tabularx}{16cm}{l|X}
	\refstepcounter{table}\label{BP und AP}
	\textbf{ID} & \textbf{FA-G \arabic{table}} \\
	\hline
	TITEL: & Bewegungspunkte (BP) und Aktionspunkte (AP) \\
	\hline
	BESCHREIBUNG: & Während eine Charakter am Zug ist, kann er BP für Bewegungen auf dem Spielfeld und AP für Aktionen einsetzen.
	Zu Beginn eines Zuges erhält er jeweils Punkte. Nach Beenden eines Zuges verfallen übrig gebliebene Punkte. BP und AP können nicht negativ sein.\\
	\hline
	BEGRÜNDUNG: & Hält fest, wie viele Bewegungen und Aktionen der Charakter in diesem Spielzug noch ausführen kann. \\
	\hline
	ABHÄNGIGKEITEN: & FA-G \ref{Flinkheit}, FA-G \ref{Schwerfaelligkeit}, FA-G \ref{Behaendigkeit}, FA-G \ref{Behaebigkeit}, FA-G \ref{Agilitaet} \todo[inline]{2.10.1 Züge, 2.5 Bewegung, 2.6 Aktionen}\\
\end{tabularx}

\begin{tabularx}{16cm}{l|X}
	\refstepcounter{table}\label{HP}
	\textbf{ID} & \textbf{FA-G \arabic{table}} \\
	\hline
	TITEL: & Health Points (HP) \\
	\hline
	BESCHREIBUNG: & HP geben Auskunft über den Gesundheitszustand des Charakters. Zu Spielbeginn werden die HP auf 100 gesetzt. Während des Spiels sorgen verschiedene Aktionen dafür, dass HP hinzugefügt oder abgezogen werden. Werden HP abgezogen, spricht man von Schaden. Die Punkte nehmen Werte zwischen 0 und 100 an.\\
	\hline
	BEGRÜNDUNG: & Zeigt Auswirkung verschiedener Aktionen auf den Charakter. \\
	\hline
	ABHÄNGIGKEITEN: & FA-G \ref{Robuster Magen}, FA-G \ref{Zaehigkeit} \todo[inline]{2.6 Aktionen, 2.4 Gadgets}\\
\end{tabularx}

\begin{tabularx}{16cm}{l|X}
	\refstepcounter{table}\label{Exfiltratation}
	\textbf{ID} & \textbf{FA-G \arabic{table}} \\
	\hline
	TITEL: & Exfiltration \\
	\hline
	BESCHREIBUNG: & Sinken die HP eines Charakter auf 0, so wird eine Exfiltration durchgeführt. Dabei wird der Charakter auf ein zufällig gewähltes freies Sitzplatz-Feld versetzt und seine HP auf 1 gesetzt. Ist kein freier Sitzplatz vorhanden, so wird ein Sitzplatz zufällig ausgewählt und die Person, die dort saß, wird auf ein zufälliges freies Nachbarfeld des Sitzplatzes platziert.\\
	\hline
	BEGRÜNDUNG: & HP sollen nicht 0 sein.\\
	\hline
	ABHÄNGIGKEITEN: & FA-G \ref{HP} \todo[inline]{2.12 Zufall und Alternativen}\\
\end{tabularx}

\begin{tabularx}{16cm}{l|X}
	\refstepcounter{table}\label{IP}
	\textbf{ID} & \textbf{FA-G \arabic{table}} \\
	\hline
	TITEL: & Intelligence Points (IP) \\
	\hline
	BESCHREIBUNG: & IP geben Auskunft, über die Spionagefähigkeiten des Charakter. Zu Beginn besitzen die IP den Wert 0.\\
	\hline
	BEGRÜNDUNG: & Punkte, die später für Sieg relevant sind.\\
	\hline
	ABHÄNGIGKEITEN: & \todo[inline]{2.4. Gaspatronen-Lippenstift, 2.4 Wanze und Ohrstöpsel, 2.4. Chicken Feed, 2.6 Roulette?, 2.7 Geheimnisse, Abhängigkeit von 2.11}\\
\end{tabularx}


\begin{tabularx}{16cm}{l|X}
	\refstepcounter{table}\label{Eigenschaften}
	\textbf{ID} & \textbf{FA-G \arabic{table}} \\
	\hline
	TITEL: & Eigenschaften \\
	\hline
	BESCHREIBUNG: & Sind entweder dauerhafte Fähigkeiten eines Charakters oder vorübergehende Zustände. Fähigkeiten kommen passiv zum Tragen oder ermöglichen dem Charakter bestimmte Aktionen. Zustände werden durch Aktionen erworben oder verloren.\\
	\hline
	BEGRÜNDUNG: & Fähigkeiten sorgen für individuelle Charaktere und Eigenschaften halten den aktuellen Zustand des Charakters fest. \\
	\hline
	ABHÄNGIGKEITEN: & FA-G \ref{Flinkheit}, FA-G \ref{Schwerfaelligkeit}, FA-G \ref{Behaendigkeit}, FA-G \ref{Behaebigkeit}, FA-G \ref{Agilitaet}, FA-G \ref{Glueckspilz}, FA-G \ref{Pechvogel}, FA-G \ref{Klamme Klamotten}, FA-G \ref{Konstant Klamme Klamotten}, FA-G \ref{Robuster Magen}, FA-G \ref{Zaehigkeit}, FA-G \ref{Babysitter}, FA-G \ref{Honey Trap}, FA-G \ref{Bang and Burn}, FA-G \ref{Flaps and Seals}, FA-G \ref{Tradecraft}, FA-G \ref{Observation} \\
\end{tabularx}

\begin{tabularx}{16cm}{l|X}
	\refstepcounter{table}\label{Inventar}
	\textbf{ID} & \textbf{FA-G \arabic{table}} \\
	\hline
	TITEL: & Inventar \\
	\hline
	BESCHREIBUNG: & Im Inventar sind alle Gadgets aufgelistet, die der Charakter aktuell bei sich trägt.\\
	\hline
	BEGRÜNDUNG: & Hält fest, welche Gadgets der Charakter nutzen kann.\\
	\hline
	ABHÄNGIGKEITEN: & \todo[inline]{2.4 Gadgets}\\
\end{tabularx}

\begin{tabularx}{16cm}{l|X}
	\refstepcounter{table}\label{haelt Cocktail}
	\textbf{ID} & \textbf{FA-G \arabic{table}} \\
	\hline
	TITEL: & hält Cocktail \\
	\hline
	BESCHREIBUNG: & Gibt an, ob der Charakter einen Cocktail in seiner Hand hält oder nicht. Ein Charakter kann maximal einen Cocktail in der Hand halten. Wird ein Cocktail in der Hand gehalten, so sind Aktionen mit diesem möglich. \\
	\hline
	BEGRÜNDUNG: & Hält fest, ob Interaktion mit einem Cocktail möglich ist oder nicht.\\
	\hline
	ABHÄNGIGKEITEN: & \todo[inline]{2.6 Cocktail}\\
\end{tabularx}

\begin{tabularx}{16cm}{l|X}
	\refstepcounter{table}\label{Spielchips}
	\textbf{ID} & \textbf{FA-G \arabic{table}} \\
	\hline
	TITEL: & Spielchips \\
	\hline
	BESCHREIBUNG: & Zu Beginn des Spiels besitzt jeder Charakter 10 Spielchips. Mit diesen kann er an Roulette-Tischen spielen und dadurch seine Anzahl an Spielchips erhöhen oder verringern.\\
	\hline
	BEGRÜNDUNG: & Werden benötigt, um Roulette zu spielen.\\
	\hline
	ABHÄNGIGKEITEN: & \todo[inline]{2.6 Roulette}\\
\end{tabularx}

\begin{tabularx}{16cm}{l|X}
	\refstepcounter{table}\label{Flinkheit}
	\textbf{ID} & \textbf{FA-G \arabic{table}} \\
	\hline
	TITEL: & Flinkheit \\
	\hline
	BESCHREIBUNG: & Besitzt ein Charakter die Fähigkeit Flinkheit, so erhält er in jeder Runde drei Bewegungspunkte.\\
	\hline
	BEGRÜNDUNG: & Unterschiedliche Anzahl an BP und AP je nach Charakter.\\
	\hline
	ABHÄNGIGKEITEN: & \\
\end{tabularx}

\begin{tabularx}{16cm}{l|X}
	\refstepcounter{table}\label{Schwerfaelligkeit}
	\textbf{ID} & \textbf{FA-G \arabic{table}} \\
	\hline
	TITEL: & Schwerfälligkeit \\
	\hline
	BESCHREIBUNG: & Besitzt ein Charakter die Fähigkeit Schwerfälligkeit, so erhält er in jeder Runde einen Bewegungspunkt.\\
	\hline
	BEGRÜNDUNG: & Unterschiedliche Anzahl an BP und AP je nach Charakter.\\
	\hline
	ABHÄNGIGKEITEN: & \\
\end{tabularx}

\begin{tabularx}{16cm}{l|X}
	\refstepcounter{table}\label{Behaendigkeit}
	\textbf{ID} & \textbf{FA-G \arabic{table}} \\
	\hline
	TITEL: & Behändigkeit \\
	\hline
	BESCHREIBUNG: & Besitzt ein Charakter die Fähigkeit Behändigkeit, so erhält er in jeder Runde zwei Aktionspunkte.\\
	\hline
	BEGRÜNDUNG: & Unterschiedliche Anzahl an BP und AP je nach Charakter.\\
	\hline
	ABHÄNGIGKEITEN: & \\
\end{tabularx}

\begin{tabularx}{16cm}{l|X}
	\refstepcounter{table}\label{Behaebigkeit}
	\textbf{ID} & \textbf{FA-G \arabic{table}} \\
	\hline
	TITEL: & Behäbigkeit \\
	\hline
	BESCHREIBUNG: & Besitzt ein Charakter die Fähigkeit Behäbigkeit, so wird ihm zu Beginn jeder Runde zufällig entweder ein Bewegungspunkt oder einer Aktionspunkt abgezogen.\\
	\hline
	BEGRÜNDUNG: & Unterschiedliche Anzahl an BP und AP je nach Charakter.\\
	\hline
	ABHÄNGIGKEITEN: & \todo[inline]{2.12 Zufall}\\
\end{tabularx}

\begin{tabularx}{16cm}{l|X}
	\refstepcounter{table}\label{Agilitaet}
	\textbf{ID} & \textbf{FA-G \arabic{table}} \\
	\hline
	TITEL: & Agilität \\
	\hline
	BESCHREIBUNG: & Besitzt ein Charakter die Fähigkeit Agilität, so wird ihm zu Beginn jeder Runde zufällig entweder ein Bewegungspunkt oder einer Aktionspunkt hinzugefügt.\\
	\hline
	BEGRÜNDUNG: & Unterschiedliche Anzahl an BP und AP je nach Charakter.\\
	\hline
	ABHÄNGIGKEITEN: & \todo[inline]{2.12 Zufall}\\
\end{tabularx}

\begin{tabularx}{16cm}{l|X}
	\refstepcounter{table}\label{Glueckspilz}
	\textbf{ID} & \textbf{FA-G \arabic{table}} \\
	\hline
	TITEL: & Glückspilz \\
	\hline
	BESCHREIBUNG: & Besitzt ein Charakter die Fähigkeit Glückspilz, so beträgt seine Gewinnchance beim Roulette $\frac{23}{37}$.\\
	\hline
	BEGRÜNDUNG: & Unterschiedliche Gewinnchancen beim Roulette je nach Charakter.\\
	\hline
	ABHÄNGIGKEITEN: & \todo[inline]{2.12 Zufall, Abhängigkeit von 2.6 Roulette}\\
\end{tabularx}

\begin{tabularx}{16cm}{l|X}
	\refstepcounter{table}\label{Pechvogel}
	\textbf{ID} & \textbf{FA-G \arabic{table}} \\
	\hline
	TITEL: & Pechvogel \\
	\hline
	BESCHREIBUNG: & Besitzt ein Charakter die Fähigkeit Pechvogel, so beträgt seine Gewinnchance beim Roulette $\frac{13}{37}$.\\
	\hline
	BEGRÜNDUNG: & Unterschiedliche Gewinnchancen beim Roulette je nach Charakter. \\
	\hline
	ABHÄNGIGKEITEN: & \todo[inline]{2.12 Zufall, Abhängigkeit von 2.6 Roulette}\\
\end{tabularx}

\begin{tabularx}{16cm}{l|X}
	\refstepcounter{table}\label{Klamme Klamotten}
	\textbf{ID} & \textbf{FA-G \arabic{table}} \\
	\hline
	TITEL: & Klamme Klamotten \\
	\hline
	BESCHREIBUNG: & Besitzt ein Charakter den Zustand Klamme Klamotten, so halbiert sich seine Erfolgswahrscheinlichkeit bei einer Wahrscheinlichkeitsprobe. \\
	\hline
	BEGRÜNDUNG: & Unterschiedliche Erfolgswahrscheinlichkeiten je nach Zustand des Charakters.\\
	\hline
	ABHÄNGIGKEITEN: & FA-G \ref{Konstant Klamme Klamotten} \todo[inline]{2.1 Kamin-Feld, 2.4 Föhn, 2.6 Cocktail, 2.12 Zufall}\\
\end{tabularx}

\begin{tabularx}{16cm}{l|X}
	\refstepcounter{table}\label{Konstant Klamme Klamotten}
	\textbf{ID} & \textbf{FA-G \arabic{table}} \\
	\hline
	TITEL: & Konstant Klamme Klamotten \\
	\hline
	BESCHREIBUNG: & Besitzt ein Charakter die Fähigkeit Konstant Klamme Klamotten, so hat er dauerhaft den Zustand Klamme Klamotten.\\
	\hline
	BEGRÜNDUNG: & Unterschiedliche Erfolgswahrscheinlichkeiten je nach Charakter.\\
	\hline
	ABHÄNGIGKEITEN: & \\
\end{tabularx}

\begin{tabularx}{16cm}{l|X}
	\refstepcounter{table}\label{Robuster Magen}
	\textbf{ID} & \textbf{FA-G \arabic{table}} \\
	\hline
	TITEL: & Robuster Magen \\
	\hline
	BESCHREIBUNG: & Besitzt ein Charakter die Fähigkeit Robuster Magen, so erhält er die doppelte Anzahl Cocktail-HP und nur den halben Gift-Cocktail-Schaden durch vergiftete Cocktails.\\
	\hline
	BEGRÜNDUNG: & Unterschiedliche Health Points je nach Charakter.\\
	\hline
	ABHÄNGIGKEITEN: & \todo[inline]{Cocktail-HP, Gift-Cocktail-Schaden}\\
\end{tabularx}

\begin{tabularx}{16cm}{l|X}
	\refstepcounter{table}\label{Zaehigkeit}
	\textbf{ID} & \textbf{FA-G \arabic{table}} \\
	\hline
	TITEL: & Zähigkeit \\
	\hline
	BESCHREIBUNG: & Besitzt ein Charakter die Fähigkeit Zähigkeit, so wird jeder Schaden, der nicht durch vergiftete Cocktails entsteht, um die Hälfte reduziert.\\
	\hline
	BEGRÜNDUNG: & Unterschiedliche Health Points je nach Charakter.\\
	\hline
	ABHÄNGIGKEITEN: & \\
\end{tabularx}

\begin{tabularx}{16cm}{l|X}
	\refstepcounter{table}\label{Babysitter}
	\textbf{ID} & \textbf{FA-G \arabic{table}} \\
	\hline
	TITEL: & Babysitter \\
	\hline
	BESCHREIBUNG: & Besitzt ein Charakter die Fähigkeit Babysitter, so wehrt er Angriffe auf benachbarte Charaktere der eigenen Fraktion mit der vorgegebenen Babysitter-Wahrscheinlichkeit ab. Bei einem Angriff wird zuerst die Wahrscheinlichkeitsprobe durch den Angreifer gemacht und wenn diese erfolgreich ist wird die Babysitter Fähigkeit eingesetzt. Ist der Charakter mit der Fähigkeit Babysitter erfolgreich, so misslingt der Angriff, ohne dass die gegnerische Fraktion etwas von der Fähigkeit erfährt, ansonsten wird der Angriff durchgeführt.\\
	\hline
	BEGRÜNDUNG: & Ermöglicht es Angriffen zu entgehen.\\
	\hline
	ABHÄNGIGKEITEN: & \todo[inline]{Babysitter-Wahrscheinlichkeit, Wahrscheinlichkeitsprobe, Abhängigkeit von 2.6 Aktion, 2.4 Gadgets}\\
\end{tabularx}

\begin{tabularx}{16cm}{l|X}
	\refstepcounter{table}\label{Honey Trap}
	\textbf{ID} & \textbf{FA-G \arabic{table}} \\
	\hline
	TITEL: & Honey Trap \\
	\hline
	BESCHREIBUNG: & Besitzt ein Charakter die Fähigkeit Honey Trap, so werden mit der vorgegebenen Honey-Trap-Wahrscheinlichkeit Angriffe anstatt auf diesen Charakter auf einen zufälligen anderen Charakter ausgeübt, insofern dieser andere Charakter ebenfalls Ziel der Aktion hätte sein können.\\
	\hline
	BEGRÜNDUNG: & Ermöglicht es Angriffen zu entgehen.\\
	\hline
	ABHÄNGIGKEITEN: & \todo[inline]{Honey-Trap-Wahrscheinlichkeit}\\
\end{tabularx}

\begin{tabularx}{16cm}{l|X}
	\refstepcounter{table}\label{Bang and Burn}
	\textbf{ID} & \textbf{FA-G \arabic{table}} \\
	\hline
	TITEL: & Bang and Burn \\
	\hline
	BESCHREIBUNG: & Besitzt ein Charakter die Fähigkeit Bang and Burn, so kann er einen benachbarten Roulette-Tisch unbrauchbar machen. Auf diesem kann dann nicht mehr gespielt werden.\\
	\hline
	BEGRÜNDUNG: & Verhindert das Sammeln von Spielchips, die später über den Sieg mitentscheiden.\\
	\hline
	ABHÄNGIGKEITEN: & \todo[inline]{Abhängigkeit von 2.1 Roulette-Tisch}\\
\end{tabularx}

\begin{tabularx}{16cm}{l|X}
\refstepcounter{table}\label{Flaps and Seals}
\textbf{ID} & \textbf{FA-G \arabic{table}} \\
\hline
TITEL: & Flaps and Seals \\
\hline
BESCHREIBUNG: & Besitzt ein Charakter die Fähigkeit Flaps and Seals, so kann er in einen Tresor spicken, der zwei Felder von ihm entfernt und somit nicht auf einem Nachbarfeld steht.\\
\hline
BEGRÜNDUNG: & Weniger Bewegungspunkte notwendig, um zu Tresor zu gelangen.\\
\hline
ABHÄNGIGKEITEN: & \todo[inline]{2.4 Maulwürfel, Abhängigkeit von 2.6 Tresor-Spicken}\\
\end{tabularx}


\begin{tabularx}{16cm}{l|X}
\refstepcounter{table}\label{Tradecraft}
\textbf{ID} & \textbf{FA-G \arabic{table}} \\
\hline
TITEL: & Tradecraft \\
\hline
BESCHREIBUNG: & Besitzt ein Charakter die Fähigkeit Tradecraft, so wiederholt er eine fehlgeschlagene Wahrscheinlichkeitsprobe einer Aktion einmal.\\
\hline
BEGRÜNDUNG: & Höhere Chance für eine erfolgreiche Aktion.\\
\hline
ABHÄNGIGKEITEN: & \todo[inline]{Wahrscheinlichkeitsprobe, 2.4 Maulwuerfel}\\
\end{tabularx}

\begin{tabularx}{16cm}{l|X}
\refstepcounter{table}\label{Observation}
\textbf{ID} & \textbf{FA-G \arabic{table}} \\
\hline
TITEL: & Observation \\
\hline
BESCHREIBUNG: & Besitzt ein Charakter die Fähigkeit Observation, so kann er diese als Aktion gegen einen anderen Charakter in Sichtlinie ausführen. Dabei wird mit der vorgegebenen Observation-Erfolg-Wahrscheinlichkeit aufgedeckt, ob der Charakter zur gegnerischen Fraktion gehört oder nicht. Die Aktion bleibt vom observierten Charakter unbemerkt. \\
\hline
BEGRÜNDUNG: & Möglichkeit um herauszufinden, welche Charaktere zur gegnerischen Fraktion gehören und welche NPCs sind.\\
\hline
ABHÄNGIGKEITEN: & \todo[inline]{Observation-Erfolg-Wahrscheinlichkeit, 2.4 Maulwuerfel}\\
\end{tabularx}

\begin{tabularx}{16cm}{l|X}
	\refstepcounter{table}\label{Gadgets}
	\textbf{ID} & \textbf{FA-G \arabic{table}} \\
	\hline
	TITEL: & Gadgets \\
	\hline
	BESCHREIBUNG: & Charaktere können Gadgets in ihrem Inventar bei sich tragen. Sie verschaffen ihrem Besitzer bestimmte Eigenschaften oder ermöglichen bestimmte Aktionen. Jedes Gadget kommt höchstens einmal im Spiel vor und kann, falls nicht anders festgelegt, mehrmals verwendet werden. \\
	\hline
	BEGRÜNDUNG: & Dadurch wird das Spiel spannender, denn die Gadgets beeinflussen den Spielverlauf. \\
	\hline
	ABHÄNGIGKEITEN: & FA-G \ref{Charakter}, FA-G \ref{Inventar}, FA-G \ref{Eigenschaften}, FA-G \ref{Aktion durchfuehren} \\
\end{tabularx}

\begin{tabularx}{16cm}{l|X}
	\refstepcounter{table}\label{Akku-Foehn}
	\textbf{ID} & \textbf{FA-G \arabic{table}} \\
	\hline
	TITEL: & Akku-Föhn \\
	\hline
	BESCHREIBUNG: & Wird der Akku-Föhn von einem Charakter als Aktion an sich selbst oder einem benachbarten Charakter angewendet, so verliert dieser Ziel-Charakter die Klamme Klamotten-Eigenschaft. Der Akku-Föhn kann beliebig oft eingesetzt werden. \\
	\hline
	BEGRÜNDUNG: & Das ermöglicht es den Charakteren die Klamme Klamotten-Eigenschaft loszuwerden.  \\
	\hline
	ABHÄNGIGKEITEN: & FA-G \ref{Gadgets}, FA-G \ref{Charakter}, FA-G \ref{Aktion durchfuehren} \\
\end{tabularx}

\begin{tabularx}{16cm}{l|X}
	\refstepcounter{table}\label{Maulwuerfel}
	\textbf{ID} & \textbf{FA-G \arabic{table}} \\
	\hline
	TITEL: & Maulwürfel \\
	\hline
	BESCHREIBUNG: & Hat ein Charakter den Maulwürfel im Inventar, so werden die Eigenschaften Tradecraft, Flaps and Seals und Observation deaktiviert, falls er sie hat. Er verliert diese Eigenschaften vorübergehend und bekommt sie wieder, wenn er den Maulwürfel losgeworden ist. Mit einer Aktion kann ein Charakter den Maulwürfel auf ein beliebiges Nicht-Wand-Feld in Sichtweite und Maulwürfel-Wurfweite werfen. Befindet sich auf diesem Zielfeld ein Charakter, nimmt dieser den Maulwürfel in sein Inventar auf. Ansonsten prallt der Maulwürfel von dem Zielfeld ab und landet im Inventar des sich am nächsten befindenden Charakters. Falls der Maulwürfel im Inventar eines NPC ist, macht dieser bei seinem nächsten Zug einen Maulwürfelwurf auf ein zufälliges Zielfeld. \\
	\hline
	BEGRÜNDUNG: &  Der Maulwürfel deaktiviert bei dem Charakter, der ihn im Inventar hat positive Eigenschaften und bringt dadurch Spannung in das Spiel. \\
	\hline
	ABHÄNGIGKEITEN: & FA-G \ref{Gadgets}, FA-G \ref{Charakter}, FA-G \ref{Inventar},FA-G \ref{Eigenschaften}, \todo[inline]{Maulwürfel-Wurfweite} \\
\end{tabularx}

\begin{tabularx}{16cm}{l|X}
	\refstepcounter{table}\label{Technicolor-Prisma}
	\textbf{ID} & \textbf{FA-G \arabic{table}} \\
	\hline
	TITEL: & Technicolor-Prisma \\
	\hline
	BESCHREIBUNG: & Durch eine Aktion kann das Technicolor-Prisma an einem Roulette-Tisch installiert werden und vertauscht dann die Farben rot und schwarz. Dadurch wird das Resultat beim Roulette-Spielen negiert, d.h. ein Charakter verliert, wenn er normalerweise gewonnen hätte und umgekehrt. Damit sind die Erfolgswahrscheinlichkeiten für Glückspilze und Pechvögel beim Roulette-Spielen vertauscht. Ein Technicolor-Prisma kann nur einmal verwendet werden. \\
	\hline
	BEGRÜNDUNG: &  Durch das Technicolor-Prisma wird das Roulette-Spielen beeinflusst. \\
	\hline
	ABHÄNGIGKEITEN: & FA-G \ref{Gadgets}, FA-G \ref{Roulette spielen}, \\
\end{tabularx}

\begin{tabularx}{16cm}{l|X}
	\refstepcounter{table}\label{Hutsichtlinie}
	\textbf{ID} & \textbf{FA-G \arabic{table}} \\
	\hline
	TITEL: & Hutsichtlinie \\
	\hline
	BESCHREIBUNG: & Die Hutsichtlinie ist eine reguläre Sichtlinie, wobei die Hutsichtlinie zusätzlich von Feldern blockiert wird, auf denen sich Charaktere befinden. \\
	\hline
	BEGRÜNDUNG: &  Man kann den Klingen-Hut nicht durch Personen durchwerfen. \\
	\hline
	ABHÄNGIGKEITEN: & FA-G \ref{Sichtlinie} \\
\end{tabularx}

\begin{tabularx}{16cm}{l|X}
	\refstepcounter{table}\label{Klingen-Hut}
	\textbf{ID} & \textbf{FA-G \arabic{table}} \\
	\hline
	TITEL: & Klingen-Hut \\
	\hline
	BESCHREIBUNG: & Der Klingen-Hut ist ein Bowler-Hut mit einer scharfen Klinge in der Hutkrempe. Ein Charakter, der den Klingen-Hut im Inventar hat, kann diesen auf eine Zielperson in Hutsichtlinie und Hut-Wurf-Reichweite werfen. Die Zielperson wird mit Hut-Treffer-Wahrscheinlichkeit verletzt und erleidet n HP Hut-Schaden. Der Klingen-Hut landet nach einem Wurf auf eine Zielperson in jedem Fall auf einem zufälligen freien Nachbarfeld der Zielperson. Betritt ein Charakter das Feld, auf welchem der Klingen-Hut liegt, so nimmt dieser Charakter den Klingen-Hut in sein Inventar auf. \\
	\hline
	BEGRÜNDUNG: &  Damit kann weiter entfernten Charakteren Schaden zugefügt werden. \\
	\hline
	ABHÄNGIGKEITEN: & FA-G \ref{Gadgets}, FA-G \ref{Hutsichtlinie} \\
\end{tabularx}

\begin{tabularx}{16cm}{l|X}
	\refstepcounter{table}\label{Magnetfeld-Armbanduhr}
	\textbf{ID} & \textbf{FA-G \arabic{table}} \\
	\hline
	TITEL: & Magnetfeld-Armbanduhr \\
	\hline
	BESCHREIBUNG: & Wird der Charakter, der die Magnetfeld-Uhr trägt mit einem Klingen-Hut beworfen, so erleidet dieser keinen Schaden und der Klingen-Hut landet auf einem zufälligen freien Nachbarfeld des Charakters. \\
	\hline
	BEGRÜNDUNG: &  Dieses Gadget schützt den Besitzer vor Angriffen mit dem Klingen-Hut. \\
	\hline
	ABHÄNGIGKEITEN: & FA-G \ref{Gadgets}, FA-G \ref{Klingen-Hut} \\
\end{tabularx}

\begin{tabularx}{16cm}{l|X}
	\refstepcounter{table}\label{Giftpillen-Flasche}
	\textbf{ID} & \textbf{FA-G \arabic{table}} \\
	\hline
	TITEL: & Giftpillen-Flasche \\
	\hline
	BESCHREIBUNG: & Die Giftpillen-Flasche enthält fünf Giftpillen. Der Besitzer der Giftpillen-Flasche kann eine Giftpille als Aktion in einen Cocktail werfen, wenn der Cocktail auf einem Nachbarfeld steht oder von einer benachbarten Person gehalten wird. Dieser Cocktail verwandelt sich dann in einen vergifteten Cocktail und die Giftpille ist damit verbraucht. Nur die Fraktion des Vergifters sieht, dass der Cocktail vergiftet ist. Wird ein vergifteter Cocktail getrunken, werden dem Charakter die Cocktail-Health Points als Schaden abgezogen. Sind alle fünf Giftpillen verbaucht, wird die Giftpillen-Flasche aus dem Inventar entfernt.  \\
	\hline
	BEGRÜNDUNG: &  Damit können Cocktails vergiftet werden. \\
	\hline
	ABHÄNGIGKEITEN: & FA-G \ref{Gadgets}, FA-G \ref{Cocktail aufnehmen} \\
\end{tabularx}

\begin{tabularx}{16cm}{l|X}
	\refstepcounter{table}\label{Laser-Puderdose}
	\textbf{ID} & \textbf{FA-G \arabic{table}} \\
	\hline
	TITEL: & Laser-Puderdose \\
	\hline
	BESCHREIBUNG: & Mit der Laser-Puderdose kann ein Charakter auf einen Cocktail schießen, der sich auf einem Feld in Sichtlinie befindet, egal ob der Cocktail herumsteht oder von einer Person gehalten wird. Die Laser-Schuss-Aktion trifft mit Laser-Treffer-Wahrscheinlichkeit und lässt den Cocktail bei einem Treffer verschwinden, indem dieser einfach verdampft. \\
	\hline
	BEGRÜNDUNG: &  Damit kann man Cocktails verschwinden lassen. \\
	\hline
	ABHÄNGIGKEITEN: & FA-G \ref{Gadgets}, FA-G \ref{Cocktail aufnehmen} \\
\end{tabularx}

\begin{tabularx}{16cm}{l|X}
	\refstepcounter{table}\label{Raketenwerfer-Füllfederhalter}
	\textbf{ID} & \textbf{FA-G \arabic{table}} \\
	\hline
	TITEL: & Raketenwerfer-Füllfederhalter \\
	\hline
	BESCHREIBUNG: & Mit dem Raketenwerfer-Füllfederhalter kann eine Rakete auf ein Zielfeld in Sichtlinie und beliebiger Entfernung verschossen werden. Alle Wände auf dem Zielfeld und auf benachbarten Feldern werden durch die Explosion zu freien Feldern und alle Personen auf dem Zielfeld oder auf benachbarten Feldern bekommen einen Raketen-Schaden. Der Raketenwerfer-Füllfederhalter kann nur einmal benutzt werden und verschwindet dann aus dem Inventar. \\
	\hline
	BEGRÜNDUNG: &  Damit können Wände entfernt werden und anderen Charakteren Schaden zugefügt werden. \\
	\hline
	ABHÄNGIGKEITEN: & FA-G \ref{Gadgets}, FA-G \ref{Wand} \\
\end{tabularx}

\begin{tabularx}{16cm}{l|X}
	\refstepcounter{table}\label{Gaspatronen-Lippenstift}
	\textbf{ID} & \textbf{FA-G \arabic{table}} \\
	\hline
	TITEL: & Gaspatronen-Lippenstift \\
	\hline
	BESCHREIBUNG: & Mit dem Gaspatronen-Lippenstift kann eine Wolke Reizgas auf eine benachbarte Zielperson gesprüht werden, diese erleidet dann n HP Reizgas-Schaden. Der Gaspatronen-Lippenstift kann nur einmal verwendet werden und verschwindet dann aus dem Inventar. \\
	\hline
	BEGRÜNDUNG: &  Mit dem Gaspatronen-Lippenstift kann anderen Charakteren Schaden zugefügt werden. \\
	\hline
	ABHÄNGIGKEITEN: & FA-G \ref{Gadgets} \\
\end{tabularx}

\begin{tabularx}{16cm}{l|X}
	\refstepcounter{table}\label{Mottenkugel-Beutel}
	\textbf{ID} & \textbf{FA-G \arabic{table}} \\
	\hline
	TITEL: & Mottenkugel-Beutel \\
	\hline
	BESCHREIBUNG: & Ein Charakter, der den Mottelkugel-Beutel im Inventar hat, kann mit einer Aktion eine Mottenkugel auf ein in Sichtlinie und Mottenkugel-Wurf-Reichweite befindliches Kaminfeld werfen. Die entstehende Explosion fügt jeder Person auf einem Nachbarfeld des Kamins einen Mottenkugel-Schaden zu. Der Mottenkugel-Beutel enthält fünf Mottenkugeln und verschwindet aus dem Inventar, sobald diese aufgebraucht sind. \\
	\hline
	BEGRÜNDUNG: &  Dadurch kann Personen in der Nähe eines Kamins Schaden zugefügt werden. \\
	\hline
	ABHÄNGIGKEITEN: & FA-G \ref{Gadgets}, FA-G \ref{Sichtlinie} \\
\end{tabularx}

\begin{tabularx}{16cm}{l|X}
	\refstepcounter{table}\label{Nebeldose}
	\textbf{ID} & \textbf{FA-G \arabic{table}} \\
	\hline
	TITEL: & Nebeldose \\
	\hline
	BESCHREIBUNG: & Die Nebeldose kann mit einer Aktion auf jedes Feld, das keine Wand ist geworfen werden, wenn sich dieses Zielfeld in Nebeldosen-Wurf-Reichweite und in Sichtlinie befindet. Die Nebeldose erzeugt auf dem Zielfeld und auf dessen Nachbarfeldern einen dichten Nebel, welcher die Sichtlinie blockiert. Charaktere, die sich im Nebel befinden können keine Aktionen ausführen. Der Nebel bleibt für den Rest der laufenden Runde und während der nächsten beiden Runden bestehen und löst sich dann auf. Die Nebeldose kann nur einmal verwendet werden und verschwindet dann aus dem Inventar. \\
	\hline
	BEGRÜNDUNG: &  Dadurch können Personen aktionsunfähig gemacht werden und Sichtlinien werden blockiert. \\
	\hline
	ABHÄNGIGKEITEN: & FA-G \ref{Gadgets}, FA-G \ref{Sichtlinie} \\
\end{tabularx}

\begin{tabularx}{16cm}{l|X}
	\refstepcounter{table}\label{Wurfhaken}
	\textbf{ID} & \textbf{FA-G \arabic{table}} \\
	\hline
	TITEL: & Wurfhaken \\
	\hline
	BESCHREIBUNG: & Objekte, die frei auf dem Boden oder auf Tischen herumliegen und sich in Wurfhaken-Reichweite und in Sichtlinie befinden, können von einem Charakter mit dem Wurfhaken zu sich herangezogen werden. Diese Aktion gelingt mit Wurfhaken-Treffer-Wahrscheinlichkeit. Die herangezogenen Objekte werden dann sofort ins Inventar aufgenommen oder im Falle eines Cocktails aufgenommen und in der Hand gehalten. \\
	\hline
	BEGRÜNDUNG: &  Mit dem Wurfhaken können weiter entfernte Objekte aufgenommen werden. \\
	\hline
	ABHÄNGIGKEITEN: & FA-G \ref{Gadgets}, FA-G \ref{Sichtlinie} \\
\end{tabularx}

\begin{tabularx}{16cm}{l|X}
	\refstepcounter{table}\label{Jetpack}
	\textbf{ID} & \textbf{FA-G \arabic{table}} \\
	\hline
	TITEL: & Jetpack \\
	\hline
	BESCHREIBUNG: & Mit dem Jetpack kann ein Charakter mit einer Aktion auf ein beliebiges freies Feld fliegen. Das Jetpack kann nur einmal verwendet werden und verschwindet dann aus dem Inventar. \\
	\hline
	BEGRÜNDUNG: &  Mit dem Jetpack können Charaktere weite Strecken zurücklegen. \\
	\hline
	ABHÄNGIGKEITEN: & FA-G \ref{Gadgets} \\
\end{tabularx}

\begin{tabularx}{16cm}{l|X}
	\refstepcounter{table}\label{Wanze und Ohrstöpsel}
	\textbf{ID} & \textbf{FA-G \arabic{table}} \\
	\hline
	TITEL: & Wanze und Ohrstöpsel \\
	\hline
	BESCHREIBUNG: & Mit einer Aktion kann einer benachbarten Zielperson eine Wanze untergeschoben werden und die Ohrstöpsel behält der Besitzer selbst. Die Zielperson bemerkt dies nicht. Während die Wanze an der Zielperson klebt, werden alle IPs, die die verwanzte Person erwirbt ebenso dem Träger der Ohrstöpsel gutgeschrieben. Zu Beginn jeder neuen Runde kann es sein, dass die Wanze mit Wanzen-Ausfalls-Wahrscheinlichkeit ausfällt. Wenn dies passiert, verschwinden die Wanze und Ohrstöpsel. \\
	\hline
	BEGRÜNDUNG: &  Damit kann ein Charakter die IPs, die eine andere Person erwirbt ebenfalls bekommen. \\
	\hline
	ABHÄNGIGKEITEN: & FA-G \ref{Gadgets}, FA-G \ref{Sichtlinie} \\
\end{tabularx}

\begin{tabularx}{16cm}{l|X}
	\refstepcounter{table}\label{Chicken Feed}
	\textbf{ID} & \textbf{FA-G \arabic{table}} \\
	\hline
	TITEL: & Chicken Feed \\
	\hline
	BESCHREIBUNG: & Das ist ein Säckchen Hühnerfutter, das einer benachbarten Zielperson gegeben werden kann. Gehört diese Zielperson nicht zu gegnerischen Fraktion, passiert nichts und das Chicken Feed ist weg. Gehört die Zielperson zur gegnerischen Fraktion, verändern sich die IPs wie folgt: Als erstes wird der Absolutbetrag der Differenz zwischen den IPs der Zielperson und den IPs des Chicken Feed-Gebers berechnet. Hat der Geber weniger IPs als die Zielperson, wird ihm die Differenz abgezogen, wobei die IPs nicht unter 0 fallen können, und falls er mehr hat wird ihm die Differenz gutgeschrieben. Das Chicken Feed wird dadurch verbraucht und verschwindet aus dem Inventar. \\
	\hline
	BEGRÜNDUNG: &  Dieses Gadget verändert die IPs des Anwenders entweder zum Positiven oder zum Negativen. \\
	\hline
	ABHÄNGIGKEITEN: & FA-G \ref{Gadgets} \\
\end{tabularx}

\begin{tabularx}{16cm}{l|X}
	\refstepcounter{table}\label{Nugget}
	\textbf{ID} & \textbf{FA-G \arabic{table}} \\
	\hline
	TITEL: & Nugget \\
	\hline
	BESCHREIBUNG: & Das Nugget kann einer benachbarten Zielperson gegeben werden. Ist diese Zielperson ein NPC, schließt sich dieser der Fraktion des Nugget-Gebers an und man kann ihn als Spieler-Charakter steuern. Dabei wird das Nugget verbraucht. Ist die Zielperson ein Mitglied der gegnerischen Fraktion, wechselt dieses nicht die Seite und bekommt das Nugget ins Inventar und kann es selbst benutzen. Die gegnerische Fraktion bekommt diese Aktion dann mit. \\
	\hline
	BEGRÜNDUNG: &  Dadurch können NPC zu der eigenen Fraktion hinzugeholt werden. \\
	\hline
	ABHÄNGIGKEITEN: & FA-G \ref{Gadgets} \\
\end{tabularx}

\begin{tabularx}{16cm}{l|X}
	\refstepcounter{table}\label{Mirror of Wilderness}
	\textbf{ID} & \textbf{FA-G \arabic{table}} \\
	\hline
	TITEL: & Mirror of Wilderness \\
	\hline
	BESCHREIBUNG: & Wendet ein Charakter den Mirror of Wilderness bei einer benachbarten Zielperson an, werden die IPs der beiden Charaktere vertauscht. Bei Personen der eigenen Fraktion funktioniert dies immer und der Spiegel ist beliebig oft anwendbar, bei Mitgliedern der gegnerischen Fraktion klappt der Tausch nur mit einer Mirror-of-Wilderness-Wahrscheinlichkeit und wird im Erfolgsfall zerstört und verschwindet aus dem Inventar.\\
	\hline
	BEGRÜNDUNG: &  Dadurch können die IPs zwischen Charakteren vertauscht werden. \\
	\hline
	ABHÄNGIGKEITEN: & FA-G \ref{Gadgets} \\
\end{tabularx}

\begin{tabularx}{16cm}{l|X}
	\refstepcounter{table}\label{Pocket Litter}
	\textbf{ID} & \textbf{FA-G \arabic{table}} \\
	\hline
	TITEL: & Pocket Litter \\
	\hline
	BESCHREIBUNG: & Hat ein Charakter Pocket Litter in seinem Inventar, ist er so gut getarnt, dass er wie ein NPC erscheint und selbt Observations-Aktionen standhält. \\
	\hline
	BEGRÜNDUNG: &  Ein Charakter mit Pocket Litter kann von gegnerischen Charakteren nicht als ein Gegner erkannt werden.  \\
	\hline
	ABHÄNGIGKEITEN: & FA-G \ref{Gadgets} \\
\end{tabularx}

\begin{tabularx}{16cm}{l|X}
	\refstepcounter{table}\label{Diamanthalsband der weißen Katze}
	\textbf{ID} & \textbf{FA-G \arabic{table}} \\
	\hline
	TITEL: & Diamanthalsband der weißen Katze \\
	\hline
	BESCHREIBUNG: & Das Diamanthalsband liegt in einem der Tresore im Casino. Wenn ein Charakter den Tresor öffnet, nimmt er das Diamanthalsband in sein Inventar auf. Das Diamanthalsband wird nicht in der Drafting Phase verwendet. \\
	\hline
	BEGRÜNDUNG: &  Wer das Halsband zu Blofelds weißer Katze bringt, bekommt von ihr ein Geheimnis, das eine Menge IPs wert ist. \\
	\hline
	ABHÄNGIGKEITEN: & FA-G \ref{Gadgets} \\
\end{tabularx}

\begin{tabularx}{16cm}{l|X}
	\refstepcounter{table}\label{Charakter Liste}
	\textbf{ID} & \textbf{FA-G \arabic{table}} \\
	\hline
	TITEL: & Charakter Liste \\
	\hline
	BESCHREIBUNG: & Eine Liste, die alle spielbaren Charaktere mit ihren Fähigkeiten beinhaltet. Die Charaktere, die im Lastenheft unter \glqq A Einige Vorschläge für Charaktere\grqq aufgeführt sind, sollen mindestens in der Liste enthalten sein.\\
	\hline
	BEGRÜNDUNG: & Aus dieser Liste können Charaktere zum Spielen gewählt werden.\\
	\hline
	ABHÄNGIGKEITEN: & FA-G \ref{Charakter} \todo[inline]{2.8.1 Wahlphase}\\
\end{tabularx}

\begin{tabularx}{16cm}{l|X}
	\refstepcounter{table}\label{Bewegung durchfuehren}
	\textbf{ID} & \textbf{FA-G\arabic{table}} \\
	\hline
	TITEL: & Bewegung durchführen \\
	\hline
	BESCHREIBUNG: & Innerhalb eines Zuges kann ein Charakter, der noch mehr als einen Bewegungspunkt besitzt, eine Bewegung durchführen. Das bedeutet, dass der Charakter sich von dem Spielfeld seiner aktuellen Position auf ein angrenzendes betretbares Spielfeld bewegen kann.
	Jedes Spielfeld, welches mit einer Seite oder einer Ecke das Spielfeld der aktuellen Position berührt, ist ein angrenzendes Spielfeld. Dementsprechend sind Bewegungen in horizontaler, vertikaler und diagonaler Linie möglich.
	Wenn der Client die Anweisung gibt, den Charakter auf ein angrenzendes Feld zu bewegen, welches nicht betretbar ist, so darf der Charakter sich danach nicht auf diesem Feld befinden, sondern muss auf dem Spielfeld der aktuellen Position bleiben. In diesem Fall wurde die  Bewegung nicht erfolgreich durchgeführt. In der graphischen Darstellung kann dem Benutzer mit einer Animation angezeigt werden, dass die Bewegung nicht erfolgreich war bzw. dass es nicht möglich ist, das gewählte Spielfeld zu betreten.
 Bei erfolgreicher Durchführung einer Bewegung muss dem Charakter am Zug die Anzahl der Bewegungspunkte um 1 reduziert werden. Bei nicht erfolgreicher Durchführung darf sich die Anzahl der Bewegungspunkte  nicht verändern.\\
	\hline
	BEGRÜNDUNG: & Die Bedingungen für das Bewegen eines Charakters müssen eindeutig definiert sein.\\
	\hline
	ABHÄNGIGKEITEN: & FA-G\ref{Charakter} \todo[inline]{2.8.1 Wahlphase}\\
\end{tabularx}

\begin{tabularx}{16cm}{l|X}
	\refstepcounter{table}\label{Draengeln}
	\textbf{ID} & \textbf{FA-G\arabic{table}} \\
	\hline
	TITEL: & Drängeln \\
	\hline
	BESCHREIBUNG: & Wenn ein Charakter sich auf ein Feld bewegt, auf dem bereits ein anderer Charakter 		befindet, so tauschen die beiden Charaktere Plätze. Das heißt, der Charakter der die Bewegung durchgeführt hat steht auf dem Feld, auf das er sich bewegen wollte. Dies ist das Feld, auf dem der andere Charakter vor der Bewegung stand. Der andere Charakter befindet sich, nach dem Durchführen der Bewegung, auf dem Feld, auf dem der aktive Charakter ursprünglich stand. \\
	\hline
	BEGRÜNDUNG: & Um das Verhalten im Fall der Bewegung auf ein Spielfeld, welches mit einem anderen 			Charakter besetzt ist, eindeutig zu definieren.\\
	\hline
	ABHÄNGIGKEITEN: & FA-G\ref{Bewegung durchfuehren} \todo[inline]{2.8.1 Wahlphase}\\
\end{tabularx}

\begin{tabularx}{16cm}{l|X}
	\refstepcounter{table}\label{Aktion durchfuehren}
	\textbf{ID} & \textbf{FA-G\arabic{table}} \\
	\hline
	TITEL: & Aktion durchführen \\
	\hline
	BESCHREIBUNG: & Innerhalb eines Zuges kann ein Charakter, der noch mehr als einen Aktionspunkt hat und die Ausführungsbedingungen einer spezifischen Aktion erfüllt, diese spezifische Aktion ausführen.
	 Wenn der Client eine gültigen Aktionsbefehl eingibt, so gilt die Ausführung der Aktion als erfolgreich. Bei der Eingabe eines ungültigen Aktionsbefehls gilt die Ausführung der Aktions als nicht erfolgreich.
	 Bei erfolgreicher Ausführung der Aktion muss dem Charakter am Zug die Anzahl der Aktionspunkte um 1 reduziert werden und die Konsequenzen der spezifischen Aktion müssen den Zustand des Spiels entsprechend ändern. Bei nicht erfolgreicher Ausführung darf sich die Anzahl der Aktionspunkte und der Zustand des Spiels nicht ändern.  \\
	\hline
	BEGRÜNDUNG: & Es muss genau definiert sein, unter welchen Bedingungen das Ausführen einer Aktion möglich ist und welche Konsequenzen die Aktion hat.\\
	\hline
	ABHÄNGIGKEITEN: & FA-G\ref{Bewegung durchfuehren} \todo[inline]{2.8.1 Wahlphase}\\
\end{tabularx}


\begin{tabularx}{16cm}{l|X}
	\refstepcounter{table}\label{Gadget verwenden}
	\textbf{ID} & \textbf{FA-G\arabic{table}} \\
	\hline
	TITEL: & Gadget verwenden \\
	\hline
	BESCHREIBUNG: & Wenn sich ein verwendbares Gadget im Inventar eines Charakters befindet, dieser Charakter am Zug ist und die Ausführungsbedingungen des Gadgets für diesen Charakter erfüllt sind, so kann der Charakter als Aktion das Gadget verwenden. \\
	 
	\hline
	BEGRÜNDUNG: & Das Verwenden von Gadgets stellt eine Aktion dar.\\
	\hline
	ABHÄNGIGKEITEN: & FA-G\ref{Aktion durchfuehren}  \todo[inline]{2.8.1 Wahlphase}\\
\end{tabularx}

\begin{tabularx}{16cm}{l|X}
	\refstepcounter{table}\label{Roulette spielen}
	\textbf{ID} & \textbf{FA-G\arabic{table}} \\
	\hline
	TITEL: & Roulette spielen \\
	\hline
	BESCHREIBUNG: & Ein Charakter, der sich innerhalb eines Zuges auf einem Feld befindet, welches an ein Feld mit Roulette-Tisch angrenzt, der kann als Aktion einmal Roulette spielen. Der Client muss dazu einen Betrag an Spielchips eingeben. Dieser Betrag hat den Mindestwert 1 und einen Höchstwert. Der Höchstwert ist der kleinere der beiden folgenden Werte: die Anzahl der Spielchips, welche der Charakter im Moment besitzt; die Anzahl der Spielchips, die am Tisch verfügbar sind. Die eingegebene Betrag stellt den Einsatz des Charakters dar. Wenn der Charakter gewinnt, dann wird der Spielchips-Wert des Charakters um den Einsatz-Wert erhöht und der Spielchips-Wert des Roulette-Tisches um den Einsatz-Wert verringert. Wenn der Charakter verliert, dann wird der Spielchips-Wert des Charakters um den Einsatz-Wert reduziert und der Spielchipswert des Roulette-Tisches um den Einsatz-Wert erhöht. Ob ein Charakter gewinnt wird durch die Roulette-Gewinn-Wahrscheinlichkeit bestimmt.\\
	 
	\hline
	BEGRÜNDUNG: & Aufgrund der Spielmechanik ist Roulette spielen eine Aktion. Das Verhalten beim Ausführen dieser Aktion muss eindeutig definiert sein.\\
	\hline
	ABHÄNGIGKEITEN: & FA-G\ref{Aktion durchfuehren} \todo[inline]{2.8.1 Wahlphase}\\
\end{tabularx}

\begin{tabularx}{16cm}{l|X}
	\refstepcounter{table}\label{Cocktail aufnehmen}
	\textbf{ID} & \textbf{FA-G\arabic{table}} \\
	\hline
	TITEL: & Cocktail aufnehmen \\
	\hline
	BESCHREIBUNG: & Wenn ein Charakter am Zug ist, noch mehr als 0 Aktionpunkte hat, sich auf einem Feld befindet, welches an einen Bar-Tisch angrenzt, sich auf diesem Bartisch zu diesem Zeitpunkt ein Cocktail befindet und er selbst im Moment keinen Cocktail in der Hand hält, so kann der Charakter als Aktion diesen Cocktail aufnehmen. Danach befindet sich der Cocktail in der Hand des Charakters. Diese Information muss für alle Clients sichtbar sein.\\
	 
	\hline
	BEGRÜNDUNG: & Aufgrund der Spielmechanik ist das Aufnehmen eines Cocktails eine Aktion. Das Verhalten beim Ausführen dieser Aktion muss eindeutig definiert sein.\\
	\hline
	ABHÄNGIGKEITEN: & FA-G\ref{Aktion durchfuehren}  \todo[inline]{2.8.1 Wahlphase}\\
\end{tabularx}
