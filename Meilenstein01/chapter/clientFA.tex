\begin{tabularx}{16cm}{l|X}
\textbf{ID} & \textbf{FA-C} \\
\hline
TITEL: & Optional: Intro \\
\hline 
BESCHREIBUNG: & Beim Starten des Programms soll ein Film ablaufen, der den Titel des Spiels präsentiert und die Teammitglieder nennt. Dieser kann jederzeit mittels Tastendruck abgebrochen werden. Nach Ende des Films oder Abbruch gelangt der Nutzer ins Hauptmenü. \\
\hline
BEGRÜNDUNG: & Der Film versteckt etwaige Ladezeiten beim Starten und informiert den Spieler über das Spiel und die Autoren.\\
\hline
ABHÄNGIGKEITEN: & \\
\end{tabularx}

\begin{tabularx}{16cm}{l|X}
\textbf{ID} & \textbf{FA-C} \\
\hline
TITEL: & Hauptmenü \\
\hline 
BESCHREIBUNG: & Im Hauptmenü hat der Nutzer die Möglichkeit, seinen Namen festzulegen, einem Server durch Eingabe der Adresse beizutreten, oder das Spiel zu beenden.\\ 
\hline
BEGRÜNDUNG: & Vor dem Spielen muss der Spieler eine Verbindung zum Server aufbauen. \\
\hline
ABHÄNGIGKEITEN: & \\
\end{tabularx}

\begin{tabularx}{16cm}{l|X}
\textbf{ID} & \textbf{FA-C} \\
\hline
TITEL: & Beitreten als Mitspieler \\
\hline 
BESCHREIBUNG: & Nach Verbinden mit dem Server kann der Spieler die Rolle als aktiver Spieler wählen. Dies wird dem Server mitgeteilt. \\
\hline
BEGRÜNDUNG: & Es muss zwischen aktiven Spielern und passiven Zuschauern unterschieden werden. \\
\hline
ABHÄNGIGKEITEN: & \\
\end{tabularx}

\begin{tabularx}{16cm}{l|X}
\textbf{ID} & \textbf{FA-C} \\
\hline
TITEL: & Beitreten als Zuschauer \\
\hline 
BESCHREIBUNG: & Nach Verbinden mit dem Server kann der Spieler die Rolle als passiver Zuschauer wählen. Dies wird dem Server mitgeteilt. \\
\hline
BEGRÜNDUNG: & Es muss zwischen aktiven Spielern und passiven Zuschauern unterschieden werden. \\
\hline
ABHÄNGIGKEITEN: & \\
\end{tabularx}

\begin{tabularx}{16cm}{l|X}
\textbf{ID} & \textbf{FA-C} \\
\hline
TITEL: & Registrieren als menschlicher Spieler \\
\hline 
BESCHREIBUNG: & Beim Verbinden mit dem Server muss diesem mitgeteilt werden, dass es sich um einen menschlichen Spieler handelt. \\
\hline
BEGRÜNDUNG: & Es muss zwischen menschlichen Spielern und KI Spielern unterschieden werden. \\
\hline
ABHÄNGIGKEITEN: & \\
\end{tabularx}

\begin{tabularx}{16cm}{l|X}
\textbf{ID} & \textbf{FA-C} \\
\hline
TITEL: & Nutzername \\
\hline 
BESCHREIBUNG: & Der Client muss es dem Nutzer ermöglichen einen Nutzernamen zu wählen. Dieser soll vom Server validiert werden und wird eventuell nicht akzeptiert. \\
\hline
BEGRÜNDUNG: & Beide Mitspieler werden durch Nutzernamen identifiziert. \\
\hline
ABHÄNGIGKEITEN: & \\
\end{tabularx}

\begin{tabularx}{16cm}{l|X}
\textbf{ID} & \textbf{FA-C} \\
\hline
TITEL: & Spiel Anzeige \\
\hline 
BESCHREIBUNG: & Der Client muss dem Benutzer den Aktuellen Zustand des Spiels jederzeit darstellen. Dazu gehört:
\begin{itemize}
    \item Charaktere der eigenen Fraktion
    \item Eigenschaften der Charaktere
    \item Gadgets der Charaktere
    \item Spielfeld mit Spielern und Gadgets
\end{itemize} \\
\hline
BEGRÜNDUNG: & Integraler Teil des Spiels \\
\hline
ABHÄNGIGKEITEN: & \\
\end{tabularx}

\begin{tabularx}{16cm}{l|X}
\textbf{ID} & \textbf{FA-C} \\
\hline
TITEL: & Interaktion mit dem Spiel \\
\hline 
BESCHREIBUNG: & Der Client muss dem Spieler ermöglichen, Aktionen auf dem Spielfeld durchzuführen, wenn der Spieler am Zug ist. \\
\hline
BEGRÜNDUNG: & Integraler Bestandteil des Spiels \\
\hline
ABHÄNGIGKEITEN: & \\
\end{tabularx}

\begin{tabularx}{16cm}{l|X}
\textbf{ID} & \textbf{FA-C} \\
\hline
TITEL: & Optional: Hilfefunktion \\
\hline 
BESCHREIBUNG: & Der Client soll den Spieler durch Vorschläge für Aktionen unterstützen. \\
\hline
BEGRÜNDUNG: & Dies vereinfacht das Erlernen des Spiels und das Kennenlernen der verschiedenen Aktionen. \\
\hline
ABHÄNGIGKEITEN: & \\
\end{tabularx}

\begin{tabularx}{16cm}{l|X}
\textbf{ID} & \textbf{FA-C} \\
\hline
TITEL: & Optional: Hotkeys \\
\hline 
BESCHREIBUNG: & Die Aktionen während dem Spiel sollen durch Hotkeys ausführbar sein. \\
\hline
BEGRÜNDUNG: & Dies dient einer komfortableren Bedienung des Spiels. \\
\hline
ABHÄNGIGKEITEN: & \\
\end{tabularx}

\begin{tabularx}{16cm}{l|X}
\textbf{ID} & \textbf{FA-C} \\
\hline
TITEL: & Animation der Aktionen \\
\hline 
BESCHREIBUNG: & Die Aktionen einer Runde sollen animiert dargestellt werden. \\
\hline
BEGRÜNDUNG: & Einfacheres Nachvollziehen der eigenen und gegnerischen Aktionen \\
\hline
ABHÄNGIGKEITEN: & \\
\end{tabularx}

\begin{tabularx}{16cm}{l|X}
\textbf{ID} & \textbf{FA-C} \\
\hline
TITEL: & Wunsch auf Pausieren \\
\hline 
BESCHREIBUNG: & Der Client ermöglicht es dem Spieler, ein Pausieren der Runde zu beantragen. \\
\hline
BEGRÜNDUNG: & Für bessere Immersion kann der Benutzer während dem Spielen selbst Cocktails trinken, was eventuelle Pausen fürs Nachfüllen bedingt. \\
\hline
ABHÄNGIGKEITEN: & \\
\end{tabularx}

\begin{tabularx}{16cm}{l|X}
\textbf{ID} & \textbf{FA-C} \\
\hline
TITEL: & Wunsch auf Wiederaufnahme des Spiels \\
\hline 
BESCHREIBUNG: & Der Client ermöglicht es dem Spieler, die Wiederaufnahme des Spiels (Beenden der Pause) zu beantragen. \\
\hline
BEGRÜNDUNG: & Ermöglicht Beenden der Pause \\
\hline
ABHÄNGIGKEITEN: & \\
\end{tabularx}

\begin{tabularx}{16cm}{l|X}
\textbf{ID} & \textbf{FA-C} \\
\hline
TITEL: & Persistente Session \\
\hline 
BESCHREIBUNG: & Ein Abbruch der Verbindung zum Server darf nicht zum Beenden des Spiels führen. Der Client muss versuchen, die Verbindung wiederherzustellen und im Erfolgsfall das Spiel fortsetzen. \\
\hline
BEGRÜNDUNG: & Mit Verbindungsabbrüchen ist z.B bei der Verwendung von WiFi zu rechnen, es soll trotzdem möglich sein zu spielen. \\
\hline
ABHÄNGIGKEITEN: & \\
\end{tabularx}

\begin{tabularx}{16cm}{l|X}
\textbf{ID} & \textbf{FA-C} \\
\hline
TITEL: & Gewinneranzeige \\
\hline 
BESCHREIBUNG: & Wenn durch den Server ein Gewinner festgestellt wird, muss dies im Client dem Spieler präsentiert werden. \\
\hline
BEGRÜNDUNG: & Integraler Bestandteil des Spiels \\
\hline
ABHÄNGIGKEITEN: & \\
\end{tabularx}

\begin{tabularx}{16cm}{l|X}
\textbf{ID} & \textbf{FA-C} \\
\hline
TITEL: & Optional: Statistik \\
\hline 
BESCHREIBUNG: & Neben der Anzeige des Gewinners sollen zusätzlich Statistiken zum Spielverlauf angezeigt werden. \\
\hline
BEGRÜNDUNG: & Information des Spielers \\
\hline
ABHÄNGIGKEITEN: & \\
\end{tabularx}

\begin{tabularx}{16cm}{l|X}
\textbf{ID} & \textbf{FA-C} \\
\hline
TITEL: & Optional: Replay \\
\hline 
BESCHREIBUNG: & Der Client kann statt eines aktiven Spiels auch eine vom Server erstellte Aufzeichnung eines vergangenen Spiels anzeigen. Hierbei ist keinerlei Interaktion möglich. \\
\hline
BEGRÜNDUNG: & Erneutes Anschauen besonders interessanter Partien \\
\hline
ABHÄNGIGKEITEN: & \\
\end{tabularx}
