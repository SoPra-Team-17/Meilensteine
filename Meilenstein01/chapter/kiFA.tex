\begin{tabularx}{16cm}{l|X}
\refstepcounter{table}\label{ki-session}
\textbf{ID} & \textbf{FA-KI \arabic{table}} \\
\hline
TITEL: & Websocket Verbindung \\
\hline 
BESCHREIBUNG: & Der KI-Client muss eine Websocket Verbindung zum Server aufbauen können. Darüber können JSON encodierte Nachrichten geschickt werden. \\
\hline
BEGRÜNDUNG: & Kommunikation mit dem Server \\
\hline
ABHÄNGIGKEITEN: & \\
\end{tabularx}

\begin{tabularx}{16cm}{l|X}
\refstepcounter{table}\label{ki-register}
\textbf{ID} & \textbf{FA-KI \arabic{table}} \\
\hline
TITEL: & Registrieren als KI \\
\hline 
BESCHREIBUNG: & Der KI-Client muss sich nach der Verbindung mit dem Server als KI identifizieren. \\
\hline
BEGRÜNDUNG: & Anzeige für den Spielers, ob er gegen eine KI oder einen menschlichen Gegner spielt. \\
\hline
ABHÄNGIGKEITEN: & FA-KI \ref{ki-session} \\
\end{tabularx}

\begin{tabularx}{16cm}{l|X}
\refstepcounter{table}\label{ki-cli}
\textbf{ID} & \textbf{FA-KI \arabic{table}} \\
\hline
TITEL: & Commandline Argumente \\
\hline 
BESCHREIBUNG: & Der KI-Client muss Kommandozeilen Argumente unterstützen um Parameter zu setzen. Die Argumente sind vom Standardisierungskomitee definiert. \\
\hline
BEGRÜNDUNG: & Nicht-interaktive Benutzung, Konfiguration des Clients. \\
\hline
ABHÄNGIGKEITEN: & \\
\end{tabularx}

\begin{tabularx}{16cm}{l|X}
\refstepcounter{table}\label{ki-config}
\textbf{ID} & \textbf{FA-KI \arabic{table}} \\
\hline
TITEL: & Konfigurationsdatei \\
\hline 
BESCHREIBUNG: & Die Parameter des Commandline Interface müssen auch über eine Konfigurationsdatei konfigurierbar sein. \\
\hline
BEGRÜNDUNG: & Persistentes Speichern der Parameter \\
\hline
ABHÄNGIGKEITEN: & FA-KI \ref{ki-cli}\\
\end{tabularx}

\begin{tabularx}{16cm}{l|X}
\refstepcounter{table}\label{ki-intelligenz}
\textbf{ID} & \textbf{FA-KI \arabic{table}} \\
\hline
TITEL: & Intelligenzstufen \\
\hline 
BESCHREIBUNG: & Der KI-Client muss die Möglichkeit besitzen, zwischen mehreren Strategien zu wechseln abhängig von der eingestellten Intelligenzstufe. Die Intelligenzstufe kann durch ein Commandline Argument oder in der Konfigurationsdatei festgelegt werden. \\
\hline
BEGRÜNDUNG: & Unterschiedlicher Schwierigkeitsgrad für Anfänger vs. fortgeschrittene Spieler \\
\hline
ABHÄNGIGKEITEN: & FA-KI \ref{ki-cli}, \ref{ki-config}\\
\end{tabularx}

\begin{tabularx}{16cm}{l|X}
\refstepcounter{table}\label{ki-actions}
\textbf{ID} & \textbf{FA-KI \arabic{table}} \\
\hline
TITEL: & Aktionen \\
\hline 
BESCHREIBUNG: & Die KI muss in jeder Runde regelkonforme und sinnvolle Aktionen bestimmen, und diese dem Server senden. \\
\hline
BEGRÜNDUNG: & Grundlegende Funktionalität der KI \\
\hline
ABHÄNGIGKEITEN: & FA-KI \ref{ki-session}\\
\end{tabularx}

\begin{tabularx}{16cm}{l|X}
\refstepcounter{table}\label{ki-api}
\textbf{ID} & \textbf{FA-KI \arabic{table}} \\
\hline
TITEL: & Optional: API \\
\hline 
BESCHREIBUNG: & Die KI soll eine Programmierschnittstelle zur Anbindung an den Benutzer Client besitzen. \\
\hline
BEGRÜNDUNG: & Verwenden der KI zum Vorschlagen von Aktionen für den Spieler \\
\hline
ABHÄNGIGKEITEN: & FA-KI \ref{ki-actions}\\
\end{tabularx}
