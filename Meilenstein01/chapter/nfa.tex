%Nicht funktionale Anforderungen
Dieser Abschnitt spezifiziert die Qualitätsanforderungen (QA) an das Softwaresystem.

\begin{tabularx}{16cm}{l|X}
	 \textbf{ID} & \textbf{QA1} \\
	 \hline
	 TITEL: & Robustheit \\
	 \hline 
	 BESCHREIBUNG: & Die Anwendung darf nicht abstürzen. Bei 100 Spielen darf maximal 1 Spiel
	aufgrund eines Fehlers abgebrochen werden. \\
\end{tabularx}


\begin{tabularx}{16cm}{l|X}
	 \textbf{ID} & \textbf{QA2} \\
	 \hline
	 TITEL: & Kommandozeileninterface für Server \\
	 \hline 
	 BESCHREIBUNG: & Der Server muss nicht-interaktiv auf einem Linux-System über ein Kommandozeileninterface innerhalb eines Docker-Containers gestartet werden können. Es müssen standardisierte Parameter unterstützt werden. 
\end{tabularx} 


\begin{tabularx}{16cm}{l|X}
	 \textbf{ID} & \textbf{QA3} \\
	 \hline
	 TITEL: & Kommandozeileninterface für KI-Client \\
	 \hline 
	 BESCHREIBUNG: & Der KI-Client muss nicht-interaktiv auf einem Linux-System über ein Kommandozeileninterface innerhalb eines Docker-Containers gestartet werden können. Es müssen standardisierte Parameter unterstützt werden. 
\end{tabularx} 


\begin{tabularx}{16cm}{l|X}
	 \textbf{ID} & \textbf{QA4} \\
	 \hline
		TITEL: & Performanz des Client \\
	 \hline 
	 BESCHREIBUNG: & Der Client muss dem Benutzer eines modernen Systems eine Aktualiserungsrate von $60\si{\hertz}$ bieten. Der Input-Delay darf maximal $50\si{\milli\second}$ betragen. 
\end{tabularx} 


\begin{tabularx}{16cm}{l|X}
	 \textbf{ID} & \textbf{QA5} \\
	 \hline
		TITEL: & Antwortzeit des Servers\\ 
	 \hline 
	 BESCHREIBUNG: & Der Server muss einem Client, welcher auf eine Antwort wartet, innerhalb von $100\si{\milli\second}$ eine Antwort senden. In der Zeit sind Verzögerungen, welche durch das Netzwerk entstehen nicht enthalten.
\end{tabularx} 


\begin{tabularx}{16cm}{l|X}
	 \textbf{ID} & \textbf{QA6} \\
	 \hline
		TITEL: & Anlegen von Log-Dateien\\ 
	 \hline 
	 BESCHREIBUNG: & Alle Komponenten der verteilten Anwendung sollen Log-Dateien erstellen. Auf Basis dieser muss der Zustand und Verhalten rekonstruiert werden können. Insbesondere müssen die Log-Dateien auch bei Absturz einer Komponente intakt sein. 
\end{tabularx} 


\begin{tabularx}{16cm}{l|X}
	 \textbf{ID} & \textbf{QA7} \\
	 \hline
		TITEL: & Testcoverage\\ 
	 \hline 
	 BESCHREIBUNG: & Alle Komponenten der verteilten Anwendung müssen getestet werden. Bei Unittests wird eine Testcoverage von mindestens 50\% erwartet. Es wird außerdem monkey-testing eingesetzt. 
\end{tabularx} 



\begin{tabularx}{16cm}{l|X}
	 \textbf{ID} & \textbf{QA8} \\
	 \hline
		TITEL: & Codeanalyse\\ 
	 \hline 
	 BESCHREIBUNG: & Der Code aller Komponenten soll analysiert werden. Das Mergen eines Pullrequests soll erst möglich sein, wenn die in SonarQube konfigurierten Gütekriterien vollständig erfüllt sind. Die Gütekriterien in SonarQube sind so zu wählen, dass eine hohe Codequalität gewährleistet werden kann. 
\end{tabularx} 


\begin{tabularx}{16cm}{l|X}
	 \textbf{ID} & \textbf{QA9} \\
	 \hline
		TITEL: & Continous Integration\\ 
	 \hline 
	 BESCHREIBUNG: & Es soll Continous Integration verwendet werden um den Code auf Funktionalität zu prüfen. Insbesondere müssen CI-Checks erfüllt sein, bevor ein Pull-Request gemerged werden kann.
\end{tabularx} 


\begin{tabularx}{16cm}{l|X}
	 \textbf{ID} & \textbf{QA10} \\
	 \hline
		TITEL: & Einfache Benutzung\\ 
	 \hline 
	 BESCHREIBUNG: & Ein Benutzer der verteilten Anwendungen soll bei allen Komponenten nach dreimaliger Benutzung ein grundlegendes Verständis für die Benutzung haben. 
\end{tabularx} 

\begin{tabularx}{16cm}{l|X}
	 \textbf{ID} & \textbf{QA11} \\
	 \hline
		TITEL: & User-Experience\\ 
	 \hline 
	 BESCHREIBUNG: & Um eine gute UX zu bieten, müssen alle sechs Teammitglieder einer Änderung der graphischen Benutzerschnittstelle zustimmen. Es soll sichergestellt werden, dass auch außenstehende mit der Benutzerschnittstelle arbeiten können. 
\end{tabularx} 


\begin{tabularx}{16cm}{l|X}
	 \textbf{ID} & \textbf{QA12} \\
	 \hline
		TITEL: & Dokumentation\\ 
	 \hline 
	 BESCHREIBUNG: & Der Code aller Komponenten muss dokumentiert werden. Es sollen dafür Doxygen-Kommentare verwendet werden. Das daraus generierte Referenzhandbuch muss zu jedem Zeitpunkt der Entwicklung auf einem akutellen und vollständigen Stand sein.  
\end{tabularx} 


\begin{tabularx}{16cm}{l|X}
	 \textbf{ID} & \textbf{QA13} \\
	 \hline
		TITEL: & Entwicklerhandbuch\\ 
	 \hline 
		BESCHREIBUNG: & Auf Basis der in \textit{QA12} beschriebenen generierten Dokumentation soll ein Entwicklerhandbuch erstellt werden. Dieses soll Details zur Architektur enthalten und vollständig alle verwendeten Technoligien und Frameworks auflisten.
\end{tabularx} 


\begin{tabularx}{16cm}{l|X}
	 \textbf{ID} & \textbf{QA14} \\
	 \hline
		TITEL: & Benutzerhandbuch\\ 
	 \hline 
	 BESCHREIBUNG: & Für jede Komponente der verteilten Anwendung soll ein Benutzerhandbuch erstellt werden. In diesem wird einem Benutzer anhand von Beispielen erklärt, wie die Anwendung zu verwenden ist. 
\end{tabularx} 


\begin{tabularx}{16cm}{l|X}
	 \textbf{ID} & \textbf{QA15} \\
	 \hline
		TITEL: & Projekttagebuch\\ 
	 \hline 
	 BESCHREIBUNG: & Es soll über den gesamten Entwicklungsprozess ein Projekttagebuch geführt werden. In diesem werden alle Tätigkeiten der Teammitglieder und der dazugehörige Zeitaufwand aufgelistet. 
\end{tabularx} 
