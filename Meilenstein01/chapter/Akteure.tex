%Hier kommen die Akteure rein

\begin{tabularx}{16cm}{l|X}
	\textbf{Akteur} & \textbf{Server} \\
	\hline
	Rolle & Zentrale Software-Einheit zur Kommunikation mit den Clients.\\ 
	\hline
	Aufgabe & Der Server ist Vermittler in der Kommunikation zwischen den Clients. Er geht auf Anfrage Verbindungen mit Clients ein und kann diese auch wieder beenden. Während ein Client mit dem Server verbunden ist, bearbeitet und validiert der Server alle Anfragen des Clients und sorgt dafür, dass der Client alle notwendigen Informationen zum aktuellen Stand erhält.\\ 
\end{tabularx}

\begin{tabularx}{16cm}{l|X}
	\textbf{Akteur} & \textbf{Client} \\
	\hline
	Rolle & Lokale Software-Einheit zur Kommunikation mit dem Server \\ 
	\hline
	Aufgabe & Da Clients nicht direkt miteinander kommunizieren können, kommuniziert der Client mit einem Server. Der Client kann eine Verbindung zum Server auf- und auch wieder abbauen. Ein verbundener Client kann Informationen weitergeben, anfragen und erhalten.\\ 
\end{tabularx}

\begin{tabularx}{16cm}{l|X}
	\textbf{Akteur} & \textbf{Benutzer-Client} \\
	\hline
	Rolle & Client, der von einem Menschen bedient wird.\\ 
	\hline
	Aufgabe & Der Benutzer-Client stellt eine graphische Oberfläche zur Bedienung zur Verfügung. Er ermöglicht die Bedienung des Editors und die Teilnahme an einer Spielpartie.\\ 
\end{tabularx}

\begin{tabularx}{16cm}{l|X}
	\textbf{Akteur} & \textbf{KI-Client} \\
	\hline
	Rolle & Client, der von einem Computer gesteuert wird.\\ 
	\hline
	Aufgabe & Der KI-Client kann von einem Benutzer-Client als Spieler zur Spielpartie hinzugefügt werden und benutzt keine graphische Oberfläche.\\ 
\end{tabularx}

\begin{tabularx}{16cm}{l|X}
	\textbf{Akteur} & \textbf{Teilnehmer} \\
	\hline
	Rolle & Alle Akteure, die an einer Spielpartie teilnehmen.\\ 
	\hline
	Aufgabe & Teilnehmer erhalten Informationen zur Spielpartie vom Server und können die Spielpartie jederzeit wieder verlassen.\\ 
\end{tabularx}

\begin{tabularx}{16cm}{l|X}
	\textbf{Akteur} & \textbf{Zuschauer} \\
	\hline
	Rolle & Benutzer-Client, der passiv an einer Spielpartie teilnimmt.\\ 
	\hline
	Aufgabe & Über die graphische Oberfläche des Benutzer-Clients, kann der Benutzer das laufende Spiel beobachten, aber nicht eingreifen. \\ 
\end{tabularx}

\begin{tabularx}{16cm}{l|X}
	\textbf{Akteur} & \textbf{Spieler} \\
	\hline
	Rolle & Client, der aktiv an einer Spielpartei teilnimmt.\\ 
	\hline
	Aufgabe & Wenn ein Spieler an der Reihe ist, führt er durch Steuerung seiner Charaktere valide Spielzüge aus und informiert den Server über seine Aktionen. Ziel des Spielers ist es, die Spielpartie zu gewinnen.\\ 
\end{tabularx}

\begin{tabularx}{16cm}{l|X}
	\textbf{Akteur} & \textbf{Benutzer-Spieler} \\
	\hline
	Rolle & Spieler, der ein Benutzer-Client ist.\\ 
	\hline
	Aufgabe & Ein Benutzer-Spieler spielt die Spielpartie über die graphische Oberfläche.\\ 
\end{tabularx}

\begin{tabularx}{16cm}{l|X}
	\textbf{Akteur} & \textbf{KI-Spieler} \\
	\hline
	Rolle & Spieler, der ein KI-Client ist.\\ 
	\hline
	Aufgabe & Ein KI-Spieler spielt die Spielpartie unter Verwendung von künstlicher Intelligenz.\\ 
\end{tabularx}

\begin{tabularx}{16cm}{l|X}
	\textbf{Akteur} & \textbf{Agent} \\
	\hline
	Rolle & Charakter in einer Spielpartie, der von einem Spieler gesteuert wird.\\ 
	\hline
	Aufgabe & Agenten sind Charaktere, die während der Spielpartie von Spielern gesteuert werden.\\ 
\end{tabularx}

\begin{tabularx}{16cm}{l|X}
	\textbf{Akteur} & \textbf{NPC} \\
	\hline
	Rolle & Charakter in einer Spielpartie, der von keinem Spieler gesteuert werden kann.\\ 
	\hline
	Aufgabe & NPCs sind Charaktere, die vom Server gesteuert werden. Der Server informiert die Teilnehmer über die Spielzüge der NPCs.\\ 
\end{tabularx}

\begin{tabularx}{16cm}{l|X}
	\textbf{Akteur} & \textbf{Auftraggeber} \\
	\hline
	Rolle & Florian Ege, stellvertretend für das Institut für Softwaretechnik und Programmiersprachen\\ 
	\hline
	Aufgabe & Gibt die Anforderungen für \glqq No Time To Spy\grqq vor und nimmt das fertige Produkt ab.\\ 
\end{tabularx}

\begin{tabularx}{16cm}{l|X}
	\textbf{Akteur} & \textbf{Vertreter des Auftraggebers} \\
	\hline
	Rolle & Tutor Ismail Temel\\ 
	\hline
	Aufgabe & Dient als Vermittler zwischen Entwicklern und Auftraggeber.\\ 
\end{tabularx}

\begin{tabularx}{16cm}{l|X}
\textbf{Akteur} & \textbf{Entwickler} \\
\hline
Rolle & SoPra-Team-17: Dominik Authaler, Lukas Bleile, Marco Deuscher, Jonas Otto, Carolin Schindler, Dominik Tabib Khoie\\ 
\hline
Aufgabe & Agile Entwicklung von \glqq No Time To Spy\grqq nach den Anforderungen des Auftraggebers.\\ 
\end{tabularx}
