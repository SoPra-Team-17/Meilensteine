\begin{tabularx}{16cm}{l|X}
\refstepcounter{table}\label{e-json-encoding}
\textbf{ID} & \textbf{FA-E \arabic{table}} \\
\hline
TITEL: & JSON Encoding \\
\hline
BESCHREIBUNG: & Der Editor muss die Funktionalität besitzen, Konfigurationsdateien für Szenarios, Partien und Charaktere im JSON Format zu encodieren. \\
\hline
BEGRÜNDUNG: & Notwendig zum Speichern der Konfiguration. \\
\hline
PRIORITÄT: & ++\\ 
\hline
ABHÄNGIGKEITEN: & \\
\end{tabularx}

\begin{tabularx}{16cm}{l|X}
\refstepcounter{table}\label{e-json-decoding}
\textbf{ID} & \textbf{FA-E \arabic{table}} \\
\hline
TITEL: & JSON Decoding \\
\hline
BESCHREIBUNG: & Der Editor muss die Funktionalität besitzen, Konfigurationsdateien für Szenarios, Partien und Charaktere aus dem JSON Format zu decodieren. \\
\hline
BEGRÜNDUNG: & Notwendig zum laden der Konfiguration. \\
\hline
PRIORITÄT: & ++\\
\hline
ABHÄNGIGKEITEN: & \\
\end{tabularx}

\begin{tabularx}{16cm}{l|X}
\refstepcounter{table}\label{e-gui}
\textbf{ID} & \textbf{FA-E \arabic{table}} \\
\hline
TITEL: & GUI \\
\hline
BESCHREIBUNG: & Der Editor muss über eine grafische Benutzeroberfläche zum Editieren der Konfiguration besitzen. \\
\hline
BEGRÜNDUNG: & Grundlegende Funktionalität \\
\hline
PRIORITÄT: & +\\
\hline
ABHÄNGIGKEITEN: & \\
\end{tabularx}

\begin{tabularx}{16cm}{l|X}
\refstepcounter{table}\label{e-szenarioedit}
\textbf{ID} & \textbf{FA-E \arabic{table}} \\
\hline
TITEL: & Szenario Editor \\
\hline
BESCHREIBUNG: & Benutzeroberfläche zum Editieren des Szenarios mit Darstellung äquivalent zum Benutzer Client des Spiels \\
\hline
BEGRÜNDUNG: & Grundlegende Funktionalität \\
\hline
PRIORITÄT: & 0\\
\hline
ABHÄNGIGKEITEN: & FA-E \ref{e-gui}\\
\end{tabularx}

\begin{tabularx}{16cm}{l|X}
\refstepcounter{table}\label{e-partieedit}
\textbf{ID} & \textbf{FA-E \arabic{table}} \\
\hline
TITEL: & Partie Editor \\
\hline
BESCHREIBUNG: & Im Partie Editor werden die Partie Konfigurationen (Zeit für Phasen, Wahrscheinlichkeiten) in tabellarischer Form editiert. \\
\hline
BEGRÜNDUNG: & Grundlegende Funktionalität \\
\hline
PRIORITÄT: & +\\
\hline
ABHÄNGIGKEITEN: & FA-E \ref{e-gui}\\
\end{tabularx}

\begin{tabularx}{16cm}{l|X}
\refstepcounter{table}\label{e-charedit}
\textbf{ID} & \textbf{FA-E \arabic{table}} \\
\hline
TITEL: & Charakter Editor \\
\hline
BESCHREIBUNG: & Der Charakter Editor zeigt die existierenden Charaktere in der Konfiguration an und erlaubt es, bestehende und neue Charaktere zu editieren. \\
\hline
BEGRÜNDUNG: & Grundlegende Funktionalität \\
\hline
PRIORITÄT: & -\\
\hline
ABHÄNGIGKEITEN: & FA-E \ref{e-gui}\\
\end{tabularx}
